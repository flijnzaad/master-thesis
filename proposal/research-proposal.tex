%TC:macro \todo [ignore]
\documentclass{article}
\usepackage[margin=1.5in]{geometry}
\usepackage[utf8]{inputenc}
\usepackage[english]{babel}

\usepackage{tgpagella}

\usepackage[
    style=authoryear,
    maxbibnames=99,
    natbib
]{biblatex}
\addbibresource{../literature.bib}
\newcommand{\poscite}[1]{\citeauthor{#1}'s (\citeyear{#1})}
\usepackage{csquotes}
% to make for example the references clickable
\usepackage{hyperref}
% puts the relevant thing in front of the reference, if you use \cref{...}
\usepackage[capitalize, noabbrev]{cleveref}

\DeclareCiteCommand{\citeyear}
    {}
    {\bibhyperref{\printdate}}
    {\multicitedelim}
    {}

\DeclareRobustCommand{\VAN}[3]{#2}

\usepackage{fancyhdr}
\pagestyle{fancy}
\fancyhead{}
\fancyhead[L]{Master of Logic thesis \\ Research proposal}
\fancyhead[R]{Flip Lijnzaad \\ \today}
\setlength{\headheight}{23pt}

\usepackage{enumitem}
\usepackage{multicol}
\usepackage{todonotes}
\usepackage{gb4e}

\title{Thesis title}
\author{Flip Lijnzaad}
\date{\today}

\begin{document}

\section*{\centering Did reasoning in humans evolve for purposes of communication?}
\subsection*{\centering Exploring the relation between reasoning and communication from an evolutionary perspective}
\medskip

Two important cognitive skills that are often considered to set humans apart from their evolutionary cousins are on the one hand our remarkable capacity for reasoning, and on the other our profound linguistic abilities.

This thesis aims to take a closer look at each of these cognitive capacities from an evolutionary perspective. Ultimately, the goal will be to establish the nature of the relation between reasoning and communication: did reasoning in humans evolve for the purpose of enabling communication?

% In order to answer this question, we would need to establish a 

\section{A quick glance at causation in evolution}

In order to answer the question of whether one thing evolved for the purposes of, or because of, another, a brief look at what this entails is in order.
I will analyze what it means for something to evolve because of another thing using findings in philosophy of biology (\citet{Ayala99}, and possibly \citet{Allen98} and \citet{Wright76}).
The purpose of this chapter is not to provide a complete overview of theories of causation in evolution, as I consider this to be out of scope. It is merely to provide a foundation to the claims on evolutionary causation later in the thesis.

\section{Reasoning: an evolutionary perspective}

In this chapter, I will delve into reasoning as a cognitive capacity by considering its evolutionary origins.
An influential view on the evolution of reasoning is the \emph{argumentative theory of reasoning}, first posited by \citet{MercierSperber11} and further expanded upon in \citet{Mercier16} and \citet{MercierSperber17}. This view states that the main function (in a biological sense, i.e.\@ the reason it evolved) of reasoning is to devise arguments and evaluate those of others in a dialectical setting. Ultimately, on their view, this mechanism serves to enhance social status.

I will scrutinize this theory and consider competing theories.

\section{Communication: an evolutionary perspective}

\section{The relation between reasoning and communication}

Now that we have sketched a view of the evolution of reasoning and communication in humans, it is time to consider the intersection of these two capacities.

\section*{Summary}

In conclusion, the preliminary table of contents of my thesis will be as follows:
\begin{itemize}
    \item Introduction
    \item[1.] A quick glance at causation in evolution
        \begin{itemize}
            \item What does it mean for a trait to evolve \emph{because of} another trait?
        \end{itemize}
    \item [2.] Reasoning: an evolutionary perspective
        \begin{itemize}
            \item Why might we have evolved to possess the reasoning skills we do?
        \end{itemize}
    \item [3.] Communication: an evolutionary perspective
        \begin{itemize}
            \item Why might we have evolved to possess the communicative skills we do?
        \end{itemize}
    \item [4.] The relation between reasoning and communication
    \item Conclusion
\end{itemize}

\nocite{*}
\printbibliography[title=Preliminary bibliography]

\end{document}
