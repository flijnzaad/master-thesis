%TC:macro \todo [ignore]
\documentclass{article}
\usepackage[margin=1.5in]{geometry}
\usepackage[utf8]{inputenc}
\usepackage[english]{babel}

\usepackage{tgpagella}

\usepackage[
    style=authoryear,
    maxbibnames=99,
    giveninits,
    natbib
]{biblatex}
\addbibresource{../literature.bib}
\newcommand{\poscite}[1]{\citeauthor{#1}'s (\citeyear{#1})}
\usepackage{csquotes}
% to make for example the references clickable
\usepackage{hyperref}
% puts the relevant thing in front of the reference, if you use \cref{...}
\usepackage[capitalize, noabbrev]{cleveref}

\DeclareCiteCommand{\citeyear}
    {}
    {\bibhyperref{\printdate}}
    {\multicitedelim}
    {}

\DeclareRobustCommand{\VAN}[3]{#2}

\usepackage{fancyhdr}
\pagestyle{fancy}
\fancyhead{}
\fancyhead[L]{Master of Logic thesis \\ Research proposal}
\fancyhead[R]{Flip Lijnzaad \\ \today}
\setlength{\headheight}{23pt}

\usepackage{enumitem}
\usepackage{multicol}
\usepackage{todonotes}
\usepackage{gb4e}

\title{Thesis title}
\author{Flip Lijnzaad}
\date{\today}

\begin{document}

\section*{\centering Exploring the relation between reasoning and communication in humans from an evolutionary perspective}

Two cognitive skills that are often considered to set humans apart from their evolutionarily closest relatives are on the one hand our outstanding capacity for reasoning, and on the other our profound communicative abilities.

This thesis aims to take a closer look at each of these cognitive capacities from an evolutionary perspective. Ultimately, the goal will be to establish the nature of the relation between reasoning and communication: did reasoning in humans evolve for the purpose of enabling communication?

\section{A quick glance at causation in evolution}

In order to answer the question of whether one thing evolved because of another, a brief look at what this entails is in order.
I will analyze what it means for one trait to evolve because of another trait using findings from philosophy of biology (\citet{Ayala99}; \citet{Allen98}; \citet{Wright76}).
The purpose of this chapter is not to provide a complete overview of theories of causation in evolution, as I consider this to be out of scope for this thesis. It is merely to provide a foundation to the claims on evolutionary causation later in the thesis.

\section{Reasoning: an evolutionary perspective}

In this chapter, I will delve into reasoning as a cognitive capacity by considering its evolutionary origins.
An influential view on the evolution of reasoning is the \emph{argumentative theory of reasoning}, first posited by \citet{MS11} and further expanded upon in \citet{Mercier16} and \citet{MS17}. This view states that the main function of reasoning (in a biological sense, i.e.\@ the reason it evolved) is to devise arguments and evaluate those of others in a dialectical setting. Ultimately, on their view, this mechanism serves to enhance social status. In this chapter, I will scrutinize this theory and consider competing theories on the evolution of reasoning.

\section{Communication: an evolutionary perspective}

In this chapter, I will give a similar treatment to communication as a cognitive skill. In doing so, I intend to draw from \citet{Tomasello08-origins} and references therein on experimental evidence regarding communicative skills in primates and in human children.

Moreover, I will consider a number of perspectives on how pragmatics in particular may have played a central role in the evolution of language and communication (\citet{Benitez21}; \citet{Moore17}; \cite{Scott-Phillips17-pragmatics}).

If the time allows for it, I would like to discuss perspectives on the 'difference' between language and communication: what role does the one play in the other, and vice versa? Are we the only species that can be said to possess language, and how does communication compare in this respect?

\section{The relation between reasoning and communication}

Now that we have sketched a view of the evolution of reasoning and communication in humans, it is time to consider the intersection of these two capacities: what is the role, if any, that reasoning plays in human communication?

After mostly having considered ontogenetic and phylogenetic origins of the two cognitive capacities (i.e. from developmental and comparative psychology respectively), in this chapter I intend to also consider perspectives from experimental evidence on (socio-)cognitive impairments. For example, empirical findings from individuals with autism-spectrum disorder (ASD) may shine a light on the relation between reasoning simpliciter and pragmatic reasoning (\citet{Geurts19}; \citet{Brosnan16}).

While pragmatics is the subfield of linguistics mostly associated with reasoning (where sometimes pragmatic content itself is even defined to be precisely the content which is inferred), it would be interesting for completeness' sake to briefly consider the cognitive processes underlying language comprehension and production, to be able to judge whether or not these can be considered to make use of reasoning.

Moreover, in this chapter I will consider the concept of \emph{epistemic vigilance}, introduced by \citet{Sperber10}, a phenomenon at the intersection of communication and reasoning which also informs \poscite{MS11} argumentative theory of reasoning.

Finally, in this chapter I will propose my account of the evolutionary relation between reasoning and communication.

\section*{Summary}

In conclusion, the preliminary table of contents of my thesis will be as follows:
\begin{itemize}
    \item Introduction
    \item[1.] A quick glance at causation in evolution
        \begin{itemize}
            \item What does it mean for a trait to evolve \emph{because of} another trait?
        \end{itemize}
    \item [2.] Reasoning: an evolutionary perspective
        \begin{itemize}
            \item Why might we have evolved to possess the reasoning skills we do?
        \end{itemize}
    \item [3.] Communication: an evolutionary perspective
        \begin{itemize}
            \item Why might we have evolved to possess the communicative skills we do?
        \end{itemize}
    \item [4.] The relation between reasoning and communication
        \begin{itemize}
            \item What kind of reasoning, if any, is involved in human communication? Is reasoning necessary for communication?
            \item Did reasoning in humans evolve for purposes of communication?
        \end{itemize}
    \item Conclusion
\end{itemize}

\pagebreak

\nocite{*}
\printbibliography[title=Preliminary bibliography]

\end{document}
