This framework warrants some explanation; although I consider the modularity of the mind to be out of scope for this thesis, Mercier and Sperber's dual system theory is best explained and understood through their views on modularity.
These modules are autonomous in their function, have distinct evolutionary and developmental histories, and they have characteristic inputs, procedures and outputs.
On this massively modular view of the mind, inferences can be performed by many different domain-specific modules. Even inferences that seem to be performed by domain-general module, such as logical inferences, are carried out by domain-specific \emph{metarepresentational modules}: modules that perform inferences on conceptual representations.
Because these conceptual representations may belong to any domain, metarepresentational inferences appear to be domain-general. However, Mercier and Sperber state, this domain-generality is indirect and virtual. In their words,
\begin{quoting}
    Metarepresentational modules are as specialized and modular as any other kind of module. It is just that the domain-specific inferences they perform may result in the fixation of beliefs in any domain.
    \hfill (p.~153)
\end{quoting}

When it comes to the exact definition of these two categories of inferences however, we unfortunately run into some ontological problems.
That is, Mercier and Sperber define intuitive and reflective inferences in two different, seemingly incompatible ways. First, they state that
\begin{quoting}
    intuitive inferences the conclusion of which are the direct output of all inferential modules (including the argumentation module), and reflective inferences the conclusions of which are an indirect output embedded in the direct output the argumentation module \emph{(sic)}
    \hfill (pp.~155--156)
\end{quoting}
In other words, the \emph{conclusion} of an intuitive inference is the direct output of an inferential module, and the \emph{conclusion} of a reflective inference is an indirect output of the argumentation module. However, later they state that
\todo[inline]{what follows from these citations? unclear}
\begin{quoting}
    Intuitive inferences are the direct output of many different modules. Reflective inferences are an indirect output of one of these modules.
    \hfill (p.~156)
\end{quoting}
We will return to these problems in \cref{sec:ont-atr}.

