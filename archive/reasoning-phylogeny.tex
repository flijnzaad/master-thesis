\section{Reasoning in non-human animals}
\label{sec:reason-phylogeny}

Before we consider experimental findings on reasoning or reasoning-like abilities in non-human animals, we first need to consider one aspect of Mercier \& Sperber's definition of reasoning -- consciousness -- in more detail.

\subsection{Consciousness}

Mercier \& Sperber take the defining difference between reasoning and inference to be whether or not the 'reasons' are \emph{consciously} attended to. However, it may not be as straight-forward as it seems to grasp this concept exactly. Moreover, once we start to consider animal minds, it becomes especially tricky to pinpoint what consciousness means in this definition.

In a seminal \citeyear{Block95} paper, Ned Block proposed a distinction between two types of consciousness: access-consciousness (or A-consciousness) and phenomenal consciousness (or P-consciousness). A-consciousness then concerns the ability to access one's own mental states, whereas P-consciousness concerns "the qualitative nature of experience" \citep[p.~52]{Andrews15}: what it 'feels like' to be in a certain mental state. \todo{Is this an accurate paraphrase? Check the sources again}
Block noted the relation between A-consciousness and reasoning, and stated that A-consciousness is fundamental to reasoning:
\begin{quoting}
    It is of the essence of A-conscious content to play a role in reasoning
\hfill (p.~232)
\end{quoting}
Although Mercier \& Sperber do not mention Block's distinction, it is presumably A-consciousness rather than P-consciousness that their definition of reasoning employs. In other words, in order to reason one must be able to have conscious access to the mental representations resulting from their inferential mechanisms.
Unfortunately for our current purposes however, studies into animal consciousness are primarily interested in the extent to which non-human animals possess \emph{phenomenal} consciousness\footnote{It may be noted that in his original paper, Ned Block wanted to leave open the possibility of non-human animals possessing access-consciousness \citep{Block95}.}, not access consciousness \citep{Andrews15, Carruthers18}.
Josep \citet{Call06} does discuss the capacity for \emph{reflection} in non-human animals; however, its relation to access consciousness and to reasoning (in the definition of Mercier \& Sperber) is unclear\todo{Is it? Compare this to \citet{MS09} on reflective inference}. Call discusses empirical evidence that some animals know when they are uncertain about something, that some monkeys can know if they have forgotten something, and that apes know what they have not seen. \todo{Re-paraphrase}
\todo{Missing conclusion}

\subsection{Inference and reasoning in animals}

Many non-human animals are thought to possess the cognitive capacity for inference. For instance, monkeys are capable of performing disjunctive syllogisms \citep{Ferrigno21}; monkeys, birds, and some fish are capable of transitive inference \citep{Premack07}\footnote{Although Premack remarks that no non-human animals possess the concept of monotonicity, concluding that their capacity for transitive inference must rely on a 'hard-wired mechanism' rather than being 'based on reasoning' (p.~13864).}.

