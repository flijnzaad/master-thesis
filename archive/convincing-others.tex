\section{How is convincing others advantageous?}

\todo[inline]{This section should probably be moved to Chapter 2, utility of communication}

The account of human communication that underpins the argumentative theory of reasoning (see \citet{Sperber01, Sperber10})
entails that for human communication to have stabilized over time, it must have been evolutionarily advantageous to both the sender and the receiver. I agree that it is reasonable to assume that -- since it depends on the participation of two parties -- communication would collapse if either party was not experiencing any benefits from the action. Let us now briefly discuss these benefits.

The possible benefit of communication to the receiver is for them to gain information (to the extent that it is genuine information).
\todo[inline]{This is not really accurate: see notes of \citet{Michaelian13}. Also, then why is gaining information beneficial?}
This benefit seems straight-forward enough: communication can enable us to gain information about the world in a similar way to how direct perception, or inference on the basis of held beliefs, yield information to us. However, we should be careful to regard communication as 'cognition by proxy', since the sender also stands to gain from communication and thus has their own interests as well \citep{Sperber01}.

On the other side of the coin, \citet{Sperber01} describes the benefits of communication to the sender as follows:
\begin{quoting}
    From the point of view of producers of messages, what makes communication, and testimony in particular, beneficial is that it allows them to have desirable effects on the receivers' attitudes and behavior. By communicating, one can cause others to do what one wants them to do and to take specific attitudes to people, objects, and so on.
    \hfill (p.~404)
\end{quoting}
The details of this point in particular require some explication in order to understand their force within the evolutionary story.

\todo[inline,caption={}]{
    To do:
    \begin{itemize}
        \item Check \citet{MS11} for a possible quote on benefits of convincing others: how do they describe it there?
        \item (It might turn out that the benefits of persuasion are already discussed in Chapter 2 once it's revised.)
    \end{itemize}
}

% On the flip side of these benefits of convincing others are the purported disadvantages to the receiver that come with being deceived by the sender. \citet{Sperber10} state that
% \begin{quoting}
%     being accidentally or intentionally misinformed (\ldots) may reduce, cancel, or even reverse [the gains received from communication with others]
%     \hfill (p.~360)
% \end{quoting}
