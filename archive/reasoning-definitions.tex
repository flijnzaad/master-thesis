\subsection{Other definitions of reasoning}
Let us now mention and compare two other definitions of reasoning, one due to cognitive scientist Vinod Goel and another due to philosopher Gilbert Harman.

\subsubsection{Goel}
In his \citeyear{Goel22} book "Reason and Less", Goel proposes and develops an account of \emph{tethered rationality}, where four different cognitive systems work in tandem to determine human behavior. The four systems, from evolutionarily 'oldest' to 'youngest', are the autonomic, instinctual, associative, and reasoning systems. These systems evolved 'on top of' each other, and are tightly integrated with one another --- tethered, as Goel calls it.
Goel then describes (and in doing so more or less defines) reasoning as follows:
\begin{quoting}
    Reasoning is a system for generating new beliefs from observations and/or existing beliefs and maintaining consistency of our beliefs (i.e., mental representations of the world).
    \hfill (p.~114)
\end{quoting}
In other words, according to Goel reasoning has a dual role: it generates inferences (in the way Mercier \& Sperber define them), and it ensures that our beliefs remain internally consistent or coherent.

\subsubsection{Classical conceptions of reasoning's function}

Classical theories of reasoning, building on centuries of philosophical work, maintain that the function of reasoning is to enhance or support individual cognition \citep{MS11}.

\begin{quoting}
    one of the functions of System 2 is to monitor the quality of both mental operations and overt behavior
    \hfill \citep[p.~699]{Kahneman03}
\end{quoting}

\begin{quoting}
    The analytic system is primarily a control system focused on the interests of the whole person. It is the primary maximizer of an individual's \emph{personal} goal satisfaction.
    (p.~64)
\end{quoting}

\citet{Goel22} also more or less adheres to this classical account of what the function of reasoning is, stating that
\begin{quoting}
    the function of the reasoning mind is to allow for greater flexibility in individual behavior, and thus more finely tuned responses to environmental stimuli than can be accommodated by the autonomic, instinctive, and associative minds.
    \hfill (p.~114)
\end{quoting}

Mercier \& Sperber state that their definition of reasoning is the one that is most commonly adhered to in the psychology of reasoning. However plausible, this claim is difficult to verify since most work in psychology of reasoning does not explicitly define reasoning within the frame of their research. Especially when one moves from psychology of reasoning to the neighboring discipline of animal cognition, what is understood by 'reasoning' becomes fuzzy. As an example, \citet{Andrews15} and \citet{Call06} discuss animal reasoning at length, yet they never explicate the definition of reasoning they adhere to.
