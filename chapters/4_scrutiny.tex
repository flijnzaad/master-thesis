\chapter{The ATR closely inspected}
\label{ch:scrutiny}

Although Mercier and Sperber's story about the evolution and function of reasoning as it relates to communication is quite convincing on the face of it, it seems to leave a number of details underexposed. In this chapter, I will scrutinize these details and highlight areas for improvement.

In \cref{sec:comm-func-scrutiny}, I will critically evaluate Mercier and Sperber's purported function of communication by stacking it against the picture painted in \cref{sec:comm:function}.
\cref{sec:EV-scrutiny} will see us discussing epistemic vigilance at length by considering a critical response paper to \citet{Sperber10}, and Sperber's reply to this criticism.
Lastly, in \cref{sec:ont-atr} I will detail some metatheoretical frustrations regarding the generally imprudent way in which Mercier and Sperber define and use the concepts they introduce.

\section{The function of communication: revisited}
\label{sec:comm-func-scrutiny}

As we have seen in \cref{sec:comm:function}, I argue that the function of communication is to facilitate cooperation. Humans are a uniquely cooperative species: we depend on each other for our survival, and communication plays a crucial role in enabling this. On this view, it is critical to consider the perspective of the whole social group, not just the individuals interacting with each other.

The function of communication as it transpires from the work of Mercier and Sperber approaches it from a different angle.
As we have seen in \cref{ch:atr}, they agree to some extent with my view, as they note the importance of small group cooperation in evolutionary history and the role communication plays in this cooperation \citep[p.~60]{MS11}.
However, for the most part, their theory divorces itself from this broader context, as they choose to focus only on how communication benefits senders and receivers.

Sperber argues that "the function of communication presents itself differently for communicator and audience" \citep[p.~411]{Sperber01}. On his view, it is advantageous for senders to deceive others, or at the very least persuade them, because this may have 'desirable effects' on the receiver. On the side of the receiver then, communication is advantageous because it allows the receiver to gain information.
While both of these claims may indeed be intuitively very plausible, they are underspecified from an evolutionary perspective. What exactly are these 'desirable effects' of deception, and how might they improve an individual's fitness? What is exactly the advantage of gaining information? Multiple key steps
are missing from the causal chain because they are taken for granted as uncontroversial, weakening the overall account.

Because Mercier and Sperber's theory is disassociated from the evolutionary context of human cooperation, the evolutionary story becomes somewhat blurry.
Catarina Dutilh Novaes also expresses reservations regarding this aspect of the ATR in her \citeyear{Novaes18} review of \citet{MS17}, criticizing Mercier and Sperber's focus on the individual and accompanying dismissal of the group-level perspective \citep[\S 3.3]{Novaes18}.

This failure to incorporate the essential background of human cooperation does, however, not count as a definitive strike against the theory: I conjecture that the ATR can be integrated with the account of human cooperation we saw in \cref{sec:comm:function}. The ATR and this account are not incompatible per se, but integrating them does most likely require small modifications or additions to the ATR. In particular, the function of communication for senders and receivers needs to be linked to the cooperative function of communication to complete the evolutionary account.
Doing this would fortify the ATR: strongly tethering the theory to this cooperative context would make the story as a whole more plausible.

\section{Epistemic vigilance: revisited}
\label{sec:EV-scrutiny}

It appears that the ATR rests on epistemic vigilance having evolved in humans. Therefore, this concept deserves some critical analysis.
In this section, I will first discuss the position of the notion of epistemic vigilance within the argumentative theory of reasoning. Then, I will discuss a critical response to \citet{Sperber10} by Kourken \citet{Michaelian13}, and turn Dan \poscite{Sperber13} response to the criticism. I will go through Michaelian's criticisms to Sperber et al.
\todo{Rewrite this summary}
Lastly, I will conclude with the consequences of this discussion for the argumentative theory of reasoning.

Before we start, let me first briefly outline Michaelian's paper and Sperber's response to it. In short, \citeauthor{Michaelian13} spells out some of the assumptions that he claims are inherent to Sperber et al.'s argument for epistemic vigilance. He contrasts these assumptions with empirical findings from deception detection research. In doing so, he concludes that epistemic vigilance does not play a major role in ensuring the stability of communication --- rather, he argues, the majority of the burden befalls \emph{speaker honesty}. In the same \citeyear{Sperber13} issue of \emph{Episteme}, Dan Sperber provides a response to the criticisms of Michaelian. Based on the argument Michaelian put forward, Sperber further explains and sometimes tweaks some of the claims he and his colleagues made in the original 2010 paper. He attributes a considerable portion of Michaelian's criticism to a misunderstanding, which effectively nullifies Michaelian's import of empirical findings.

\subsection{Epistemic vigilance and the ATR}
\label{sec:epi-vigil-atr}

From our discussions of \citet{Sperber10} and \citet{MS11} in \cref{ch:atr}, it transpires that the notion of epistemic vigilance plays some role of importance in the argumentative theory of reasoning. However, in order to assess how gripes with epistemic vigilance as a concept would impact the ATR's credibility, we should consider the exact position of epistemic vigilance within the ATR.

To start off, the paper coining the ATR has the following to say about epistemic vigilance:

\begin{quoting}
    For communication to be stable, it has to benefit both senders and receivers (\ldots) To avoid being victims of misinformation, receivers must therefore exercise some degree of what may be called epistemic vigilance (Sperber et al. 2010).
    \hfill \citep[p.~60]{MS11}
\end{quoting}
In other words, the stability of communication depends on its benefits for both the sender and the receiver. According to Mercier \& Sperber, the benefits for the receiver depend on, or are mediated by, epistemic vigilance.

The evolutionary arms race (recall \cref{sec:Sperber01}) is foundational to the argumentative theory of reasoning. Epistemic vigilance is one of the steps in this arms race, having evolved as a defense against misinformation and deception. Consequently, epistemic vigilance is a critical component of the argumentative theory of reasoning: for, if receivers were not vigilant, senders need not have the ability to display the coherence of their arguments in order to be able to convince receivers. Thus, any serious issues with epistemic vigilance as a concept would constitute a significant blow to the evolutionary arms race and consequently to the ATR as a whole.

\todo{Old transition here: modify}
Let us now go through the points of scrutiny that Kourken \citet{Michaelian13} subjects Sperber and colleagues' epistemic vigilance to, and consider \poscite{Sperber13} response to the scrutiny.

\subsection{What is epistemic vigilance exactly?}
\label{sec:EV-def}

\citet{Sperber13} chalks up a considerable share of the disagreement between him and Michaelian to an alleged misunderstanding between the two authors on the exact definition of epistemic vigilance:
\begin{quoting}
    Michaelian seems to attribute to us the view that ‘epistemic vigilance is a matter of processes devoted to screening out incoming false information on the basis of available behavioural cues’. Showing that vigilance in this narrow sense is not efficient would, he holds, be quite damaging to our conjecture. This is a misunderstanding.
    \hfill \citep[p.~65]{Sperber13}
\end{quoting}
While I agree with Sperber that Michaelian seems to attack a narrower version of epistemic vigilance, I do not blame Michaelian for the misunderstanding. Sperber and colleagues are (intentionally or unintentionally) vague in their original paper about exactly what epistemic vigilance is.
Their flexible use of the notion of epistemic vigilance might not be inherently problematic, but given the central role epistemic vigilance plays in much of Sperber and Mercier's work, I believe we are long overdue an exact definition of epistemic vigilance. In this section, I will gather the details that together may constitute a definition of epistemic vigilance according to \citet{Sperber10} and \citet{Sperber13}.

Let us first consider the ontological status of epistemic vigilance. Although on the face of it, one may want to construe epistemic vigilance as a set of mechanisms for filtering incoming information, \citet{Sperber10} describe humans as having evolved a 'suite of mechanisms' \emph{for} -- not \emph{of} -- epistemic vigilance. This leaves open the question of what epistemic vigilance itself could be. One candidate is a cognitive capability or skill, which seems to be supported by \citet[p.~60]{MS11} who describe epistemic vigilance as something that can be 'exercise[d to] some degree'. Moreover, \citet[\S 5]{Sperber10} import empirical evidence on the development of epistemic vigilance in children, which would point to vigilance being a capacity or skill as well.
Also possibly pointing us in the right direction, Sperber and colleagues contrast vigilance with blind trust:
\begin{quoting}
    Vigilance (unlike distrust) is not the opposite of trust; it is the opposite
of blind trust
    \hfill \citep[p.~363]{Sperber10},
\end{quoting}
implying that trust and vigilance are of the same kind. Although the ontological status of trust does not seem to transpire from their \citeyear{Sperber10} paper, I would argue that this can be excused, as trust has been described by others as difficult to define \citep{Simpson12,McKnight00}.
\todo[inline]{Possibly remove this discussion of trust, Karolina had some reservations}

A possibly useful perspective on the definition of epistemic vigilance comes from sleep science, a field of research related to neurology and neurophysiology. \citet{VanSchie21} define vigilance \emph{per se} as follows:
\begin{quoting}
    Vigilance is defined as the capability to be sensitive to potential changes in one's environment, ie the capability to reach a level of alertness above a threshold for a certain period of time rather than the state of alertness itself.
    \hfill (p.~175)
\end{quoting}
It may not be immediately obvious how a definition from sleep science may at all be applicable in our attempt to define epistemic vigilance. However, I do believe that we may assume some degree of overlap between neurology and psychology, and that epistemic vigilance must relate in some way to vigilance \emph{per se}.

All things considered, I believe epistemic vigilance would be best defined as \emph{the capability to be sensitive to the trustworthiness of communicated information and informants}.
\todo[inline]{Tweak this definition?}

Next, let us consider the specifics of the processes in the 'suite of mechanisms for epistemic vigilance'. Just like Sperber and colleagues are somewhat vague about the ontological status of epistemic vigilance, they are rarely straight-forward about the exact mechanisms they consider to fall under the umbrella of epistemic vigilance. This however may not be as easily forgiven as their opacity surrounding epistemic vigilance's ontological status. The nature of the mechanisms for epistemic vigilance is crucial for the cost-benefit analysis that underlies the evolutionary argument that they are making. \citet{Michaelian13} already implictly picks up on this (and in my opinion, \citet{Sperber13} does little to alleviate these concerns).
\todo[inline]{Discussion of epistemic vigilance, dual processes, costs and benefits. In short: they're vague about the costs, and we see with the strong/weak reading stuff that they're also vague about the benefits. Ironic and/or hypocritical}

Lastly, let us consider the relation between epistemic vigilance and reasoning. \citet{Sperber10} describe reasoning as "a tool for epistemic vigilance, and for communication with vigilant addressees" (p.~378). This would imply that reasoning is an item in the suite of mechanisms for epistemic vigilance. However, in a similar way to how Sperber and colleagues are unspecific about epistemic vigilance's contribution to the stability of communication, they are unspecific about reasoning's contribution to epistemic vigilance.
\todo[inline]{Conclusion to this paragraph?}

All in all, Sperber and colleagues' definition and description of epistemic vigilance leaves a lot to be desired when it comes to the details. We return to this issue in \cref{sec:ont-atr} when we consider the broader picture of metatheoretical issues with the ATR as a whole.

\subsection{Strong vs. weak readings of the argument for epistemic vigilance}

In their \citeyear{Sperber10} paper, Sperber and colleagues hint at different ways to interpret their argument --- at different roles to attribute to epistemic vigilance:

\begin{quoting}
    It is because of the risk of deception that epistemic vigilance may be not merely advantageous but indispensable if communication itself is to remain advantageous.
\hfill (p.~360)
\end{quoting}
Sperber et al. seem to prefer to remain agnostic (or -- put differently -- vague) about the exact role of epistemic vigilance in explaining the stability of communication: is epistemic vigilance "merely advantageous", or is it "indispensable"? In other words, is it just the case that the benefits of epistemic vigilance outweigh its costs, leading to its having evolved in humans; or, stronger, would communication collapse if it were not for receivers' epistemic vigilance?
\example{Add an illustrative, intuitive example about this}
\todo[inline]{Add in more quotes that point to Sperber's vagueness?}
These different readings of epistemic vigilance are also implicit in the following statement Sperber and colleagues make:
\begin{quoting}
    People stand to gain immensely from communication with others, but this leaves them open to the risk of being accidentally or intentionally misinformed, which may reduce, cancel, or even reverse these gains.
\hfill (p.~360)
\end{quoting}
If being misinformed reduces or cancels the benefits an addressee receives from the communicative event, then it would stand to reason that it would be advantageous for the addressee to be epistemically vigilant so as to maintain the positive benefits of communication. If misinformation however reverses the gains one receives from communication, then epistemic vigilance would be \emph{necessary} for the receiver in order to not be negatively affected.

Kourken Michaelian picks up on these different views of epistemic vigilance in his response paper (\citeyear{Michaelian13}). He distinguishes between a strong and weak reading of Sperber et al.'s argument for epistemic vigilance, in line with this distinction between vigilance being indispensable or advantageous, respectively.
He then outlines the assumptions that underpin each reading of the argument, and purports to show that these assumptions are unfounded, or at the very least too strong.
According to Michaelian, the strong reading carries with it the assumption that dishonesty is sufficiently prevalent to necessitate vigilance on the side of the receiver; since, if epistemic vigilance is indispensable, non-vigilance must then yield a "dramatic reduction in the fitness of receivers" \citep[p.~39]{Michaelian13}. He presents empirical findings that maintain that lying is infrequent \citep{Serota10}, and that therefore the strong reading of Sperber et al.'s argument cannot hold.
Further, he argues that the weak reading of the argument carries with it the assumption that the benefits of epistemic vigilance outweigh its costs.\todo[inline]{What is Michaelian's conclusion from this?}

In his response to Michaelian, \citet{Sperber13} states that "I now believe that we could and should have been even less definite" (p. 63) in recognizing a stronger and weaker reading of their argument. Sperber argues that in the recognition of the two readings of the argument, one mistakenly regards communication as a static enterprise. The degree to which vigilance is advantageous or even indispensable to a receiver, varies greatly from situation to situation, he argues:
\begin{quoting}
    The benefits of vigilance may be negligible in some communicative interactions and essential in other interactions. All I feel confident to say is that, without vigilance, human communication would be a very different and probably much more restricted affair.
    \hfill \citep[p.~63]{Sperber13}
\end{quoting}
This conclusion, however plausible, leaves a lot to be desired when it comes to the details for the evolutionary story, and in particular for the picture of the evolutionary arms race.
It is especially puzzling that Sperber, in the same 2013 paper, emphasizes how crucial costs and benefits are to the whole story:
\begin{quoting}
    The probability of a biological trait evolving is contingent on its costs-benefits balance. Only if the benefits are greater than the costs, is it likely to evolve at all, and it is likely to evolve in a manner that, within the local range of possibilities, optimizes this balance.
    \hfill \citep[p.~62]{Sperber13}
\end{quoting}
How can this importance ascribed to the cost-benefits balance be reconciled with his non-committal stance on how beneficial epistemic vigilance actually is?

In the original \citeyear{Sperber10} paper, Sperber and colleagues maintain that it is sufficient that communication is \emph{on average} advantageous to both parties:
\begin{quoting}
    The fact that communication is so pervasive despite [the risk of misinformation] suggests that people are able to calibrate their trust well enough to make it advantageous on average to both communicator and audience
    \hfill \citep[p.~360]{Sperber10}
\end{quoting}
However, I believe that this does not constitute a full solution to the problem. 
In order to provide a satisfactory answer, one would need to provide a detailed sketch of costs and benefits.
\todo[inline]{This point is a bit incoherent as of now: more thinking}

\subsection{Honesty or dishonesty as prior}

\todo[inline]{Is honesty vs. dishonesty the same debate as trust vs. vigilance? Make sure you correctly interpret the quotes and POVs of the authors}

Ultimately, a fundamentally different outlook on human communication and cooperation seems to transpire from the accounts of Sperber, Mercier and colleagues on the one hand, and Michaelian on the other.

Sperber's 'evolutionary arms race' account takes dishonesty as prior.
In short, dishonesty creates vigilance, vigilance then creates honesty, and honesty creates trust. As Sperber and colleagues write,
\begin{quoting}
    We could not be mutually trustful \emph{unless} we were mutually vigilant.
    \hfill \citep[p.~364]{Sperber10}
\end{quoting}
In other words, vigilance is prior to trust. Vigilance is necessitated by dishonesty, and honesty and trust are contingent on vigilance.
For this account to work, one must convincably argue for the evolutionary benefits of dishonesty; if dishonesty is the first step in the evolutionary arms race, it must be beneficial \emph{per se}.

Michaelian, on the other hand, refutes this account and instead proposes that communication is stable just because speakers are honest; in other words, honesty is prior. This view is consistent with Michael Tomasello's account of the evolution of human cooperation, as we have seen in \cref{sec:comm:cooperation}. In discussing how children altruistically share information with others, Tomasello writes
\begin{quoting}
    Of course children soon learn to lie also, but that comes only some years later and presupposes preexisting cooperation and trust. If people did not have a tendency to trust one another’s helpfulness, lying could never get off the ground.
    \hfill \citep[p.~21]{Tomasello09}
\end{quoting}
In other words, trust is prior to deception: deception could not have emerged without trust.
Analogously to Sperber's account, it then remains for Michaelian, Tomasello and the like to show how honesty is by itself beneficial.

As is often the case with 'chicken-or-egg' problems such as these, it could very well be that the truth lies somewhere in the middle. Perhaps the answer to the question is not as straight-forward as choosing between the options of 'we are fundamentally vigilant' and 'we are fundamentally trustful'. For now, I must say that the latter is more plausible than the former, due to Tomasello's evolutionary account being a more coherent, complete and thus convincing one than Mercier and Sperber's arms race account.

\todo[inline]{Maybe add some more ramblings}

\subsection{Conclusion}

\todo[inline]{For example: mention the issues with emphasizing cost-benefit analysis but not making committed claims as to the details of either the costs or the benefits. So the evolutionary story is not as strong as hoped.}

\section{Metatheoretical problems of the ATR}
\label{sec:ont-atr}

One issue that we have seen come up in \cref{sec:MS09} as well as \cref{sec:EV-def}, is that Mercier and Sperber are oftentimes intentionally or unintentionally vague about particularities of their theory.
For one, \citet{MS09} leave a lot to be desired with regards to the ontological details of intuitive and reflective inference.
% They introduce this dual system approach because they are dissatisfied with classical dual system theories of reasoning, accusing these of relying on "a bundle of constrasting features" rather than offering a "principled distinction" \citep[p.~149]{MS09}, which therefore allows these theories some leeway in ascribing characteristic to the two systems.
There are moreover small differences between the way they characterize these concepts in different writings \citep{MS09, MS11}, which contributes to an overall impression of the authors being unsure of how they want to define their concepts.

As we discussed in \cref{sec:EV-def}, \citet{Sperber10} are imprecise on the ontological status of epistemic vigilance, the contents of the suite of mechanisms that contribute to vigilance, and the relation between reasoning and epistemic vigilance.
What's more, the relation between coherence checking and reasoning is not clear; \citet{Sperber01} offers two observations that may lead us in the right direction:
\begin{quoting}
    Coherence checking involves a high processing cost, it cannot be done on a large scale because it would lead to a computational explosion, and it is fallible.
    \\ (\ldots) \\
    Coherence checking (\ldots) involves metarepresentational attention to logical and evidential relationships between representations
    \hfill \citep[p.~410]{Sperber01}
\end{quoting}
Later on, in \poscite{MS11}'s response to the in-paper peer commentary, they are a bit more explicit in the relation between coherence checking and reasoning, stating that
\begin{quoting}
    Coherence checking, we argue, is the second major heuristic used in filtering communicated information, and is at the basis of reasoning proper.
    \hfill \citep[p.~96]{MS11}
\end{quoting}
(the first heuristic being trust calibration). However, this is still 
\todo{Unfinished sentence, lol}

\todo[inline]{Bottom two paragraphs have good things to say, but could be formulated more convincingly}

All things considered, the way in which Mercier and Sperber define and use their terminology is vague at its best and confusing at its worst. The problem may partly stem from them changing their mind on details of the ATR as it developed into the full-fledged theory as expounded in \citet{MS17}.
For example, in their \citeyear{MS09} paper they explicate their dichotomy between intuitive and reflective inference, but slightly confusingly, they never use the term 'reflective inference' in \citet{MS11} anymore. This gives the appearance that something changed about their views between \citeyear{MS09} and \citeyear{MS11}. While this might very well not be the case, it does leave a reader wonder.
\todo[inline]{Also in \citet{MS17} I think they change a lot of terminology: for example, they're talking about the 'reason module' instead of 'argumentation module'. Possibly add things here from \citet{MS17}}

This is of course understandable, as slight modifications and updates to one's theory constitute a natural part of the scientific process. However, Mercier and Sperber do not explicate these changes in terminology or other details of their theory, which leaves the reader confused.

Regarding the overall vagueness of details of the ATR, the following quote from Karl Popper's \emph{Conjectures and Refutations} comes to mind:
\begin{quoting}
    by making their interpretations and prophecies sufficiently vague they were able to explain away anything that might have been a refutation of the theory had the theory and the prophecies been more precise. In order to escape falsification they destroyed the testability of their theory. It is a typical soothsayer's trick to predict things so vaguely that the predictions can hardly fail: that they become irrefutable.
    \citep[p.~37]{Popper62}
\end{quoting}
I believe that accusing Mercier and Sperber's theory of being unfalsifiable, and consequently unscientific, would be a step too far. Besides, I consider the metatheoretic analysis required for such a serious accusation to be out of scope for this thesis. All I feel confident to say is that the ATR has considerable metatheoretical issues that demand attention.
% that keep it from being more than an intuitively plausible speculation.

As in \cref{sec:comm-func-scrutiny}, I will argue that these problems do not constitute a fatal flaw of the ATR; but they do leave a lot to be desired from an otherwise intuitively convincing theory.
\todo{Is this comment up to date?}

\section{Conclusion}

\todo[inline]{Add chapter summary, concluding remarks, and final 'verdict' of the ATR}
