\chapter{The argumentative theory of reasoning closely inspected}
\label{ch:scrutiny}

Although Mercier \& Sperber's story seems very convincing on the face of it, the theory seems to leave a number of details undiscussed. Moreover, the argumentative theory of reasoning makes some assumptions that require some exposition.
In this chapter, I will scrutinize a number of assumptions of the argumentative theory of reasoning and make explicit some details that require spelling out.

\section{What is reasoning?}

\citet{MS11} take the following to be their definition of reasoning:

\begin{quoting}
    \emph{Reasoning}, as commonly understood, refers to a very special form of inference at the conceptual level, where not only is a new mental representation (or \emph{conclusion}) consciously produced, but the previously held representations (or \emph{premises}) that warrant it are also consciously entertained.
    \hfill (p.~57)
\end{quoting}

In this definition, \emph{inference} is taken to be "the production of new mental representations on the basis of previously held representations" (p.~57).

Mercier \& Sperber maintain that their definition of reasoning is the one that is most commonly adhered to in the psychology of reasoning.
They go on to state that this definition of reasoning excludes non-human animals and preverbal children from the realm of reasoners.

However, much more remains to be said about this very definition. Let us first briefly consider other definitions of reasoning, and then place some question marks around the position of Mercier \& Sperber's definition within their argumentative theory.

\subsection{Definitions of reasoning}

\todo[inline]{Mention definitions of \citet{Harman86} and maybe others (perhaps references from \citet[\S~1.2, par.~2]{MS11}), and compare these to that of Mercier \& Sperber}

\subsection{Accusations of circularity}

Especially through reference to the terms 'premise' and 'conclusion' within the definition, the definition of reasoning given by Mercier \& Sperber is reminiscent of definitions of argumentation. In a \citeyear{Sperber01} paper, a precursory work to \citet{MS11}, Sperber uses the following definition of argumentation:
\begin{quoting}
    the defense of some conclusion by appeal to a set of premises that provide support for it
    \hfill (p.~401)
\end{quoting}
and somewhat less precisely, \citet{MS11} define arguments as
\begin{quoting}
    representations of relationships between premises and conclusions
    \hfill (p.~58)
\end{quoting}

Judging by these definitions of argumentation and reasoning, what is reasoning if not internalized argumentation? Or, in the same vein, what is argumentation if not externalized reasoning? And, if this is the case, does this not render the argumentative theory of reasoning circular? For then the theory would state that the main function of internalized argumentation is argumentative.
\todo[inline,caption={}]{To do:
    \begin{itemize}
        \item Closely read \S 1.1 of \citet{MS11} for their ontological considerations on inference and argument
        \item Work out the details of what the definitions entail, and spell out what "internalized argumentation" would entail
        \item Work out whether this is indeed a circularity in the theory
        \item And whether that is a fatal flaw of this theory
    \end{itemize}
}

\section{The necessity of epistemic vigilance to survival}
\label{sec:epi-vigil-crit}

Since the argumentative theory of reasoning depends on epistemic vigilance having evolved in humans, this concept itself deserves some extra attention.
In this section, I will discuss in more detail how epistemic vigilance could have been necessary or critical to the survival of humans. This partly ties into the discussion in \cref{sec:arms-race} as well.\todo{I will have to see if this order makes sense, or if I maybe should merge these two sections}
Moreover, I will go into some of the criticisms leveraged at epistemic vigilance by \citet{Michaelian13} and examine their merit.

\todo[inline,caption={}]{
    To do:
    \begin{itemize}
        \item Consider points of scrutiny to \citet{Sperber10} from notes
        \item Recap criticisms of \citet{Michaelian13} here
        \item Reinforce or rebut these criticisms
        \item Consider the consequences (if any) for the argumentative theory of reasoning
    \end{itemize}
}

\section{How is convincing others advantageous?}

The account of human communication that underpins the argumentative theory of reasoning (see \citet{Sperber01, Sperber10})
entails that for human communication to have stabilized over time, it must have been evolutionarily advantageous to both the sender and the receiver. I agree that it is reasonable to assume that -- since it depends on the participation of two parties -- communication would collapse if either party was not receiving benefits of the action.

\todo[inline]{The following paragraph is horribly written}
The possible benefit of communication to the receiver is to gain information (to the extent that it is genuine information).\todo{Citation needed?}
\todo{Does this need to be spelled out more still?}
This benefit seems straight-forward enough: communication can enable us to gain information about the world similarly to how direct perception does. However, we should be careful to regard communication as 'cognition by proxy', since the sender also stands to gain from communication and thus has their own interests as well \citep{Sperber01}.

On the other side of things, \citet{Sperber01} describes the benefits of communication to the sender as follows:
\begin{quoting}
    From the point of view of producers of messages, what makes communication, and testimony in particular, beneficial is that it allows them to have desirable effects on the receivers' attitudes and behavior. By communicating, one can cause others to do what one wants them to do and to take specific attitudes to people, objects, and so on.
    \hfill (p.~404)
\end{quoting}
Although these benefits of communication to the sender of a message seem convincing enough, the details of this remain to be spelled out.

\todo[inline,caption={}]{
    To do:
    \begin{itemize}
        \item Check \citet{MS11} for a possible quote on benefits of convincing others: how do they describe it there?
        \item Make explicit what details need to be spelled out and why this is not sufficient as-is
    \end{itemize}
}

On the flip side of these benefits of convincing others are the purported disadvantages to the receiver that come with being deceived by the sender. \citet{Sperber10} state that
\begin{quoting}
    being accidentally or intentionally misinformed (\ldots) may reduce, cancel, or even reverse [the gains received from communication with others]
    \hfill (p.~360)
\end{quoting}

\section{The evolutionary 'arms race' between communicator and addressee}
\label{sec:arms-race}

\citet{Sperber01} sketches out the steps in what he calls the 'evaluation-persuasion arms race', i.e.\@ the causal chain of events that has led to our current argumentation and reasoning mechanisms.\todo{Weird definition}
He argues that the first step in this arms race is for the addressee to develop coherence checking, as a defense against the risks of deception by the communicator. The second step is then for the communicator to anticipate this coherence-checking by overtly displaying the coherence of their message to their addressee, which requires argumentative form. The next steps are on the side of the addressee to develop skills for examining these displays of coherence (i.e., arguments), and on the side of the communicator to 'improve their argumentative skills'.

In this section I will work out this sketch of the evolutionary arms race in more detail. In doing so, I will supplement and possibly amend this account with findings from \citet{Sperber10}, \citet{MS11}, \citet{Reboul17}, and the conclusion of my discussion of \citet{Michaelian13} in \cref{sec:epi-vigil-crit}.

\todo[inline]{I believe this section will build upon multiple other sections in this chapter (criticisms of epistemic vigilance, benefitis of persuasion, etc.), so it should come later.}

\section{Motivations, dispositions and a common interest in the truth}

Throughout their \citeyear{MS11} article, Mercier \& Sperber allude to the dispositions\todo{'Dispositions' is not meant as a technical term, and I don't think it is; is it?}
of interlocutors in argumentative settings\footnote{Emphasis in quotes added.}:
\begin{quoting}
    This experiment illustrates the more general finding stemming from this literature that, \emph{when they are \textbf{motivated}, participants are able to use reasoning to evaluate arguments accurately}
    \hfill (p.~61)
\end{quoting}
\begin{quoting}
    Most participants are \textbf{willing} to change their mind only once they have been thoroughly convinced that their initial answer was wrong
    \hfill (p.~63)
\end{quoting}
\begin{quoting}
    this [experimental finding] should not be interpreted as revealing a lack of ability but only a lack of \textbf{motivation}. When participants \textbf{want} to prove a conclusion wrong, they will find ways to falsify it.
    \hfill (p.~65)
\end{quoting}
\begin{quoting}
    people are good at assessing arguments and are quite able to do so in an unbiased way, \textbf{provided they have no particular axe to grind}. In group reasoning experiments where participants \textbf{share an interest in discovering the right answer}, it has been shown that \emph{truth wins}
    \hfill (p.~72)
\end{quoting}
This leaves open some questions as to the specifics of the 'argumentative setting' that is mentioned multiple times throughout the paper. It seems that the disposition of the interlocutors going into an argument plays an important role in Mercier \& Sperber's account of argumentation, yet they do not expand on this.
How plausible is the assumption that people engaging in argumentation have a 'common interest in the truth', as Mercier \& Sperber call it? What happens (or what would happen) when interlocutors do \emph{not} share this interest?

\todo[inline,caption={}]{
    To do:
    \begin{itemize}
        \item I'm pretty sure this criticism is about something different than 'motivated reasoning', but check this
        \item Define what an argumentative setting is, according to Mercier \& Sperber (close-read \citet{MS11} for this)
        \item Possibly find some empirical work on arguers' dispositions
    \end{itemize}
}
