\chapter{The argumentative theory of reasoning closely inspected}
\label{ch:scrutiny}

Although Mercier \& Sperber's story seems very convincing on the face of it, the theory seems to leave a number of details undiscussed. Moreover, the argumentative theory of reasoning makes some assumptions that require some exposition.
In this chapter, I will scrutinize a number of assumptions of the argumentative theory of reasoning and make explicit some details that require spelling out.

\section{What is reasoning?}

\citet{MS11} take the following to be their definition of reasoning:

\begin{quoting}
    \emph{Reasoning}, as commonly understood, refers to a very special form of inference at the conceptual level, where not only is a new mental representation (or \emph{conclusion}) consciously produced, but the previously held representations (or \emph{premises}) that warrant it are also consciously entertained.
    \hfill (p.~57)
\end{quoting}

In this definition, \emph{inference} is taken to be "the production of new mental representations on the basis of previously held representations" (p.~57).

Mercier \& Sperber maintain that their definition of reasoning is the one that is most commonly adhered to in the psychology of reasoning.
They go on to state that this definition of reasoning excludes non-human animals and preverbal children from the realm of reasoners.

However, much more remains to be said about this very definition. Let us first briefly consider other definitions of reasoning, and then place some question marks around the position of Mercier \& Sperber's definition within their argumentative theory.

\subsection{Definitions of reasoning}

\todo[inline]{Mention definitions of \citet{Harman86} and maybe others (perhaps references from \citet[\S~1.2, par.~2]{MS11}), and compare these to that of Mercier \& Sperber}

\subsection{Accusations of circularity}

Especially through reference to the terms 'premise' and 'conclusion' within the definition, the definition of reasoning given by Mercier \& Sperber is reminiscent of definitions of argumentation. In a \citeyear{Sperber01} paper, a precursory work to \citet{MS11}, Sperber uses the following definition of argumentation:
\begin{quoting}
    the defense of some conclusion by appeal to a set of premises that provide support for it
    \hfill (p.~401)
\end{quoting}
and somewhat less precisely, \citet{MS11} define arguments as
\begin{quoting}
    representations of relationships between premises and conclusions
    \hfill (p.~58)
\end{quoting}

Judging by these definitions of argumentation and reasoning, what is reasoning if not internalized argumentation? Or, in the same vein, what is argumentation if not externalized reasoning? And, if this is the case, does this not render the argumentative theory of reasoning circular? For then the theory would state that the main function of internalized argumentation is argumentative.
\todo[inline,caption={}]{To do:
    \begin{itemize}
        \item Closely read \S 1.1 of \citet{MS11} for their ontological considerations on inference and argument
        \item Work out the details of what the definitions entail, and spell out what "internalized argumentation" would entail
        \item Work out whether this is indeed a circularity in the theory
        \item And whether that is a fatal flaw of this theory
    \end{itemize}
}

\section{The necessity of epistemic vigilance to survival}

\section{How is convincing others advantageous?}

The account of human communication that underpins the argumentative theory of reasoning (see \citet{Sperber01, Sperber10})
entails that for human communication to have stabilized over time, it must have been evolutionarily advantageous to both the sender and the receiver. I agree that it is reasonable to assume that -- since it depends on the participation of two parties -- communication would collapse if either party was not receiving benefits of the action.

\todo[inline]{The following paragraph is horribly written}
The possible benefit of communication to the receiver is to gain information (to the extent that it is genuine information).\todo{Citation needed?}
\todo{Does this need to be spelled out more still?}
This benefit seems straight-forward enough: communication can enable us to gain information about the world similarly to how direct perception does. However, we should be careful to regard communication as 'cognition by proxy', since the sender also stands to gain from communication and thus has their own interests as well \citep{Sperber01}.

On the other side of things, \citet{Sperber01} describes the benefits of communication to the sender as follows:
\begin{quoting}
    From the point of view of producers of messages, what makes communication, and testimony in particular, beneficial is that it allows them to have desirable effects on the receivers' attitudes and behavior. By communicating, one can cause others to do what one wants them to do and to take specific attitudes to people, objects, and so on.
    \hfill (p.~404)
\end{quoting}
Although these benefits of communication to the sender of a message seem convincing enough, the details of this remain to be spelled out.
\todo[inline,caption={}]{
    \begin{itemize}
        \item Check \cite{MS11} for a possible quote on benefits of convincing others: how do they describe it?
        \item Make explicit what details need to be spelled out and why this is not sufficient as-is
    \end{itemize}
}

On the flip side of these benefits of convincing others are the purported disadvantages to the receiver that come with being deceived by the sender. \citet{Sperber10} state that
\begin{quoting}
    being accidentally or intentionally misinformed (\ldots) may reduce, cancel, or even reverse [the gains received from communication with others]
    \hfill (p.~360)
\end{quoting}

\section{The evolutionary 'arms race' between sender and receiver}

\section{Common interest in the truth}
