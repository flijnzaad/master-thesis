\chapter{The argumentative theory of reasoning closely inspected}
\label{ch:scrutiny}

Although Mercier \& Sperber's story about the evolution and function of reasoning is quite convincing on the face of it, their theory seems to leave a number of details underexposed. Moreover, the argumentative theory of reasoning makes a number of assumptions that are in need of explication and detailed discussion.

In this chapter, I will scrutinize a number of assumptions of the argumentative theory of reasoning (ATR) and make explicit some details that require spelling out.

\todo[inline]{(Add discussion of structure of the chapter after filling in the sections more)}

\section{What is reasoning? Revisited}
\label{sec:def-scrutiny}

However, much more remains to be said about this very definition. Let us first briefly consider other definitions of reasoning from the literature, and then place some question marks around the position of Mercier \& Sperber's definition of reasoning within their argumentative theory.

\subsection{Other definitions of reasoning}

\todo[inline,caption={}]{
    To do:
    \begin{itemize}
        \item Explicate \poscite{Harman86} definition
        \item Maybe find other definitions, perhaps references from \citet[\S~1.2, par.~2]{MS11}
        \item Compare these definitions to that of \citet{MS11}
    \end{itemize}
}

\subsection{Accusations of circularity}

\thinkL{It's not really a circularity but more that the evolutionary claim is void if reasoning equals arugmentation. Work this out}

The definition of reasoning given and used by Mercier \& Sperber is reminiscent of definitions of argumentation, in particular in its use of the terms 'premise' and 'conclusion'. In a \citeyear{Sperber01} paper, which can be considered to be a precursory work to \citet{MS11}, Dan Sperber uses the following definition of argumentation:
\begin{quoting}
    the defense of some conclusion by appeal to a set of premises that provide support for it
    \hfill (p.~401)
\end{quoting}
and somewhat less precisely, \citet{MS11} define arguments as
\begin{quoting}
    representations of relationships between premises and conclusions
    \hfill (p.~58)
\end{quoting}

Comparing these definitions of argumentation and of reasoning raises a question: what is reasoning, if not internalized argumentation? Or, in a similar vein, what is argumentation, if not externalized reasoning? And, if this is the case, does this not render the argumentative theory of reasoning circular? For then the theory would state that the main function of internalized argumentation is argumentative.
\todo[inline,caption={}]{To do:
    \begin{itemize}
        \item Closely read \citet[\S~1.1]{MS11} for their ontological considerations on inference and argument
        \item Work out the details of what the definitions entail, and spell out what "internalized argumentation" would entail
        \item Work out whether this is indeed a circularity in the theory
        \item And whether that would be a fatal flaw of the theory
    \end{itemize}
}

\section{Epistemic vigilance, revisited}
\label{sec:EV-scrutiny}

\todo[inline]{Update this paragraph once the section is updated. What is exactly the direction and purpose of this section?}
It seems that the argumentative theory of reasoning (ATR) rests on the notion of epistemic vigilance having evolved in humans. Therefore, this concept deserves some extra attention.
In this section, I will first discuss the position of the notion of epistemic vigilance within the argumentative theory of reasoning. Then, I will discuss a critical response to \citet{Sperber10} by Kourken \citet{Michaelian13}, and in turn Dan \poscite{Sperber13} response to the criticism. Lastly, I will conclude with the consequences of this discussion for the argumentative theory of reasoning.

\subsection*{(Short summary of both papers)}

\todo[inline]{The below paragraph is weirdly placed, and it's not detailed enough to make a lot of sense. You can decide to keep this paragraph; then you'd need to take the reader by the hand some more, and it would need to be more detailed.}
In short, \citet{Michaelian13} spells out some of the assumptions that he claims are inherent to the argument for epistemic vigilance. He contrasts these assumptions with evidence from deception detection research. In doing so, he maintains that epistemic vigilance does not play a major role in ensuring the stability of communication. Rather, he argues, the majority of the burden befalls \emph{speaker honesty}. In the same \citeyear{Sperber13} issue of \emph{Episteme}, Dan Sperber provides a response to the criticisms of Michaelian. He resolves some misunderstandings and provides further details to reemphasize his perceived importance of epistemic vigilance to the stability of communication.

\subsection{Epistemic vigilance and the ATR}
\label{sec:epi-vigil-atr}

It is clear that the notion of epistemic vigilance is important for the argumentative theory of reasoning, but it may be illuminating to consider exactly the role or position of epistemic vigilance within the ATR.

Returning focus to the paper coining the ATR, \citet{MS11} have the following to say about epistemic vigilance:

\begin{quoting}
    For communication to be stable, it has to benefit both senders and receivers (\ldots) To avoid being victims of misinformation, receivers must therefore exercise some degree of what may be called epistemic vigilance (Sperber et al. 2010).
\hfill (p.~60)
\end{quoting}
In other words, the stability of communication depends on its benefits for both the sender and the receiver. According to Mercier \& Sperber, the benefits for the receiver depend on, or are mediated by, epistemic vigilance.

\poscite{Sperber01}\todo{Also refer back to the section in Chapter 2 where you introduce this concept} 'evolutionary arms race' is foundational to the argumentative theory of reasoning. Epistemic vigilance is one of the steps in this arms race, having evolved as a defense against misinformation and deception. Consequently \todo{Right word?}, epistemic vigilance is a critical component of the argumentative theory of reasoning: for, if receivers were not vigilant, senders need not have the ability to display the coherence of their arguments in order to be able to convince receivers.

\todo{This transition could be smoother} Let us now go through the points of scrutiny that Kourken \citet{Michaelian13} subjects Sperber and colleagues' epistemic vigilance to, and consider \poscite{Sperber13} response to the scrutiny.

\subsection{What is epistemic vigilance exactly?}

\todo[inline]{This section needs to be at the start of the discussion of epistemic vigilance}

\citet{Sperber13} chalks up a considerable share of the disagreement between him and Michaelian to an alleged misunderstanding between the two authors on the exact definition of epistemic vigilance:
\begin{quoting}
    Michaelian seems to attribute to us the view that ‘epistemic vigilance is a matter of processes devoted to screening out incoming false information on the basis of available behavioural cues’. Showing that vigilance in this narrow sense is not efficient would, he holds, be quite damaging to our conjecture. This is a misunderstanding.
    \hfill \citep[p.~65]{Sperber13}
\end{quoting}
While I agree with Sperber that Michaelian seems to attack a more narrowly defined version of epistemic vigilance, I do not blame Michaelian for the misunderstanding. Sperber and colleagues are (intentionally or unintentionally) vague in their original paper about exactly what epistemic vigilance is.
Their flexible use of the notion of epistemic vigilance might not be inherently problematic, but given the central role epistemic vigilance plays in much of Sperber and Mercier's work, I believe we are long overdue an exact definition of epistemic vigilance. In this section, I will gather the details that together may constitute a definition of epistemic vigilance according to \citet{Sperber10} and \citet{Sperber13}.

Let us first consider the ontological status of epistemic vigilance. Although on the face of it, one may want to describe epistemic vigilance as a set of mechanisms for filtering incoming information, \citet{Sperber10} describe humans as having evolved a 'suite of mechanisms' \emph{for} -- not \emph{of} -- epistemic vigilance. This leaves open the question of what epistemic vigilance itself could be. One candidate is a cognitive capacity or skill, which seems to be supported by \citet[p.~60]{MS11} who describe epistemic vigilance as something that can be 'exercise[d to] some degree'. Moreover, \citet[\S 5]{Sperber10} bring in empirical evidence on the development of epistemic vigilance in children, which would point to vigilance being a capacity or skill as well.
Possibly pointing us in the right direction, Sperber and colleagues contrast vigilance with trust:
\begin{quoting}
    Vigilance (unlike distrust) is not the opposite of trust; it is the opposite
of blind trust
    \hfill \citep[p.~363]{Sperber10},
\end{quoting}
\todo{Karolina highlighted this phrase, why?} implying that trust and vigilance are of the same kind. Although the ontological status of trust does not seem to transpire from their \citeyear{Sperber10} paper, this may be excused, as trust has been described by others as difficult to define \citep{Simpson12,McKnight00}.
\todo[inline]{Finish this thought: is their vagueness w.r.t. the ontological status of EV forgiven or not? And what is the consequence (if any) of the vagueness for their arguments?}

Next, let us consider the specifics of the processes in that 'suite of mechanisms for epistemic vigilance'.
\todo[inline]{Discuss the specifics of how epistemic vigilance fits into dual processes: this is a very important part of the evolutionary account, since it relates to the costs of epistemic vigilance, and a cost-benefit analysis underlies the evolutionary account. But, since Sperber and colleagues are so opaque in their characterization of epistemic vigilance, this is tough to figure out.}

Now slightly shifting focus to the ATR, let us consider the relation between epistemic vigilance and reasoning. \citet{Sperber10} describe reasoning as "a tool for epistemic vigilance, and for communication with vigilant addressees" (p.~378). This would imply that reasoning is an item in the suite of mechanisms for epistemic vigilance. However, in a similar way to how Sperber and colleagues are unspecific about epistemic vigilance's contribution to the stability of communication, they are unspecific about reasoning's contribution to epistemic vigilance.
\todo[inline]{This point begs some clarification, and this paragraph still needs a conclusion. I think it might be the case that reasoning is something more general than the other mechanisms they mention that belong to the suite}

\subsection{Strong vs. weak readings of the argument for epistemic vigilance}

In their \citeyear{Sperber10} paper, Sperber and colleagues hint at different ways to interpret their argument, different roles to attribute to epistemic vigilance:

\begin{quoting}
    It is because of the risk of deception that epistemic vigilance may be not merely advantageous but indispensable if communication itself is to remain advantageous.
\hfill (p.~360)
\end{quoting}
Sperber et al. seem to prefer to remain agnostic (or -- put differently -- vague) about the exact role of epistemic vigilance in explaining the stability of communication: is epistemic vigilance "merely advantageous", or is it "indispensable"? In other words, is it just the case that the benefits of epistemic vigilance outweigh its costs, leading to its having evolved in humans; or, stronger, would communication collapse if it were not for receivers' epistemic vigilance?
\example{Add an illustrative, intuitive example about this}
\todo[inline]{Maybe add in here quotes that point to Sperber's vagueness}
These different readings of epistemic vigilance are also implicit in the following statement Sperber and colleagues make:
\begin{quoting}
    People stand to gain immensely from communication with others, but this leaves them open to the risk of being accidentally or intentionally misinformed, which may reduce, cancel, or even reverse these gains.
\hfill (p.~360)
\end{quoting}
If being misinformed reduces or cancels the benefits an addressee receives from the communicative event, then it would stand to reason that it would be advantageous for the addressee to be epistemically vigilant so as to maintain the positive benefits of communication. If misinformation however reverses the gains one receives from communication, then epistemic vigilance would be \emph{necessary} for the receiver in order to not be negatively affected.

Kourken Michaelian picks up on these different views of epistemic vigilance in his response paper (\citeyear{Michaelian13}). He distinguishes between a strong and weak reading of Sperber et al.'s argument for epistemic vigilance, in line with this distinction between vigilance being indispensable or advantageous, respectively.
He then outlines the assumptions that are needed for each reading of the argument, and apparently shows that these assumptions are unfounded, or at the very least too strong.
According to Michaelian, the strong reading carries with it the assumption that dishonesty is sufficiently prevalent to necessitate vigilance on the side of the receiver; since, if epistemic vigilance is indispensable, non-vigilance must then yield a "dramatic reduction in the fitness of receivers" \citep[p.~39]{Michaelian13}. He argues that, since there is empirical evidence that lying is infrequent \citep{Serota10}, the strong reading of Sperber et al.'s argument cannot hold.
\todo[inline]{Insert discussion of prevalence of lying here. Really explicate the findings, and draw a conclusion about what this means for the story: why is lying prevalent, or why not? Also, this paraphrase of Michaelian's view could be clearer.}

Further, Michaelian argues that the weak reading of the argument carries with it the assumption that the benefits of epistemic vigilance outweigh its costs.
\todo[inline]{Missing conclusion from Michaelian: what is his conclusion to all this?}

In his response to Michaelian, \citet{Sperber13} states that "I now believe that we could and should have been even less definite" (p. 63) in recognizing a stronger and weaker reading of their argument. Sperber argues that in the recognition of the two readings of the argument, one mistakenly regards communication as a static enterprise. The degree to which vigilance is advantageous or even indispensable to a receiver, varies greatly from situation to situation, he argues:

\begin{quoting}
    The benefits of vigilance may be negligible in some communicative interactions and essential in other interactions. All I feel confident to say is that, without vigilance, human communication would be a very different and probably much more restricted affair.
    \hfill \citep[p.~63]{Sperber13}
\end{quoting}
This conclusion, however plausible, leaves a lot to be desired when it comes to the details for the evolutionary story, and in particular for the picture of the evolutionary arms race.
\todo[inline]{Discuss: why is it problematic that this is a vague conclusion?}

In the original \citeyear{Sperber10} paper, Sperber and colleagues maintain that it is sufficient that communication is \emph{on average} advantageous to both parties:
\begin{quoting}
    The fact that communication is so pervasive despite [the risk of misinformation] suggests that people are able to calibrate their trust well enough to make it advantageous on average to both communicator and audience
    \hfill \citep[p.~360]{Sperber10}
\end{quoting}

\todo[inline]{Continue here: draw a conclusion that either defends Sperber, or conclude that it's a strike against the ATR. Change the tone of the subsection accordingly}

\subsection{Honesty or dishonesty as prior}

\todo[inline]{To fit in still: discussion of ignorance, competence, that stuff}

Ultimately, a fundamentally different outlook on human communication and cooperation seems to transpire from the accounts of Sperber, Mercier and colleagues on the one hand, and Michaelian on the other. Sperber's 'evolutionary arms race' account takes dishonesty as prior. On this account, dishonesty is prior to vigilance, and in turn vigilance is prior to honesty:
\begin{quoting}
    We could not be mutually trustful \emph{unless} we were mutually vigilant.
    \hfill \citep[p.~364]{Sperber10}
\end{quoting}
For this account to work, one must convincably argue for the evolutionary benefits of dishonesty; if dishonesty is the first step in the evolutionary arms race, it must be beneficial by itself.

Michaelian, on the other hand, refutes this account and instead proposes that communication is stable just because speakers are honest; in other words, honesty is prior. In a similar fashion, it then remains to show how honesty is by itself beneficial.

\todo[inline]{Discuss, or at least mention, Grice's cooperative principle}
\todo[inline]{Discuss and contrast these using \citet{Tomasello09}, and draw conclusions on the most plausible account}

\todo[inline]{Make also terminological note of dishonesty vs. lying vs. deception: \citet{Levine10} has good nuance on how lying is used. For example, intuitively I would say that something like a 'white lie' doesn't really amount to deception (see Appendix of \citet{Levine10}), so this might warrant a bit of explication. This discussion probably also ties into the next section.}

\subsection{Concluding remarks}

\todo[inline]{To do: draw a conclusion about the argument for epistemic vigilance, its viability, and the consequences of all of this for the ATR}

\section{How is convincing others advantageous?}

\todo[inline]{This section should probably be moved to Chapter 2, utility of communication}

The account of human communication that underpins the argumentative theory of reasoning (see \citet{Sperber01, Sperber10})
entails that for human communication to have stabilized over time, it must have been evolutionarily advantageous to both the sender and the receiver. I agree that it is reasonable to assume that -- since it depends on the participation of two parties -- communication would collapse if either party was not experiencing any benefits from the action. Let us now briefly discuss these benefits.

The possible benefit of communication to the receiver is for them to gain information (to the extent that it is genuine information).
\todo[inline]{This is not really accurate: see notes of \citet{Michaelian13}. Also, then why is gaining information beneficial?}
This benefit seems straight-forward enough: communication can enable us to gain information about the world in a similar way to how direct perception, or inference on the basis of held beliefs, yield information to us. However, we should be careful to regard communication as 'cognition by proxy', since the sender also stands to gain from communication and thus has their own interests as well \citep{Sperber01}.

On the other side of the coin, \citet{Sperber01} describes the benefits of communication to the sender as follows:
\begin{quoting}
    From the point of view of producers of messages, what makes communication, and testimony in particular, beneficial is that it allows them to have desirable effects on the receivers' attitudes and behavior. By communicating, one can cause others to do what one wants them to do and to take specific attitudes to people, objects, and so on.
    \hfill (p.~404)
\end{quoting}
The details of this point in particular require some explication in order to understand their force within the evolutionary story.

\todo[inline,caption={}]{
    To do:
    \begin{itemize}
        \item Check \citet{MS11} for a possible quote on benefits of convincing others: how do they describe it there?
        \item (It might turn out that the benefits of persuasion are already discussed in Chapter 2 once it's revised.)
    \end{itemize}
}

% On the flip side of these benefits of convincing others are the purported disadvantages to the receiver that come with being deceived by the sender. \citet{Sperber10} state that
% \begin{quoting}
%     being accidentally or intentionally misinformed (\ldots) may reduce, cancel, or even reverse [the gains received from communication with others]
%     \hfill (p.~360)
% \end{quoting}

\section{The evolutionary 'arms race' between communicator and addressee}
\label{sec:arms-race}

In his discussion of the evolution of testimony and argumentation,\citet{Sperber01} sketches out the steps in what he calls the 'evaluation-persuasion arms race', i.e.\@ the chain of evolutionary adaptations that has resulted in our mechanisms for argument production and evaluation.
He argues that the first step in this arms race was for the addressee to develop coherence checking, as a defense against the risks of deception by the communicator. The second step was then for the communicator to anticipate this coherence-checking by overtly displaying the coherence of their message to their addressee, which requires argumentative form. The next steps were on the side of the addressee to develop skills for examining these displays of coherence (i.e., arguments), and on the side of the communicator to 'improve their argumentative skills'.

In this section I will work out this sketch of the evolutionary arms race in more detail. In doing so, I will supplement and possibly amend the sketch with findings from \citet{Sperber10}, \citet{MS11}, \citet{Reboul17}, and the conclusion of my discussion of \citet{Michaelian13} in \cref{sec:epi-vigil-crit}.

\section{Motivations and dispositions of interlocutors}

Throughout their \citeyear{MS11} article, Mercier \& Sperber allude to the dispositions\todo{'Dispositions' is not meant as a technical term, and I don't think it is; is it?}
of interlocutors in argumentative settings (emphasis in quotes added):
\begin{quoting}
    This experiment illustrates the more general finding stemming from this literature that, \emph{when they are \textbf{motivated}, participants are able to use reasoning to evaluate arguments accurately}
    \hfill (p.~61)
\end{quoting}
\begin{quoting}
    Most participants are \textbf{willing} to change their mind only once they have been thoroughly convinced that their initial answer was wrong
    \hfill (p.~63)
\end{quoting}
\begin{quoting}
    this [experimental finding] should not be interpreted as revealing a lack of ability but only a lack of \textbf{motivation}. When participants \textbf{want} to prove a conclusion wrong, they will find ways to falsify it.
    \hfill (p.~65)
\end{quoting}
\begin{quoting}
    people are good at assessing arguments and are quite able to do so in an unbiased way, \textbf{provided they have no particular axe to grind}. In group reasoning experiments where participants \textbf{share an interest in discovering the right answer}, it has been shown that \emph{truth wins}
    \hfill (p.~72)
\end{quoting}
This reference to the motivations and disposition of interlocutors opens up some questions as to the specifics of the 'argumentative setting' that Mercier \& Sperber mention multiple times throughout the paper. It seems that the disposition of the interlocutors going into an argument plays an important role in Mercier \& Sperber's account of argumentation, yet they do not expand on this.
How plausible is the assumption that people engaging in argumentation have a 'common interest in the truth', as Mercier \& Sperber call it? And what happens (or what would happen) when interlocutors do \emph{not} share this interest?

\todo[inline,caption={}]{
    To do:
    \begin{itemize}
        \item I'm pretty sure this criticism is about something different than 'motivated reasoning', but check this
        \item Define what an argumentative setting is, according to Mercier \& Sperber (close-read \citet{MS11} for this)
        \item Possibly find some empirical work on arguers' dispositions
    \end{itemize}
}
