\chapter{The argumentative theory of reasoning closely inspected}
\label{ch:scrutiny}

Although Mercier \& Sperber's story about the evolution and function of reasoning is quite convincing on the face of it, their theory seems to leave a number of details underexposed. Moreover, the argumentative theory of reasoning makes a number of assumptions that are in need of explication and detailed discussion.

In this chapter, I will scrutinize a number of assumptions of the argumentative theory of reasoning and make explicit some details that require spelling out.

\todo[inline]{(Add discussion of structure of the chapter after filling in the sections more)}

\section{What is reasoning? Revisited}
\label{sec:def-scrutiny}

However, much more remains to be said about this very definition. Let us first briefly consider other definitions of reasoning from the literature, and then place some question marks around the position of Mercier \& Sperber's definition of reasoning within their argumentative theory.

\subsection{Other definitions of reasoning}

\todo[inline,caption={}]{
    To do:
    \begin{itemize}
        \item Explicate \poscite{Harman86} definition
        \item Maybe find other definitions, perhaps references from \citet[\S~1.2, par.~2]{MS11}
        \item Compare these definitions to that of \citet{MS11}
    \end{itemize}
}

\subsection{Accusations of circularity}

\thinkL{It's not really a circularity but more that the evolutionary claim is void if reasoning equals arugmentation. Work this out}

The definition of reasoning given and used by Mercier \& Sperber is reminiscent of definitions of argumentation, in particular in its use of the terms 'premise' and 'conclusion'. In a \citeyear{Sperber01} paper, which can be considered to be a precursory work to \citet{MS11}, Dan Sperber uses the following definition of argumentation:
\begin{quoting}
    the defense of some conclusion by appeal to a set of premises that provide support for it
    \hfill (p.~401)
\end{quoting}
and somewhat less precisely, \citet{MS11} define arguments as
\begin{quoting}
    representations of relationships between premises and conclusions
    \hfill (p.~58)
\end{quoting}

Comparing these definitions of argumentation and of reasoning raises a question: what is reasoning, if not internalized argumentation? Or, in a similar vein, what is argumentation, if not externalized reasoning? And, if this is the case, does this not render the argumentative theory of reasoning circular? For then the theory would state that the main function of internalized argumentation is argumentative.
\todo[inline,caption={}]{To do:
    \begin{itemize}
        \item Closely read \citet[\S~1.1]{MS11} for their ontological considerations on inference and argument
        \item Work out the details of what the definitions entail, and spell out what "internalized argumentation" would entail
        \item Work out whether this is indeed a circularity in the theory
        \item And whether that would be a fatal flaw of the theory
    \end{itemize}
}

\section{The necessity of epistemic vigilance to survival}
\label{sec:epi-vigil-crit}

Since the argumentative theory of reasoning depends on the notion of epistemic vigilance and its mechanisms having evolved in humans, this concept deserves some extra attention.
In this section, I will discuss in more detail how epistemic vigilance could have been necessary or critical to the survival of humans. This partly ties into the discussion in \cref{sec:arms-race} as well.\todo{Maybe merge this with \cref{sec:arms-race}}
Moreover, I will delve into some of the criticisms leveraged against epistemic vigilance by \citet{Michaelian13} and examine their merit.

\subsection{Epistemic vigilance and the ATR}
\label{sec:epi-vigil-atr}

\todo[inline]{To do: discuss the position and importance of epistemic vigilance in the ATR}

\subsection{Kourken Michaelian vs. Dan Sperber}

\citet{Michaelian13} leverages a number of criticisms against epistemic vigilance as a concept. In the paper, he considers additional evidence from deception detection research, and draws on additional evolutionary considerations. In doing so, he maintains that epistemic vigilance does not play a major role in ensuring the stability of communication. Rather, he argues, the lion's share of the burden comes to \emph{speaker honesty}. In the same \citeyear{Sperber13} issue of \emph{Episteme}, Dan Sperber provides a response to the criticisms of Michaelian. In this section, I will scrutinize both Michaelian's criticisms as well as Sperber's response to them.

\todo[inline,caption={}]{
    To do:
    \begin{itemize}
        \item Recap criticisms of \citet{Michaelian13} here
        \item Recap \poscite{Sperber13} reply
        \item Consider what's left (if any) of \poscite{Michaelian13} criticism
        \item Consider points of criticism I myself had for \citet{Sperber10} from notes about necessity of epistemic vigilance
        \item Consider the consequences of the criticisms (if any) for the argumentative theory of reasoning (also depending on findings of \cref{sec:epi-vigil-atr})
    \end{itemize}
}

\section{How is convincing others advantageous?}

The account of human communication that underpins the argumentative theory of reasoning (see \citet{Sperber01, Sperber10})
entails that for human communication to have stabilized over time, it must have been evolutionarily advantageous to both the sender and the receiver. I agree that it is reasonable to assume that -- since it depends on the participation of two parties -- communication would collapse if either party was not experiencing any benefits from the action. Let us now briefly discuss these benefits.

The possible benefit of communication to the receiver is for them to gain information (to the extent that it is genuine information).
\todo{Does this need to be spelled out more still?}
\todo[inline]{This is not really accurate: see notes of \citet{Michaelian13}}
This benefit seems straight-forward enough: communication can enable us to gain information about the world in a similar way to how direct perception, or inference on the basis of held beliefs, yield information to us. However, we should be careful to regard communication as 'cognition by proxy', since the sender also stands to gain from communication and thus has their own interests as well \citep{Sperber01}.

On the other side of the coin, \citet{Sperber01} describes the benefits of communication to the sender as follows:
\begin{quoting}
    From the point of view of producers of messages, what makes communication, and testimony in particular, beneficial is that it allows them to have desirable effects on the receivers' attitudes and behavior. By communicating, one can cause others to do what one wants them to do and to take specific attitudes to people, objects, and so on.
    \hfill (p.~404)
\end{quoting}
The details of this point in particular require some explication in order to understand their force within the evolutionary story.

\todo[inline,caption={}]{
    To do:
    \begin{itemize}
        \item Check \citet{MS11} for a possible quote on benefits of convincing others: how do they describe it there?
        \item Make explicit what details need to be spelled out and why they is not sufficient as-is
        \item (It might turn out that the benefits of persuasion are already discussed in Chapter 2 once it's revised.)
    \end{itemize}
}

% On the flip side of these benefits of convincing others are the purported disadvantages to the receiver that come with being deceived by the sender. \citet{Sperber10} state that
% \begin{quoting}
%     being accidentally or intentionally misinformed (\ldots) may reduce, cancel, or even reverse [the gains received from communication with others]
%     \hfill (p.~360)
% \end{quoting}

\section{The evolutionary 'arms race' between communicator and addressee}
\label{sec:arms-race}

In his discussion of the evolution of testimony and argumentation,\citet{Sperber01} sketches out the steps in what he calls the 'evaluation-persuasion arms race', i.e.\@ the chain of evolutionary adaptations that has resulted in our mechanisms for argument production and evaluation.
He argues that the first step in this arms race was for the addressee to develop coherence checking, as a defense against the risks of deception by the communicator. The second step was then for the communicator to anticipate this coherence-checking by overtly displaying the coherence of their message to their addressee, which requires argumentative form. The next steps were on the side of the addressee to develop skills for examining these displays of coherence (i.e., arguments), and on the side of the communicator to 'improve their argumentative skills'.

In this section I will work out this sketch of the evolutionary arms race in more detail. In doing so, I will supplement and possibly amend the sketch with findings from \citet{Sperber10}, \citet{MS11}, \citet{Reboul17}, and the conclusion of my discussion of \citet{Michaelian13} in \cref{sec:epi-vigil-crit}.

\todo[inline]{I believe this section will build upon multiple other sections in this chapter (criticisms of epistemic vigilance, benefits of persuasion, etc.), so I should fill it in later than the others.}
\todo[inline]{Also discuss here: how it works on evolutionary theory that senders and receivers are the same people.}

\section{Motivations and dispositions of interlocutors}

Throughout their \citeyear{MS11} article, Mercier \& Sperber allude to the dispositions\todo{'Dispositions' is not meant as a technical term, and I don't think it is; is it?}
of interlocutors in argumentative settings\footnote{Emphasis in quotes added.}:
\begin{quoting}
    This experiment illustrates the more general finding stemming from this literature that, \emph{when they are \textbf{motivated}, participants are able to use reasoning to evaluate arguments accurately}
    \hfill (p.~61)
\end{quoting}
\begin{quoting}
    Most participants are \textbf{willing} to change their mind only once they have been thoroughly convinced that their initial answer was wrong
    \hfill (p.~63)
\end{quoting}
\begin{quoting}
    this [experimental finding] should not be interpreted as revealing a lack of ability but only a lack of \textbf{motivation}. When participants \textbf{want} to prove a conclusion wrong, they will find ways to falsify it.
    \hfill (p.~65)
\end{quoting}
\begin{quoting}
    people are good at assessing arguments and are quite able to do so in an unbiased way, \textbf{provided they have no particular axe to grind}. In group reasoning experiments where participants \textbf{share an interest in discovering the right answer}, it has been shown that \emph{truth wins}
    \hfill (p.~72)
\end{quoting}
This reference to the motivations and disposition of interlocutors opens up some questions as to the specifics of the 'argumentative setting' that Mercier \& Sperber mention multiple times throughout the paper. It seems that the disposition of the interlocutors going into an argument plays an important role in Mercier \& Sperber's account of argumentation, yet they do not expand on this.
How plausible is the assumption that people engaging in argumentation have a 'common interest in the truth', as Mercier \& Sperber call it? And what happens (or what would happen) when interlocutors do \emph{not} share this interest?

\todo[inline,caption={}]{
    To do:
    \begin{itemize}
        \item I'm pretty sure this criticism is about something different than 'motivated reasoning', but check this
        \item Define what an argumentative setting is, according to Mercier \& Sperber (close-read \citet{MS11} for this)
        \item Possibly find some empirical work on arguers' dispositions
    \end{itemize}
}
