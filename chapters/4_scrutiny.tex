\chapter{The ATR closely inspected}
\label{ch:scrutiny}

Although Mercier \& Sperber's story about the evolution and function of reasoning is quite convincing on the face of it, their theory seems to leave a number of details underexposed. Moreover, the argumentative theory of reasoning makes a number of assumptions that are in need of explication and detailed discussion.

In this chapter, I will scrutinize a number of assumptions of the argumentative theory of reasoning (ATR) and make explicit some details that require spelling out.

\section{What is the function of communication? Revisited}
\label{sec:comm-func-scrutiny}

\section{Epistemic vigilance: revisited}
\label{sec:EV-scrutiny}

It seems that the argumentative theory of reasoning (ATR) rests on epistemic vigilance having evolved in humans. Therefore, this concept deserves some extra attention.
In this section, I will first discuss the position of the notion of epistemic vigilance within the argumentative theory of reasoning. Then, I will discuss a critical response to \citet{Sperber10} by Kourken \citet{Michaelian13}, and in turn Dan \poscite{Sperber13} response to the criticism. Lastly, I will conclude with the consequences of this discussion for the argumentative theory of reasoning.
Before we do that, let me first briefly summarize Michaelian's and Sperber's papers. In short, \citet{Michaelian13} spells out some of the assumptions that he claims are inherent to the argument for epistemic vigilance. He contrasts these assumptions with evidence from deception detection research. In doing so, he maintains that epistemic vigilance does not play a major role in ensuring the stability of communication. Rather, he argues, the majority of the burden befalls \emph{speaker honesty}. In the same \citeyear{Sperber13} issue of \emph{Episteme}, Dan Sperber provides a response to the criticisms of Michaelian. He resolves some misunderstandings and provides further details to reemphasize his perceived importance of epistemic vigilance to the stability of communication.

\subsection{Epistemic vigilance and the ATR}
\label{sec:epi-vigil-atr}

It is clear that the notion of epistemic vigilance is important for the argumentative theory of reasoning, but it may be illuminating to consider exactly the role or position of epistemic vigilance within the ATR.

Returning focus to the paper coining the ATR, \citet{MS11} have the following to say about epistemic vigilance:

\begin{quoting}
    For communication to be stable, it has to benefit both senders and receivers (\ldots) To avoid being victims of misinformation, receivers must therefore exercise some degree of what may be called epistemic vigilance (Sperber et al. 2010).
\hfill (p.~60)
\end{quoting}
In other words, the stability of communication depends on its benefits for both the sender and the receiver. According to Mercier \& Sperber, the benefits for the receiver depend on, or are mediated by, epistemic vigilance.

The 'evolutionary arms race' (recall \cref{sec:Sperber01}) is foundational to the argumentative theory of reasoning. Epistemic vigilance is one of the steps in this arms race, having evolved as a defense against misinformation and deception. Consequently, epistemic vigilance is a critical component of the argumentative theory of reasoning: for, if receivers were not vigilant, senders need not have the ability to display the coherence of their arguments in order to be able to convince receivers.

Let us now go through the points of scrutiny that Kourken \citet{Michaelian13} subjects Sperber and colleagues' epistemic vigilance to, and consider \poscite{Sperber13} response to the scrutiny.

\subsection{What is epistemic vigilance exactly?}
\label{sec:EV-def}

\citet{Sperber13} chalks up a considerable share of the disagreement between him and Michaelian to an alleged misunderstanding between the two authors on the exact definition of epistemic vigilance:
\begin{quoting}
    Michaelian seems to attribute to us the view that ‘epistemic vigilance is a matter of processes devoted to screening out incoming false information on the basis of available behavioural cues’. Showing that vigilance in this narrow sense is not efficient would, he holds, be quite damaging to our conjecture. This is a misunderstanding.
    \hfill \citep[p.~65]{Sperber13}
\end{quoting}
While I agree with Sperber that Michaelian seems to attack a more narrowly defined version of epistemic vigilance, I do not blame Michaelian for the misunderstanding. Sperber and colleagues are (intentionally or unintentionally) vague in their original paper about exactly what epistemic vigilance is.
Their flexible use of the notion of epistemic vigilance might not be inherently problematic, but given the central role epistemic vigilance plays in much of Sperber and Mercier's work, I believe we are long overdue an exact definition of epistemic vigilance. In this section, I will gather the details that together may constitute a definition of epistemic vigilance according to \citet{Sperber10} and \citet{Sperber13}.

Let us first consider the ontological status of epistemic vigilance. Although on the face of it, one may want to describe epistemic vigilance as a set of mechanisms for filtering incoming information, \citet{Sperber10} describe humans as having evolved a 'suite of mechanisms' \emph{for} -- not \emph{of} -- epistemic vigilance. This leaves open the question of what epistemic vigilance itself could be. One candidate is a cognitive capability or skill, which seems to be supported by \citet[p.~60]{MS11} who describe epistemic vigilance as something that can be 'exercise[d to] some degree'. Moreover, \citet[\S 5]{Sperber10} bring in empirical evidence on the development of epistemic vigilance in children, which would point to vigilance being a capacity or skill as well.
Also possibly pointing us in the right direction, Sperber and colleagues contrast vigilance with trust:
\begin{quoting}
    Vigilance (unlike distrust) is not the opposite of trust; it is the opposite
of blind trust
    \hfill \citep[p.~363]{Sperber10},
\end{quoting}
implying that trust and vigilance are of the same kind. Although the ontological status of trust does not seem to transpire from their \citeyear{Sperber10} paper, I would argue that this can be excused, as trust has been described by others as difficult to define \citep{Simpson12,McKnight00}.
A possibly useful perspective on the definition of epistemic vigilance comes from sleep science, relating to neurology and neurophysiology. \citet{VanSchie21} define vigilance \emph{per se} as follows:
\begin{quoting}
    Vigilance is defined as the capability to be sensitive to potential changes in one's environment, ie the capability to reach a level of alertness above a threshold for a certain period of time rather than the state of alertness itself.
    \hfill \citep[p.~175]{VanSchie21}
\end{quoting}
It may not be immediately obvious how a definition from sleep science may at all be applicable in our attempt to define epistemic vigilance. However, I do believe that we may assume some degree of overlap between neurology and psychology, and that epistemic vigilance must relate in some way to vigilance \emph{per se}.

All things considered, I believe epistemic vigilance would be best described as the capability to be sensitive to the trustworthiness of communicated information and informants.

Next, let us consider the specifics of the processes in the 'suite of mechanisms for epistemic vigilance'. Just like Sperber and colleagues are somehat vague about the ontological status of epistemic vigilance, they are rarely straight-forward about the exact mechanisms they consider to fall under the umbrella of epistemic vigilance. This however may not be as easily forgiven as their opacity surrounding epistemic vigilance's ontological status. The nature of the mechanisms for epistemic vigilance is crucial for the cost-benefit analysis that underlies the evolutionary argument that they are making. \citet{Michaelian13} already implictly picks up on this (and in my opinion, \citet{Sperber13} does little to alleviate these concerns).
\todo{Discuss epistemic vigilance and dual processes}

Now slightly shifting focus to the ATR, let us consider the relation between epistemic vigilance and reasoning. \citet{Sperber10} describe reasoning as "a tool for epistemic vigilance, and for communication with vigilant addressees" (p.~378). This would imply that reasoning is an item in the suite of mechanisms for epistemic vigilance. However, in a similar way to how Sperber and colleagues are unspecific about epistemic vigilance's contribution to the stability of communication, they are unspecific about reasoning's contribution to epistemic vigilance.

\subsection{Strong vs. weak readings of the argument for epistemic vigilance}

In their \citeyear{Sperber10} paper, Sperber and colleagues hint at different ways to interpret their argument, different roles to attribute to epistemic vigilance:

\begin{quoting}
    It is because of the risk of deception that epistemic vigilance may be not merely advantageous but indispensable if communication itself is to remain advantageous.
\hfill (p.~360)
\end{quoting}
Sperber et al. seem to prefer to remain agnostic (or -- put differently -- vague) about the exact role of epistemic vigilance in explaining the stability of communication: is epistemic vigilance "merely advantageous", or is it "indispensable"? In other words, is it just the case that the benefits of epistemic vigilance outweigh its costs, leading to its having evolved in humans; or, stronger, would communication collapse if it were not for receivers' epistemic vigilance?
\example{Add an illustrative, intuitive example about this}
\todo{Add in more quotes that point to Sperber's vagueness?}
These different readings of epistemic vigilance are also implicit in the following statement Sperber and colleagues make:
\begin{quoting}
    People stand to gain immensely from communication with others, but this leaves them open to the risk of being accidentally or intentionally misinformed, which may reduce, cancel, or even reverse these gains.
\hfill (p.~360)
\end{quoting}
If being misinformed reduces or cancels the benefits an addressee receives from the communicative event, then it would stand to reason that it would be advantageous for the addressee to be epistemically vigilant so as to maintain the positive benefits of communication. If misinformation however reverses the gains one receives from communication, then epistemic vigilance would be \emph{necessary} for the receiver in order to not be negatively affected.

Kourken Michaelian picks up on these different views of epistemic vigilance in his response paper (\citeyear{Michaelian13}). He distinguishes between a strong and weak reading of Sperber et al.'s argument for epistemic vigilance, in line with this distinction between vigilance being indispensable or advantageous, respectively.
He then outlines the assumptions that are needed for each reading of the argument, and apparently shows that these assumptions are unfounded, or at the very least too strong.
According to Michaelian, the strong reading carries with it the assumption that dishonesty is sufficiently prevalent to necessitate vigilance on the side of the receiver; since, if epistemic vigilance is indispensable, non-vigilance must then yield a "dramatic reduction in the fitness of receivers" \citep[p.~39]{Michaelian13}. He argues that, since there is empirical evidence that lying is infrequent \citep{Serota10}, the strong reading of Sperber et al.'s argument cannot hold.
\todo{Discuss prevalence of lying}

Further, Michaelian argues that the weak reading of the argument carries with it the assumption that the benefits of epistemic vigilance outweigh its costs.\todo{And so?}

In his response to Michaelian, \citet{Sperber13} states that "I now believe that we could and should have been even less definite" (p. 63) in recognizing a stronger and weaker reading of their argument. Sperber argues that in the recognition of the two readings of the argument, one mistakenly regards communication as a static enterprise. The degree to which vigilance is advantageous or even indispensable to a receiver, varies greatly from situation to situation, he argues:

\begin{quoting}
    The benefits of vigilance may be negligible in some communicative interactions and essential in other interactions. All I feel confident to say is that, without vigilance, human communication would be a very different and probably much more restricted affair.
    \hfill \citep[p.~63]{Sperber13}
\end{quoting}
This conclusion, however plausible, leaves a lot to be desired when it comes to the details for the evolutionary story, and in particular for the picture of the evolutionary arms race.

In the original \citeyear{Sperber10} paper, Sperber and colleagues maintain that it is sufficient that communication is \emph{on average} advantageous to both parties:
\begin{quoting}
    The fact that communication is so pervasive despite [the risk of misinformation] suggests that people are able to calibrate their trust well enough to make it advantageous on average to both communicator and audience
    \hfill \citep[p.~360]{Sperber10}
\end{quoting}

\subsection{Honesty or dishonesty as prior}

Ultimately, a fundamentally different outlook on human communication and cooperation seems to transpire from the accounts of Sperber, Mercier and colleagues on the one hand, and Michaelian on the other. Sperber's 'evolutionary arms race' account takes dishonesty as prior. On this account, dishonesty is prior to vigilance, and in turn vigilance is prior to honesty:
\begin{quoting}
    We could not be mutually trustful \emph{unless} we were mutually vigilant.
    \hfill \citep[p.~364]{Sperber10}
\end{quoting}
For this account to work, one must convincably argue for the evolutionary benefits of dishonesty; if dishonesty is the first step in the evolutionary arms race, it must be beneficial by itself.

Michaelian, on the other hand, refutes this account and instead proposes that communication is stable just because speakers are honest; in other words, honesty is prior. In a similar fashion, it then remains to show how honesty is by itself beneficial.

\section{Ontological and definitional problems of the ATR}

One issue that we have seen come up in \cref{sec:MS09} as well as \cref{sec:EV-def}, is that Mercier and Sperber are oftentimes either intentionally or unintentionally vague about particularities of their theory. As we have seen, \citet{MS09} leaves a lot to be desired with regards to the ontological details of intuitive and reflective inference. Moreover, \citet{Sperber10} is imprecise on the ontological status of epistemic vigilance, the contents of the suite of mechanisms that contribute to vigilance, and the relation between reasoning and epistemic vigilance.
This vagueness may be partly due to Mercier and Sperber changing their mind on details of the ATR as it developed into the full-fledged theory as expounded in \citet{MS17}.
For example, in their \citeyear{MS09} paper they explicate their dichotomy between intuitive and reflective inference, but slightly confusingly, they never use the term 'reflective inference' in \citet{MS11} anymore. This gives the appearance that something changed about their views between \citeyear{MS09} and \citeyear{MS11}. While this might very well not be the case, it does leave a reader to wonder.
\todo[inline]{Also in \citet{MS17} I think they changed a lot of terminology, talking about the reason module instead of argumentation module. Add to this after reading it}

This is of course understandable, as it constitutes a natural part of the scientific process. They however do not explicate these changes in terminology or other details of their theory, which leaves the reader confused.
This need not be construed as a moral failing nor as a definitive strike against their theory, but it does leave a lot to be desired from an otherwise intuitively convincing theory.

\section{Conclusion}

