\chapter{Why do we communicate?}
\label{ch:communication}

\largeTodo{In general: this chapter is very dense, you can go at a bit of a slower pace most of the time. But pace will also be better once elaborations and examples are added.}
In order to answer the question of how advanced reasoning \smallTodo{Terminology: probably drop "advanced" here. Possibly get back to this after writing Chapter 3} may have evolved to further communication, we will first need to examine communication in its own right: why do we communicate?
In order to answer this question, we will discuss each of the methodological questions raised in \cref{sec:evol-method} as they pertain to communication\smallTodo{Refer to the numbers or codes of the Tinbergen questions}. But before we can take a look at the evolutionary history, the developmental origins and the functions of communication, we must first fix a definition of communication, since this determines the frame of our research question.
\largeTodo{This whole introduction could be clearer: why do we look at communication now? Can use a bit more words.}

\section{Conceptions of communication}
\label{sec:comm:definition}

\largeTodo{Missing from this section: what Mercier (\& Sperber) define as, and/or have to say about, communication}

There are many different ways organisms may communicate with each other, and indeed many different ways in which one may define communication. In any case, communication is an process necessarily involving a signaler (a sender) and at least one receiver (a listener).

Some authors regard communication to inherently be a tool of persuasion, which then translates to their very definition of communication: for example, on the manipulative model of communication, communication can be taken to occur "when an animal, the actor, does something which appears to be the result of selection to influence the sense organs of another animal, the reactor, so that the reactor's behavior changes to the advantage of the actor" \citep[p.~283]{DawkinsKrebs78}.\smallTodo{Rewrite this so that the quote is actually typeset as a quote, for emphasis}
\largeTodo{Say something about this quote in your own words: comment on it, this also justifies you using the quote. Point the reader to why you use it.}
One may also notice that this definition has a teleological explanation embedded in it as well (see \cref{sec:teleology}).\smallTodo{Elaborate on this: make it more explicit}
I mention this definition only for completeness' sake \smallTodo{Maybe not necessary to mention this}, because I believe this definition to be insufficiently parsimonious in its assumptions about the function of communication.
\largeTodo{Elaborate on this: what are the assumptions they make, in what sense are they strong, and why are they too strong for my liking (i.e. what's wrong with them)?}

In their discussion of communication as it relates to social cognition, \citet{Freeberg19} define communication as follows:
\begin{quoting}
    Communication involves an action or characteristic of one individual that influences the behaviour, behavioural tendency or physiology of at least one other individual in a fashion typically adaptive to both
    \hfill (p.~281)
\end{quoting}
\largeTodo{Explain this in own words; and explain the difference with the definition of \citet{DawkinsKrebs78}. The difference lies in "to the advantage of"}
This is a very broad conception of communication; on this definition, all organisms, from bacteria to fungi to plants to animals, communicate.
\largeTodo{Provide examples for each of these kingdoms here, nice illustration}

\citet{Scott-Phillips18-communication, Scott-Phillips15-primate} contrasts two different models of communication with each other: the classical \emph{code model} of communication, and the \emph{ostensive-inferential} model of communication.
In the former model, communication involves processes of coding and decoding messages. The coding, on the side of the sender, involves a mapping between the state of the world and a behavior (namely the signal they send). The decoding, on the side of the receiver, involves a mapping between two behaviors: the signal sent, and a subsequent response of the receiver. If the mappings are properly calibrated to each other, communication between sender and receiver can be said to have occurred.

\largeTodo{This paragraph could do with a lot more philosophical discussion: Quine should definitely be mentioned, even if it's just in a footnote, because it's like this milestone philosophy thing that would be a glaring omission to philosophers reading this.
Also, the usage of "meaning" and "content" here is a big philosophical no-no, either explicate what you mean by them or change your terminology. You could also just acknowledge or specify that you use the intuitive, colloquial meaning, not the technical one. Read also the Putnam paper on the ants and Winston Churchill?}
However, in order to capture human communication, the code model is too simplistic, because it fails to account for the \emph{underdeterminacy of meaning}: in merely looking at the content of the message, one cannot account for the meaning that the message conveys to the sender \citep{Scott-Phillips18-communication}. \smallTodo{Add example}
Therefore, a move away from the code model of communication towards the \emph{ostensive-inferential} model of communication would be in order. This model takes into account the intentionality inherent in human communication.

\largeTodo{This paragraph also needs a lot more work:
(1) mention Sperber \& Wilson and relevance theory (credit where credit's due!), and that their theory is neo-Gricean, because most philosophers will be familiar with Grice. (2) Add an example, this needs to be way clearer and elaborated more if this will be my definition of human communication. (3) Define ostensive behavior, also with an example. It's apparent right now that I don't fully understand this model myself. (4) Add a comment on how the underdeterminacy is captured better by the ostensive-inferential model}
In the ostensive-inferential model, one may speak of a sender's \emph{informative intention}, which is their intending for the receiver to believe something. The sender's \emph{communicative intention} is then their intending for the receiver to believe that they have an informative intention. The sender may then express or convey this communicative intention to their receiver with an \emph{ostensive} behavior. If their receiver receives their communicative intention, then ostensive-inferential communication has occurred.

Currently, there is no evidence that any species other than humans communicate ostensively \citep{Scott-Phillips18-communication}. As a result, not only may one distinguish between the code model and the ostensive-inferential model to define what communication entails, one may also conceptualize these two models as two different types of communication. The code model then captures the way that non-human animals communicate, and the ostensive-inferential model then captures the way that humans communicate with each other.

It is this ostensive-inferential model that I will consider to form the definition of \emph{human communication} throughout this thesis. When I speak of communication broadly construed, I will adhere to the definition of communication by \citet{Freeberg19}. This definition is compatible with the code model outlined by \citet{Scott-Phillips18-communication};
\largeTodo{This is a weird comment, because it implicates maybe that the ostensive-inferential model is not compatible with the Freeberg definition. Make this implicature explicit, say something about how the Freeberg definition compares to the two models we have.}
the code model, however, provides a level of detail that will not be necessary for our discussions of non-human animal communication.
\largeTodo{Explicate the relation between the two models/definitions, because now it's very unclear why and how I'm using two different definitions at the same time}
\largeTodo{Actually, this section could do with a lot more work on intentional and non-intentional communication: read up on this, and make things explicit here.}

\largeTodo{Gricean objection: talking to the self is also communicating, albeit to an imaginary audience, or you yourself are the addressee. The point I make about using language without communicating is a bit slippery, philosophically controversial, and is maybe not relevant. Communicating without language \emph{is} relevant, but maybe this is not the place to point that out; it can also become implicitly or explicitly clear throughout or at the end of the chapter.}
As a last side note: while we would often equate human communication with linguistic communication, humans can easily communicate without language -- for example, using glances or gestures\footnote{One might even speculate that any human behavior can be used to communicate.}, see also \cref{sec:comm:ontogeny} -- and can use language without communicating -- for example, when one is talking to oneself. Therefore, although language will come up now and then throughout this chapter, it is not our object of focus at the present moment.

\section{Communication in non-human animals}
\label{sec:comm:phylogeny}

\largeTodo{This section deserves a mention of Chomsky's work. Also, it could do with more references to experimental evidence.}

Now we turn to the first methodological question and we look at the communication of other animals, especially those that we are evolutionarily closely related to.
As already mentioned, one fundamental difference between the communication of non-human animals and humans is by which model their communication is best described: the code model and the ostensive-inferential model, respectively \citep{Scott-Phillips15-primate, Scott-Phillips18-communication}.
\smallTodo{The model that best describes them is a \emph{consequence} of the difference between the two: so the difference is that which makes one model work better for NHA and the other better for humans. Reformulate}

Communication is used by non-human animals for a wide range of purposes, and it can be elicited by a number of stimuli. Moreover, communicative behaviors can manifest themselves in different modalities: not only can animals communicate through vocalizations, they may also communicate through gestures or glances.
\todo[inline]{Write more on gestural communication in apes (is mentioned in \citet{Tomasello08-origins})}

One can broadly distinguish between communication in aggressive and cooperative interactions \citep{SeyfarthCheney03}. In aggressive interactions, primates may for example use communication in order to intimidate, by using it to signal their size and willingness to fight. This minimizes the chances of a physical altercation or fight actually happening, which minimizes the chance of injury for both the dominant and the subordinate animal.
In cooperative interactions on the other hand, where the interests of the signaler and the receiver overlap, communication can be used to alert others of predators, to coordinate foraging activities and to facilitate social interactions:
\begin{quoting}
    {[information acquired by listeners]} may include, but is not limited to, information about predators or the urgency of a predator’s approach, group movements, intergroup interactions, or the identities of individuals involved in social events
    \hfill \citep[p.~168]{SeyfarthCheney03}
\end{quoting}

In animals in general, vocalizations are most often elicited not by just one stimulus, but rather a complex combination of them. Moreover, the "history of interactions between the individuals involved" \citep[p.~151]{SeyfarthCheney03} can also play a role in eliciting vocalizations. As for the 'immediate' stimuli eliciting vocalizations, we may distinguish between sensory stimuli on the one hand and mental stimuli on the other. Sensory stimuli then refer to stimuli received through the external senses, such as visual, auditory and olfactory senses.
For example, if I stop petting my dog\todo{Reconsider whether using domesticated animals as examples is valid for my point: maybe a bit messy, evolutionarily speaking} (sensory stimulus), she will direct her gaze at me (communicative behavior) to indicate that she would like me to continue.
Mental stimuli on the other hand can be viewed as the mental states an animal attributes to another animal.
For example, \todo{Add example}.
This type of stimulus elicits the majority of vocalizations in human conversation, but there is no evidence that the attribution of mental states to others causes vocalizations in other animals, except for possibly chimpanzees \citep{SeyfarthCheney03}.

\section{Communication in children's development}
\label{sec:comm:ontogeny}

\todo[inline]{This section needs to be structured better}

Now that we have seen how communication works in non-human animals, let us turn to how children start communicating throughout their development.
Around their first birthdays, children start communicating ostensively by pointing \citep{Tomasello08-origins}. Although at first glance pointing may seem like a simple behavior, it may be used in a number of communicative contexts to convey a fairly wide range of messages and intentions.
For example, infants may point at a cup to indicate that they want to drink from it (i.e., pointing to request), but they may also point to a hidden object that their parent is searching for (i.e. pointing to inform).

On the classic account, pointing can serve either of two communicative motives: an imperative motive, in which the pointer requests things from someone, and a declarative motive, in which the pointer shares their experiences and emotions with someone.
This account can be extended upon by distinguishing between declaratives as expressives (sharing attitudes and emotions) and declaratives as informatives (providing information), and by furthermore conceiving of imperatives as a continuum, with the underlying motive ranging on a scale from individualistic -- e.g. forcing someone to do something -- to cooperative, e.g. indirectly making a request to someone by informing them of some desire \citep{Tomasello08-origins}.

The fact that pointing is a fairly complex communicative act is underscored by the fact that non-human animals are not able to understand pointing in the same way humans are. \todo{Add reference for this}
The hypothesis is that in order to communicate intentionally, like children begin doing around their first birthday, first the skills and motivations for \emph{shared intentionality} need to be present in the infant; that without skills of shared intentionality, infants could only communicate intentionally, but not cooperatively.
Shared intentionality is the "ability to participate with others in interactions involving joint goals, intentions, and attention" \citep[p.~139]{Tomasello08-origins}. Communicative pointing behaviors in infants emerge around the same time as skills and motivations of shared intentionality do, which according to Tomasello confirms this hypothesis of dependency between them.

Tomasello further investigates what he calls \emph{pantomiming} or \emph{iconic gestures}, which are symbolic or representational gestures. \todo{Give example}
He presents empirical evidence\todo{Discuss this empirical evidence(?)} that these kinds of gestures rely heavily on convention for their meaning, and that the acquisition and usage of these conventions bears a strong resemblance to the acquisition and usage of language.

In short, infants first acquire the skills and motivations needed for shared intentionality; then they acquire the skills and motivations for communicative pointing; and then they acquire the ability to use iconic gestures and language around the same time.

\section{What is the function of communication?}
\label{sec:comm:function}

Finally, let us have a look at different conceptions of what the function of communication might be. We have already seen that we can broadly distinguish between competitive and cooperative functions of communication \citep{SeyfarthCheney03}.
For competitive, or aggressive, motives, one might use communication for attempting to intimidate a rival in the competition for food or for a mate. It would be evolutionarily beneficial to the communicator to intimidate verbally rather than physically because of a reduced risk of injury.
According to the theory of human self-domestication, a number of human traits can be explained by a process of 'self-domestication', in which females select sexually for less aggressive males. It has been proposed that because of this reduction in physical aggression, interactions between individuals became longer and more frequent throughout evolutionary history, which allowed for our language and communication to become more complex \citep{Benitez21}.

As for the cooperative functions of communication, \citet{Tomasello08-origins} proposes three different motives underlying communication: sharing (of emotions and attitudes), informing, and requesting (help to achieve goals).

It is apparent that communication crucially enables individuals to cooperate and collaborate \todo{I feel like this could use more argumentation?}. In addition, more sophisticated communication would make more sophisticated forms of cooperation and collaboration possible, from coordinated hunting activities to the social institutions we depend on nowadays.
But in order for us to complete the causal chain, we need to have a look at what makes cooperation and collaboration something evolution has selected for at all.

In cooperative behaviors, one can distinguish between altruistic behavior (sacrificing something in some way for the benefit of another individual) and collaborative behavior (working together with another individual for mutual benefit) \citep{Tomasello09}. In the case of the latter, the evolutionary causation is straight-forward: the mutual benefit of the outcome constitutes the evolutionary pressure that has selected for it. In the case of the former though, it is not immediately obvious that helping another person without benefit to the self would be evolutionarily advantageous. \citep{Tomasello09} hypothesizes that altruism can be best explained in the broader picture of the social group, considering the norms that are enforced and being conformed to within the group and the shared intentionality inherent in the group.
Moreover, he emphasizes the role of both human biology and human culture in shaping humans' cooperative tendencies; see also \cref{ch:evolution}.

\todo[inline]{Elaborate more on this? Or is the argument clear enough}
