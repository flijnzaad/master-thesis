\chapter{Why do we communicate?}
\label{ch:communication}

In order to answer the question of how advanced reasoning may have evolved to further communication, we will first need to examine communication in its own right. In doing so, we will first fix our definition of communication, and then we will consider each of the methodological questions raised in \cref{sec:evol-method} as they pertain to communication.

\section{Conceptions of communication}
\label{sec:comm:definition}

Before we can take a look at the evolutionary history, developmental origins and functions of communication, we should settle on a definition of communication. This will turn out to not only be trickier than one perhaps might expect, but it is also very important since it determines the frame of our research question.
\todo[inline]{I should have a look at what Mercier (\& Sperber) define as communication}

There are many different ways organisms may communicate with each other, and indeed many different ways in which one may define communication. In all cases, communication is an process necessarily involving a signaler (a sender) and at least one receiver (a listener).

In their discussion of communication as it relates to social cognition, \citet{Freeberg19} define communication as follows:
\begin{quoting}
    Communication involves an action or characteristic of one individual that influences the behaviour, behavioural tendency or physiology of at least one other individual in a fashion typically adaptive to both
    \hfill (p.~281)
\end{quoting}
This is a very broad conception of communication; on this definition, all organisms communicate. And indeed, this is a definition that is followed by a number of other researchers.\todo{weird sentence}
\todo[inline]{provide some examples of communication here}

% scott-phillips definition (from sperber & wilson)
However, when it comes to human communication, the definition can be refined a bit. \citet{Scott-Phillips18-communication, Scott-Phillips15-primate}\todo{reformulate} discusses two different models of communication: the classical \emph{code model} of communication, and the \emph{ostensive-inferential} model of communication.
On the former model, communication involves processes of coding (on the side of the sender) and decoding (on the side of the receiver) messages. The coding on the side of the sender involves a mapping between the state of the world and a behavior (namely the signal they send), and the decoding on the side of the receiver involves a mapping between two behaviors: the signal received, and a subsequent response. If the mappings are properly calibrated to each other, communication between sender and receiver can be said to have occurred.
\todo{this is a really bad explanation}
However, in order to exactly capture human communication, the code model is too simplistic, because it fails to account for the \emph{underdeterminacy of meaning}: in merely looking at the behavior of the sender, at the content of the message, one cannot account for the meaning that the message conveys to the sender \citep{Scott-Phillips18-communication}. \todo{add example}
Therefore, a move away from the code model of communication towards the ostensive-inferential model of communication would be in order. This model takes into account the intentionality inherent in human communication.
On the ostensive-inferential model, one may speak of a speaker's \emph{informative intention}, which is when they intend for their listener to believe something. The speaker's \emph{communicative intention} is then their intention for the listener to believe that they have an informative intention. The speaker may then express or convey this communicative intention to their listener with an \emph{ostensive} behavior. If their listener receives their communicative intention, then ostensive-inferential communication has occurred.

Currently, there is no evidence that any species other than humans communicate ostensively \citep{Scott-Phillips18-communication}. Not only may one distinguish between the code model and the ostensive-inferential model, one can also conceptualize these as two different kinds of communication: the former model captures the way that non-human animals communicate, and the latter captures the way that humans communicate with each other. This complicates the view a bit.\todo{add something here}

Important to note is that while we most associate human communication with language, humans can easily communicate without language -- for example, using glances or gestures\footnote{One might even speculate that any human behavior can be used to communicate.} -- and can use language without communicating -- for example, when one is talking to oneself. Therefore, language will come up now and then throughout this chapter, but it is not our object of focus at the present moment.

\section{Communication in non-human animals}
\label{sec:comm:phylogeny}

In order to illuminate communication in humans, it is necessary to look at the communication of other animals, especially those that we are evolutionarily closely related to.
As already mentioned, one fundamental difference between the communication of non-human animals and humans is by which model their communication is best described (the code model and the ostensive-inferential model, respectively) \citep{Scott-Phillips15-primate, Scott-Phillips18-communication}.

Communication is used by non-human animals for a wide range of purposes, and it can be elicited by a number of stimuli.
One can broadly distinguish between communication in aggressive and cooperative interactions \citep{SeyfarthCheney03}. In aggressive interactions, primates may for example use communication in order to intimidate, by using it to signal their size and willingness to fight. This minimizes the chances of a physical altercation or fight actually happening, which minimizes the chance of injury for the animal.
In cooperative interactions on the other hand, where the interests of the signaler and the receiver overlap more, communication can be used to alert others of predators, to coordinate foraging activities and to facilitate social interactions:
\begin{quoting}
    {[information acquired by listeners]} may include, but is not limited to, information about predators or the urgency of a predator’s approach, group movements, intergroup interactions, or the identities of individuals involved in social events
    \hfill \citep[p.~168]{SeyfarthCheney03}
\end{quoting}

In animals in general, vocalizations are most often elicited by a "complex combination of stimuli" \citep[p.~150]{SeyfarthCheney03}, which may be 'immediate', or dependent on the "history of interactions between the individuals involved" \citep[p.~151]{SeyfarthCheney03}. As for the immediate features, we may distinguish between sensory stimuli on the one hand and mental stimuli on the other. Sensory stimuli then refer to stimuli received through the external senses, such as visual, auditory and olfactory senses. Mental stimuli on the other hand, can be viewed as the mental states an animal attributes to another animal. This type of stimuli elicits the majority of vocalizations in human conversation, but there is no evidence that the attribution of mental states to others causes vocalizations in other animals, except for possibly chimpanzees \citep{SeyfarthCheney03}.

\section{Communication in children's development}
\label{sec:comm:ontogeny}

\section{What is the function of communication?}
\label{sec:comm:function}
