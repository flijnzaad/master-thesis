\chapter{Why do we communicate?}
\label{ch:communication}

\largeTodo{In general: this chapter is very dense, you can go at a bit of a slower pace most of the time. But pace will also be better once elaborations and examples are added.}
In order to answer the question of how advanced reasoning \smallTodo{Terminology: probably drop "advanced" here. Possibly get back to this after writing Chapter 3} may have evolved to further communication, we will first need to examine communication in its own right: why do we communicate?
In order to answer this question, we will discuss each of the methodological questions raised in \cref{sec:evo-conclusion} as they pertain to communication\smallTodo{Refer to the numbers or codes of the Tinbergen questions}. But before we can take a look at the evolutionary history, the developmental origins and the functions of communication, we must first fix a definition of communication, since this determines the frame of our research question.
\largeTodo{This whole introduction could be clearer: why do we look at communication now? Can use a bit more words.}

\section{Conceptions of communication}
\label{sec:comm:definition}

\largeTodo{Missing from this section: what Mercier (\& Sperber) define as, and/or have to say about, communication}

There are many different ways organisms may communicate with each other, and indeed many different ways in which one may define communication. In any case, communication is an process necessarily involving a signaler (a sender) and at least one receiver (a listener).

Some authors regard communication to inherently be a tool of persuasion, which then translates to their very definition of communication: for example, on the manipulative model of communication, communication can be taken to occur "when an animal, the actor, does something which appears to be the result of selection to influence the sense organs of another animal, the reactor, so that the reactor's behavior changes to the advantage of the actor" \citep[p.~283]{DawkinsKrebs78}.\smallTodo{Rewrite this so that the quote is actually typeset as a quote, for emphasis}
\largeTodo{Say something about this quote in your own words: comment on it, this also justifies you using the quote. Point the reader to why you use it.}
One may also notice that this definition has a teleological explanation embedded in it as well (see \cref{sec:teleology}).\smallTodo{Elaborate on this: make it more explicit}
I mention this definition only for completeness' sake \smallTodo{Maybe not necessary to mention this}, because I believe this definition to be insufficiently parsimonious in its assumptions about the function of communication.
\largeTodo{Elaborate on this: what are the assumptions they make, in what sense are they strong, and why are they too strong for my liking (i.e. what's wrong with them)?}

In their discussion of communication as it relates to social cognition, \citet{Freeberg19} define communication as follows:
\begin{quoting}
    Communication involves an action or characteristic of one individual that influences the behaviour, behavioural tendency or physiology of at least one other individual in a fashion typically adaptive to both
    \hfill (p.~281)
\end{quoting}
\largeTodo{Explain this in own words; and explain the difference with the definition of \citet{DawkinsKrebs78}. The difference lies in "to the advantage of"}
This is a very broad conception of communication; on this definition, all organisms, from bacteria to fungi to plants to animals, communicate.
\example{Provide examples for each of these kingdoms here, nice illustration}

\citet{Scott-Phillips18-communication, Scott-Phillips15-primate} contrasts two different models of communication with each other: the classical \emph{code model} of communication, and the \emph{ostensive-inferential} model of communication.
In the former model, communication involves processes of coding and decoding messages. The coding, on the side of the sender, involves a mapping between the state of the world and a behavior (namely the signal they send). The decoding, on the side of the receiver, involves a mapping between two behaviors: the signal sent, and a subsequent response of the receiver. If the mappings are properly calibrated to each other, communication between sender and receiver can be said to have occurred.

\largeTodo{This paragraph could do with a lot more philosophical discussion: Quine should definitely be mentioned, even if it's just in a footnote, because it's like this milestone philosophy thing that would be a glaring omission to philosophers reading this.
Also, the usage of "meaning" and "content" here is a big philosophical no-no, either explicate what you mean by them or change your terminology. You could also just acknowledge or specify that you use the intuitive, colloquial meaning, not the technical one. Read also the Putnam paper on the ants and Winston Churchill?}
However, in order to capture human communication, the code model is too simplistic, because it fails to account for the \emph{underdeterminacy of meaning}: in merely looking at the content of the message, one cannot account for the meaning that the message conveys to the sender \citep{Scott-Phillips18-communication}. \example{Add example}
Therefore, a move away from the code model of communication towards the \emph{ostensive-inferential} model of communication would be in order. This model takes into account the intentionality inherent in human communication.

\largeTodo{This paragraph also needs a lot more work:
(1) mention Sperber \& Wilson and relevance theory (credit where credit's due!), and that their theory is neo-Gricean, because most philosophers will be familiar with Grice. (2) Add an example, this needs to be way clearer and elaborated more if this will be my definition of human communication. (3) Define ostensive behavior, also with an example. It's apparent right now that I don't fully understand this model myself. (4) Add a comment on how the underdeterminacy is captured better by the ostensive-inferential model}
In the ostensive-inferential model, one may speak of a sender's \emph{informative intention}, which is their intending for the receiver to believe something. The sender's \emph{communicative intention} is then their intending for the receiver to believe that they have an informative intention. The sender may then express or convey this communicative intention to their receiver with an \emph{ostensive} behavior. If their receiver receives their communicative intention, then ostensive-inferential communication has occurred.

Currently, there is no evidence that any species other than humans communicate ostensively \citep{Scott-Phillips18-communication}. As a result, not only may one distinguish between the code model and the ostensive-inferential model to define what communication entails, one may also conceptualize these two models as two different types of communication. The code model then captures the way that non-human animals communicate, and the ostensive-inferential model then captures the way that humans communicate with each other.

It is this ostensive-inferential model that I will consider to form the definition of \emph{human communication} throughout this thesis. When I speak of communication broadly construed, I will adhere to the definition of communication by \citet{Freeberg19}. This definition is compatible with the code model outlined by \citet{Scott-Phillips18-communication};
\largeTodo{This is a weird comment, because it implicates maybe that the ostensive-inferential model is not compatible with the Freeberg definition. Make this implicature explicit, say something about how the Freeberg definition compares to the two models we have.}
the code model, however, provides a level of detail that will not be necessary for our discussions of non-human animal communication.
\largeTodo{Explicate the relation between the two models/definitions, because now it's very unclear why and how I'm using two different definitions at the same time}
\largeTodo{Actually, this section could do with a lot more work on intentional and non-intentional communication: read up on this, and make things explicit here. Could choose to define communication and non-intentional communication, or intentional communication and communication. Need to make a choice in this, explain it, and then be clear and \emph{consistent!} about using this terminology.}

\largeTodo{Gricean objection: talking to the self is also communicating, albeit to an imaginary audience, or you yourself are the addressee. The point I make about using language without communicating is a bit slippery, philosophically controversial, and is maybe not relevant. Communicating without language \emph{is} relevant, but maybe this is not the place to point that out; it can also become implicitly or explicitly clear throughout or at the end of the chapter.}
As a last side note: while we would often equate human communication with linguistic communication, humans can easily communicate without language -- for example, using glances or gestures\footnote{One might even speculate that any human behavior can be used to communicate.}, see also \cref{sec:comm:ontogeny} -- and can use language without communicating -- for example, when one is talking to oneself. Therefore, although language will come up now and then throughout this chapter, it is not our object of focus at the present moment.

\section{Communication in non-human animals}
\label{sec:comm:phylogeny}

\largeTodo{This section deserves a mention of Chomsky's work on animal communication: something about humans having language, and NHA only having stimulus-dependent responses? Also, this section could do with more references to experimental evidence.}

Now we turn to the first methodological question\smallTodo{Are they numbered?} and we look at the communication of other animals, especially those that we are evolutionarily closely related to.
As already mentioned, one fundamental difference between the communication of non-human animals and humans is by which model their communication is best described: the code model and the ostensive-inferential model, respectively \citep{Scott-Phillips15-primate, Scott-Phillips18-communication}.
\smallTodo{The model that best describes them is a \emph{consequence} of the difference between the two: so the difference is that which makes one model work better for NHA and the other better for humans. Reformulate}

Communication is used by non-human animals for a wide range of purposes, and it can be elicited by a number of stimuli. Moreover, communicative behaviors can manifest themselves in different modalities: not only can animals communicate through vocalizations, they may also communicate through gestures or glances.
\largeTodo{Write more on gestural communication in apes (is mentioned in \citet{Tomasello08-origins})}

One can broadly distinguish between communication in aggressive and cooperative interactions \citep{SeyfarthCheney03}. In aggressive interactions, primates may for example use communication in order to intimidate, by using it to signal their size and willingness to fight. This minimizes the chances of a physical altercation or fight actually happening, which minimizes the chance of injury for both the dominant and the subordinate animal.
In cooperative interactions on the other hand, where the interests of the signaler and the receiver overlap, communication can be used to alert others of\smallTodo{About?} predators, to coordinate foraging activities and to facilitate social interactions:
\begin{quoting}
    {[information acquired by listeners]} may include, but is not limited to, information about predators or the urgency of a predator’s approach, group movements, intergroup interactions, or the identities of individuals involved in social events
    \hfill \citep[p.~168]{SeyfarthCheney03}
\end{quoting}

In animals in general, vocalizations are most often elicited not by just one stimulus, but rather a complex combination of them. Moreover, the "history of interactions between the individuals involved" \citep[p.~151]{SeyfarthCheney03} can also play a role in eliciting vocalizations. As for the 'immediate' stimuli eliciting vocalizations, we may distinguish between sensory stimuli on the one hand and mental stimuli on the other. Sensory stimuli then refer to stimuli received through the external senses, such as visual, auditory and olfactory senses.
\example{Take a different example: this is too much anecdotal, and with domesticated animals there's issues of evolutionary tractability as well as anthropomorphization. Get a reference on this}
For example, if I stop petting my dog (sensory stimulus), she will direct her gaze at me (communicative behavior) to indicate that she would like me to continue.
Mental stimuli on the other hand can be viewed as the mental states an animal attributes to another animal.
\largeTodo{Glaring omission: the large controversy surrounding theory of mind. You can make assumptions that are controversial, so long as you acknowledge the controversy and give good reasons for making the assumption. Show that you're aware of the debates. Consider taking the consensus in the field from Kristin Andrews' textbook "Animal minds"}
For example, \example{Add example from the literature}.
This type of stimulus elicits the majority of vocalizations in human conversation, but there is no evidence that the attribution of mental states to others causes vocalizations in other animals, except for possibly chimpanzees \citep{SeyfarthCheney03}.

\section{Communication in children's development}
\label{sec:comm:ontogeny}

\largeTodo{This section tries to fit a lot of information in little space: it is too dense. Also, the structure is weird and unclear. It could do with a whole overhaul.}

Now that we have seen how communication works in non-human animals, let us turn to how children start communicating throughout their development.
Around their first birthdays, children start communicating ostensively by pointing \citep{Tomasello08-origins}.
\largeTodo{Check the exact timing of this: Tomasello (1999) talks about the "nine-month revolution", and the difference between 9 and 12 months is very big in human children. Possibly directly cite experimental sources.}
Although at first glance pointing may seem like a simple behavior, it may be used in a number of communicative contexts to convey a fairly wide range of messages and intentions.
For example, infants may point at a cup to indicate that they want to drink from it (i.e., pointing to request), but they may also point to a hidden object that their parent is searching for (i.e. pointing to inform).
\largeTodo{Why do we focus on pointing? Elaborate more on why pointing constitutes communication, and other earlier stuff doesn’t. For example, raising arms to parent: a behavior that we inherited from our ape ancestors (instinct to climb on parent). Is that communication? What distinguishes these earlier from the later interactions? And consider for example the communicative function of crying, is this mentioned by Tomasello or by other authors? How is that different from vocalizations in animals? Simple stimulus-response? This basically all comes down/back to the definition of communication I adhere to, and intentionality in it.}

\largeTodo{This paragraph is way too dense: she doesn't understand this upon reading. It's a matter of unpacking the notions.}
On the classic account, pointing can serve either of two communicative motives: an imperative motive, in which the pointer requests things from someone, and a declarative motive, in which the pointer shares their experiences and emotions with someone.
This account can be extended upon by distinguishing between declaratives as expressives (sharing attitudes and emotions) and declaratives as informatives (providing information), and by furthermore conceiving of imperatives as a continuum, with the underlying motive ranging on a scale from individualistic -- e.g. forcing someone to do something -- to cooperative, e.g. indirectly making a request to someone by informing them of some desire \citep{Tomasello08-origins}.

The fact that pointing is a fairly complex communicative act is underscored by the fact that non-human animals are not able to understand pointing in the same way humans are. \smallTodo{Add reference for this}
The hypothesis is that in order to communicate intentionally,
\largeTodo{On the current definition, isn't all communication intentional? Should really settle on this, and then change the wording here as applicable. Read up also on Putnam's Brain in a Vat article, this might be illuminating philosophically. Grice also has some readings on natural and non-natural meaning, distinguishing between different types of communication. Something about intentionality? Make the distinction between intentional and non-intentional stuff very explicit, because it's one of the most important things in this chapter.}
like children begin doing around their first birthday, first the skills and motivations for \emph{shared intentionality} need to be present in the infant; that without skills of shared intentionality, infants could only communicate intentionally, but not cooperatively.
\largeTodo{Elaborate more on intentional vs. cooperative communication: does Tomasello (1999) have something on this? Something to do with joint attention.}
Shared intentionality is the "ability to participate with others in interactions involving joint goals, intentions, and attention" \citep[p.~139]{Tomasello08-origins}. Communicative pointing behaviors in infants emerge around the same time as skills and motivations of shared intentionality do, which according to Tomasello confirms this hypothesis of dependency between them.
\largeTodo{Disentangle communication and shared intentionality, because up until this point, they seem to be the same thing. Go a bit slower in this section.}
\largeTodo{Reconsider the usage of the terminology "skills and motivation", and make explicit exactly what you mean by them. Would abilities be a better word? Why does \citet{Tomasello08-origins} use skills? Why are motivations included? What are motivations?}

Tomasello further investigates what he calls \emph{pantomiming} or \emph{iconic gestures}, which are symbolic or representational gestures. \example{Give example}
He presents empirical evidence\largeTodo{Discuss this empirical evidence} that these kinds of gestures rely heavily on convention for their meaning, and that the acquisition and usage of these conventions bears a strong resemblance to the acquisition and usage of language.

\largeTodo{This trajectory should also be more clear throughout the section}
In short, infants first acquire the skills and motivations needed for shared intentionality; then they acquire the skills and motivations for communicative pointing; and then they acquire the ability to use iconic gestures and language around the same time.

\section{What is the function of communication?}
\label{sec:comm:function}

\largeTodo{Make a terminological choice between aggressive and competitive, and be consistent in your usage of them, also in \cref{sec:comm:phylogeny}; maybe even define the terms.}
Finally, let us have a look at the function of communication.
We have already seen that we can broadly distinguish between competitive and cooperative functions of communication \citep{SeyfarthCheney03}.
For competitive, or aggressive, motives, one might use communication to intimidate a rival in the competition for food or for a mate. It would be evolutionarily beneficial to the communicator to intimidate verbally rather than physically, because of a reduced risk of injury for verbal intimidation compared to physical intimidation.
\smallTodo{See meeting notes about how to cite sources here!}
According to the theory of human self-domestication, a number of human traits can be explained by a process of \emph{self-domestication}, in which females prefer -- and thus sexually select for -- less aggressive males. It has been proposed that because of this reduction in physical aggression, interactions between humans became longer and more frequent throughout evolutionary history, which allowed for our language and communication to become more complex \citep{Benitez21}.
\todo[inline]{Maybe restructure this whole section, because I also mention self-domestication later: reconsider the usefulness of pulling apart the competitive and cooperative function, maybe they're two sides of the same coin?}

As for the cooperative functions of communication, it is apparent that communication crucially enables individuals to cooperate and collaborate.\todo[inline]{Elaborate on the above statement: not very convincing right now}
% Furthermore, more sophisticated communication would make more sophisticated forms of cooperation and collaboration possible, from coordinated hunting activities to the social institutions we depend on nowadays.
% \todo{Cite Tomasello on this?}
% \largeTodo{To what extent do animals participate in coordinated hunting activities? I remember reading about it, about monkeys launching a coordinated attack, was that in \citet{Tomasello09}? Find this reference}
To appreciate how communication facilitates cooperation, let us now consider what makes cooperation itself evolutionarily beneficial. Moreover, in order to complete the causal chain, we will have a look at how cooperation could have evolved and the role that communication plays in it. We will do so by drawing extensively from Michael Tomasello's comprehensive \citeyear{Tomasello09} book \emph{Why We Cooperate}.

\subsection{Human cooperation and its evolution}

\todo[inline]{Define cooperation vs. collaboration and be consistent in this terminology}

% In cooperative behaviors, one can distinguish between altruistic behavior (sacrificing something in some way for the benefit of another individual) and collaborative behavior (working together with another individual for mutual benefit) \citep{Tomasello09}. In the case of the latter, the evolutionary causation is more straight-forward: the mutual benefit of the outcome constitutes the evolutionary pressure that has selected for it. In the case of the former though, it is not immediately obvious that helping another person without benefit to the self would be evolutionarily advantageous. 
% \citet{Tomasello09} hypothesizes that altruism can be best explained in the broader picture of the social group, considering the norms that are enforced and being conformed to within the group and the shared intentionality inherent in the group.

% For reasons that will become clearer in \cref{ch:scrutiny}, it will also be useful to have a look at how human cooperation could have evolved at all.
Tomasello argues that somewhere along the evolutionary timeline, humans must have been "put under some kind of selective pressure to collaborate in their gathering of food--they become obligate collaborators--in a way that their closest primate relatives were not" \citep[p.~75]{Tomasello09}. (Notably, this selective pressure is a missing link in his otherwise very convincing story.)
He elaborates by noting that in general, evolution may select for sociality in animals because living together in a social group protects the group's members against predation: it is easier to defend oneself in the context of a group. The group however also brings disadvantages with it when it comes to foraging for food, since the members of the group are competitors in the acquisition of food. This is especially the case when the source of food is 'clumped', such as in a prey animal, rather than dispersed, such as in a plain of grass. The clumped source of food raises the issue of how to share the food amongst the members of the social group.
Tomasello enumerates a number of different hypotheses to explain how humans could have broken out of what he calls "the great-ape pattern of strong competition for food, low tolerance for food sharing, and no food offering at all" \citep[p.~83]{Tomasello09}; in other words, how humans could have evolved to be more tolerant and trusting, and less competitive about food.
Firstly, as (due to the aforementioned missing link) it became necessary for humans to forage collaboratively, it could have been evolutionarily advantageous to be more tolerant and less competitive, which would explain its having evolved.
Secondly, Tomasello notes it could be the case that humans went through a process of self-domestication, which eliminated aggressive, predatory or greedy individuals from the group; see \citet{Benitez21} for more on this.
Thirdly, the evolution of tolerance and trust could be related to what is called cooperative breeding, where the responsibility of child-rearing falls on more individuals than just the mother of the child. This cooperative breeding may have selected for pro-social skills and motivations.\todo{This might warrant more explanation}

Tolerance and trust then constitute a foundation upon which coordination and communication can be 'built', so to speak: they provide an environment in which more elaborate collaboration can evolve. In Tomasello's words,
\begin{quoting}
    there had to be some initial emergence of tolerance and trust (\ldots) to put a population of our ancestors in a position where selection for sophisticated collaborative skills was viable
    \hfill (p.~77)
\end{quoting}

In order to then arrive at the full picture of human cooperative activity, the final step to consider is that of social norms and institutions. As before, there is a missing link in this story, in this case it concerns how mutual expectations between individuals arise and eventually become norms. (Tomasello describes it as "one of the most fundamental questions in all of the social sciences" (p. 89).)
Norms may be defined as "socially agreed-upon and mutually known expectations bearing social force, monitored and enforced by third parties" \citep[p.~87]{Tomasello09}. Norms receive their force not only from the threat of punishment by others if the norm is violated, but also from a kind of social rationality within the collaborative activity. Individuals recognize their dependence on each other for reaching their joint goal. Just as it would be individually irrational to act in a way that thwarts your own goal, it would be socially irrational to act in a way that thwarts your joint goal.

Let us now briefly summarize the evolutionary timeline of human cooperation according to Michael Tomasello.
\todo[inline]{This needs to be laid out still}
% At some point, for reasons as of yet unknown to us, foraging for food collaboratively rather than individualistically became beneficial for humans. Somewhere 

Now that we know a bit more about how human cooperation might have evolved, and how communication could have featured in it, we will touch on some foundational concepts for the argumentative theory of reasoning, that relate to the evolution of communication.

\subsection{The stability of communication}

\todo[inline]{How to go about citing Scott-Phillips here? The whole section is a discussion of his \citeyear{Scott-Phillips08} paper}

If communication between individuals of a species persists throughout evolution, we may speak of it as stable. The stability of communication is considered by some as the 'defining problem' of animal signalling research \citep{Scott-Phillips08}. It is not a trivial problem by any means: the stability of a communication system is threatened by evolutionary pressures on the communicator to 'defect', as it were. As \citet{Scott-Phillips08} describes it,
\begin{quoting}
    If one can gain through the use of an unreliable signal then we should expect natural selection to favour such behaviour. Consequently, signals will cease to be of value, since receivers have no guarantee of their reliability. This will, in turn, produce listeners who do not attend to signals, and the system will thus collapse in an evolutionary retelling of Aesop’s fable of the boy who cried wolf.
    \hfill (p.~275)
\end{quoting}
In the context of human communication: if it can be advantageous for me to lie, deceive or mislead someone, then it would evolutionarily make sense for me to do so; yet then it would make evolutionary sense for you to stop listening to me, and as a consequence our system of communication would collapse.

There have been a number of attempts at explaining the reliability of animal communication in general. One such attempt is the \emph{handicap principle}, by which the signaller incurs costs (i.e., a handicap) for signalling, which thus guarantees the reliability of the signal. The paradigmatic example for the handicap principle is that of the peacock's tail. This tail is like a handicap for the peacock: not only does it take a lot of resources to grow the tail and carry it around, it also leaves the bird more vulnerable to predation because it is less agile with a large unwieldy tail. Yet \todo{right connective?}, a large tail signals to peahens that the peacock is fit enough to be able to incur these costs, and thus has a sexual advantage. \todo{cite the original paper coining the handicap principle}

However useful in explaining some cases of the reliability of animal communication, the handicap principle is not able to explain all of those cases: often, it is not the case that reliable signals are costly to produce \citep{Scott-Phillips08} \example{Add example}. Especially in the case of human communication, the handicap principle cannot account for its reliability, since it is in general not costly to produce utterances \citep{Scott-Phillips08}.
Thus, it remains to be shown how communication can be stable if signals are cost-free.

On the handicap principle, reliable signals are costly to produce, thus ensuring their reliability. An alternative explanation of the reliability of animal communication is the principle of \emph{deterrence}, whereby \emph{un}reliable signals are costly to produce, and consequently signallers are deterred from producing unreliable signals.
There are a number of ways in which producing unreliable signals may be costly to the signaller. Firstly, this is the case in a coordination game, where the signaller and receiver share some common interest with regard to the outcome of the interaction.\todo{possibly explain this more}
Secondly, if two individuals have repeated interactions, it may also be costly in the long term to produce unreliable signals, because it may hinder cooperation in the future.
Thirdly, producing unreliable signals may be costly to the signaller if false signals are punished by the receiver.

The 'logic of deterrents' applied to the case of human communication poses the following constraint \todo{terminology?} in order for the story about stability to work:
\begin{quoting}
    Sufficient conditions for cost-free signalling in which reliability is ensured through deterrents are that signals be verified with relative ease (if they are not verifiable then individuals will not know who is and who is not worthy of future attention) and that costs be incurred when unreliable signalling is revealed.
    \hfill \citep[p.~?]{Scott-Phillips08}
\end{quoting}
\todo{Figure out page numbers of \citet{Scott-Phillips08}}
In other words, if unreliable signals are caught relatively easily, and unreliable signallers incur costs for their unreliability, the reliability of communication is secured through deterrents.

Scott-Phillips goes on to state that these sufficient conditions are met in the case of human communication, since people may refrain from interacting with unreliable individuals in the future, which can be very costly for a social species such as humans.
Notably however, he does not explicate how the first sufficient condition is met in the case of human communication; we will return to this in \cref{sec:EV-scrutiny}.\todo{I think?}

\todo[inline]{Make a note about terminology: reliability vs. honesty}

\subsection{Benefits and costs of (dis)honesty}

\todo[inline]{This should be discussed somewhere; but where, and how?}
\todo[inline]{Dangers of misinformation?}

\subsection{Sperber on the evolution of testimony and argumentation}

\citet{Sperber01} considers testimony and argumentation from an evolutionary perspective and in doing so, provides important groundwork for his later work with Mercier (and others) on the relation between reasoning, argumentation and the stability of communication.

\todo[inline]{Later: check overlap with \cref{sec:arms-race}}

Testimony -- "the transmission of observed (or allegedly observed) information from one person to others" (p.~401) -- and argumentation -- "the defense of some conclusion by appeal to a set of premises that provide support for it" (ibid.) -- are two concepts central to human communication.
Sperber puts these two concepts in the perspective of evolution, and discusses in particular how they have figured in stabilizing communication over the course of evolutionary history.

A tempting way to look at communication is as a kind of 'cognition by proxy': through communication, one organism may access information another organism has obtained from its own perception or inference. For instance, if you tell me that there is milk in the fridge, I can through this act of communication benefit from the information derived from your perception of the milk carton in the fridge.
However, Sperber argues that, at least in the case of human communication, testimony does not amount to cognition by proxy. This is because testimony has different effects than direct perception does. Going back to our example, upon receiving your testimony stating that the milk is in the fridge, I am in a different cognitive state than if I would have perceived the milk carton there myself. Moreover, in human communication, interpretation and acceptance of utterances are two separate processes: recognizing what a speaker meant by their utterance is not the same as accepting it as true.
\todo[inline]{And so? Connect this to the ostensive-inferential model: is interpretation part of communication?}

The classical account of animal signalling (i.e.\@ communication) by \citet{DawkinsKrebs78} focuses only on the side of the communicator in the story, maintaining that the function of communication is to manipulate others. Sperber rejects this classical approach, arguing that the interests of the sender cannot be the only driving force in the evolution of communication.\todo{Paraphrase better}
He argues more or less in line with what we have already seen \citet{Scott-Phillips08} argue about the stability of communication.
Sperber states that for communication to have stabilized and continued to be stable between senders and receivers, both parties must have benefited from the action. In other -- game-theoretic -- terms, communication must (at least in the long run) be a positive-sum game, where both senders and receivers gain from the interaction.

In the case of receiving testimony from others, the receiver stands to gain from this testimony "only to the extent that it is a source of genuine (\ldots) information" (p.~404).
\todo[inline]{Where to talk about why gaining information is itself beneficial? Here or Ch. 4?}
On the side of the production of testimony, the sender stands to gain from this testimony because
\begin{quoting}
    it allows them to have desirable effects on the receivers' attitudes and behavior. By communicating, one can cause others to do what one wants them to do and to take specific attitudes to people, objects, and so on
    \hfill (p.~404)
\end{quoting}
He later elaborates on this by saying that getting others to accept your communicated message is not intrinsically beneficial. Rather, it is \emph{indirectly} beneficial, through bringing about these 'desirable effects' in others.
We briefly return to these observations (particularly the 'desirable effects') in \cref{ch:scrutiny} \todo{Where exactly?}.

Sperber goes on to cast these observations in game-theoretic terms by sketching out a payoff matrix for a one-off communicative event. In it, he considers that senders may be truthful or untruthful, and receivers may be trusting or distrusting. According to Sperber, the sender's gain amounts to whether they have the 'desired' effect on the receiver; therefore, the sender gains from the interaction if the receiver is trusting (since this means the sender's message is accepted), and loses from the interaction if the receiver is distrusting. This payoff is independent of the truthfulness of the sender. On the side of the receiver, their payoff \emph{is} dependent on the truthfulness of the sender: the receiver gains if they accept a truthful message, loses if they accept an untruthful message, and incurs no gain nor loss if they are distrusting and thus don't accept a message.
\todo[inline]{Should I reproduce this payoff matrix here? Not sure how relevant it is, but it's probably illuminating because the text is a bit chaotic}

Sperber notes that the optimal strategy for such a game varies with the circumstances for both 'players': it is not always beneficial to be truthful, nor always untruthful; nor is it beneficial to be always trusting, nor always distrusting. In other words, there is no one stable solution to this game.
This is especially the case once we move away from this simple one-off communicative event to an iterated game of communication, where not only short-term payoffs but also long-term payoffs determine the optimal strategy.

\todo[inline]{(Some parts of \citet{Sperber01} still missing here)}

Next, Sperber sketches out the steps in what he calls the 'evaluation-persuasion arms race', i.e.\@ the chain of evolutionary adaptations that has resulted in our mechanisms for argument production and evaluation.
He argues that the first step in this arms race was for the addressee to develop coherence checking, as a defense against the risks of deception by the communicator. The second step was then for the communicator to anticipate this coherence-checking by overtly displaying the coherence of their message to their addressee, which requires argumentative form. The next steps were on the side of the addressee to develop skills for examining these displays of coherence (i.e., arguments), and on the side of the communicator to 'improve their argumentative skills'.

\todo[inline]{Finish the discussion of \citet{Sperber01}}
\todo[inline]{Possibly revisit the arms race in a dedicated section in Ch. 4}

\subsection{Sperber and colleagues on epistemic vigilance}

\todo[inline]{Make this fit with the previous section once that's done}

One of the cornerstones of the argumentative theory of reasoning is the concept of \emph{epistemic vigilance}, which Sperber and colleagues (among whom Hugo Mercier) introduced in a seminal \citeyear{Sperber10} paper.

In the paper, Sperber and colleagues note that humans are dependent on communication, and they argue that this dependence leaves humans vulnerable to being deceived by others.
They state that misinformation or deception may "reduce, cancel, or even reverse" the gains that communication might bring to the addressee (p.~360).
Consequently, the information that addressees receive from communicators is only advantageous to them to the extent that the information is genuine.
Sperber and colleagues thus conclude that for this purpose, humans have evolved a suite of cognitive mechanisms for \emph{epistemic vigilance}.

A communicative act triggers not only comprehension in the addressee, but also epistemic vigilance alongside it.
In the grand scheme of human communication, vigilance maintains a balance between honesty and dishonesty; as Sperber and colleagues put it, "the audience's vigilance limits the range of situations where dishonesty might be in the communicators' best interest" (p.~368), which results in communication being honest most of the time.
Sperber and colleagues argue that vigilance is not a nicety, something that is only invoked sometimes; they maintain that vigilance is the default disposition of interlocutors in communicative settings.

One may distinguish between vigilance towards the \emph{source} of a message, and vigilance towards the \emph{content} of the message.
In regards to the source, the authors note that reliable sources must be both competent and benevolent (and this is dependent on the context).
Empirical evidence suggests that deceiving people can be quite beneficial, since experiments from deception detection research show that people are not good at detecting lies based on non-verbal behavioral cues.

\todo[inline]{Finish this: discuss \citet{Sperber10} in great detail for \cref{sec:EV-scrutiny} to make sense}

\section{Conclusion}

\todo[inline]{Summarize the conclusions from each section and segue into the next chapter}
