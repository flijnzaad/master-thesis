\chapter{Conclusion}
\label{ch:conclusion}

The aim of this thesis has been to critically evaluate Mercier and Sperber's argumentative theory of reasoning, which posits that the biological function of reasoning is to produce arguments and evaluate those of others.

To buttress this critical evaluation, we first had a look at evolutionary theory to provide some background to the evolutionary perspective perused in this thesis.
We considered biological and cultural evolution, causation in evolution, evolutionary psychology, terminological issues, and finally outlined a methodology for an evolutionary view on human communication.
This evolutionary view on human communication saw us discuss definitions of communication, consider empirical work on communication in non-human animals and in human development, and (arguably most importantly) analyze the function of communication. We introduced the problem of the stability of communication,discussed the evolution of human cooperation and how this relates to communication, and talked about lying and deception.
Next, we laid out Mercier and Sperber's views in detail. We discussed the 'evolutionary arms race' of communication and epistemic vigilance, and saw how these components came together in the argumentative theory of reasoning.
Finally, we used the findings so far to critically evaluate the ATR. Firstly, the function of human communication as it transpired from my research, is not easily reconciled with the ATR as-is. The ATR's focus and context is too narrow, leaving it weak from an evolutionary perspective.
Secondly, combining my own findings with criticisms by \citet{Michaelian13}, I conclude that some aspects of the epistemic vigilance story are unconvincing beyond a superficial glance.
Thirdly, as a theory, the ATR is unsatisfactory because it is oftentimes vague or underspecified, and it lacks important details.

All in all, the argumentative theory of reasoning is intuitively attractive, but rests on implicit assumptions that are currently unconvincing. Moreover, it is plagued by vagueness and lacks important details. These issues are not fatal, in the sense that the theory would be unsalvageable, but there is some serious work that needs to happen before the ATR can really be taken seriously.

\section{Further research}

These issues naturally give rise to avenues for further investigation.

For starters, it would be nice to see Mercier and Sperber (or perhaps others) resolve the vagueness in the ontological aspects of the ATR, that we discussed in \cref{sec:ont-atr}. This would require additional philosophical work, critically analyzing the ATR from a metatheoretical perspective.

Another interesting line of inquiry concerns the reconciliation of the ATR with a broader context. In \cref{sec:comm-func-scrutiny}, I mentioned that the ATR would be stronger if it were integrated within a broader evolutionary context of human cooperation à la \citet{Tomasello09}, and conjectured that it would be possible to do so. The obvious next step is then to determine whether it is indeed the case that these two stories can be reconciled. This endeavor would be one in evolutionary anthropology mostly, with a philosophical touch. The open questions posed by the questionable assumptions discussed in \cref{sec:comm-func-scrutiny} could perhaps be answered by findings from evolutionary anthropology.
If it turns out from this line of research that the ATR in its current version cannot be successfully integrated into a Tomasellan view on cooperation, then it could be fruitful to determine if the ATR could be modified in a such a way that it \emph{can} be integrated.

The last avenue for further research concerns a glaring and quite serious question that transpires from the discussion in \cref{sec:ont-atr}: is the ATR unfalsifiable?
To answer this question, one could carry out a detailed metatheoretical analysis into multiple aspects of the theory, such as the ontological and definitional rigor with which concepts are used, and how empirical work is imported to support the theory. Mercier and Sperber ascribe empirical believability to the ATR by formulating hypotheses that they take to be entailed by the theory. It would be justified to critically evaluate the status of these hypotheses against the whole theory, to assess how convincing it is that they are generated by the theory. Moreover, one could survey the field on evidence that refutes the hypotheses and evaluate its merit, to determine whether it could be the case that Mercier and Sperber are cherry-picking the empirical work they reference.

\section{Concluding words}
I believe that viewing the particularities of the human experience through an evolutionary lens is an infinitely fascinating and rewarding endeavor. It allows us to see things in a broader context, especially since with each passing day our lived experiences move further from those of our ancestors. Reconsidering our roots in this way might be some form of escapism, but it also allows us to understand ourselves and our place in the world better.

When I first read about the argumentative theory of reasoning, Mercier and Sperber's ideas felt optimistic to me. The ATR told me that we're not bad at reasoning; the so-called flaws of human reasoning are a feature, not a bug! Now that I have analyzed the theory in more detail, I feel differently about the story they tell. Their focus on deception and vigilance has a more pessimistic flair (see also \cref{sec:honesty-dishonesty}), which is (seemingly) at odds with the general outlook on human evolution that Michael Tomasello emanates (cf.~\citet{Tomasello09}). It's of course not the job of science to make us feel good about ourselves; but it is a pleasant side effect.
