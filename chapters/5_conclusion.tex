\chapter{Conclusion}
\label{ch:conclusion}

\section{Further research}
\label{sec:further-research}
\subsection{What is reasoning? Revisited}
\label{sec:def-scrutiny}

However, much more remains to be said about this very definition. Let us first briefly consider other definitions of reasoning from the literature, and then place some question marks around the position of Mercier \& Sperber's definition of reasoning within their argumentative theory.

The definition of reasoning given and used by Mercier \& Sperber is reminiscent of definitions of argumentation, in particular in its use of the terms 'premise' and 'conclusion'. In a \citeyear{Sperber01} paper, which can be considered to be a precursory work to \citet{MS11}, Dan Sperber uses the following definition of argumentation:
\begin{quoting}
    the defense of some conclusion by appeal to a set of premises that provide support for it
    \hfill (p.~401)
\end{quoting}
and somewhat less precisely, \citet{MS11} define arguments as
\begin{quoting}
    representations of relationships between premises and conclusions
    \hfill (p.~58)
\end{quoting}

Comparing these definitions of argumentation and of reasoning raises a question: what is reasoning, if not internalized argumentation? Or, in a similar vein, what is argumentation, if not externalized reasoning? And, if this is the case, does this not render the argumentative theory of reasoning void? For then the theory would state that the main function of internalized argumentation is argumentative.

\subsection{Motivations and dispositions of interlocutors}

Throughout their \citeyear{MS11} article, Mercier \& Sperber allude to the dispositions\todo{'Dispositions' is not meant as a technical term, and I don't think it is; is it?}
of interlocutors in argumentative settings (emphasis in quotes added):
\begin{quoting}
    This experiment illustrates the more general finding stemming from this literature that, \emph{when they are \textbf{motivated}, participants are able to use reasoning to evaluate arguments accurately}
    \hfill (p.~61)
\end{quoting}
\begin{quoting}
    Most participants are \textbf{willing} to change their mind only once they have been thoroughly convinced that their initial answer was wrong
    \hfill (p.~63)
\end{quoting}
\begin{quoting}
    this [experimental finding] should not be interpreted as revealing a lack of ability but only a lack of \textbf{motivation}. When participants \textbf{want} to prove a conclusion wrong, they will find ways to falsify it.
    \hfill (p.~65)
\end{quoting}
\begin{quoting}
    people are good at assessing arguments and are quite able to do so in an unbiased way, \textbf{provided they have no particular axe to grind}. In group reasoning experiments where participants \textbf{share an interest in discovering the right answer}, it has been shown that \emph{truth wins}
    \hfill (p.~72)
\end{quoting}
This reference to the motivations and disposition of interlocutors opens up some questions as to the specifics of the 'argumentative setting' that Mercier \& Sperber mention multiple times throughout the paper. It seems that the disposition of the interlocutors going into an argument plays an important role in Mercier \& Sperber's account of argumentation, yet they do not expand on this.
How plausible is the assumption that people engaging in argumentation have a 'common interest in the truth', as Mercier \& Sperber call it? And what happens (or what would happen) when interlocutors do \emph{not} share this interest?

\todo[inline,caption={}]{
    To do:
    \begin{itemize}
        \item I'm pretty sure this criticism is about something different than 'motivated reasoning', but check this
        \item Define what an argumentative setting is, according to Mercier \& Sperber (close-read \citet{MS11} for this)
        \item Possibly find some empirical work on arguers' dispositions
    \end{itemize}
}
