\chapter{Conclusion}
\label{ch:conclusion}

The aim of this thesis was to critically evaluate Mercier and Sperber's argumentative theory of reasoning.
Along the way, we have looked at the considerations needed for an evolutionary perspective, provided an evolutionary perspective on human communication, considered a bunch of papers by Sperber, Mercier, and colleagues, and used all of these findings (and more) to distill a number of criticisms for the theory. 
As concluded in \cref{ch:scrutiny}, none of the criticisms I have for the ATR are fatal in the sense that the theory is unsalvageable, but there is some serious work that needs to happen before the ATR can really be taken seriously.

\section{Further research}

\section{Concluding words}
I believe that viewing the particularities of the human experience through an evolutionary lens is an infinitely fascinating and rewarding endeavor. It allows us to see things in a broader context, especially since with each passing day our lived experiences move further from those of our ancestors. Reconsidering our roots in this way might be some form of escapism, but it also allows us to understand ourselves and our place in the world better.

When I first read about the argumentative theory of reasoning, Mercier and Sperber's ideas felt optimistic to me. The ATR told me that we're not bad at reasoning; the so-called flaws of human reasoning are a feature, not a bug! Now that I have analyzed the theory in more detail, I feel differently about the story they tell. Their focus on deception and vigilance has a more pessimistic flair (see also \cref{sec:honesty-dishonesty}), which is (seemingly) at odds with the general outlook on human evolution that Michael Tomasello emanates (cf.~\citet{Tomasello09}). It's of course not the job of science to make us feel good about ourselves; but it is a pleasant side effect.
