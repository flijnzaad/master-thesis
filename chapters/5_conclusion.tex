\chapter*{Conclusion}
\label{ch:conclusion}
\addcontentsline{toc}{chapter}{\nameref{ch:conclusion}}
\markboth{CONCLUSION}{}

The aim of this thesis has been to critically evaluate Mercier and Sperber's argumentative theory of reasoning, which posits that the function of reasoning is to produce arguments and evaluate those of others, which stabilizes human communication.

To support this critical evaluation, we first had a look at evolutionary theory to provide some background to the evolutionary perspective of the ATR.
We discussed biological and cultural evolution, causation in evolution, evolutionary psychology, terminological issues, and finally outlined a methodology for an evolutionary analysis of human communication.
This evolutionary analysis of human communication saw us discuss definitions of communication, consider empirical work on communication in non-human animals and in human development, and (arguably most importantly) analyze the function of communication. We reviewed the problem of the stability of communication, examined the evolution of human cooperation and how this relates to communication, and discussed lying and deception.
Next, we laid out Mercier and Sperber's views in detail. We discussed their 'evolutionary arms race' of communication and the related concept of epistemic vigilance, and saw how these components came together in the argumentative theory of reasoning.

Finally, this investigation culminated in a critical evaluation of the ATR, resulting in a couple of conclusions. Firstly, the ATR is divorced from the broader context of cooperative communication, which makes it rather weak from an evolutionary perspective.
Secondly, combining my own findings with criticisms by \citet{Michaelian13}, I conclude that particular aspects of the epistemic vigilance story are unconvincing beyond a superficial glance.
Thirdly, metatheoretically the ATR is unsatisfactory because it is oftentimes vague or underspecified, and it lacks important details.

All in all, the argumentative theory of reasoning is intuitively attractive, but rests on implicit assumptions that are unconvincing as-is. Moreover, the theory is plagued by vagueness and is underspecified in crucial aspects. I hypothesize that the ATR is not unsalvageable; however, substantial modifications would be needed before the ATR can really be taken seriously.

\section*{Further research}

The ATR's issues naturally give rise to avenues for further investigation.

For starters, it would be nice to see Mercier and Sperber (or perhaps others) resolve the vagueness in the ontological aspects of the ATR as discussed in \cref{sec:ont-atr}. This would require additional philosophical work, critically evaluating the characterization and use of the concepts of the ATR, and subsequently explicating them clearly and straightforwardly.

Another interesting line of inquiry concerns the reconciliation of the ATR within a broader context. In \cref{sec:comm-func-scrutiny}, I argued that the ATR would be stronger if it were integrated within a broader evolutionary context of human cooperation à la \citet{Tomasello09}, and conjectured that it would be possible to do so. The obvious next step is then to determine whether it is indeed the case that these two stories can be reconciled. This endeavor would be one of evolutionary anthropology mostly, with a philosophical touch. The open questions posed by the questionable assumptions discussed in \cref{sec:comm-func-scrutiny} could perhaps be answered by findings from evolutionary anthropology.
If this research would conclude that the ATR in its current version cannot be successfully integrated with a Tomasellan view on human cooperation, then it could be fruitful to determine if the ATR could be modified in a such a way that it \emph{can} be integrated. Though, it might turn out that such a modified version of the ATR barely resembles the original theory, as noted in \cref{sec:scrutiny-conclusion}.

The last avenue for further research I will mention here concerns a significant question that transpires from the discussion in \cref{sec:ont-atr}: is the ATR unfalsifiable?
To answer this question, one would carry out a detailed metatheoretical analysis into multiple aspects of the theory, such as the ontological and definitional rigor with which concepts are used, and how empirical work is imported to support the theory. As we saw in \cref{sec:empirical}, Mercier and Sperber ascribe empirical believability to the ATR by formulating predictions that they argue to be entailed by the theory. It would be good to critically evaluate the status of these hypotheses against the whole theory, to assess whether they are indeed generated by the theory. Moreover, one could survey the field for evidence that refutes these hypotheses and evaluate its merit, to determine whether it could be the case that Mercier and Sperber are cherry-picking the empirical work they reference. If they are indeed cherry-picking data, they would, ironically enough, be acting consistently with their own theory by exhibiting a confirmation bias.

\section*{Concluding remarks}

Let me end this thesis on a general note.

I have personally found viewing the particularities of the human experience through an evolutionary lens to be an infinitely fascinating and rewarding endeavor. It allows us to see things in a broader context, especially since with each passing day our lived experiences seemingly move further away from those of our ancestors. Focusing on our 'roots' in this way might even constitute some form of escapism -- but equally, it allows us to understand ourselves and our place in the modern world better.
I have especially found Michael Tomasello's work on human cooperation (\citeyear{Tomasello09}) to be incredibly illuminating to this end, and I enjoy the positive general outlook on human nature that radiates from his views. Ultimately, it is of course not the purpose of science to make us feel good about ourselves; yet, it is a pleasant side effect nonetheless.
