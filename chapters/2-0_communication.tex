\chapter{Why do we communicate?}
\label{ch:communication}

\largeTodo{In general: this chapter is very dense, you can go at a bit of a slower pace most of the time. But pace will also be better once elaborations and examples are added.}
In order to answer the question of how advanced reasoning \smallTodo{Terminology: probably drop "advanced" here. Possibly get back to this after writing Chapter 3} may have evolved to further communication, we will first need to examine communication in its own right: why do we communicate?
In order to answer this question, we will discuss each of the methodological questions raised in \cref{sec:evo-conclusion} as they pertain to communication\smallTodo{Refer to the numbers or codes of the Tinbergen questions}. But before we can take a look at the evolutionary history, the developmental origins and the functions of communication, we must first fix a definition of communication, since this determines the frame of our research question.
\largeTodo{This whole introduction could be clearer: why do we look at communication now? Can use a bit more words.}

\section{Conceptions of communication}
\label{sec:comm:definition}

\largeTodo{Missing from this section: what Mercier (\& Sperber) define as, and/or have to say about, communication}

\todo[inline]{Address lSignalling vs. communication}

There are many different ways organisms may communicate with each other, and indeed many different ways in which one may define communication. In any case, communication is an process necessarily involving a signaler (a sender) and at least one receiver (a listener).

Some authors regard communication to inherently be a tool of persuasion, which then translates to their very definition of communication: for example, on the manipulative model of communication, communication can be taken to occur "when an animal, the actor, does something which appears to be the result of selection to influence the sense organs of another animal, the reactor, so that the reactor's behavior changes to the advantage of the actor" \citep[p.~283]{DawkinsKrebs78}.\smallTodo{Rewrite this so that the quote is actually typeset as a quote, for emphasis}
\largeTodo{Say something about this quote in your own words: comment on it, this also justifies you using the quote. Point the reader to why you use it.}
One may also notice that this definition has a teleological explanation embedded in it as well (see \cref{sec:teleology}).\smallTodo{Elaborate on this: make it more explicit}
I mention this definition only for completeness' sake \smallTodo{Maybe not necessary to mention this}, because I believe this definition to be insufficiently parsimonious in its assumptions about the function of communication.
\largeTodo{Elaborate on this: what are the assumptions they make, in what sense are they strong, and why are they too strong for my liking (i.e. what's wrong with them)?}

In their discussion of communication as it relates to social cognition, \citet{Freeberg19} define communication as follows:
\begin{quoting}
    Communication involves an action or characteristic of one individual that influences the behaviour, behavioural tendency or physiology of at least one other individual in a fashion typically adaptive to both
    \hfill (p.~281)
\end{quoting}
\largeTodo{Explain this in own words; and explain the difference with the definition of \citet{DawkinsKrebs78}. The difference lies in "to the advantage of"}
This is a very broad conception of communication; on this definition, all organisms, from bacteria to fungi to plants to animals, communicate.
\example{Provide examples for each of these kingdoms here, nice illustration}

\citet{Scott-Phillips18-communication, Scott-Phillips15-primate} contrasts two different models of communication with each other: the classical \emph{code model} of communication, and the \emph{ostensive-inferential} model of communication.
In the former model, communication involves processes of coding and decoding messages. The coding, on the side of the sender, involves a mapping between the state of the world and a behavior (namely the signal they send). The decoding, on the side of the receiver, involves a mapping between two behaviors: the signal sent, and a subsequent response of the receiver. If the mappings are properly calibrated to each other, communication between sender and receiver can be said to have occurred.

\largeTodo{This paragraph could do with a lot more philosophical discussion: Quine should definitely be mentioned, even if it's just in a footnote, because it's like this milestone philosophy thing that would be a glaring omission to philosophers reading this.
Also, the usage of "meaning" and "content" here is a big philosophical no-no, either explicate what you mean by them or change your terminology. You could also just acknowledge or specify that you use the intuitive, colloquial meaning, not the technical one. Read also the Putnam paper on the ants and Winston Churchill?}
However, in order to capture human communication, the code model is too simplistic, because it fails to account for the \emph{underdeterminacy of meaning}: in merely looking at the content of the message, one cannot account for the meaning that the message conveys to the sender \citep{Scott-Phillips18-communication}. \example{Add example}
Therefore, a move away from the code model of communication towards the \emph{ostensive-inferential} model of communication would be in order. This model takes into account the intentionality inherent in human communication.

\largeTodo{This paragraph also needs a lot more work:
(1) mention Sperber \& Wilson and relevance theory (credit where credit's due!), and that their theory is neo-Gricean, because most philosophers will be familiar with Grice. (2) Add an example, this needs to be way clearer and elaborated more if this will be my definition of human communication. (3) Define ostensive behavior, also with an example. It's apparent right now that I don't fully understand this model myself. (4) Add a comment on how the underdeterminacy is captured better by the ostensive-inferential model}
In the ostensive-inferential model, one may speak of a sender's \emph{informative intention}, which is their intending for the receiver to believe something. The sender's \emph{communicative intention} is then their intending for the receiver to believe that they have an informative intention. The sender may then express or convey this communicative intention to their receiver with an \emph{ostensive} behavior. If their receiver receives their communicative intention, then ostensive-inferential communication has occurred.

Currently, there is no evidence that any species other than humans communicate ostensively \citep{Scott-Phillips18-communication}. As a result, not only may one distinguish between the code model and the ostensive-inferential model to define what communication entails, one may also conceptualize these two models as two different types of communication. The code model then captures the way that non-human animals communicate, and the ostensive-inferential model then captures the way that humans communicate with each other.

It is this ostensive-inferential model that I will consider to form the definition of \emph{human communication} throughout this thesis. When I speak of communication broadly construed, I will adhere to the definition of communication by \citet{Freeberg19}. This definition is compatible with the code model outlined by \citet{Scott-Phillips18-communication};
\largeTodo{This is a weird comment, because it implicates maybe that the ostensive-inferential model is not compatible with the Freeberg definition. Make this implicature explicit, say something about how the Freeberg definition compares to the two models we have.}
the code model, however, provides a level of detail that will not be necessary for our discussions of non-human animal communication.
\largeTodo{Explicate the relation between the two models/definitions, because now it's very unclear why and how I'm using two different definitions at the same time}
\largeTodo{Actually, this section could do with a lot more work on intentional and non-intentional communication: read up on this, and make things explicit here. Could choose to define communication and non-intentional communication, or intentional communication and communication. Need to make a choice in this, explain it, and then be clear and \emph{consistent!} about using this terminology.}

\largeTodo{Gricean objection: talking to the self is also communicating, albeit to an imaginary audience, or you yourself are the addressee. The point I make about using language without communicating is a bit slippery, philosophically controversial, and is maybe not relevant. Communicating without language \emph{is} relevant, but maybe this is not the place to point that out; it can also become implicitly or explicitly clear throughout or at the end of the chapter.}
As a last side note: while we would often equate human communication with linguistic communication, humans can easily communicate without language -- for example, using glances or gestures\footnote{One might even speculate that any human behavior can be used to communicate.}, see also \cref{sec:comm:ontogeny} -- and can use language without communicating -- for example, when one is talking to oneself. Therefore, although language will come up now and then throughout this chapter, it is not our object of focus at the present moment.

\section{Communication in non-human animals}
\label{sec:comm:phylogeny}

\largeTodo{This section deserves a mention of Chomsky's work on animal communication: something about humans having language, and NHA only having stimulus-dependent responses? Also, this section could do with more references to experimental evidence.}

Now we turn to the first methodological question\smallTodo{Are they numbered?} and we look at the communication of other animals, especially those that we are evolutionarily closely related to.
As already mentioned, one fundamental difference between the communication of non-human animals and humans is by which model their communication is best described: the code model and the ostensive-inferential model, respectively \citep{Scott-Phillips15-primate, Scott-Phillips18-communication}.
\smallTodo{The model that best describes them is a \emph{consequence} of the difference between the two: so the difference is that which makes one model work better for NHA and the other better for humans. Reformulate}

Communication is used by non-human animals for a wide range of purposes, and it can be elicited by a number of stimuli. Moreover, communicative behaviors can manifest themselves in different modalities: not only can animals communicate through vocalizations, they may also communicate through gestures or glances.
\largeTodo{Write more on gestural communication in apes (is mentioned in \citet{Tomasello08})}

One can broadly distinguish between communication in aggressive and cooperative interactions \citep{SeyfarthCheney03}. In aggressive interactions, primates may for example use communication in order to intimidate, by using it to signal their size and willingness to fight. This minimizes the chances of a physical altercation or fight actually happening, which minimizes the chance of injury for both the dominant and the subordinate animal.
In cooperative interactions on the other hand, where the interests of the signaler and the receiver overlap, communication can be used to alert others of\smallTodo{About?} predators, to coordinate foraging activities and to facilitate social interactions:
\begin{quoting}
    {[information acquired by listeners]} may include, but is not limited to, information about predators or the urgency of a predator’s approach, group movements, intergroup interactions, or the identities of individuals involved in social events
    \hfill \citep[p.~168]{SeyfarthCheney03}
\end{quoting}

In animals in general, vocalizations are most often elicited not by just one stimulus, but rather a complex combination of them. Moreover, the "history of interactions between the individuals involved" \citep[p.~151]{SeyfarthCheney03} can also play a role in eliciting vocalizations. As for the 'immediate' stimuli eliciting vocalizations, we may distinguish between sensory stimuli on the one hand and mental stimuli on the other. Sensory stimuli then refer to stimuli received through the external senses, such as visual, auditory and olfactory senses.
\example{Take a different example: this is too much anecdotal, and with domesticated animals there's issues of evolutionary tractability as well as anthropomorphization. Get a reference on this}
For example, if I stop petting my dog (sensory stimulus), she will direct her gaze at me (communicative behavior) to indicate that she would like me to continue.
Mental stimuli on the other hand can be viewed as the mental states an animal attributes to another animal.
\largeTodo{Glaring omission: the large controversy surrounding theory of mind. You can make assumptions that are controversial, so long as you acknowledge the controversy and give good reasons for making the assumption. Show that you're aware of the debates. Consider taking the consensus in the field from Kristin Andrews' textbook "Animal minds"}
For example, \example{Add example from the literature}.
This type of stimulus elicits the majority of vocalizations in human conversation, but there is no evidence that the attribution of mental states to others causes vocalizations in other animals, except for possibly chimpanzees \citep{SeyfarthCheney03}.

\todo[inline]{This should be woven in better}
\citet{Tomasello09} notes that in the case of pointing gestures, humans use these mainly to be informative (i.e.\@ cooperative), whereas primates use this gesture mainly or perhaps even exclusively for imperative motives. Experimentally, Tomasello and colleagues found that primates only used a pointing gesture when they benefited from this act of communication, while 25-month-old infants pointed regardless of whether they themself benefited from this action \citep{Bullinger11}. As \citet{Tomasello09} concludes, "[the infants] could not help but be informative" (p.~17). Speaking of which, let us now consider communication in human infants.

\todo[inline]{This is a rough part}
As we will later discuss deception in communication a great deal, let us briefly touch on deception in animals as well. As noted by \citet{Dor17} in his discussion of deception, primates possess the ability to deceive others.\example{Example or elaborate}
However, compared to humans, their ability to deceive is rather primitive. Dor argues that this is due to the fact that language uniquely enables humans to communicate with each others' imaginations. This opens up countless avenues for deception by the fabrication of stories. For primates on the other hand, without language, it is difficult to fabricate stories to deceive others. They can hide information: for example, they might suppress the food call they would usually expel upon finding food in order to keep this food for themself. However, their ability to fabricate information is very limited.

\section{Communication in children's development}
\label{sec:comm:ontogeny}

\largeTodo{This section tries to fit a lot of information in little space: it is too dense. Also, the structure is weird and unclear. It could do with a whole overhaul.}

Now that we have seen how communication works in non-human animals, let us turn to how children start communicating throughout their development.
Around their first birthdays, children start communicating ostensively by pointing \citep{Tomasello08}.
\largeTodo{Check the exact timing of this: Tomasello (1999) talks about the "nine-month revolution", and the difference between 9 and 12 months is very big in human children. Possibly directly cite experimental sources.}
Although at first glance pointing may seem like a simple behavior, it may be used in a number of communicative contexts to convey a fairly wide range of messages and intentions.
For example, infants may point at a cup to indicate that they want to drink from it (i.e., pointing to request), but they may also point to a hidden object that their parent is searching for (i.e. pointing to inform).
\largeTodo{Why do we focus on pointing? Elaborate more on why pointing constitutes communication, and other earlier stuff doesn’t. For example, raising arms to parent: a behavior that we inherited from our ape ancestors (instinct to climb on parent). Is that communication? What distinguishes these earlier from the later interactions? And consider for example the communicative function of crying, is this mentioned by Tomasello or by other authors? How is that different from vocalizations in animals? Simple stimulus-response? This basically all comes down/back to the definition of communication I adhere to, and intentionality in it.}

\largeTodo{This paragraph is way too dense: she doesn't understand this upon reading. It's a matter of unpacking the notions.}
On the classic account, pointing can serve either of two communicative motives: an imperative motive, in which the pointer requests things from someone, and a declarative motive, in which the pointer shares their experiences and emotions with someone.
This account can be extended upon by distinguishing between declaratives as expressives (sharing attitudes and emotions) and declaratives as informatives (providing information), and by furthermore conceiving of imperatives as a continuum, with the underlying motive ranging on a scale from individualistic -- e.g. forcing someone to do something -- to cooperative, e.g. indirectly making a request to someone by informing them of some desire \citep{Tomasello08}.

The fact that pointing is a fairly complex communicative act is underscored by the fact that non-human animals are not able to understand pointing in the same way humans are. \smallTodo{Add reference for this}
The hypothesis is that in order to communicate intentionally,
\largeTodo{On the current definition, isn't all communication intentional? Should really settle on this, and then change the wording here as applicable. Read up also on Putnam's Brain in a Vat article, this might be illuminating philosophically. Grice also has some readings on natural and non-natural meaning, distinguishing between different types of communication. Something about intentionality? Make the distinction between intentional and non-intentional stuff very explicit, because it's one of the most important things in this chapter.}
like children begin doing around their first birthday, first the skills and motivations for \emph{shared intentionality} need to be present in the infant; that without skills of shared intentionality, infants could only communicate intentionally, but not cooperatively.
\largeTodo{Elaborate more on intentional vs. cooperative communication: does Tomasello (1999) have something on this? Something to do with joint attention.}
Shared intentionality is the "ability to participate with others in interactions involving joint goals, intentions, and attention" \citep[p.~139]{Tomasello08}. Communicative pointing behaviors in infants emerge around the same time as skills and motivations of shared intentionality do, which according to Tomasello confirms this hypothesis of dependency between them.
\largeTodo{Disentangle communication and shared intentionality, because up until this point, they seem to be the same thing. Go a bit slower in this section.}
\largeTodo{Reconsider the usage of the terminology "skills and motivation", and make explicit exactly what you mean by them. Would abilities be a better word? Why does \citet{Tomasello08} use skills? Why are motivations included? What are motivations?}

Tomasello further investigates what he calls \emph{pantomiming} or \emph{iconic gestures}, which are symbolic or representational gestures. \example{Give example}
He presents empirical evidence\largeTodo{Discuss this empirical evidence} that these kinds of gestures rely heavily on convention for their meaning, and that the acquisition and usage of these conventions bears a strong resemblance to the acquisition and usage of language.

\largeTodo{This trajectory should also be more clear throughout the section}
In short, infants first acquire the skills and motivations needed for shared intentionality; then they acquire the skills and motivations for communicative pointing; and then they acquire the ability to use iconic gestures and language around the same time.

Let us, like in \cref{sec:comm:phylogeny}, touch briefly upon deception in ontogeny. Children acquire the skills for lying around the age of four \citep{Lee13}.\todo{Build upon this with stuff from \citet{Meibauer18}}

\section{What is the function of communication?}
\label{sec:comm:function}

Finally and arguably most importantly for our endeavor, let us have a look at the function of communication.

Essentially, communication facilitates interaction between individuals.
This interaction may be either cooperative or competitive in nature, as we have seen in \cref{sec:comm:phylogeny} when discussing \poscite{SeyfarthCheney03} review of animal communication.
Whether the communicative event is cooperative or competitive in nature depends on the interests of the interlocutors. If the interests of interlocutors overlap or align, their communication can be considered to be cooperative; if their interests do not overlap, or even oppose each other, their communication can be considered to be competitive.
For example, if two individuals engage in collaborative hunting of a large prey animal, their interests (catching the prey together and sharing it) align and they will thus use communication for cooperative purposes -- i.e., to coordinate their hunting activity. On the other hand, if two individuals compete for a smaller prey animal, their interests (catching the prey by themselves and keeping it for themselves) oppose each other, and their communication would thus be competitive. They might for example intimidate each other verbally, which may be evolutionarily more advantageous than physical intimidation (i.e. fight) because of a reduced risk of injury.

As argued by \citet{Tomasello08, Tomasello09} and echoed by \citet{Dor17}, the cooperative setting constitutes the 'birthplace' of the unique features of human communication; the competitive use of human-style communication must have emerged later. As \citet{Tomasello08} writes:
\begin{quoting}
    The use of skills of cooperative communication outside of collaborative activities (e.g., for lying), came only later.
    \hfill (p.~325)
\end{quoting}
Especially the emergence of language could only have occurred in cooperative settings, \citet{Tomasello08} and \citet{Dor17} argue. I return to this line of thought in a bit.

Let us now consider pragmatic communication; specifically, how the cooperative function of communication relates to Grice's cooperative principle:
\begin{quoting}
    Make your conversational contribution such as is required, at the stage at which it occurs, by the accepted purpose or direction of the talk exchange in which you are engaged.
    \hfill \citep[p.~45]{Grice75}
\end{quoting}
Earlier along the evolutionary timeline, an interlocutor's Gricean cooperativeness may very well have coincided with her cooperative intention or disposition. This is of course especially the case with animals communicating (see \cref{sec:comm:phylogeny}).
However, it is apparent that these two 'dimensions' of cooperativeness need not always coincide.
Daniel \poscite{Dor17} discussion of lying and the evolution of language relates to this distinction between what we might refer to as Gricean cooperativeness and cooperative intent. Let us briefly consider some of the points he makes in this paper before we return to the distinction between Gricean cooperativeness and cooperative intent.

\citet{Dor17} notes that the distinction between honesty and deception might be interpreted in two ways. On the one hand, one can consider the honesty of a signal to be its truthfulness: an honest signal is a true signal, and a deceitful signal is a false signal. On the other hand, one may consider the honesty of a signal to refer to not its truthfulness, but rather the benefits and costs the sender and receiver incur as a result of the communicated signal. In the case of animals communicating, these two conceptions of honesty might very well coincide -- i.e., truthful signals benefit receivers, and false signals harm receivers. However, it should be apparent that they do not always coincide in the human case: truthful signals may hurt or cause harm to receivers, and false signals may benefit the receiver. \todo{Add example}

Dor then outlines four possible communicative options one might choose: "co-operative honesty, harmful honesty, co-operative lying and harmful lying" \citep[p.~45]{Dor17}.
He argues that while anti-social, exploitative lies -- lies with the intention to profit at the expense of the receiver, i.e.\@ harmful lies -- constitute the most intuitive, salient conception of a lie, they are not at all the most prevalent kind of lie. \citet{Meibauer18} has useful additions to this point: he notes that prosocial lying is connected to politeness. The notion of politeness, in turn, can be connected with the notion of benefits and costs:
\begin{quoting}
    In antisocial (mendacious) lying, only the speaker profits from lying. In prosocial lying, lying is either altruistic (only the hearer profits from the speaker's lie) or polite ("Pareto-white," as Erat \& Gneezy 2012 call it) in the sense that both speaker and hearer profit from the lie.
    \hfill \citep[p.~371]{Meibauer18}
\end{quoting}

Going back to Grice's cooperative principle, note that Dor in his analysis only considers Grice's maxim of Quality ('Try to make your contribution one that is true'). If we extend his distinction to the whole cooperative principle -- in other words, if we also incorporate the maxims of Quantity, Relation and Manner -- we get the two 'dimensions' of cooperativeness we considered earlier: Gricean cooperativeness, and cooperative intent.
These two dimensions combine to give us four communicative options one might choose: Gricean cooperative cooperation, Gricean noncooperative cooperation, Gricean cooperative noncooperation, and Gricean noncooperative noncooperation.
For example, lying constitues Gricean noncooperation, because it violates the maxim of Quality. As noted before, lying may be used with competitive intent (i.e. antisocial lying), or cooperative intent (i.e. prosocial lying, such as lying for politeness or white lies).

\subsubsection{A small note on language}

Let me briefly address the elephant in the room: when considering human communication, human language and its evolution cannot remain unmentioned. In this thesis, I only consider human communication in general, because the emergence and evolution of symbolic communication in the form of language is an entire field of research of its own.
I will finish by noting one important thing about the evolution of language as it relates to cooperation and trust. It has been argued that only in cooperative settings could our complex language have emerged at all. This is because for such complexity to arise, more frequent and prolonged interactions are necessary \citep{Benitez21}. As \citet{Dor17} writes,
\begin{quoting}
    The collective effort of the invention and stabilization of the new technology [namely, language] must have been based on high levels of reliability and trust between the inventors: otherwise, indeed, they would not have been able to get the system going.
    \hfill (p.~50)
\end{quoting}
We return to a discussion of reliability, trust and the idea of 'getting the system going' in \cref{sec:S-P08}. For more discussion on the evolution of language, see for example \citet{Tomasello08} and \citet{Dor17}.

To fully appreciate the cooperative function of communication, let us now consider what makes cooperation itself evolutionarily beneficial. Moreover, in order to complete the causal chain, we will have a look at how cooperation could have evolved and the role that communication plays in it. We will do so by drawing extensively from Michael Tomasello's comprehensive \citeyear{Tomasello09} book \emph{Why We Cooperate}.

\subsection{Human cooperation and its evolution}
\label{sec:comm:cooperation}

Let me start off with a brief terminological aside: although colloquially the terms 'cooperation' and 'collaboration' are more or less synonymous, Tomasello does not use them interchangeably. He defines collaboration as working together for mutual benefit (p.~xvii). Implicitly, he takes cooperation to be an overarching term which also encompasses for example altruism, in which one individual sacrifices something to help another individual. For the remainder of this thesis, I will adhere to his terminological conventions.

Tomasello argues that somewhere along the evolutionary timeline, humans must have been "put under some kind of selective pressure to collaborate in their gathering of food--they become obligate collaborators--in a way that their closest primate relatives were not" \citep[p.~75]{Tomasello09}\footnote{Notably, what exactly this selective pressure is, is a missing link in his otherwise very convincing story.}.
He elaborates by noting that in general, evolution may select for sociality in animals because living together in a social group protects the group's members against predation: it is easier to defend oneself in the context of a group. The group however also brings disadvantages with it when it comes to foraging for food, since the members of the group are competitors in the acquisition of food. This is especially the case when the source of food is 'clumped', such as in a prey animal, rather than dispersed, such as in a plain of grass. The clumped source of food raises the issue of how to share the food amongst the members of the social group.
Tomasello enumerates a number of different hypotheses to explain how humans could have broken out of what he calls "the great-ape pattern of strong competition for food, low tolerance for food sharing, and no food offering at all" \citep[p.~83]{Tomasello09}; in other words, how humans could have evolved to be more tolerant and trusting, and less competitive about food.
Firstly, as due to a certain selective pressure it became necessary for humans to forage collaboratively, it could have been evolutionarily advantageous to be more tolerant and less competitive.
Secondly, Tomasello notes it could be the case that humans went through a process of self-domestication, which eliminated aggressive, predatory or greedy individuals from the group; see \citet{Benitez21} for more on this.
Thirdly, the evolution of tolerance and trust could be related to what is called \emph{cooperative breeding}, also known as \emph{alloparenting}. In cooperative breeding, the responsibility of child-rearing falls on more individuals than just the mother of the child; these individuals help by providing food for the child and engaging in other acts of childcare. This cooperative breeding may have selected for pro-social skills and motivations; see \citet{Hrdy09} for an elaboration of this argument.

Tolerance and trust then constitute a foundation upon which coordination and communication can be 'built', so to speak: they provide an environment in which more elaborate collaboration can evolve. In Tomasello's words,
\begin{quoting}
    there had to be some initial emergence of tolerance and trust (\ldots) to put a population of our ancestors in a position where selection for sophisticated collaborative skills was viable
    \hfill (p.~77)
\end{quoting}

In order to then arrive at the full picture of human cooperative activity, the final step to consider is that of social norms and institutions. As before, there is a missing link in this story, in this case it concerns how mutual expectations between individuals arise and eventually become norms. (Tomasello describes it as "one of the most fundamental questions in all of the social sciences" (p. 89).)
Norms may be defined as "socially agreed-upon and mutually known expectations bearing social force, monitored and enforced by third parties" \citep[p.~87]{Tomasello09}. Norms receive their force not only from the threat of punishment by others if the norm is violated, but also from a kind of social rationality within the collaborative activity. Individuals recognize their dependence on each other for reaching their joint goal. Just as it would be individually irrational to act in a way that thwarts your own goal, it would be socially irrational to act in a way that thwarts your joint goal.

Let us now briefly summarize the evolutionary timeline of human cooperation according to Michael Tomasello.
At some point, for reasons as of yet unknown to us, foraging for food collaboratively rather than individualistically became beneficial -- perhaps even necessary -- for humans.
During this evolutionary process, some degree of tolerance and trust must have emerged between those collaborating individuals.
In the process of adapting to this collaborative foraging, humans evolved certain skills and motivations specifically for cooperation -- for example, abilities for establishing joint goals as well as a role division for the joint activity.
This kind of collaborative activity then constituted the breeding ground for human cooperative communication.
These joint goals and role divisions later evolved into the superindividual norms, rights and responsibilities that we see within our social institutions today.

As a brief aside: it has been argued that communication is not necessary nor sufficient for the coordination of activities. \citet{Goldstone24} propose a framework of five features characterizing the specialization of roles in group activities; communication is only one of these five features. This is corroborated by experiments they review in which people "spontaneously differentiate themselves into stable roles" (p.~264) in group activity without communicating with each other.
However, the authors note that communication does play a very central role in coordinating group activities, stating that "direct communication of plans is often the single most potent tool of collective coordination" \citep[p.~276]{Goldstone24}.
See also \citet{Vorobeychik17} for a discussion of communication and coordination.

Now, armed with the ins and outs of human cooperation, and an inkling of how communication relates to the story, we turn our attention to a crucial aspect of understanding human communication: how it can have persisted despite evolutionary pressures treathening its stability. To this end, let us consider at length a paper by \citet{Scott-Phillips08}, who convincingly brings findings from animal signaling research into the realm of human communication. This will provide a good background for discussing two precursory papers to the argumentative theory of reasoning, by Sperber (and others) in \cref{sec:Sperber01,sec:Sperber10}.

\subsection{The stability of communication}
\label{sec:S-P08}

If communication between individuals of a species persists throughout evolution, we may speak of it as stable. The stability of communication is considered by some as the 'defining problem' of animal signaling research \citep{Scott-Phillips08}. It is not a trivial problem by any means: the stability of a communication system is threatened by evolutionary pressures on the communicator to 'defect', as it were. As \citet{Scott-Phillips08} describes it,
\begin{quoting}
    If one can gain through the use of an unreliable signal then we should expect natural selection to favour such behaviour. Consequently, signals will cease to be of value, since receivers have no guarantee of their reliability. This will, in turn, produce listeners who do not attend to signals, and the system will thus collapse in an evolutionary retelling of Aesop’s fable of the boy who cried wolf.
    \hfill (p.~275)
\end{quoting}
In the context of human communication: if it can be advantageous for me to lie, deceive or mislead someone, then it would evolutionarily make sense for me to do so; yet then it would make evolutionary sense for you to stop listening to me, and as a consequence our system of communication would collapse.

There have been a number of attempts at explaining the reliability of animal communication in general. One such attempt is the \emph{handicap principle} \citep{Zahavi75, Zahavi99}, which might be best understood through the paradigmatic example of the peacock's tail.
This tail is like a handicap for the peacock: not only does it take a lot of resources to grow the tail and carry it around, it also leaves the bird more vulnerable to predation because it is less agile with a large unwieldy tail. At the same time, a large tail signals to peahens that the peacock is fit enough to be able to incur these costs, and thus has a sexual advantage.
The handicap principle then describes this process of communication, by which the signaler incurs costs (i.e., a handicap) for signaling, which thus guarantees the reliability of the signal.

However useful in explaining some cases of the reliability of animal communication, the handicap principle is not able to explain all of those cases: often, it is not the case that reliable signals are costly to produce \citep{Scott-Phillips08} \todo{Add example}. Especially in the case of human communication, the handicap principle cannot account for its reliability, since it is in general not costly to produce utterances \citep{Scott-Phillips08}.
Thus, it remains to be shown how communication can be stable if signals are cost-free.

On the handicap principle, reliable signals are costly to produce, thus ensuring their reliability. An alternative explanation of the reliability of animal communication is the principle of \emph{deterrence}, whereby \emph{un}reliable signals are costly to produce, and consequently signalers are deterred from producing unreliable signals.
There are a number of ways in which producing unreliable signals may be costly to the signaler. Firstly, this is the case in a coordination game, where the signaler and receiver share some common interest with regard to the outcome of the interaction.\todo{Add 1 sentence to elaborate}
Secondly, if two individuals have repeated interactions, it may also be costly in the long term to produce unreliable signals, because it may hinder cooperation in the future.
Thirdly, producing unreliable signals may be costly to the signaler if false signals are punished by the receiver.

The 'logic of deterrents' applied to the case of human communication poses the following demands in order for the story about stability to work:
\begin{quoting}
    Sufficient conditions for cost-free signalling in which reliability is ensured through deterrents are that signals be verified with relative ease (if they are not verifiable then individuals will not know who is and who is not worthy of future attention) and that costs be incurred when unreliable signalling is revealed.
    \hfill \citep[p.~279]{Scott-Phillips08}
\end{quoting}
In other words, if unreliable signals are recognized as unreliable relatively easily, and unreliable signalers incur costs for their unreliability, the reliability of communication is secured through deterrents.

Scott-Phillips goes on to state that these sufficient conditions are met in the case of human communication, since people may refrain from interacting with unreliable individuals in the future, which can be very costly for a social species such as humans.
Notably however, he does not explicate how the first sufficient condition is met in the case of human communication; we will return to this in \cref{sec:EV-scrutiny}.

Now, before we consider Sperber's precursory concepts to the argumentative theory of reasoning, it will be good to have a closer look at deception. The 'stability of communication' problem hinges on the assumption that it is advantageous to deceive others. This is intuitively plausible; however, it deserves some extra attention, as this is such a fundamental assumption to the ATR.

\subsection{Deception and lying}
\label{sec:deception}

Let us start off this section by clarifying and examining some of the terminology surrounding honesty and dishonesty.

Briefly returning to \citet{Scott-Phillips08}, he uses the term 'reliable' when explicating his story about the stability of communication, rather than 'honest'. As he explains in the paper's introduction, this is a principled choice, to do with a difference between humans and non-human animals. He argues that one may want to steer clear from anthropomorphically ascribing intentions to animals, and meanings to their behavior. He maintains that thus 'reliability' would be a more neutral term than 'honesty', because it refrains from ascribing intentions and meanings to individuals.
However, I find that talking about 'reliable communicators' blurs an important distinction between honest, benevolent communicators and competent communicators. We return to this distinction in \cref{sec:Sperber10}.

Deception, lying and persuasion may be defined in a number of different ways. Let us now briefly look to the literature to obtain a working definition of these concepts.

First off, deception may be defined as "deliberately leading someone into a false belief" \citep[p.~358]{Meibauer18}.

Lying, on the other hand, might not be as easily defined: Jennifer Mather Saul even dedicates the whole first chapter of her \citeyear{Saul12} book to obtaining a viable definition that includes the relevant examples and excludes the irrelevant ones. An intricate discussion of her definition is out of scope for this thesis; let us for now consider a definition of lying due to \citet{Williams02} that \citet{Meibauer18} calls 'standard':
\begin{quoting}
    an assertion, the content of which the speaker believes to be false, which is made with the intention to deceive the hearer with respect to that content
    \hfill \citep[p.~96]{Williams02}
\end{quoting}
Meibauer notes -- and this also transpires from \poscite{Saul12} discussion of the definition of lying -- that each of the components of this definition can be, and has been, challenged.

Let us now discuss some of the points that Daniel \citet{Dor17} makes about lying and the stability of communication.

Especially important are the notes he makes regarding what he terms the "paradox of honest signaling" \citep[p.~46]{Dor17} -- i.e., the theoretical issues plaguing the stability of communication that we discussed in \cref{sec:S-P08}. 
Dor notes that this foretold collapse of communication due to unreliability of the speaker, does not hinge on whether or not the speaker is truthful, but whether her \emph{intention} is benevolent.
In other words, this story appeals not to receivers evaluating the truthfulness of incoming information, but rather the receivers evaluating the \emph{intention} of the sender. Listeners care whether speakers intend to be harmful, not whether or not they are truthful, Dor argues.
Moreover, he notes that the paradox of honest signaling mostly appeals to situations in which interests between interlocutors conflict; however, these situations might not be the most pertinent or prevalent kind of communicative situation.
According to Dor, at the point in evolutionary time when language emerged, humans were already crucially dependent on cooperation and coordinated action, and thus their interests overlapped more often than not.

Briefly returning to the utility of communication, another avenue Dor explores is how communication is used. As we will see in \cref{sec:Sperber01} and in \cref{sec:Sperber10}, Sperber and his colleagues focus a lot on the transmission of information between individuals, in the form of testimony and argumentation. However, Dor argues that within the paradox of honest signaling, this transmission of information is not the only relevant use of communication. Communication is also used for cooperation, and in that situation lying is not really an issue; Dor writes
\begin{quoting}
    Language is extremely useful in the coordination of collective work, collective defense and so on, where it is used not just for the exchange of information but also for collective planning, division of labor, ordering and requesting, where lying as such does not seem to play a major role.
    \hfill \citep[p.~51]{Dor17}
\end{quoting}
Convincingly, Dor goes on to argue that due to this dual role of communication (transmitting information on the one hand, and facilitating cooperation on the other), the stability of communication is not threatened by lying. He writes:
\begin{quoting}
    Even in the very unlikely doomsday scenario, then, where all the members of a community lie to each other in their factual statements, and eventually refrain from sharing information with each other, there is no reason to assume that they would stop using language for all these other purposes, especially where their survival, whether they like it or not, depends on collective action.
    \hfill \citep[p.~52]{Dor17}
\end{quoting}

Lastly, we will have a brief look at persuasion, since it naturally plays a considerable role in Mercier \& Sperber's argumentative theory of reasoning.
\citet{Brinol09} broadly define persuasion as "any procedure with the potential to change someone's mind" (p.~50),
whether that be changing someone's emotional state, beliefs, behaviors or attitudes.
They describe persuasion as "the most frequent and ultimately efficient approach to social influence" (pp.~49--50). Put crudely, persuasion is a tool for getting what you want, and it serves this end better than the alternatives of using force, threats or violence.\footnote{In this, one may see a parallel with \poscite{SeyfarthCheney03} conception of aggressive communication as a low-risk alternative to fighting.}
From this observation, the conclusion emerges that persuading someone is beneficial to an individual exactly to the extent that the corresponding gain in social influence is beneficial to the individual.
For limitations of time and space we will not go into the benefits of gaining social influence.

\section{Precursory concepts to the ATR}
\label{sec:comm-ATR}

Now that we have discussed the utility of communication, and in doing so also considered the utility of cooperation and deception, and the stability of communication, we should have our first look at some basic foundations for Mercier \& Sperber's argumentative theory of reasoning. Here, we will consider \citet{Sperber01} and \citet{Sperber10}, since they mostly concern communication; in \cref{ch:reasoning}, we will consider \citet{MS09} and \citet{MS11}, since they mostly concern reasoning.

\subsection{Sperber on the evolution of testimony and argumentation}
\label{sec:Sperber01}

In a \citeyear{Sperber01} paper, Dan Sperber analyzes testimony and argumentation from an evolutionary perspective. In doing so, he provides important groundwork for his later work with Mercier (and others) on the relation between reasoning, argumentation and the stability of communication.

Testimony and argumentation are two concepts central to human communication. Sperber borrows his definitions for these concepts from epistemologist Alvin Goldman, who defines testimony as "the transmission of observed (or allegedly observed) information from one person to others" \citep[p.~401]{Sperber01} and argumentation as "the defense of some conclusion by appeal to a set of premises that provide support for it" (ibid.).
Sperber puts these two concepts in an evolutionary perspective, and discusses in particular how they have figured in stabilizing communication over the course of evolutionary history.

A tempting way to look at communication is as a kind of 'cognition by proxy': through communication, one organism may access information another organism has obtained from its own perception or inference.
For instance, if you tell me that there is milk in the fridge, I can through this act of communication benefit from the information derived from your perception of the milk carton in the fridge.
\example{Replace this example by an animal communication example}
However, Sperber argues that, at least in the case of human communication, testimony does not amount to cognition by proxy. This is because testimony has different effects than direct perception does. Going back to our example, upon receiving your testimony stating that the milk is in the fridge, I am in a different cognitive state than if I would have perceived the milk carton there myself. Moreover, in human communication, Sperber argues that interpretation and acceptance of utterances are two separate processes: recognizing what a speaker meant by their utterance is not the same as accepting it as true.\footnote{This may very well be the case philosophically or epistemologically speaking, but psychologically speaking, they may be more intertwined than Sperber implies. In a later paper, he elaborates more on his stance, making even stronger claims about how comprehension always precedes the acceptance (or rejection) of a claim \citep[\S 3]{Sperber10}. Although this is intuitively plausible, \citet{Lewandowsky12} points out that empirical evidence suggests that for someone to comprehend an utterance, they must (at least temporarily) accept it.}

The classical account of animal communication by \citet{DawkinsKrebs78} focuses only on the side of the communicator in the story, maintaining that the function of communication is to manipulate others. Sperber rejects this classical approach, arguing that the interests of the sender cannot be the only driving force in the evolution of communication.
He outlines a similar line of argumentation as we have seen in \cref{sec:S-P08},
arguing that for communication to have stabilized and continued to be stable between senders and receivers, both parties must have benefited from the action. In other -- game-theoretic -- terms, communication must (at least in the long run) be a positive-sum game, where both senders and receivers gain from the interaction.

In the case of receiving testimony from others, the receiver gains from testimony "only to the extent that it is a source of genuine (\ldots) information" (p.~404).
On the side of the production of testimony, the sender stands to gain from this testimony because
\begin{quoting}
    it allows them to have desirable effects on the receivers' attitudes and behavior. By communicating, one can cause others to do what one wants them to do and to take specific attitudes to people, objects, and so on
    \hfill (p.~404)
\end{quoting}
He later elaborates on this by saying that getting others to accept your communicated message is not intrinsically beneficial. Rather, it is \emph{indirectly} beneficial, through bringing about these 'desirable effects' in others, as a way of 'cognitive manipulation'.
Sperber notes that it is exactly this self-interest of the sender that renders this 'cognition by proxy' view as inapplicable to human communication.
Moreover, he concludes from these observations of his that
\begin{quoting}
    the function of communication presents itself differently for communicator and audience
    \hfill (p.~411)
\end{quoting}

Sperber goes on to cast his observations in game-theoretic terms by sketching out a payoff matrix for a one-off communicative event. In it, he considers that senders may be truthful or untruthful, and receivers may be trusting or distrusting. According to Sperber, the sender's gain amounts to whether they have the 'desired' effect on the receiver; therefore, the sender gains from the interaction if the receiver is trusting (since this means the sender's message is accepted), and loses from the interaction if the receiver is distrusting. The payoff of this event for the sender is thus independent of the truthfulness of the sender. On the side of the receiver, their payoff \emph{is} dependent on the truthfulness of the sender: the receiver gains if they accept a truthful message, loses if they accept an untruthful message, and incurs no gain nor loss if they are distrusting and thus don't accept a message (truthful or not).

Sperber notes that the optimal strategy for such a game varies with the circumstances for both players: it is not always beneficial to be truthful, nor always untruthful; nor is it beneficial to be always trusting, nor always distrusting. In other words, there is no one stable solution to this game.
This is especially the case once we move away from this simple one-off communicative event to an iterated game of communication, where not only short-term payoffs but also long-term payoffs determine the optimal strategy.
Therefore, it is in the receiver's interest to calibrate their trust towards senders as accurately as possible; in fact, Sperber argues, this trust calibration is necessary for explaining the stability of communication.

Unlike non-human animals, humans have another way to communicate facts, other than testimony, namely \emph{argumentation}. Senders may provide receivers with reasons to accept their testimony, which the receiver may evaluate and accept or reject, independent of their trust in the sender.
Sperber sketches out the steps in what he calls the 'evaluation-persuasion arms race', i.e.\@ the chain of evolutionary adaptations that has resulted in our mechanisms for argument production and evaluation.
He argues that the first step in this 'arms race' was for the receiver to develop \emph{coherence checking}. Coherence checking involves attending to both the internal coherence of the communicated message, and the external coherence with what the receiver already believes. Coherence checking, Sperber argues, is a useful defense against the risks of deception by the sender, because lies and other false claims are often externally or internally incoherent.
The second step in the arms race was then for the sender to anticipate this coherence-checking by overtly displaying the coherence of their message to their receiver, which requires argumentative form; thus, testimony becomes argument.
The next steps were on the side of the receiver to develop skills for examining these displays of coherence (i.e., arguments), and on the side of the sender to 'improve their argumentative skills'.
\todo{Add forward reference to the relevant Chapter 3 section}
As Mercier and Sperber nicely describe it a decade later,
\begin{quoting}
    receivers' coherence checking creates selective pressure for communicators' coherence displays in the form of arguments, which in turn creates selective pressure for adequate evaluation of arguments on the part of receivers
    \hfill \citep[p.~96]{MS11}
\end{quoting}

\subsection{Sperber and colleagues on epistemic vigilance}
\label{sec:Sperber10}

% my prologue
Further building upon his \citeyear{Sperber01} views, Sperber and his colleagues (among whom, notably, Hugo Mercier) introduced the concept of \emph{epistemic vigilance} in a seminal \citeyear{Sperber10} paper. This concept constitutes a cornerstone of Sperber \& Mercier's later argumentative theory of reasoning. Therefore, a comprehensive discussion is in order here -- not in the least because this concept will be one of my targets of scrutiny in \cref{ch:scrutiny}.

% introduction
In their \citeyear{Sperber10} paper, Sperber and colleagues start by emphasizing that humans are dependent on communication, and they argue that this dependence leaves humans vulnerable to being deceived by others.
They state that misinformation or deception may "reduce, cancel, or even reverse" the gains that communication can bring to the addressee (p.~360).
Consequently, the information that an addressee receives from a communicator is only advantageous to her to the extent that the information is genuine.
Sperber and colleagues thus conclude that for this purpose, humans have evolved a suite of cognitive mechanisms for \emph{epistemic vigilance}.
Moreover, this suite of mechanisms must have evolved alongside, and is used in tandem with, abilities for ostensive-inferential communication\footnote{Slightly confusingly, Sperber et al. call the ostensive-inferential communication that we saw in \cref{sec:comm:definition} 'overt intentional communication' in this paper. However, they refer to Sperber \& Wilson's \emph{Relevance Theory}, which calls this ostensive-inferential communication. So ultimately, they are talking about the same thing.}, because they work in tandem to facilitate trust calibration on the side of the receiver.

% epistemic trust and vigilance
In order to illustrate and somewhat demarcate the concept of epistemic vigilance, Sperber and colleagues discuss some work in philosophy and psychology related to trust and vigilance. Specifically, they consider different views on the question of whether humans are 'per default' trusting or vigilant.
Of this discussion, two points stand out to me as noteworthy, especially as they pertain to the discussion to come in \cref{sec:EV-scrutiny}. The first of them concerns a characterization of vigilance that nicely captures its spirit:
\begin{quoting}
    Vigilance (unlike distrust) is not the opposite of trust; it is the opposite
of blind trust
    \hfill \citep[p.~363]{Sperber10}
\end{quoting}
The second concerns a strong claim, which we will critically assess in \cref{sec:EV-scrutiny}:
\begin{quoting}
    in communication, it is not that we can generally be trustful and therefore need to be vigilant only in rare and special circumstances. We could not be mutually trustful unless we were mutually vigilant.
    \hfill \citep[p.~364]{Sperber10}
\end{quoting}

% comprehension and acceptance
Next, the authors move on to discussing comprehension and acceptance of utterances in communication, and how these relate to epistemic vigilance and trust.
They argue that a communicative act does not only trigger comprehension in the addressee, but it also triggers epistemic vigilance alongside it. If epistemic vigilance then "does not come up with reasons to doubt" (p.~369), this comprehension leads to acceptance.
They go on to argue that comprehension of an utterance is not "guided by a presumption of truth", as other theorists state, but rather by an "expectation of relevance" (p.~367); see \citet{SperberWilson86}. This expectation of relevance requires a 'stance of trust' of the addressee regarding the speaker (this relates to the Gricean cooperativeness we discussed in \cref{sec:comm:function}).
This stance of trust of the addressee is "tentative and labile" (p.~368), and epistemic vigilance is (as mentioned) active alongside this stance of trust.
\todo{Tie this to the discussion of the ostensive-inferential model in §1}

% Sperber and colleagues maintain that vigilance is not a nicety, something that is only invoked sometimes; they maintain that vigilance is the default disposition of interlocutors in communicative settings.

To further explicate epistemic vigilance as a concept, Sperber and colleagues outline a distinction between vigilance towards the \emph{source} of a message (the 'who'), and vigilance towards the \emph{content} of the message (the 'what').
% vigilance towards the source
As for vigilance towards the source, they note that the reliability of a source depends on two factors: a reliable source must be competent, and a reliable source must be benevolent.
Moreover (and importantly), a receiver's vigilance towards the sender as a source of information -- in other words, the sender's perceived trustworthiness -- is dependent on the context: it varies per topic and per situation.
Because of this, it is important for a receiver to accurately calibrate her trust in the sender depending on the context.
They go on to discuss empirical evidence that corroborates that trust, and calibrating trust to the situation, is indeed important to us.
Moreover, on the other side of the coin, they note that deceiving people can be quite beneficial: experiments from deception detection research show that people are not good at detecting lies based on non-verbal behavioral cues.
They end this particular discussion by noting that more empirical research is needed about how people calibrate their trust in everyday communication, outlining some desiderata for this research.

% the development of epistemic vigilance and mindreading: SKIPPED THIS

% vigilance towards the content
Moving on now to vigilance towards the \emph{content} of a message, Sperber and colleagues restate that comprehension and epistemic vigilance are two processes that are intertwined to some extent. Specifically, they note that one mechanism of comprehension, namely the search for relevance, provides a basis for an "imperfect but cost-effective epistemic assessment" (p.~374).
They discuss belief revision and the role that coherence checking plays in it. We already saw \poscite{Sperber01} discussion of coherence checking; Sperber and colleagues now describe coherence checking a mechanism for epistemic vigilance. They note that coherence checking "takes advantage of the limited background information activated by the comprehension process itself" (p.~375). They argue that the search for relevance "automatically involves the making of inferences which may turn up inconsistencies or incoherences relevant to epistemic assessment" (p.~376).

% epistemic vigilance and reasoning
Next, the authors return to and expand upon an idea we have seen \citet{Sperber01} propose, concerning the emergence of argumentation as a demonstration of coherence. I will discuss this in much more detail in \cref{sec:exp-atr}, as this part of \citet{Sperber10} basically constitutes a rudimentary explication of the argumentative theory of reasoning.

To summarize, according to Sperber and his colleagues humans have developed a suite of mechanisms for epistemic vigilance, filtering incoming information in order to avoid being deceived by others. A communicative act triggers both comprehension and epistemic vigilance, and the epistemic assessment of the communicative act draws upon some of the inferential steps that are carried out in the search for relevance, which makes the assessment relatively cost-effective. Epistemic vigilance can be directed towards the source of a message or towards the content of the message. This amounts to the calibration of trust and coherence checking, respectively.


\section{Conclusion}

\todo[inline]{Summarize the conclusions from each section and segue into the next chapter}
