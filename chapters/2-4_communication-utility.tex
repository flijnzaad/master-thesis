\section{What is the function of communication?}
\label{sec:comm:function}

Finally and arguably most importantly for our endeavor, let us have a look at the function of communication.

Essentially, communication facilitates interaction between individuals.
This interaction may be either cooperative or competitive in nature, as we have seen in \cref{sec:comm:phylogeny} when discussing \poscite{SeyfarthCheney03} review of animal communication.
Whether the communicative event is cooperative or competitive in nature depends on the interests of the interlocutors. If the interests of interlocutors overlap or even align, their communication can be considered to be cooperative; if they do not overlap, or even oppose each other, their communication can be considered to be competitive.
For example, if two individuals engage in collaborative hunting of a large prey animal, their interests (catching the prey together and sharing it) align and they will thus use communication for cooperative purposes -- i.e., to coordinate their hunting activity. On the other hand, if two individuals compete for a smaller prey animal, their interests (catching the prey alone and keeping it for themselves) oppose each other, and their communication would thus be competitive. They might for example intimidate each other verbally, which may be evolutionarily more advantageous than physical intimidation (i.e. fight) because of a reduced risk of injury.

As argued by \citet{Tomasello08, Tomasello09} and echoed by \citet{Dor17}, the cooperative setting constitutes the 'birthplace' of the unique features of human communication; the competitive use of human-style communication must have emerged later. As \citet{Tomasello08} writes:
\begin{quoting}
    The use of skills of cooperative communication outside of collaborative activities (e.g., for lying), came only later.
    \hfill (p.~325)
\end{quoting}
Especially the emergence of language could only have occurred in cooperative settings, \citet{Tomasello08} and \citet{Dor17} argue. I return to this line of thought in a bit.

Let us now consider pragmatic communication, specifically how all of this relates to Grice's cooperative principle: \todo{Is quoting the principle redudant for my audience?}
\begin{quoting}
    Make your conversational contribution such as is required, at the stage at which it occurs, by the accepted purpose or direction of the talk exchange in which you are engaged.
    \hfill \citep[p.~45]{Grice75}
\end{quoting}
Earlier along the evolutionary timeline, an interlocutor's Gricean cooperativeness may very well have coincided with her cooperative intention or disposition. This is of course especially the case with animals communicating (see \cref{sec:comm:phylogeny}).
However, it is apparent that at the very least nowadays, these two dimensions of cooperativeness need not coincide.
Daniel \poscite{Dor17} discussion of lying and the evolution of language relates to this distinction between what we might refer to as Gricean cooperativeness and behavioral cooperativeness.\todo{Think of a better term}
Dor notes that the distinction between honesty and deception might be interpreted in two ways. On the one hand, one can consider the honesty of a signal to be its truthfulness; so an honest signal is a true one, and a deceitful signal is a false one. On the other hand, one may consider the honesty of a signal to refer to not its truthfulness, but rather the benefits and costs the sender and receiver incur as a result of the communicated signal. In the case of animals communicating, these two conceptions of honesty might very well coincide -- i.e., truthful signals benefit receivers and false signals harm receivers. However, it should be apparent that they do not always coincide in the human case: truthful signals may hurt or cause harm to receivers, and false signals may benefit the receiver. \example{Give example}
He then outlines four possible communicative options one might choose, namely "co-operative honesty, harmful honesty, co-operative lying and harmful lying" \citep[p.~45]{Dor17}.
He argues that while anti-social, exploitative lies -- lies with the intention to profit at the expense of the receiver, i.e.\@ harmful lies -- are like the 'default' conception of a lie, they are not at all the most prevalent kind of lie. \citet{Meibauer18} has useful additions to this point: he notes that prosocial lying is connected to politeness. The notion of politeness, in turn, can be connected with the notion of benefits and costs:
\begin{quoting}
    In antisocial (mendacious) lying, only the speaker profits from lying. In prosocial lying, lying is either altruistic (only the hearer profits from the speaker's lie) or polite ("Pareto-white," as Erat \& Gneezy 2012 call it) in the sense that both speaker and hearer profit from the lie.
    \hfill \citep[p.~371]{Meibauer18}
\end{quoting}

Going back to Grice's cooperative principle, we note that Dor in his analysis only considers Grice's maxim of Quality ('Try to make your contribution one that is true'). If we extend his distinction to the whole cooperative principle -- in other words, if we also incorporate the maxims of Quantity, Relation and Manner -- we get two axes of cooperation (Gricean and intentional). This leads us to four communicative options one might choose: Gricean cooperative cooperation, Gricean noncooperative cooperation, Gricean cooperative noncooperation, and Gricean noncooperative noncooperation.

For example, lying constitues Gricean noncooperation, because it violates the maxim of Quality. As noted before, lying may be used with a competitive intention (i.e. antisocial lying), or a cooperative intention (i.e. prosocial lying, such as lying for politeness or white lies).

\subsubsection{A small note on language}

Let me briefly address the elephant in the room: when considering human communication, human language and its evolution cannot remain unmentioned. In this thesis, I will only consider human communication in general, because the emergence and evolution of symbolic communication in the form of language is a whole field of research of its own.
I will finish by noting one important thing about the evolution of language as it relates to cooperation and trust. It has been argued that only in cooperative settings could our complex language have emerged at all. This is because for such complexity to arise, more frequent and prolonged interactions are necessary \citep{Benitez21}. As \citet{Dor17} writes,
\begin{quoting}
    The collective effort of the invention and stabilization of the new technology [namely, language] must have been based on high levels of reliability and trust between the inventors: otherwise, indeed, they would not have been able to get the system going.
    \hfill (p.~50)
\end{quoting}
We return to a discussion of reliability, trust and 'getting the system going' in \cref{sec:S-P08}. For more discussion on the evolution of language, see for example \citet{Tomasello08} and \citet{Dor17}.

\todo{Rewrite this segue}
Now, we have seen that communication may be used cooperatively or competitively. To fully appreciate the cooperative function of communication, let us now consider what makes cooperation itself evolutionarily beneficial. Moreover, in order to complete the causal chain, we will have a look at how cooperation could have evolved and the role that communication plays in it. We will do so by drawing extensively from Michael Tomasello's comprehensive \citeyear{Tomasello09} book \emph{Why We Cooperate}.

\subsection{Human cooperation and its evolution}
\label{sec:comm:cooperation}

Let me start off with a brief terminological aside: although colloquially the terms 'cooperation' and 'collaboration' are more or less synonymous, Tomasello does not use them interchangeably. He defines collaboration as working together for mutual benefit (p.~xvii). Implicitly, he takes cooperation to be an overarching term which also encompasses for example altruism, in which one individual sacrifices something to help another individual. For the remainder of this thesis, I will adhere to his terminological conventions.

Tomasello argues that somewhere along the evolutionary timeline, humans must have been "put under some kind of selective pressure to collaborate in their gathering of food--they become obligate collaborators--in a way that their closest primate relatives were not" \citep[p.~75]{Tomasello09}.
\footnote{Notably, what exactly this selective pressure is, is a missing link in his otherwise very convincing story.}
He elaborates by noting that in general, evolution may select for sociality in animals because living together in a social group protects the group's members against predation: it is easier to defend oneself in the context of a group. The group however also brings disadvantages with it when it comes to foraging for food, since the members of the group are competitors in the acquisition of food. This is especially the case when the source of food is 'clumped', such as in a prey animal, rather than dispersed, such as in a plain of grass. The clumped source of food raises the issue of how to share the food amongst the members of the social group.
Tomasello enumerates a number of different hypotheses to explain how humans could have broken out of what he calls "the great-ape pattern of strong competition for food, low tolerance for food sharing, and no food offering at all" \citep[p.~83]{Tomasello09}; in other words, how humans could have evolved to be more tolerant and trusting, and less competitive about food.
Firstly, as due to a certain selective pressure it became necessary for humans to forage collaboratively, it could have been evolutionarily advantageous to be more tolerant and less competitive, which would explain its having evolved.
Secondly, Tomasello notes it could be the case that humans went through a process of self-domestication, which eliminated aggressive, predatory or greedy individuals from the group; see \citet{Benitez21} for more on this.
Thirdly, the evolution of tolerance and trust could be related to what is called \emph{cooperative breeding}, also known as \emph{alloparenting}. In cooperative breeding, the responsibility of child-rearing falls on more individuals than just the mother of the child; these individuals help by providing food for the child and engaging in other acts of childcare. This cooperative breeding may have selected for pro-social skills and motivations; see \citet{Hrdy09} for an elaboration of this argument.

Tolerance and trust then constitute a foundation upon which coordination and communication can be 'built', so to speak: they provide an environment in which more elaborate collaboration can evolve. In Tomasello's words,
\begin{quoting}
    there had to be some initial emergence of tolerance and trust (\ldots) to put a population of our ancestors in a position where selection for sophisticated collaborative skills was viable
    \hfill (p.~77)
\end{quoting}

In order to then arrive at the full picture of human cooperative activity, the final step to consider is that of social norms and institutions. As before, there is a missing link in this story, in this case it concerns how mutual expectations between individuals arise and eventually become norms. (Tomasello describes it as "one of the most fundamental questions in all of the social sciences" (p. 89).)
Norms may be defined as "socially agreed-upon and mutually known expectations bearing social force, monitored and enforced by third parties" \citep[p.~87]{Tomasello09}. Norms receive their force not only from the threat of punishment by others if the norm is violated, but also from a kind of social rationality within the collaborative activity. Individuals recognize their dependence on each other for reaching their joint goal. Just as it would be individually irrational to act in a way that thwarts your own goal, it would be socially irrational to act in a way that thwarts your joint goal.

Let us now briefly summarize the evolutionary timeline of human cooperation according to Michael Tomasello.
At some point, for reasons as of yet unknown to us, foraging for food collaboratively rather than individualistically became beneficial -- perhaps even necessary -- for humans.
During this evolutionary process, some degree of tolerance and trust must have emerged between those collaborating individuals.
In the process of adapting to this collaborative foraging, humans evolved certain skills and motivations specifically for cooperation -- for example, abilities for establishing joint goals as well as a role division for the joint activity.
This kind of collaborative activity then constituted the breeding ground for human cooperative communication.
These joint goals and role divisions later evolved into the superindividual norms, rights and responsibilities that we see within our social institutions today.

As a brief aside: it has been argued that communication is not necessary nor sufficient for the coordination of activities. \citet{Goldstone24} propose a framework of five features characterizing the specialization of roles in group activities; communication is only one of these five features. This is corroborated by experiments they review in which people "spontaneously differentiate themselves into stable roles" (p.~264) in group activity without communicating with each other.
However, the authors note that communication does play a very central role in coordinating group activities, stating that
\begin{quoting}
    direct communication of plans is often the single most potent tool of collective coordination
    \hfill \citep[p.~276]{Goldstone24}
\end{quoting}
See also \citet{Vorobeychik17} for a discussion of communication and coordination.

Now, armed with the ins and outs of human cooperation, and an inkling of how communication relates to the story, we turn our attention to a crucial aspect of understanding human communication: how it can have persisted despite evolutionary pressures treathening its stability. To this end, let us consider at length a paper by \citet{Scott-Phillips08}, who convincingly brings findings from animal signaling research into the realm of human communication. This will provide a good background for discussing two precursory papers to the argumentative theory of reasoning, by Sperber (and others) in \cref{sec:Sperber01,sec:Sperber10}.

\subsection{The stability of communication}
\label{sec:S-P08}

If communication between individuals of a species persists throughout evolution, we may speak of it as stable. The stability of communication is considered by some as the 'defining problem' of animal signaling research \citep{Scott-Phillips08}. It is not a trivial problem by any means: the stability of a communication system is threatened by evolutionary pressures on the communicator to 'defect', as it were. As \citet{Scott-Phillips08} describes it,
\begin{quoting}
    If one can gain through the use of an unreliable\footnote{See \cref{sec:deception} for a terminological comment on reliability versus honesty.} signal then we should expect natural selection to favour such behaviour. Consequently, signals will cease to be of value, since receivers have no guarantee of their reliability. This will, in turn, produce listeners who do not attend to signals, and the system will thus collapse in an evolutionary retelling of Aesop’s fable of the boy who cried wolf.
    \hfill (p.~275)
\end{quoting}
In the context of human communication: if it can be advantageous for me to lie, deceive or mislead someone, then it would evolutionarily make sense for me to do so; yet then it would make evolutionary sense for you to stop listening to me, and as a consequence our system of communication would collapse.

There have been a number of attempts at explaining the reliability of animal communication in general. One such attempt is the \emph{handicap principle} \citep{Zahavi75, Zahavi99}, which might be best understood through the paradigmatic example of the peacock's tail. This tail is like a handicap for the peacock: not only does it take a lot of resources to grow the tail and carry it around, it also leaves the bird more vulnerable to predation because it is less agile with a large unwieldy tail. At the same time, a large tail signals to peahens that the peacock is fit enough to be able to incur these costs, and thus has a sexual advantage.
The handicap principle then describes this process of communication, by which the signaler incurs costs (i.e., a handicap) for signaling, which thus guarantees the reliability of the signal.
\todo{Make reference back to the Freeberg definition: a tail is communication}

However useful in explaining some cases of the reliability of animal communication, the handicap principle is not able to explain all of those cases: often, it is not the case that reliable signals are costly to produce \citep{Scott-Phillips08} \example{Add example}. Especially in the case of human communication, the handicap principle cannot account for its reliability, since it is in general not costly to produce utterances \citep{Scott-Phillips08}.
Thus, it remains to be shown how communication can be stable if signals are cost-free.

On the handicap principle, reliable signals are costly to produce, thus ensuring their reliability. An alternative explanation of the reliability of animal communication is the principle of \emph{deterrence}, whereby \emph{un}reliable signals are costly to produce, and consequently signalers are deterred from producing unreliable signals.
There are a number of ways in which producing unreliable signals may be costly to the signaler. Firstly, this is the case in a coordination game, where the signaler and receiver share some common interest with regard to the outcome of the interaction.\todo{possibly explain this more}
Secondly, if two individuals have repeated interactions, it may also be costly in the long term to produce unreliable signals, because it may hinder cooperation in the future.
Thirdly, producing unreliable signals may be costly to the signaler if false signals are punished by the receiver.

The 'logic of deterrents' applied to the case of human communication poses the following demands in order for the story about stability to work:
\begin{quoting}
    Sufficient conditions for cost-free signalling in which reliability is ensured through deterrents are that signals be verified with relative ease (if they are not verifiable then individuals will not know who is and who is not worthy of future attention) and that costs be incurred when unreliable signalling is revealed.
    \hfill \citep[p.~?]{Scott-Phillips08}
\end{quoting}
In other words, if unreliable signals are recognized as unreliable relatively easily, and unreliable signalers incur costs for their unreliability, the reliability of communication is secured through deterrents.

Scott-Phillips goes on to state that these sufficient conditions are met in the case of human communication, since people may refrain from interacting with unreliable individuals in the future, which can be very costly for a social species such as humans.
Notably however, he does not explicate how the first sufficient condition is met in the case of human communication; we will return to this in \cref{sec:EV-scrutiny}.

Now, before we consider Sperber's precursory concepts to the argumentative theory of reasoning, it will be good to have a closer look at deception. The 'stability of communication' problem hinges on the assumption that it is advantageous to deceive others. This is intuitively plausible; however, it deserves some extra attention, as this is such a fundamental assumption.

\subsection{Deception and lying}
\label{sec:deception}

Let us start off this section by clarifying and examining some of the terminology surrounding honesty and dishonesty.

Briefly returning to \citet{Scott-Phillips08}, he uses the term 'reliable' when explicating his story about the stability of communication, rather than 'honest'. As he explains in the paper's introduction, this is a principled choice, to do with a difference between humans and non-human animals. He argues that one may want to steer clear from anthropomorphically ascribing intentions to animals, and meanings to their behavior. He maintains that thus 'reliability' would be a more neutral term than 'honesty', because it refrains from ascribing intentions and meanings to individuals.
However, I find that talking about 'reliable communicators' blurs an important distinction between honest, benevolent communicators and competent communicators. We return to this distinction in \cref{sec:Sperber10}.

Deception, lying and persuasion may be defined in a number of different ways. Let us now briefly look to the literature to obtain a working definition of these concepts.

First off, deception may be defined as
\begin{quoting}
    deliberately leading someone into a false belief
    \hfill \citep[p.~358]{Meibauer18}
\end{quoting}

Lying, on the other hand, might not be as easily defined: Jennifer Mather Saul even dedicates the whole first chapter of her \citeyear{Saul12} book to obtaining a viable definition that includes the relevant examples and excludes the irrelevant ones. An intricate discussion of her definition is out of scope for this thesis; let us for now consider a definition of lying due to \citet{Williams02} that \citet{Meibauer18} calls 'standard':
\begin{quoting}
    an assertion, the content of which the speaker believes to be false, which is made with the intention to deceive the hearer with respect to that content
    \hfill \citep[p.~96]{Williams02}
\end{quoting}
Meibauer notes -- and this also transpires from \poscite{Saul12} discussion of the definition of lying -- that each of the components of this definition can be, and has been, challenged.

Let us now discuss some of the points that Daniel \citet{Dor17} makes about lying and the stability of communication.

Especially important are the notes he makes regarding what he terms the "paradox of honest signaling" \citep[p.~46]{Dor17} -- i.e., the theoretical issues plaguing the stability of communication that we discussed in \cref{sec:S-P08}. 
Dor notes that this foretold collapse of communication due to unreliability of the speaker, does not hinge on whether or not the speaker is truthful, but whether her \emph{intention} is benevolent.
In other words, this story appeals not to receivers evaluating the truthfulness of incoming information, but rather the receivers evaluating the \emph{intention} of the sender. Listeners care whether speakers intend to be harmful, not whether or not they are truthful, Dor argues.
Moreover, he notes that the paradox of honest signaling mostly appeals to situations in which interests between interlocutors conflict; however, these situations might not be the most pertinent or prevalent kind of communicative situation.
According to Dor, at the point in evolutionary time when language emerged, humans were already crucially dependent on cooperation and coordinated action, and thus their interests overlapped more often than not.

Moreover, (and briefly returning to the utility of communication), another avenue Dor explores is how communication is used. As we will see in \cref{sec:Sperber01} and in \cref{sec:Sperber10}, Sperber and his colleagues focus a lot on the transmission of information between individuals, in the form of testimony and argumentation. However, Dor argues that within the paradox of honest signaling, this transmission of information is not the only relevant use of communication. Communication is also used for cooperation, and in that situation lying is not really an issue; Dor writes
\begin{quoting}
    Language is extremely useful in the coordination of collective work, collective defense and so on, where it is used not just for the exchange of information but also for collective planning, division of labor, ordering and requesting, where lying as such does not seem to play a major role.
    \hfill \citep[p.~51]{Dor17}
\end{quoting}
Convincingly, Dor goes on to argue that due to this dual role of communication (transmitting information on the one hand, and facilitating cooperation on the other), the stability of communication is not threatened by lying. He writes:
\begin{quoting}
    Even in the very unlikely doomsday scenario, then, where all the members of a community lie to each other in their factual statements, and eventually refrain from sharing information with each other, there is no reason to assume that they would stop using language for all these other purposes, especially where their survival, whether they like it or not, depends on collective action.
    \hfill \citep[p.~52]{Dor17}
\end{quoting}

Lastly, it would be good to have a brief look at persuasion, since it naturally plays a considerable role in Mercier \& Sperber's argumentative theory of reasoning.
\citet{Brinol09} broadly define persuasion as
\begin{quoting}
    any procedure with the potential to change someone's mind
    \hfill (p.~50),
\end{quoting}
whether that be changing someone's emotional state, beliefs, behaviors or attitudes.
They describe persuasion as "the most frequent and ultimately efficient approach to social influence" (pp.~49--50). Put crudely, persuasion is a tool for getting what you want, and it serves this end better than the alternatives of using force, threats or violence.\footnote{In this, one may see a parallel with \poscite{SeyfarthCheney03} conception of aggressive communication as a low-risk alternative to fighting.}
From this observation, the conclusion emerges that persuading someone is beneficial to an individual exactly to the extent that the corresponding gain in social influence is beneficial to the individual.
For limitations of time and space we will not go into the benefits of gaining social influence. However, I believe the discussion of cooperation in \cref{sec:comm:cooperation} is sufficient for the present purposes.
