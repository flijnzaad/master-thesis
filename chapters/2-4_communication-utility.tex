\section{What is the function of communication?}
\label{sec:comm:function}

Finally and arguably most importantly for our endeavor, let us have a look at the function of communication.

\todo[inline]{Terminology: competitive vs. noncooperative?}

Essentially, communication facilitates interaction between individuals.
This interaction may be either cooperative or competitive in nature, as we have seen in \cref{sec:comm:phylogeny} when discussing \poscite{SeyfarthCheney03} review of animal communication.
Whether the communicative event is cooperative or competitive in nature depends on the interests of the interlocutors. If the interests of interlocutors overlap or even align, their communication can be considered to be cooperative; if they do not overlap, or even oppose each other, their communication can be considered to be competitive.
For example, if two individuals engage in collaborative hunting of a large prey animal, their interests (catching the prey together and sharing it) align and they will thus use communication for cooperative purposes -- i.e., to coordinate their hunting activity. On the other hand, if two individuals compete for a smaller prey animal, their interests (catching the prey alone and keeping it for themselves) oppose each other, and their communication would thus be competitive. They might for example intimidate each other verbally, which may be evolutionarily more advantageous than physical intimidation (i.e. fight) because of a reduced risk of injury.

As argued by \citet{Tomasello08, Tomasello09} and echoed by \citet{Dor17}, the cooperative setting constitutes the 'birthplace' of the unique features of human communication; the competitive use of human-style communication must have emerged later. As \citet{Tomasello08} writes:
\begin{quoting}
    The use of skills of cooperative communication outside of collaborative activities (e.g., for lying), came only later.
    \hfill (p.~325)
\end{quoting}
Especially the emergence of language could only have occurred in cooperative settings, \citet{Tomasello08} and \citet{Dor17} argue. I continue this line of thought in \cref{sec:comm:language}.

\todo[inline]{This section could be a bit more coherent. Also, improve the terminology: it's a damn mess}
Let us now consider pragmatic communication, specifically how all of this relates to Grice's cooperative principle \todo{Cite 'Logic and conversation' here, and quote the principle}.
Earlier along the evolutionary timeline, an interlocutor's Gricean cooperativeness may very well have coincided with her cooperative intention or disposition. This is of course especially the case with animals communicating (see \cref{sec:comm:phylogeny}).
However, it is apparent that at the very least nowadays, these two dimensions of cooperativeness need not coincide.
\todo[inline]{This discussion of Dor needs to be more tailored to this section}
Daniel \poscite{Dor17} discussion of lying and the evolution of language relates to this distinction between what we might refer to as Gricean cooperativeness and behavioral cooperativeness.\todo{Think of a better term}
Dor notes that the distinction between honesty and deception might be interpreted in two ways. On the one hand, one can consider the honesty of a signal to be its truthfulness; so an honest signal is a true one, and a deceitful signal is a false one. On the other hand, one may consider the honesty of a signal to refer to not its truthfulness, but rather the benefits and costs the sender and receiver incur as a result of the communicated signal. In the case of animals communicating, these two conceptions of honesty might very well coincide -- i.e., truthful signals benefit receivers and false signals harm receivers. However, it should be apparent that they do not always coincide in the human case: truthful signals may hurt or cause harm to receivers, and false signals may benefit the receiver. \example{Give example?}
He then outlines four possible communicative options one might choose, namely "co-operative honesty, harmful honesty, co-operative lying and harmful lying" \citep[p.~45]{Dor17}.
He argues that while anti-social, exploitative lies -- lies with the intention to profit at the expense of the receiver, i.e.\@ harmful lies -- are like the 'default' conception of a lie, they are not at all the most prevalent kind of lie. \citet{Meibauer18} has useful additions to this point: he notes that prosocial lying is connected to politeness. The notion of politeness, in turn, can be connected with the notion of benefits and costs:
\begin{quoting}
    In antisocial (mendacious) lying, only the speaker profits from lying. In prosocial lying, lying is either altruistic (only the hearer profits from the speaker's lie) or polite ("Pareto-white," as Erat \& Gneezy 2012 call it) in the sense that both speaker and hearer profit from the lie.
    \hfill \citep[p.~371]{Meibauer18}
\end{quoting}

Going back to Grice's cooperative principle, we note that Dor in his analysis only considers Grice's maxim of Quality ('Try to make your contribution one that is true'). If we extend his distinction to the whole cooperative principle, we get two axes of cooperation (Gricean and intentional), leading to four communicative options one might choose: Gricean cooperative cooperation, Gricean noncooperative cooperation, Gricean cooperative noncooperation, and Gricean noncooperative noncooperation.

For example, lying constitues Gricean noncooperation, because it violates the maxim of Quality. As noted before, lying may be used with a competitive intention (i.e. antisocial lying), or a cooperative intention (i.e. prosocial lying, such as lying for politeness or white lies).\todo{Does more need to be added? Matrix for clarity?}

\todo[inline]{How does the following paragraph fit in? Elaborate and/or update after updating \cref{sec:comm:ontogeny}}
Moving on from the most basic distinction between cooperative and competitive uses of communication, we may consider three basic motives that drive communicative acts\todo{Does this hold only for cooperative communication?}:
requesting, informing and sharing \citep{Tomasello08}.

\todo[inline]{Where to put this subsection}
\subsubsection{A small note on language}
\label{sec:comm:language}

Let me briefly address the elephant in the room: when considering human communication, human language and its evolution cannot remain unmentioned. In this thesis, I will only consider human communication in general, because the emergence and evolution of symbolic communication in the form of language is a whole field of research of its own.
I will finish by noting one important thing about the evolution of language as it relates to cooperation and trust. It has been argued that only in cooperative settings could our complex language have emerged at all. This is because for such complexity to arise, more frequent and prolonged interactions are necessary \citep{Benitez21}. As \citet{Dor17} writes,
\begin{quoting}
    The collective effort of the invention and stabilization of the new technology [namely, language] must have been based on high levels of reliability and trust between the inventors: otherwise, indeed, they would not have been able to get the system going.
    \hfill (p.~50)
\end{quoting}
We return to a discussion of reliability, trust and 'getting the system going' in \cref{sec:S-P08}. For more discussion on the evolution of language, see for example \citet{Tomasello08} and \citet{Dor17}.

\todo[inline]{Rewrite this segue}
Now, we have seen that communication may be used cooperatively or competitively. To appreciate how communication facilitates cooperation, let us now consider what makes cooperation itself evolutionarily beneficial. Moreover, in order to complete the causal chain, we will have a look at how cooperation could have evolved and the role that communication plays in it. We will do so by drawing extensively from Michael Tomasello's comprehensive \citeyear{Tomasello09} book \emph{Why We Cooperate}.

\subsection{Human cooperation and its evolution}
\label{sec:comm:cooperation}

Let me start off with a brief terminological aside: although colloquially the terms 'cooperation' and 'collaboration' are more or less synonymous, Tomasello does not use them interchangeably. He defines collaboration as working together for mutual benefit (p.~xvii). Implicitly, he takes cooperation to be an overarching term which also encompasses for example altruism, in which one individual sacrifices something to help another individual. For the remainder of this thesis, I will adhere to his terminological conventions.

Tomasello argues that somewhere along the evolutionary timeline, humans must have been "put under some kind of selective pressure to collaborate in their gathering of food--they become obligate collaborators--in a way that their closest primate relatives were not" \citep[p.~75]{Tomasello09}.
\footnote{Notably, what exactly this selective pressure is, is a missing link in his otherwise very convincing story.}
He elaborates by noting that in general, evolution may select for sociality in animals because living together in a social group protects the group's members against predation: it is easier to defend oneself in the context of a group. The group however also brings disadvantages with it when it comes to foraging for food, since the members of the group are competitors in the acquisition of food. This is especially the case when the source of food is 'clumped', such as in a prey animal, rather than dispersed, such as in a plain of grass. The clumped source of food raises the issue of how to share the food amongst the members of the social group.
Tomasello enumerates a number of different hypotheses to explain how humans could have broken out of what he calls "the great-ape pattern of strong competition for food, low tolerance for food sharing, and no food offering at all" \citep[p.~83]{Tomasello09}; in other words, how humans could have evolved to be more tolerant and trusting, and less competitive about food.
Firstly, as due to a certain selective pressure it became necessary for humans to forage collaboratively, it could have been evolutionarily advantageous to be more tolerant and less competitive, which would explain its having evolved.
Secondly, Tomasello notes it could be the case that humans went through a process of self-domestication, which eliminated aggressive, predatory or greedy individuals from the group; see \citet{Benitez21} for more on this.
Thirdly, the evolution of tolerance and trust could be related to what is called \emph{cooperative breeding}, also known as \emph{alloparenting}. In cooperative breeding, the responsibility of child-rearing falls on more individuals than just the mother of the child; these individuals help by providing food for the child and engaging in other acts of childcare. This cooperative breeding may have selected for pro-social skills and motivations; see \citet{Hrdy09} for an elaboration of this argument.

Tolerance and trust then constitute a foundation upon which coordination and communication can be 'built', so to speak: they provide an environment in which more elaborate collaboration can evolve. In Tomasello's words,
\begin{quoting}
    there had to be some initial emergence of tolerance and trust (\ldots) to put a population of our ancestors in a position where selection for sophisticated collaborative skills was viable
    \hfill (p.~77)
\end{quoting}

In order to then arrive at the full picture of human cooperative activity, the final step to consider is that of social norms and institutions. As before, there is a missing link in this story, in this case it concerns how mutual expectations between individuals arise and eventually become norms. (Tomasello describes it as "one of the most fundamental questions in all of the social sciences" (p. 89).)
Norms may be defined as "socially agreed-upon and mutually known expectations bearing social force, monitored and enforced by third parties" \citep[p.~87]{Tomasello09}. Norms receive their force not only from the threat of punishment by others if the norm is violated, but also from a kind of social rationality within the collaborative activity. Individuals recognize their dependence on each other for reaching their joint goal. Just as it would be individually irrational to act in a way that thwarts your own goal, it would be socially irrational to act in a way that thwarts your joint goal.

\todo[inline]{This could/should be improved upon}
Let us now briefly summarize the evolutionary timeline of human cooperation according to Michael Tomasello.
At some point, for reasons as of yet unknown to us, foraging for food collaboratively rather than individualistically became beneficial -- perhaps even necessary -- for humans.
During this evolutionary process, some degree of tolerance and trust must have emerged between those collaborating individuals.
In the process of adapting to this collaborative foraging, humans evolved certain skills and motivations specifically for cooperation -- for example, abilities for establishing joint goals as well as a role division for the joint activity.
This kind of collaborative activity then constituted the breeding ground for human cooperative communication.
These joint goals and role divisions later evolved into the superindividual norms, rights and responsibilities that we see within our social institutions today.\todo{Weird sentence}

As a brief aside: it has been argued that communication is not necessary nor sufficient for the coordination of activities. \citet{Goldstone24} propose a framework of five features characterizing the specialization of roles in group activities; communication is only one of these five features. This is corroborated by experiments they review in which people "spontaneously differentiate themselves into stable roles" (p.~264) in group activity without communicating with each other.
However, the authors note that communication does play a very central role in coordinating group activities, stating that
\begin{quoting}
    direct communication of plans is often the single most potent tool of collective coordination
    \hfill \citep[p.~276]{Goldstone24}
\end{quoting}
See also \citet{Vorobeychik17} for a discussion of communication and coordination.

Now, armed with the ins and outs of human cooperation, and an inkling of how communication relates to the story, we turn our attention to a crucial aspect of understanding human communication: how it can have persisted despite evolutionary pressures treathening its stability. To this end, let us consider at length a paper by \citet{Scott-Phillips08}, who convincingly brings findings from animal signalling research into the realm of human communication. This will prove a steady foundation for discussing two precursory papers to the argumentative theory of reasoning, by Sperber (and others) in \cref{sec:Sperber01,sec:Sperber10}.

\subsection{The stability of communication}
\label{sec:S-P08}

\todo[inline]{(Possibly) add page numbers throughout section}

\todo[inline]{Make a comment here about terminology: reliability vs. honesty}

If communication between individuals of a species persists throughout evolution, we may speak of it as stable. The stability of communication is considered by some as the 'defining problem' of animal signalling research \citep{Scott-Phillips08}. It is not a trivial problem by any means: the stability of a communication system is threatened by evolutionary pressures on the communicator to 'defect', as it were. As \citet{Scott-Phillips08} describes it,
\begin{quoting}
    If one can gain through the use of an unreliable signal then we should expect natural selection to favour such behaviour. Consequently, signals will cease to be of value, since receivers have no guarantee of their reliability. This will, in turn, produce listeners who do not attend to signals, and the system will thus collapse in an evolutionary retelling of Aesop’s fable of the boy who cried wolf.
    \hfill (p.~275)
\end{quoting}
In the context of human communication: if it can be advantageous for me to lie, deceive or mislead someone, then it would evolutionarily make sense for me to do so; yet then it would make evolutionary sense for you to stop listening to me, and as a consequence our system of communication would collapse.

There have been a number of attempts at explaining the reliability of animal communication in general. One such attempt is the \emph{handicap principle} \citep{Zahavi75, Zahavi99}, which might be best understood through the paradigmatic example of the peacock's tail. This tail is like a handicap for the peacock: not only does it take a lot of resources to grow the tail and carry it around, it also leaves the bird more vulnerable to predation because it is less agile with a large unwieldy tail. At the same time, a large tail signals to peahens that the peacock is fit enough to be able to incur these costs, and thus has a sexual advantage.
The handicap principle then describes this process of communication, by which the signaller incurs costs (i.e., a handicap) for signalling, which thus guarantees the reliability of the signal.
\todo{How does this tail constitute, not only signalling, but communication?}

However useful in explaining some cases of the reliability of animal communication, the handicap principle is not able to explain all of those cases: often, it is not the case that reliable signals are costly to produce \citep{Scott-Phillips08} \example{Add example}. Especially in the case of human communication, the handicap principle cannot account for its reliability, since it is in general not costly to produce utterances \citep{Scott-Phillips08}.
Thus, it remains to be shown how communication can be stable if signals are cost-free.

On the handicap principle, reliable signals are costly to produce, thus ensuring their reliability. An alternative explanation of the reliability of animal communication is the principle of \emph{deterrence}, whereby \emph{un}reliable signals are costly to produce, and consequently signallers are deterred from producing unreliable signals.
There are a number of ways in which producing unreliable signals may be costly to the signaller. Firstly, this is the case in a coordination game, where the signaller and receiver share some common interest with regard to the outcome of the interaction.\todo{possibly explain this more}
Secondly, if two individuals have repeated interactions, it may also be costly in the long term to produce unreliable signals, because it may hinder cooperation in the future.
Thirdly, producing unreliable signals may be costly to the signaller if false signals are punished by the receiver.

The 'logic of deterrents' applied to the case of human communication poses the following demands in order for the story about stability to work:\todo{This sentence is still a bit weird}
\begin{quoting}
    Sufficient conditions for cost-free signalling in which reliability is ensured through deterrents are that signals be verified with relative ease (if they are not verifiable then individuals will not know who is and who is not worthy of future attention) and that costs be incurred when unreliable signalling is revealed.
    \hfill \citep[p.~?]{Scott-Phillips08}
\end{quoting}
In other words, if unreliable signals are recognized as unreliable relatively easily, and unreliable signallers incur costs for their unreliability, the reliability of communication is secured through deterrents.

Scott-Phillips goes on to state that these sufficient conditions are met in the case of human communication, since people may refrain from interacting with unreliable individuals in the future, which can be very costly for a social species such as humans.
Notably however, he does not explicate how the first sufficient condition is met in the case of human communication; we will return to this in \cref{sec:EV-scrutiny}.

\subsection{Honesty, deception and lying}

\todo[inline]{Dangers of misinformation?}

\todo[inline]{This section was written in order to be bullshit: be very critical of this one when revising. Also, it feels like some of this, if not all of this, information belongs in Chapter 4 possibly}

\todo[inline]{Still to add: benefits and costs of honesty/deception}

\todo[inline]{Relate this section to the functions of communication as you laid them out at the start of this section}

Let us start off this section by discussing some terminology around honesty.

\citet{Scott-Phillips08} uses the term 'reliable' when explicating his story about the stability of communication, rather than 'honest'. This is a principled choice, as he explains in the paper's introduction, to do with a difference between humans and non-human animals. He argues that one may want to steer clear from anthropomorphically ascribing intentions to animals and meanings to their behavior. He maintains that the term 'reliability' rather than 'honesty' would be a more neutral option, because it refrains from ascribing intentions and meanings to people. \todo{Is this a correct interpretation of what he says?}
However, I find that his use of the term 'reliable' confuses two important axes with each other: one ranging from competent to incompetent signallers, and one ranging from benevolent to malevolent signallers. We return to this point later.

Lying, deception, honesty and persuasion may be defined in a great number of ways. Let us here look to the literature to obtain a working definition for the remainder of this thesis.

Especially when it comes to lying, finding a definition might not be as easy as expected: \citet{Saul12} even dedicates the whole first chapter of her book to obtaining a viable definition that includes and/or excludes the relevant examples. An intricate discussion of her definition is out of scope for this thesis; let us for now consider a definition of lying due to \citet{Williams02} that \citet{Meibauer18} calls 'standard':
\begin{quoting}
    an assertion, the content of which the speaker believes to be false, which is made with the intention to deceive the hearer with respect to that content
    \hfill \citep[p.~96]{Williams02}
\end{quoting}
Meibauer notes -- and this also transpires from \poscite{Saul12} discussion of the definition of lying -- that each of the components of this definition may be and have been challenged.
Additionally, \citet{Meibauer18} notes that deception may be defined as
\begin{quoting}
    deliberately leading someone into a false belief
    \hfill \citep[.~358]{Meibauer18}
\end{quoting}

Let us now discuss some of the points that Daniel \citet{Dor17} makes about how the lie may have figured in the evolution of language. Although our present endeavor may be focused more on deception, not lies, and communication in general, not language, this will still prove to be a relevant undertaking.

Especially important are the notes he makes regarding the so-called "paradox of honest signaling" \citep[p.~45]{Dor17} -- i.e., the theoretical issues plaguing the stability of communication that we discussed in \cref{sec:S-P08}. Dor notes that the collapse of communication as it is described in the 'paradox of honest signaling', appeals not to receivers evaluating the truthfulness of incoming information, but rather the receivers evaluating the \emph{intention} of the sender. Listeners care whether speakers intend to be harmful, not whether or not they are truthful, Dor argues.
Moreover, he notes that the paradox of honest signaling appeals greatly to situations in which interests between interlocutors conflict, which may not be as prevalent or plausible. As he argues, at the point in evolution where language emerged, humans were already crucially dependent on cooperation and coordinated action, and thus their interests overlapped more often than not.

\todo[inline]{Too many quotes? I just think he worded it nicely :(}
Moreover, another very important avenue that Dor explores is how communication is used. As we will see in \cref{sec:Sperber01} and in \cref{sec:Sperber10}, Sperber and his colleagues focus a lot on the transmission of information between individuals, in the forms of testimony and argumentation. However, within the 'paradox of honest signaling', this transmission of information is not the only relevant use of communication. Communication is also used for cooperation, and in that situation lying is not really an issue; Dor writes
\begin{quoting}
    Language is extremely useful in the coordination of collective work, collective defense and so on, where it is used not just for the exchange of information but also for collective planning, division of labor, ordering and requesting, where lying as such does not seem to play a major role.
    \hfill \citep[p.~51]{Dor17}
\end{quoting}
Convincingly, Dor goes on to argue that due to this dual role of communication (transmitting information and facilitating cooperation), the stability of communication is not threatened by lying. He writes:
\begin{quoting}
    Even in the very unlikely doomsday scenario, then, where all the members of a community lie to each other in their factual statements, and eventually refrain from sharing information with each other, there is no reason to assume that they would stop using language for all these other purposes, especially where their survival, whether they like it or not, depends on collective action.
    \hfill \citep[p.~52]{Dor17}
\end{quoting}

\todo[inline]{How does persuasion figure in this?:}
Lastly, it would be good to have a brief look at what \emph{persuasion} is, since it naturally plays a considerable role in Mercier \& Sperber's argumentative theory of reasoning.
\citet{Brinol09} broadly define persuasion as
\begin{quoting}
    any procedure with the potential to change someone's mind
    \hfill (p.~50),
\end{quoting}
whether that be changing someone's emotional state, beliefs, behaviors or attitudes.
They describe persuasion as "the most frequent and ultimately efficient appraoch to social influence" (pp.~49--50). Put crudely, persuasion is a tool for getting what you want, and better than its alternatives of using force, threats or violence to get what you want.\footnote{In this, one may see a parallel with \poscite{SeyfarthCheney03} conception of aggressive communication as a low-risk alternative to fighting.}
From this, I believe it is fair to conclude that persuading someone is beneficial to an individual exactly to the extent that the corresponding gain in social influence is beneficial to the individual.
For limitations of time and space we will not go into the benefits of gaining social influence; however, (conclusion).

\subsection{Sperber on the evolution of testimony and argumentation}
\label{sec:Sperber01}

In a \citeyear{Sperber01} paper, Dan Sperber analyzes testimony and argumentation from an evolutionary perspective. In doing so, he provides important groundwork for his later work with Mercier (and others) on the relation between reasoning, argumentation and the stability of communication.

Testimony and argumentation are two concepts central to human communication. Sperber borrows his definitions for these concepts from epistemologist Alvin Goldman, who defines testimony as "the transmission of observed (or allegedly observed) information from one person to others" \citep[p.~401]{Sperber01} and argumentation as "the defense of some conclusion by appeal to a set of premises that provide support for it" (ibid.).
Sperber puts these two concepts in an evolutionary perspective, and discusses in particular how they have figured in stabilizing communication over the course of evolutionary history.

\todo[inline]{Replace this example by an animal communication example}
A tempting way to look at communication is as a kind of 'cognition by proxy': through communication, one organism may access information another organism has obtained from its own perception or inference. For instance, if you tell me that there is milk in the fridge, I can through this act of communication benefit from the information derived from your perception of the milk carton in the fridge.
However, Sperber argues that, at least in the case of human communication, testimony does not amount to cognition by proxy. This is because testimony has different effects than direct perception does. Going back to our example, upon receiving your testimony stating that the milk is in the fridge, I am in a different cognitive state than if I would have perceived the milk carton there myself. Moreover, in human communication, Sperber argues that interpretation and acceptance of utterances are two separate processes: recognizing what a speaker meant by their utterance is not the same as accepting it as true.\footnote{As a sidenote, this may very well be the case philosophically or epistemologically speaking, but psychologically speaking they may be more intertwined than Sperber implies. Although intuitively one would say that comprehension of an utterance always precedes acceptance (or rejection), \citet{Lewandowsky12} points out that empirical evidence suggests that for someone to comprehend an utterance, they must (at least temporarily) accept it.}
\todo[inline]{And so? Connect this to the ostensive-inferential model: is interpretation part of communication?}

The classical account of animal communication by \citet{DawkinsKrebs78} focuses only on the side of the communicator in the story, maintaining that the function of communication is to manipulate others. Sperber rejects this classical approach, arguing that the interests of the sender cannot be the only driving force in the evolution of communication.
He outlines a similar line of argumentation as we have seen in \cref{sec:S-P08}\todo{Is it too much overlap with that section?},
arguing that for communication to have stabilized and continued to be stable between senders and receivers, both parties must have benefited from the action. In other -- game-theoretic -- terms, communication must (at least in the long run) be a positive-sum game, where both senders and receivers gain from the interaction.

In the case of receiving testimony from others, the receiver gains from testimony "only to the extent that it is a source of genuine (\ldots) information" (p.~404).
\todo[inline]{Where to talk about why gaining information is itself beneficial? Here or Ch. 4?}
On the side of the production of testimony, the sender stands to gain from this testimony because
\begin{quoting}
    it allows them to have desirable effects on the receivers' attitudes and behavior. By communicating, one can cause others to do what one wants them to do and to take specific attitudes to people, objects, and so on
    \hfill (p.~404)
\end{quoting}
He later elaborates on this by saying that getting others to accept your communicated message is not intrinsically beneficial. Rather, it is \emph{indirectly} beneficial, through bringing about these 'desirable effects' in others, as a way of 'cognitive manipulation'.
We briefly return to these observations (particularly the 'desirable effects') in \cref{ch:scrutiny}.
\todo[inline]{Where exactly? Will I talk about how this paper stacks up to the other things I found and concluded about communication? If so, where? Here?}
Sperber notes that it is exactly this self-interest of the sender that renders this 'cognition by proxy' view as inapplicable to human communication.
Moreover, he concludes from these observations of his that
\begin{quoting}
    the function of communication presents itself differently for communicator and audience
    \hfill (p.~411)
\end{quoting}
\todo[inline]{Conclusion about how this fits with what I wrote?? Or should this quote and comment be somewhere else?}

Sperber goes on to cast his observations in game-theoretic terms by sketching out a payoff matrix for a one-off communicative event. In it, he considers that senders may be truthful or untruthful, and receivers may be trusting or distrusting. According to Sperber, the sender's gain amounts to whether they have the 'desired' effect on the receiver; therefore, the sender gains from the interaction if the receiver is trusting (since this means the sender's message is accepted), and loses from the interaction if the receiver is distrusting. The payoff of this event for the sender is thus independent of the truthfulness of the sender. On the side of the receiver, their payoff \emph{is} dependent on the truthfulness of the sender: the receiver gains if they accept a truthful message, loses if they accept an untruthful message, and incurs no gain nor loss if they are distrusting and thus don't accept a message (truthful or not).

Sperber notes that the optimal strategy for such a game varies with the circumstances for both players: it is not always beneficial to be truthful, nor always untruthful; nor is it beneficial to be always trusting, nor always distrusting. In other words, there is no one stable solution to this game.
This is especially the case once we move away from this simple one-off communicative event to an iterated game of communication, where not only short-term payoffs but also long-term payoffs determine the optimal strategy.\todo{Should I discuss somewhere how reputation works in mass society?}
Therefore, it is in the receiver's interest to calibrate their trust towards senders as accurately as possible; in fact, Sperber argues, this trust calibration is necessary for explaining the stability of communication.

Unlike non-human animals, humans have another way to communicate facts, other than testimony, namely \emph{argumentation}. Senders may provide receivers with reasons to accept their testimony, which the receiver may evaluate and accept or reject, independent of their trust in the sender.
Sperber sketches out the steps in what he calls the 'evaluation-persuasion arms race', i.e.\@ the chain of evolutionary adaptations that has resulted in our mechanisms for argument production and evaluation.
He argues that the first step in this 'arms race' was for the receiver to develop \emph{coherence checking}. Coherence checking involves attending to both the internal coherence of the communicated message, and the external coherence with what the receiver already believes. Coherence checking, Sperber argues, is a useful defense against the risks of deception by the sender, because lies and other false claims are often externally or internally incoherent.
The second step in the arms race was then for the sender to anticipate this coherence-checking by overtly displaying the coherence of their message to their receiver, which requires argumentative form; thus, testimony becomes argument. The next steps were on the side of the receiver to develop skills for examining these displays of coherence (i.e., arguments), and on the side of the sender to 'improve their argumentative skills'.

\subsection{Sperber and colleagues on epistemic vigilance}
\label{sec:Sperber10}

\todo[inline]{Make this fit with the previous section once that's done}

One of the cornerstones of the argumentative theory of reasoning is the concept of \emph{epistemic vigilance}, which Sperber and colleagues (among whom Hugo Mercier) introduced in a seminal \citeyear{Sperber10} paper.

In the paper, Sperber and colleagues note that humans are dependent on communication, and they argue that this dependence leaves humans vulnerable to being deceived by others.
They state that misinformation or deception may "reduce, cancel, or even reverse" the gains that communication might bring to the addressee (p.~360).
Consequently, the information that addressees receive from communicators is only advantageous to them to the extent that the information is genuine.
Sperber and colleagues thus conclude that for this purpose, humans have evolved a suite of cognitive mechanisms for \emph{epistemic vigilance}.

A communicative act triggers not only comprehension in the addressee, but also epistemic vigilance alongside it.
In the grand scheme of human communication, vigilance maintains a balance between honesty and dishonesty; as Sperber and colleagues put it, "the audience's vigilance limits the range of situations where dishonesty might be in the communicators' best interest" (p.~368), which results in communication being honest most of the time.
Sperber and colleagues argue that vigilance is not a nicety, something that is only invoked sometimes; they maintain that vigilance is the default disposition of interlocutors in communicative settings.
\todo[inline]{Talk about the duality Sperber et al. also mention, of trust vs. vigilance. Move the blind trust quote from Ch. 4 to here}

One may distinguish between vigilance towards the \emph{source} of a message, and vigilance towards the \emph{content} of the message.
In regards to the source, the authors note that reliable sources must be both competent and benevolent (and this is dependent on the context).
Empirical evidence suggests that deceiving people can be quite beneficial, since experiments from deception detection research show that people are not good at detecting lies based on non-verbal behavioral cues.

\todo[inline]{Finish this: discuss \citet{Sperber10} in great detail for \cref{sec:EV-scrutiny} to make sense}


