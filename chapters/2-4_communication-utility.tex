\section{The utility of communication}
\label{sec:comm:function}

Finally and arguably most importantly for our endeavor, let us have a look at the utility, or function, of communication.

Essentially, communication facilitates interaction between individuals.
This interaction may be either cooperative or competitive in nature, as we have seen in \cref{sec:comm:phylogeny} when discussing non-human animal communication.
Whether the communicative event is cooperative or competitive in nature depends on the interests of the interlocutors. If the interests of interlocutors overlap or align, their communication can be considered to be cooperative.
For example, if two individuals engage in collaborative hunting of a large prey animal, their interests (catching the prey together and sharing it) align and they will thus use communication for cooperative purposes -- i.e., to coordinate their hunting activity.
On the other hand, if interlocutors' interests do not overlap, or they even oppose each other, communication can be considered to be competitive.
For example, if two individuals compete for a smaller prey animal, their interests (catching the prey by themselves and keeping it for themselves) oppose each other, and their communication would thus be competitive. This competitive communication could for example entail verbal intimidation, which may be evolutionarily more advantageous than physical intimidation (i.e. fight) because of a reduced risk of injury.

As argued by \citet{Tomasello08, Tomasello09}, the cooperative setting constitutes the 'birthplace' of the unique features of human communication, and the competitive use of human-style communication must have emerged later:
\begin{quoting}
    The use of skills of cooperative communication outside of collaborative activities (e.g., for lying), came only later.
    \signed{\citep[p.~325]{Tomasello08}}
\end{quoting}
In particular, it has been argued that language could only have emerged in cooperative settings \citep{Tomasello08, Dor17}.
This is because for the complexity of language to arise, more frequent and prolonged interactions are necessary \citep{Benitez21}. As \citet{Dor17} writes,
\begin{quoting}
    The collective effort of the invention and stabilization of the new technology [namely, language] must have been based on high levels of reliability and trust between the inventors: otherwise, indeed, they would not have been able to get the system going.
    \signed{\citep[p.~50]{Dor17}}
\end{quoting}
We return to a discussion of reliability, trust and this idea of 'getting the system going' in \cref{sec:S-P08}.
% Any further discussion out of scope for this thesis, because the emergence and evolution of symbolic communication in the form of language is an entire field of research of its own.
This discussion of cooperativeness naturally connects to pragmatics; more specifically, it will be illuminating to consider how the aforementioned cooperative function of communication relates to Grice's cooperative principle:
\begin{quoting}
    Make your conversational contribution such as is required, at the stage at which it occurs, by the accepted purpose or direction of the talk exchange in which you are engaged.
    \hfill \citep[p.~45]{Grice75}
\end{quoting}
It appears that cooperativeness in a Gricean sense is distinct from the cooperativeness we have been talking about so far.
Regarding this, \citet{Dor17} notes that 'honesty' as a concept might be interpreted in two ways. On the one hand, one can consider the honesty of a signal to be its truthfulness: an honest signal is a true signal, and a deceitful signal is a false signal. Alternatively however, one may conceptualize the honesty of a signal not as its truthfulness, but rather as its 'helpfulness'. In this interpretation, the honesty of a signal corresponds to the benefits and costs the sender and receiver incur as a result of the communicated signal. In the case of animals communicating, these two conceptions of honesty might very well coincide -- i.e., truthful signals benefit receivers, and false signals harm receivers. However, it should be apparent that they do not always coincide in the human case: truthful signals may hurt or cause harm to receivers, and false signals may benefit the receiver.

These two conceptions, or 'dimensions', of honesty then give rise to four possible communicative options one might choose: "co-operative honesty, harmful honesty, co-operative lying and harmful lying" \citep[p.~45]{Dor17}.
While anti-social, exploitative lies -- lies by which the communicator profits at the expense of the receiver, i.e.\@ harmful lies -- constitute the most intuitive, salient conception of a lie, they are not at all the most prevalent kind of lie. Moreover, \citet{Meibauer18} notes that prosocial lying is connected to politeness. The notion of politeness, in turn, can be connected with the notion of benefits and costs:
\begin{quoting}
    In antisocial (mendacious) lying, only the speaker profits from lying. In prosocial lying, lying is either altruistic (only the hearer profits from the speaker's lie) or polite ("Pareto-white," as Erat \& Gneezy 2012 call it) in the sense that both speaker and hearer profit from the lie.
    \signed{\citep[p.~371]{Meibauer18}}
\end{quoting}

Returning to Grice's cooperative principle, note that the above analysis only considers Grice's maxim of Quality ('Try to make your contribution one that is true'). If we extend the above account to the whole cooperative principle -- in other words, if we also incorporate the maxims of Quantity, Relation and Manner -- we end up with two 'dimensions' of cooperativeness that we might term \emph{Gricean cooperativeness} and \emph{dispositional cooperativeness}. One is then cooperative or noncooperative in a Gricean way depending on whether or not they adhere to Grice's cooperative principle. On the other hand, one is helpful or harmful in their disposition depending on whether their act of communication benefits the receiver or harms them.
As with \poscite{Dor17} account of honesty,
these two dimensions then combine to give us four communicative options; see \cref{tab:dimensions-of-cooperation} for an overview with examples for each option.

\begin{table}[ht]
  \centering
\begin{tabular}{lll|l}
  \cline{3-4}
  & & \multicolumn{2}{c}{\emph{Gricean cooperativeness}} \\ \cline{3-4}
  & & \textbf{cooperative} & \textbf{noncooperative} \\ \hline
  \emph{dispositional} &
  \textbf{helpful} & informative true statement & white lie \\ \cline{2-4}
  \emph{cooperativeness} &
  \textbf{harmful} & harsh true statement & malevolent lie \\ \hline
\end{tabular}
\caption{Examples for each of the four communicative options from the two dimensions of cooperation.}
\label{tab:dimensions-of-cooperation}
\end{table}

% For example, lying constitutes Gricean noncooperation, because it violates the maxim of Quality. As noted before, lying may be used harmfully (i.e. antisocial lying), or helpfully (i.e. prosocial lying, such as lying for politeness or white lies).

To fully appreciate the cooperative function of communication, let us now consider what makes cooperation itself evolutionarily beneficial. Moreover, in order to complete the causal chain, we will have a look at how cooperation could have evolved and the role that communication plays in it. We will do so by drawing extensively from Michael Tomasello's \emph{Why We Cooperate} (\citeyear{Tomasello09}).

\subsection{Human cooperation and its evolution}
\label{sec:comm:cooperation}

Let me start off with a brief terminological aside: although colloquially the terms 'cooperation' and 'collaboration' are more or less synonymous, Tomasello does not use these terms interchangeably. He defines collaboration as working together for mutual benefit (\citeyear{Tomasello09}, p.~xvii). Implicitly, he takes cooperation to be an overarching term which also encompasses for example altruism, in which one individual sacrifices something to help another individual. For the remainder of this thesis, I will adhere to these terminological conventions.

Tomasello argues that somewhere along the evolutionary timeline, humans must have been "put under some kind of selective pressure to collaborate in their gathering of food--they become obligate collaborators--in a way that their closest primate relatives were not" \citep[p.~75]{Tomasello09}.\footnote{Notably, what exactly this selective pressure is, is a missing link in this story. However, this is not of immediate concern to us. Although filling in this 'blank' is necessary for a full account of the evolution of human cooperation, for the current purposes it is only relevant that humans at some point \emph{did} become obligate collaborators -- which I take to be an uncontroversial assumption.}
In general, evolution may select for sociality in animals because living together in a social group protects the group's members against predation: it is easier to defend oneself in the context of a group. The group however also has its disadvantages when it comes to foraging for food, since group's members have to compete with each other for food. This is especially the case when the source of food is 'clumped', such as in a prey animal, rather than dispersed, such as in a plain of grass. The clumped source of food raises the issue of how to share the food amongst the members of the social group.
Tomasello enumerates a number of different hypotheses to explain how humans could have broken out of what he calls "the great-ape pattern of strong competition for food, low tolerance for food sharing, and no food offering at all" \citep[p.~83]{Tomasello09}; in other words, how humans could have evolved to be more tolerant and trusting, and less competitive about food.
Firstly, as due to a certain selective pressure it became necessary for humans to forage collaboratively, it would have been evolutionarily advantageous to be more tolerant and less competitive.
Secondly, Tomasello notes it could be the case that humans went through a process of self-domestication, by which aggressive, predatory or greedy individuals were eliminated from the group; see also \citet{Hare17, Benitez21}.
Thirdly, the evolution of tolerance and trust could be related to what is called \emph{cooperative breeding}, also known as \emph{alloparenting}. In cooperative breeding, the responsibility of child-rearing falls on more individuals than just the mother of the child; these individuals help by providing food for the child and engaging in other acts of childcare. This cooperative breeding may have selected for pro-social skills and motivations; see \citet{Hrdy09} for an elaboration of this argument.

Tolerance and trust then constitute a foundation upon which coordination and communication could be 'built', so to speak: they provide an environment in which more elaborate collaboration could evolve. In Tomasello's words,
\begin{quoting}
    there had to be some initial emergence of tolerance and trust (\ldots) to put a population of our ancestors in a position where selection for sophisticated collaborative skills was viable
    \signed{\citep[p.~77]{Tomasello09}}
\end{quoting}

In order to then arrive at the full picture of human cooperative activity, the final step to consider is that of social norms and institutions.
In this part of the story, there is also a missing link, concerning how mutual expectations between individuals arise and eventually become norms; Tomasello describes this as "one of the most fundamental questions in all of the social sciences" (\citeyear{Tomasello09}, p.~89).
Norms may be defined as "socially agreed-upon and mutually known expectations bearing social force, monitored and enforced by third parties" \citep[p.~87]{Tomasello09}. Norms receive their force not only from the threat of punishment by others if the norm is violated, but also from a kind of 'social rationality' in collaborative activity, where individuals recognize their dependence on each other in reaching their joint goal. Just as it would be individually irrational to act in a way that thwarts your own goal, it would be socially irrational to act in a way that thwarts your joint goal.

Let us now summarize the evolutionary timeline of human cooperation according to Michael Tomasello.
At some point in time, foraging for food collaboratively rather than individualistically became beneficial -- perhaps even necessary -- for humans.
As a result of this, some degree of tolerance and trust must have emerged between these collaborating individuals.
In the process of adapting to this collaborative foraging, humans evolved certain skills and motivations specifically for cooperation -- for example, abilities for establishing joint goals and role divisions for joint activity.
This kind of collaborative activity then constituted the breeding ground for human cooperative communication.
Joint goals and role divisions later evolved into the superindividual norms, rights and responsibilities that we see within our social institutions today.

As a brief aside: it has been argued that communication is not necessary nor sufficient for the coordination of activities. \citet{Goldstone24} propose a framework of five features characterizing the specialization of roles in group activities; communication is only one of these five features. This is corroborated by experiments they review in which people "spontaneously differentiate themselves into stable roles" \citep[p.~264]{Goldstone24} in group activity without communicating with each other.
However, the authors note that communication does play a very central role in coordinating group activities, stating that "direct communication of plans is often the single most potent tool of collective coordination" \citep[p.~276]{Goldstone24}.
See also \citet{Vorobeychik17} for a discussion of communication and coordination.

Now, armed with the ins and outs of human cooperation, and an inkling of how communication relates to it, we turn our attention to a crucial aspect in understanding the evolution of human communication: the problem of the stability of communication.
This problem in a way constitutes the starting point for the argumentative theory of reasoning, as we will see in \cref{sec:Sperber01}.

\subsection{The stability of communication}
\label{sec:S-P08}

If communication between individuals of a species persists throughout evolution, we may speak of it as stable. The stability of communication is considered by some to be the 'defining problem' of animal signaling research \citep{Scott-Phillips08}. It is not a trivial problem by any means: the stability of a communication system is threatened by evolutionary pressures on the communicator to 'defect', as it were. As \citet{Scott-Phillips08} describes it,
\begin{quoting}
    If one can gain through the use of an unreliable signal then we should expect natural selection to favour such behaviour. Consequently, signals will cease to be of value, since receivers have no guarantee of their reliability. This will, in turn, produce listeners who do not attend to signals, and the system will thus collapse in an evolutionary retelling of Aesop’s fable of the boy who cried wolf.
    \signed{\citep[p.~275]{Scott-Phillips08}}
\end{quoting}
In the context of human communication, this problem may be interpreted as follows: if it can be advantageous for me to deceive you, then it would evolutionarily speaking make sense for me to do so; yet, it would then evolutionary speaking make sense for you to stop listening to me, and as a consequence our system of communication would collapse.

There have been a number of attempts at explaining the stability of animal communication in general. One influential attempt is the \emph{handicap principle} \citep{Zahavi75, Zahavi99}, which might be best understood through the paradigmatic example of the peacock's tail.
This tail is like a handicap for the peacock: not only does it take a lot of resources to grow the tail and carry it around, it also leaves the bird more vulnerable to predation because it is less agile with a large unwieldy tail. At the same time, a large tail signals to peahens that the peacock is fit enough to be able to incur these costs, and thus the peacock has a sexual advantage.
The handicap principle then describes this process of communication, by which the signaler incurs costs (i.e., a handicap) for signaling, which thus guarantees the reliability of the signal.

While influential, the handicap principle has been argued to fall short of explaining the stability of communication \citep{Penn20, Scott-Phillips08}.
One reason for this is that reliable signals are not always costly to produce. For example, Harris sparrows signal their social status to other sparrows by the amount of black feathering in their plumage, and this signal is not costly to produce \citep[p.~13191]{Lachmann01}.
Moreover, human linguistic communication is considered to be generally\footnote{See \citet[p.~13193]{Lachmann01} for a discussion of when and why humans may use costly forms of communication.} a low-cost endeavor: "Although the capacity for speech may itself be costly, once language has been acquired the production costs vary little among alternative signals" \citep[p.~13192]{Lachmann01}.
Thus, it remains to be shown how communication can be stable if signals are cheap.

An alternative explanation of the reliability of animal communication is the principle of \emph{deterrence}, where \emph{un}reliable signals are costly to produce, and consequently signalers are deterred from producing unreliable signals \citep{Scott-Phillips08}.
There are a number of ways in which producing unreliable signals may be costly to the signaler. Firstly, this is the case in a coordination game, where the signaler and receiver share some common interest with regard to the outcome of the interaction \citep[see][]{Smith94}.
Secondly, if two individuals have repeated interactions, it may also be costly in the long term to produce unreliable signals, because it may hinder cooperation in the future.
Thirdly, producing unreliable signals may be costly to the signaler if false signals are punished by the receiver. In the case of the Harris sparrow for example, subordinate sparrows that are dyed black (and are thus signalling falsely) are socially persecuted by other sparrows \citep{Rohwer78}.

This 'logic of deterrents' can be applied to the case of human communication to account for its stability, if the following demands are met:
\begin{quoting}
    Sufficient conditions for cost-free signalling in which reliability is ensured through deterrents are that signals be verified with relative ease (if they are not verifiable then individuals will not know who is and who is not worthy of future attention) and that costs be incurred when unreliable signalling is revealed.
    \hfill \citep[p.~279]{Scott-Phillips08}
\end{quoting}
In other words, if unreliable signals are 'caught' relatively easily, and unreliable signalers incur costs for their unreliability, then the reliability of communication is secured through deterrents.
\citet{Scott-Phillips08} argues that these sufficient conditions are met in the case of human communication, since people may refrain from interacting with unreliable individuals in the future, which can be very costly for a social species such as humans.
Notably however, he does not explicate how exactly the first sufficient condition is met in the case of human communication. We will return to this in \cref{sec:EV-scrutiny}, as it turns out that the ease with which unreliable signals are caught, is a contentious point of the argumentative theory of reasoning.

The problem of the stability of communication hinges on the assumption that it is advantageous to deceive others through the use of unreliable signals. Let us explore this further by turning our discussion to deception and lying.

\subsection{Deception and lying}
\label{sec:deception}

Let us start off this section by clarifying and examining some terminology.
In our discussion of the stability of communication, we have so far been talking about 'reliability' rather than 'honesty'. This is because 'reliability' is a more neutral term than 'honesty', in the sense that it refrains from ascribing intentions and meanings to individuals; and especially in the case of animal communication, we should steer clear of ascribing intentions and meanings to individuals \citep{Scott-Phillips08}.
This ascription of intentions and meanings is not problematic when discussing human communication however; moreover, I find that talking about 'reliable communicators' blurs an important distinction between honest, benevolent, and competent communicators. We return to this distinction in \cref{sec:Sperber10}.

Deception and lying and may be defined in a number of different ways.
First off, deception may be defined as "deliberately leading someone into a false belief" \citep[p.~358]{Meibauer18}.
Lying, on the other hand, might not be as easily defined. Consider the following 'standard' definition of lying \citep{Meibauer18}, due to \citet{Williams02}:
\begin{quoting}
    an assertion, the content of which the speaker believes to be false, which is made with the intention to deceive the hearer with respect to that content
    \signed{\citep[p.~96]{Williams02}}
\end{quoting}
There is, however, a case to be made for excluding 'intention to deceive' from this definition. In the case of bald-faced lying, where the liar knows he will not be believed, or in cases where people are forced to lie (in totalitarian states, for example), lying occurs without intention to deceive \citep[\S 1.5]{Saul12}.
For the purposes of this thesis and the general nature of its discussion however, the exclusion of these edge cases is not problematic. Thus, we will proceed using \poscite{Williams02} definition and assume that lying is by definition done with deceitful intent.

Let us now connect lying with the problem of the stability of communication. In this narrower context, let us refer to this problem more concisely as the 'paradox of honest signaling' \citep[following][]{Dor17}.
\citet{Dor17} argues that the foretold collapse of communication due to unreliability of the speaker does not hinge on whether or not the speaker is truthful, but whether her \emph{intention} is benevolent.
In other words, it matters to the story not that receivers evaluate the truthfulness of incoming information, but rather the \emph{intention} of the sender. Listeners care whether speakers intend to be harmful, not whether or not they are truthful, Dor argues.
Moreover, he notes that the paradox of honest signaling mostly applies to situations in which interests between interlocutors conflict; however, these situations might not be the most pertinent or prevalent kind of communicative situation.
He argues that, at the point in evolutionary time when language emerged, humans were already crucially dependent on cooperation and coordinated action, and thus their interests overlapped more often than not.

There is thus an interesting interaction to observe between the paradox of honest signaling and the utility of communication. As we will see in \cref{ch:atr}, the argumentative reasoning mostly focuses on communication as the transmission of information between individuals, in the form of testimony and argumentation. However, within the paradox of honest signaling this transmission of information is not the only relevant use of communication. Communication is also used for cooperation, and in that situation lying is not really an issue, as argued by \citet{Dor17}:
\begin{quoting}
    Language is extremely useful in the coordination of collective work, collective defense and so on, where it is used not just for the exchange of information but also for collective planning, division of labor, ordering and requesting, where lying as such does not seem to play a major role.
    \signed{\citep[p.~51]{Dor17}}
\end{quoting}
Therefore, due to this dual role of communication (transmitting information on the one hand, and facilitating cooperation on the other), the stability of communication is not threatened by lying:
\begin{quoting}
    Even in the very unlikely doomsday scenario, then, where all the members of a community lie to each other in their factual statements, and eventually refrain from sharing information with each other, there is no reason to assume that they would stop using language for all these other purposes, especially where their survival, whether they like it or not, depends on collective action.
    \signed{\citep[p.~52]{Dor17}}
\end{quoting}
We return to these considerations in \cref{ch:scrutiny}.

Lastly, we will have a brief look at persuasion, since it naturally plays a considerable role in Mercier \& Sperber's argumentative theory of reasoning.
\citet{Brinol09} broadly define persuasion as "any procedure with the potential to change someone's mind" (p.~50),
whether that be changing someone's emotional state, beliefs, behaviors or attitudes.
They describe persuasion as "the most frequent and ultimately efficient approach to social influence" (pp.~49--50). Put crudely, persuasion is a tool for getting what you want, and it serves this end better than the alternatives of using force, threats or violence.\footnote{In this, one may see a parallel with \poscite{SeyfarthCheney03} conception of aggressive communication (intimidation) as a low-risk alternative to fighting.}
From this observation, the conclusion emerges that persuading someone is beneficial to an individual exactly to the extent that the corresponding gain in social influence is beneficial to the individual.
% For limitations of time and space we will not go into the benefits of gaining social influence.
