\chapter*{Introduction}
\label{sec:introduction}
\largeTodo{Don't overuse passive voice. Don't make sentences too long (rule of thumb: not longer than two lines). Don't abuse semicolons. Don't have too long NPs before the VP comes. Watch out for Dutch word order. Explain technical terms always, to fix their meaning. Give examples (and keep them as familiar as possible). Don't undersell your points, be confident! Don't vary terminology for the sake of variation.}

% Why are reasoning and communication special?
\smallTodo{This paragraph is saying the same thing twice, and word order is funky}
Two cognitive skills that are often considered
to set humans apart from their evolutionarily closest relatives are on the one hand our outstanding capacity for reasoning, and on the other our profound communicative abilities.
% in what way is our reasoning superior?
Broadly considered to be unmatched in the animal kingdom \citep{CheneySeyfarth98} are on the one hand
our sophisticated reasoning abilities\largeTodo{Explicate this; cite a source}
and
% in what way is our communication superior?
on the other hand
our communication using languages that are infinitely creative in enabling the production of complex sentences.

% Why are reasoning and communication intertwined?
Our reasoning and communication are intertwined with each other in different ways; it is hard to imagine our communication without reasoning. In our everyday lives, a lot of the content we intend to convey to others, we relay pragmatically:
% \Todo{Source?}
we do not literally spell out these things, but rather hope and expect our interlocutors to infer the intended message from the communicated content.
% example
When I ask my dinner partner if they can pass me the salt, they infer that I am not interested in learning about their ability to pass me the salt but rather that I am requesting to be passed the salt.
% another example
When I give feedback on an interlocutor's behavior, I first reason about how my words will come across to her in order to minimize social conflict.

% Why might reasoning and communication be evolutionarily causally related?
It is thus easy to see that reasoning and communication are intricately linked. But what exactly is the extent and nature of this link?
% explain about MS11
In \citeyear{MS11}, Hugo Mercier and Dan Sperber proposed a revolutionary
% \Todo{Can I say that?}
theory of reasoning that intended to account for a number of long-standing issues in the experimental psychology of reasoning.
According to their \emph{argumentative theory of reasoning}, the main function of reasoning in humans is argumentative; that is, reasoning evolved in humans in order to devise arguments and evaluate those of others. Their theory is able to explain a number of purported 'flaws' of human reasoning, such as poor performance on standard reasoning tasks such as the Wason selection task; confirmation bias; and the phenomenon of motivated reasoning leading to attitude polarization.

In the words of Mercier and Sperber,
% argument your way to the research question
\begin{quote}
    Reasoning has evolved and persisted mainly because it makes human communication more effective and advantageous.
    \citep[p.~60]{MS11}
\end{quote}
\largeTodo{Add a few words about why this thesis is worth scrutinizing: for example, that others also disagree (see the MS11 commentary), or already hint at your own objections}
\largeTodo{Add a few words on that the "why" of communication is an important question, explain why this is needed to ultimately answer the RQ. It's more primitive, or primary; address this}

% The research question
In this thesis, I will scrutinize this position in order to ultimately
answer the question of whether advanced reasoning skills in humans evolved because they facilitate more advanced communication.
\largeTodo{Could do with some more explication: how will I scrutinize this position?}

\thinkL{To explore in the introduction: generic question of why an evolutionary approach is worhtwhile. Motivate why you're interested in Mercier \& Sperber}
