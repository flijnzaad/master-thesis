\chapter*{Introduction}
\label{sec:introduction}

% Why are reasoning and communication special?

Two cognitive skills that are often considered\todo{source?} to set humans apart from their evolutionarily closest relatives are on the one hand our outstanding capacity for reasoning, and on the other our profound communicative abilities.
% in what way is our reasoning superior?
Broadly considered to be unmatched in the animal kingdom are on the one hand
our sophisticated reasoning abilities\todo{explicate this; source?}
and
% in what way is our communication superior?
on the other hand
our communication using languages that are infinitely creative in enabling the production of complex sentences \citep{CheneySeyfarth98}.

% Why are reasoning and communication intertwined?
Our reasoning and communication are intertwined with each other in different ways; it is hard to imagine our communication without reasoning. In our everyday lives, a lot of the content we intend to convey to others, we relay pragmatically\todo{source?}: we do not literally spell out these things, but rather hope and expect our interlocutors to infer the intended message from the communicated content.
% example
When I ask my dinner partner if they can pass me the salt, they infer that I am not interested in learning about their ability to pass me the salt but rather that I am requesting to be passed the salt.
% another example
When I give feedback on an interlocutor's behavior, I first reason about how my words will come across to her in order to minimize social conflict.

% Why might reasoning and communication be evolutionarily causally related?
It is thus easy to see that reasoning and communication are intricately linked. But what is exactly the extent and nature of this link?
% explain about MS11
In \citeyear{MS11}, Hugo Mercier and Dan Sperber proposed a revolutionary\todo{can I say that?} theory of reasoning that intended to account for a number of issues in the experimental psychology of reasoning.
According to their \emph{argumentative theory of reasoning}, the main function of reasoning in humans is argumentative; that is, reasoning evolved in humans in order to devise arguments and evaluate those of others. Their theory is able to explain a number of properties of human reasoning, such as poor performance on historically standard reasoning tasks such as the Wason selection task;
confirmation bias; and the phenomenon of motivated reasoning leading to attitude polarization.\todo{add sources?}

In the words of Mercier and Sperber,
% argument your way to the research question
\begin{quote}
    Reasoning has evolved and persisted mainly because it makes human communication more effective and advantageous.
    \citep[p.~60]{MS11}
\end{quote}

% The research question

In this thesis, I intend to scrutinize this position and take it further, in order to ultimately
answer the question of whether advanced reasoning skills in humans evolved because they facilitate more advanced communication.

In order to answer this research question, after addressing methodological considerations on functions and explanations in evolutionary biology, I will at length consider the argumentative theory of Hugo Mercier and Dan Sperber \citep{MS11, Mercier16, MS17}. Then, an exploration of the origins of human communication is in order \citep{Tomasello08-origins, Moore17, Scott-Phillips17-pragmatics, Scott-Phillips18-communication, Benitez21}.
