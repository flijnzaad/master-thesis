\chapter*{Introduction}
\label{ch:introduction}
\addcontentsline{toc}{chapter}{\nameref{ch:introduction}}

% Introduction & motivation

Communication is one of the most fundamental parts of the human experience: it's hard, nigh impossible, to imagine life without it. Compared to our closest evolutionary relatives, our communicative abilities are very sophisticated; human language is incomparable to any non-human animal's system of communication \citep{CheneySeyfarth98}.
One large difference between human and non-human animal communication is our ability to communicate more than just the truth-conditional content of our sentences. More often than not, the communicated content of the sentences we utter extends way beyond the words we speak. For example, take the following exchange \citep[p.~102]{Levinson83}:
\begin{quoting}
    A: Where's Bill? \\
    B: There's a yellow VW outside Sue's house.
\end{quoting}
Taking the utterances at face value, this exchange is quite nonsensical: the utterances seem to have nothing to do with each other. However, this is a completely normal interaction in real life, given our skills for inferring that this yellow VW might belong to Bill, pointing at his current location being Sue's house.

It should be obvious that our abilities for inference play a large role in our communication. This motivates one to ponder on the nature and history of the relationship between our profound communicative abilities and our outstanding inferential skills. Is there some evolutionary reason why we communicate and infer things in the way we do?

% Hugo Mercier and Dan Sperber have proposed a theory of the relationship between reasoning and communication, that intends to demonstrate the evolution of both of these cognitive capacities.

As a small aside: much of the pragmatics literature uses the terms reasoning and inference interchangeably (see for example Stephen Levinson's usage of the term 'reasoning' in \citet[p.~218]{Levinson83}). However, Mercier and Sperber do sharply distinguish between reasoning and inference (see the very first paragraph of \citet{MS11}).
In any case, it should be clear that reasoning and inference are closely related cognitive abilities, and that their (evolutionary) relationship to communication is one worth investigating.

% The ATR

In \citeyear{MS11}, Hugo Mercier and Dan Sperber proposed a revolutionary
theory of reasoning, in response to a puzzling problem on the function of reasoning.
It has been a longstanding view in philosophy that the function of reasoning is to enhance an individual's knowledge. However, experimental findings from the psychology of reasoning apparently show again and again that humans are not good at reasoning. This raises the question of how this can be the case, evolutionarily speaking; why would reasoning have evolved to serve its function poorly?

Mercier and Sperber argue that reasoning actually serves its function well.
According to their \emph{argumentative theory of reasoning} (ATR), the main function of reasoning in humans is argumentative; that is, reasoning evolved in humans in order to devise arguments and evaluate those of others. In this way, reasoning serves to further communication, the stability of which is threatened by dishonest communicators. Speakers can argue for their case, and addressees can critically evaluate these arguments, which allows us to move beyond accepting others' testimony merely on trust.

Mercier and Sperber buttress the ATR by formulating testable hypotheses that are entailed by it, and they gather empirical evidence to corroborate these hypotheses. They argue for example that the phenomenon of confirmation bias, usually construed as a flaw of reasoning, makes perfect sense when considering reasoning's argumentative function: if you want to convince your audience, it is perfectly reasonable to only seek out evidence that confirms your opinion.

To summarize, in the words of Mercier and Sperber:
\begin{quote}
    Reasoning has evolved and persisted mainly because it makes human communication more effective and advantageous.
    \citep[p.~60]{MS11}
\end{quote}

This view, while thought-provoking and intuitively attractive, is certainly not uncontroversial. Mercier and Sperber's theory has been criticized by philosophers and cognitive scientists alike. Critiques range from objections to their characterization of reasoning \citep{Koren23} or their evolutionary framework underpinning the theory \citep{Novaes18} to the cognitive-scientific framework underpinning their theory \citep{Sterelny18, Chater18}.

% This thesis

This brings us to our current research endeavor.
The purpose of this thesis is to critically analyze Mercier and Sperber's argumentative theory of reasoning by evaluating its (implicit) assumptions and scrutinizing its details against the background of the evolution of communication.

In order to do so, I will first provide some needed context from evolutionary theory. This context will then inform a comprehensive analysis of human communication from an evolutionary perspective, and in particular the function of communication. These findings will constitute the foundations for my criticism of the ATR.
After dissecting the ATR in detail, we are then able to assess the plausibility of its component parts and come to a conclusion on the theory's tenability.

The consideration and combination of different disciplines, from epistemology to psychology of reasoning and from the philosophy of biology to evolutionary anthropology,
situates this thesis as an endeavor in \emph{synthetic philosophy}, as described by Eric Schliesser:
\begin{quoting}
    {[synthetic philosophy is]} a style of philosophy that brings together insights, knowledge, and arguments from the special sciences with the aim to offer a coherent account of complex systems and connect these to a wider culture or other philosophical projects (or both).
    \hfill \citep[pp.~1--2]{Schliesser19}
\end{quoting}

This thesis will provide constructive criticism for Mercier and Sperber's argumentative theory of reasoning. It is, however, way beyond the scope of this thesis to propose an alternative theory to the ATR; I will highlight problems with the theory and propose possible ways to fix these problems, but I conjecture that much additional philosophical and/or empirical work is needed to 'fix' the argumentative theory of reasoning.

% Chapters summary

The structure of this work is as follows.
\cref{ch:evolution} provides some necessary background on evolution. In it, I will discuss evolutionary causation, provide context about evolutionary psychology, and discuss terminological issues associated with function in evolution. This chapter concludes by laying out a methodology for the next chapter.

\cref{ch:communication} considers human communication from an evolutionary perspective, following the methodological steps outlined in the previous chapter. In particular, this chapter will see us discussing at length the function of communication, which will provide us with some important anchor points for scrutinizing the ATR.

Then in \cref{ch:atr}, I will expound the argumentative theory of reasoning by discussing a number of foundational papers for the theory.
All of this culminates in \cref{ch:scrutiny} with the critical analysis of the ATR, combining the findings from \cref{ch:communication} on the evolution of communication with the details and implicit assumptions that became apparent from \cref{ch:atr}. Moreover, I critically examine a key concept of the ATR, \emph{epistemic vigilance}, by discussing a critical response to Sperber's work and his reply to it.

Finally, \cref{ch:conclusion} recounts the whole story and highlights avenues to continue this research.
