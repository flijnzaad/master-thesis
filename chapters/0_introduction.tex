\chapter*{Introduction}
\label{ch:introduction}
\addcontentsline{toc}{chapter}{\nameref{ch:introduction}}

% Why are reasoning and communication special?
\smallTodo{This paragraph is saying the same thing twice, and word order is funky}

\section{Introduction \& motivation}

Communication is one of the most fundamental parts of the human experience: it's hard, nigh impossible, to imagine life without it. Compared to our closest evolutionary relatives, our communicative abilities are very sophisticated; almost incomparable. One reason for this is our ability to communicate more than just the truth-conditional content of our sentences: more often than not, the communicated content of the sentences we utter extends way beyond the words we speak. For example,
\todo[inline]{insert example}
It is now obvious that our abilities for inference play a large role in the way we communicate. This was noted first by Paul Grice, giving birth to the field of pragmatics. And later by the neo-Gricean relevance theory, composed by Dan Sperber and Deirdre Wilson.

This motivates one to ponder on the relationship between our profound communicative abilities and our outstanding inferential skills. Is there some evolutionary reason why we communicate and infer things in the way we do?
Hugo Mercier and Dan Sperber have proposed a theory of the relationship between reasoning and communication, that intends to demonstrate the evolution of both of these cognitive capacities.
As a small aside: much of the pragmatics literature uses the terms reaosning and inference interchangeably (cf. Levinson ...). Mercier and Sperber however do sharply distinguish between reasoning and inference (cf. MS11). In any case, it should be clear that reasoning and inference are closely related cognitive abilities.

Mercier and Sperber's argumentative theory of reasoning was borne out of decades of philosophical work [cite], and intends to reply to the research from the field of psychology of reasoning on why humans are bad reasoners.
The argumentative theory of reasoning states that, contrary to this paradigmatic opinion, reasoning actually serves its function well. The ATR argues that the (biological) function of reasoning is to devise arguments and evaluate the arguments of others. This serves to further communication, the stability of which is threatened by dishonest communicators. By being able to argue for their case, and subsequently being able to critically evaluate the arguments of others, people are able to move beyond accepting others' testimony merely on trust.
Mercier and Sperber buttress this theory empirically by formulating testable hypotheses that are entailed by their theory, and gather empirical evidence to corroborate these hypotheses. As an example, they argue that the phenomenon of confirmation bias, usually construed as a flaw of reasoning, makes perfect sense when considering reasoning's argumentative function: if you want to convince your audience, it is perfectly reasonable to only seek out evidence that confirms your opinion.

This theory, while thought-provoking and intuitively attractive, is certainly a controversial one. [cite the works]

This brings us to our current research endeavor.
The purpose of this thesis is to critically analyze the argumentative theory of reasoning by evaluating its (implicit) assumptions and scrutinizing its details.
\todo[inline]{This could be more explicit still}

In order to do so, I will first provide some needed context from evolutionary theory. This context will then inform a comprehensive analysis of human communication from an evolutionary perspective, and in particular the function of communication. These findings will then constitute the foundations for my criticisms at the ATR.

This consideration and combination of different disciplines [name them] situates this thesis as an endeavor in \emph{synthetic philosophy}, as coined by Schliesser:
\begin{quoting}
    a style of philosophy that brings together insights, knowledge, and arguments from the special sciences with the aim to offer a coherent account of complex systems and connect these to a wider culture or other philosophical projects (or both) 
\end{quoting}

This thesis will provide constructive criticism for Mercier and Sperber's argumentative theory of reasoning. It is however way beyond the scope of this thesis to propose an alternative theory; I will highlight problems with the theory and enumerate points of interest for fixing these problems, but I conjecture that much additional philosophical and/or empirical work is needed to 'fix' the argumentative theory of reasoning.

The structure of this work is as follows.
In \cref{ch:evolution}, I will provide some necessary background on evolution. I will discuss evolutionary causation, provide context about evolutionary psychology, discuss how apt the terminology of 'function' is when talking about evolution. This chapter concludes by laying out a methodology for the next chapter.

\cref{ch:communication} then considers human communication from an evolutionary perspective, following the methodological steps outlined in the previous chapter. In particular, this chapter will see us discussing at length the function of communication, which will provide us with important context for scrutinizing the ATR.

Then in \cref{ch:atr}, I will expound the argumentative theory of reasoning by discussing a number of foundational papers for the theory.
All of this to culminate in \cref{ch:scrutiny} with the critical analysis of their theory, combining the findings from \cref{ch:communication} on the evolution of communication with the details and implicit assumptions that became apparent from \cref{ch:atr}.

\section{Old stuff}

% Why are reasoning and communication intertwined?
Our reasoning and communication are intertwined with each other in different ways; it is hard to imagine our communication without reasoning. In our everyday lives, a lot of the content we intend to convey to others, we relay pragmatically:
% \Todo{Source?}
we do not literally spell out these things, but rather hope and expect our interlocutors to infer the intended message from the communicated content.
% example
When I ask my dinner partner if they can pass me the salt, they infer that I am not interested in learning about their ability to pass me the salt but rather that I am requesting to be passed the salt.
% another example
When I give feedback on an interlocutor's behavior, I first reason about how my words will come across to her in order to minimize social conflict.

\section{The ATR}

% Why might reasoning and communication be evolutionarily causally related?
It is thus easy to see that reasoning and communication are intricately linked. But what exactly is the extent and nature of this link?
% explain about MS11
In \citeyear{MS11}, Hugo Mercier and Dan Sperber proposed a revolutionary
% \Todo{Can I say that?}
theory of reasoning that intended to account for a number of long-standing issues in the experimental psychology of reasoning.
According to their \emph{argumentative theory of reasoning}, the main function of reasoning in humans is argumentative; that is, reasoning evolved in humans in order to devise arguments and evaluate those of others. Their theory is able to explain a number of purported 'flaws' of human reasoning, such as poor performance on standard reasoning tasks such as the Wason selection task; confirmation bias; and the phenomenon of motivated reasoning leading to attitude polarization.

In the words of Mercier and Sperber,
% argument your way to the research question
\begin{quote}
    Reasoning has evolved and persisted mainly because it makes human communication more effective and advantageous.
    \citep[p.~60]{MS11}
\end{quote}
\largeTodo{Add a few words on that the "why" of communication is an important question, explain why this is needed to ultimately answer the RQ. It's more primitive, or primary; address this}

\section{This thesis}

% The research question
In this thesis, I will scrutinize this position in order to ultimately
answer the question of whether advanced reasoning skills in humans evolved because they facilitate more advanced communication.
\largeTodo{Could do with some more explication: how will I scrutinize this position?}

\thinkL{To explore in the introduction: generic question of why an evolutionary approach is worhtwhile. Motivate why you're interested in Mercier \& Sperber}

\section{Chapters summary}
