\chapter{Why do we reason?}
\label{ch:reasoning}

Reason is considered by many to be that which sets human animals apart from their non-human fellow animals. In this chapter, we will consider reasoning from the different angles laid out in \cref{sec:evo-conclusion}, in particular in relation to the theory under scrutiny, Mercier \& Sperber's argumentative theory of reasoning. First, we consider their definition of reasoning and compare it with some other definitions of reasoning. Then, we look at reasoning in developing children and in non-human animals. Finally, we consider the utility of reasoning, and expound the argumentative theory of reasoning.

\section{What is reasoning?}

Before anything else can be said about reasoning, it is critical to get clear what we mean exactly when we talk about reasoning. \citet{MS11} take the following as their definition of reasoning:

\begin{quoting}
    \emph{Reasoning}, as commonly understood, refers to a very special form of inference at the conceptual level, where not only is a new mental representation (or \emph{conclusion}) consciously produced, but the previously held representations (or \emph{premises}) that warrant it are also consciously entertained.
    \hfill (p.~57)
\end{quoting}
In other words, what distinguishes reasoning proper \todo{terminology?} from inference, is the conscious attending to the representations.
In this definition, Mercier \& Sperber take \emph{inference} to be "the production of new mental representations on the basis of previously held representations" (ibid.).

Mercier \& Sperber note that their definition of reasoning is the one that is most commonly adhered to in the psychology of reasoning.
Moreover, they go on to state that this definition of reasoning excludes non-human animals and preverbal children from the realm of reasoners.
We will get back to the role of the position of this definition within their theory in \cref{sec:def-scrutiny}.

For completeness' sake, let us mention and compare two other definitions of reasoning, one due to cognitive scientist Vinod Goel and another due to philosopher Gilbert Harman.

In his \citeyear{Goel22} book "Reason and Less", Goel proposes and develops an account of \emph{tethered rationality}, where four different systems work in tandem to determine human behavior. The four systems, in order of evolutionary 'age', are the autonomic, instinctual, associative, and reasoning systems. These systems evolved 'on top of' each other, and are tightly integrated with one another --- tethered, as Goel calls them.
Goel then describes (and in doing so more or less defines) reasoning as follows:
\begin{quoting}
    Reasoning is a system for generating new beliefs from observations and/or existing beliefs and maintaining consistency of our beliefs (i.e., mental representations of the world).
    \hfill (p.~114)
\end{quoting}
In other words, according to Goel reasoning has a dual role: it generates inferences (in the way Mercier \& Sperber define them), and it ensures that our beliefs remain internally consistent or coherent.
\todo{Make note of coherence?}

\todo[inline]{Discussion of Harman's reasoned change in view}

\section{Reasoning in children}

\section{Reasoning in non-human animals}

\section{The utility of reasoning}

Let us now look at the utility of reasoning, and in particular, dive head-first into the argumentative theory of reasoning.
We briefly consider classical accounts of the utility of reasoning, and then we will consider Mercier \& Sperber's account at length.

\subsection{Classical theories of reasoning}

Classical theories of reasoning maintain that the function of reasoning is to enhance or further individual cognition.
\todo[inline]{Detail them, quote some here}

\citet{Goel22} also more or less adheres to this classical account of what the function of reasoning is, stating that
\begin{quoting}
    the function of the reasoning mind is to allow for greater flexibility in individual behavior, and thus more finely tuned responses to environmental stimuli than can be accommodated by the autonomic, instinctive, and associative minds.
    \hfill (p.~114)
\end{quoting}

\subsection{Mercier \& Sperber's theory of reasoning}

\todo[inline]{The tone and content of this section may change slightly on what has already been said in the introduction.}

The argumentative theory of reasoning is Hugo Mercier and Dan Sperber's influential, but not uncontroversial, account of the function of reasoning from an evolutionary perspective. They introduced and coined the theory in a \citeyear{MS11} paper, a culmination of more than a decade worth of experimental and philosophical research \todo{Cite those works here}.
Briefly, the argumentative theory of reasoning states that the main function of reasoning is to produce arguments and evaluate arguments of others, in order to stabilize communication.
Before we can tackle the theory in detail, it is useful to look at one foundational concept in isolation first.

\subsubsection{Testimony, argumentation, and stability of communication}

\todo[inline]{Discussion of \citet{Sperber01}}

\subsubsection{Epistemic vigilance}
\todo[inline]{This section I think belongs in the communication chapter, but I've put it here for now to make it clearer that it's new. I'll move it to Chapter 2 once I start revising that chapter (??)}

\todo[inline]{Is this true? Reconsider the relative importance of epistemic vigilance to the ATR}
One of the cornerstones of the argumentative theory of reasoning is the concept of \emph{epistemic vigilance}, which Sperber and colleagues (among whom Hugo Mercier) introduced in a seminal \citeyear{Sperber10} paper.

In the paper, Sperber and colleagues note that humans are dependent on communication, and they argue that this dependence leaves humans vulnerable to being deceived by others.
They state that misinformation or deception may "reduce, cancel, or even reverse" the gains that communication might bring to the addressee (p.~360).
Consequently, the information that addressees receive from others is only advantageous to them to the extent that the information is genuine.
Sperber and colleagues thus conclude that for this purpose, humans have evolved a suite of cognitive mechanisms for \emph{epistemic vigilance}.

A communicative act triggers not only comprehension in the addressee, but also epistemic vigilance alongside it.
In the grand scheme of human communication, vigilance in a way maintains a balance between honesty and dishonesty; as Sperber and colleagues put it, "the audience's vigilance limits the range of situations where dishonesty might be in the communicators' best interest" (p.~368), which results in communication being honest most of the time.
Sperber and colleagues argue that vigilance is not a nicety, something that is only invoked sometimes; they maintain that vigilance is the default disposition.

One may distinguish between vigilance towards the \emph{source} of a message, and vigilance towards the \emph{content} of the message.
In regards to the source, the authors note that reliable sources must be both competent and benevolent (and this is dependent on the context).
Empirical evidence suggests that trust is important to us, and that lying to or deceiving people can be quite beneficial, since experiments from deception detection research show that people are not good at detecting lies based on non-verbal behavioral cues.
\todo[inline]{Add some stuff here as well}

\subsubsection{The argumentative theory of reasoning}

\subsection{The utility of argumentation}
\todo[inline]{Why did I put this heading here?}
