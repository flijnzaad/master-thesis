\chapter{Why do we reason?}
\label{ch:reasoning}

% \section*{Introduction}

\section{Reasoning in non-human animals}

\section{Reasoning in children}

\section{The utility of reasoning}

Let us now look at the utility of reasoning, and in particular, dive head-first into the argumentative theory of reasoning.
We briefly consider classical accounts of the utility of reasoning, and then we will consider Mercier \& Sperber's account at length.

\subsection{Classical theories of reasoning}

\subsection{Mercier \& Sperber's theory of reasoning}

\todo[inline]{The tone and content of this section may change slightly on what has already been said in the introduction.}

The argumentative theory of reasoning is Hugo Mercier and Dan Sperber's influential, but not uncontroversial, account of the function of reasoning from an evolutionary perspective. They introduced and coined the theory in a \citeyear{MS11} paper, a culmination of more than a decade worth of experimental and philosophical research \todo{Cite those works here}.
Briefly, the argumentative theory of reasoning states that the main function of reasoning is to produce arguments and evaluate arguments of others, in order to stabilize communication.
Before we can tackle the theory in detail, it is useful to look at one foundational concept in isolation first.

\subsubsection{Epistemic vigilance}
\todo[inline]{This section I think belongs in the communication chapter, but I've put it here for now to make it clearer that it's new. I'll move it to Chapter 2 once I start revising that chapter (??)}

One of the cornerstones of the argumentative theory of reasoning is the concept of \emph{epistemic vigilance}, which Sperber and colleagues (among whom Hugo Mercier) introduced in a seminal \citeyear{Sperber10} paper.

In the paper, Sperber and colleagues note that humans are dependent on communication, and they argue that this dependence leaves humans vulnerable to being deceived by others.
They state that misinformation or deception may "reduce, cancel, or even reverse" the possible gains of communication (p.~360) .
Consequently, the information that addressees receive from others is only advantageous to them to the extent that the information is genuine.
Sperber and colleagues thus conclude that for this purpose, humans have evolved a suite of cognitive mechanisms for \emph{epistemic vigilance}.

\subsubsection{The argumentative theory of reasoning}

\subsection{The utility of argumentation}
\todo[inline]{Why did I put this heading here?}
