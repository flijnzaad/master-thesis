\chapter{Why do we reason?}
\label{ch:reasoning}

Reason is considered by many to be that which sets human animals apart from their non-human fellow animals. In this chapter, we will consider reasoning from the different angles laid out in \cref{sec:evo-conclusion} --- in particular in relation to the theory under scrutiny, Mercier \& Sperber's argumentative theory of reasoning. First, we consider Mercier \& Sperber's definition of reasoning and compare it with some other definitions of reasoning. Then, we will look at reasoning in developing children and in non-human animals. Finally, we will consider the utility of reasoning, and expound the argumentative theory of reasoning.

\section{What is reasoning?}
\label{sec:reasoning-def}

Before anything else can be said about reasoning, it is critical to get clear what we mean exactly when we talk about reasoning. \citet{MS11} take the following as their definition of reasoning:

\begin{quoting}
    \emph{Reasoning}, as commonly understood, refers to a very special form of inference at the conceptual level, where not only is a new mental representation (or \emph{conclusion}) consciously produced, but the previously held representations (or \emph{premises}) that warrant it are also consciously entertained.
    \hfill (p.~57)
\end{quoting}
In other words, that what distinguishes reasoning from inference is the conscious attending to the representations.
In their definition of reasoning, Mercier \& Sperber take \emph{inference} to be "the production of new mental representations on the basis of previously held representations" (ibid.). They make explicit that this definition of reasoning excludes non-human animals and preverbal children from the realm of reasoners.
We will return to the role of the position of their definition of reasoning within the argumentative theory of reasoning in \cref{sec:def-scrutiny}.

Mercier \& Sperber state that their definition of reasoning is the one that is most commonly adhered to in the psychology of reasoning. However plausible, this claim is difficult to verify since most work in psychology of reasoning does not explicitly define reasoning within the frame of their research. Especially when one moves from psychology of reasoning to the neighboring discipline of animal cognition, what is understood by 'reasoning' becomes fuzzy. As an example, \citet{Andrews15} and \citet{Call06} discuss animal reasoning at length, yet they never explicate the definition of reasoning they adhere to.

\subsection{Other definitions of reasoning}
Let us now mention and compare two other definitions of reasoning, one due to cognitive scientist Vinod Goel and another due to philosopher Gilbert Harman.

\subsubsection{Goel}
In his \citeyear{Goel22} book "Reason and Less", Goel proposes and develops an account of \emph{tethered rationality}, where four different cognitive systems work in tandem to determine human behavior. The four systems, from evolutionarily 'oldest' to 'youngest', are the autonomic, instinctual, associative, and reasoning systems. These systems evolved 'on top of' each other, and are tightly integrated with one another --- tethered, as Goel calls it.
Goel then describes (and in doing so more or less defines) reasoning as follows:
\begin{quoting}
    Reasoning is a system for generating new beliefs from observations and/or existing beliefs and maintaining consistency of our beliefs (i.e., mental representations of the world).
    \hfill (p.~114)
\end{quoting}
In other words, according to Goel reasoning has a dual role: it generates inferences (in the way Mercier \& Sperber define them), and it ensures that our beliefs remain internally consistent or coherent.

\subsubsection{Harman}

\subsubsection{Stenning \& van Lambalgen, or Oaksford \& Chater}

\section{Reasoning in non-human animals}
\label{sec:reason-phylogeny}

Before we consider experimental findings on reasoning or reasoning-like abilities in non-human animals, we first need to consider one aspect of Mercier \& Sperber's definition of reasoning -- consciousness -- in more detail.

\subsection{Consciousness}

Mercier \& Sperber take the defining difference between reasoning and inference to be whether or not the 'reasons' are \emph{consciously} attended to. However, it may not be as straight-forward as it seems to grasp this concept exactly. Moreover, once we start to consider animal minds, it becomes especially tricky to pinpoint what consciousness means in this definition.

In a seminal \citeyear{Block95} paper, Ned Block proposed a distinction between two types of consciousness: access-consciousness (or A-consciousness) and phenomenal consciousness (or P-consciousness). A-consciousness then concerns the ability to access one's own mental states, whereas P-consciousness concerns "the qualitative nature of experience" \citep[p.~52]{Andrews15}: what it 'feels like' to be in a certain mental state. \todo{Is this an accurate paraphrase? Check the sources again}
Block noted the relation between A-consciousness and reasoning, and stated that A-consciousness is fundamental to reasoning:
\begin{quoting}
    It is of the essence of A-conscious content to play a role in reasoning
\hfill (p.~232)
\end{quoting}
Although Mercier \& Sperber do not mention Block's distinction, it is presumably A-consciousness rather than P-consciousness that their definition of reasoning employs. In other words, in order to reason one must be able to have conscious access to the mental representations resulting from their inferential mechanisms.
Unfortunately for our current purposes however, studies into animal consciousness are primarily interested in the extent to which non-human animals possess \emph{phenomenal} consciousness\footnote{It may be noted that in his original paper, Ned Block wanted to leave open the possibility of non-human animals possessing access-consciousness \citep{Block95}.}, not access consciousness \citep{Andrews15, Carruthers18}.
Josep \citet{Call06} does discuss the capacity for \emph{reflection} in non-human animals; however, its relation to access consciousness and to reasoning (in the definition of Mercier \& Sperber) is unclear\todo{Is it? Compare this to \citet{MS09} on reflective inference}. Call discusses empirical evidence that some animals know when they are uncertain about something, that some monkeys can know if they have forgotten something, and that apes know what they have not seen. \todo{Re-paraphrase}
\todo{Missing conclusion}

\subsection{Inference and reasoning in animals}

Many non-human animals are thought to possess the cognitive capacity for inference. For instance, monkeys are capable of performing disjunctive syllogisms \citep{Ferrigno21}; monkeys, birds, and some fish are capable of transitive inference \citep{Premack07}\footnote{Although Premack remarks that no non-human animals possess the concept of monotonicity, concluding that their capacity for transitive inference must rely on a 'hard-wired mechanism' rather than being 'based on reasoning' (p.~13864).}.

\section{Reasoning in children}
\label{sec:reason-ontogeny}

\section{The utility of reasoning}
\label{sec:reason-function}

Let us now look at the utility of reasoning, and in particular, dive head-first into the argumentative theory of reasoning.
We briefly consider classical accounts of the utility of reasoning, and then we will consider Mercier \& Sperber's account at length.

\subsubsection{Classical and dual-process theories of reasoning}

Classical theories of reasoning, building on centuries of philosophical work, maintain that the function of reasoning is to enhance or support individual cognition \citep{MS11}.

\begin{quoting}
    one of the functions of System 2 is to monitor the quality of both mental operations and overt behavior
    \hfill \citep[p.~699]{Kahneman03}
\end{quoting}

\begin{quoting}
    The analytic system is primarily a control system focused on the interests of the whole person. It is the primary maximizer of an individual's \emph{personal} goal satisfaction.
    (p.~64)
\end{quoting}

\citet{Goel22} also more or less adheres to this classical account of what the function of reasoning is, stating that
\begin{quoting}
    the function of the reasoning mind is to allow for greater flexibility in individual behavior, and thus more finely tuned responses to environmental stimuli than can be accommodated by the autonomic, instinctive, and associative minds.
    \hfill (p.~114)
\end{quoting}

\subsection{Introductory thoughts on reasoning by Mercier \& Sperber}

\subsection{Mercier \& Sperber's dual-process theory}

\subsection{Mercier \& Sperber's argumentative theory of reasoning}
\label{sec:exp-atr}

The argumentative theory of reasoning is Hugo Mercier and Dan Sperber's influential, but not uncontroversial, account of the function of reasoning from an evolutionary perspective. They introduced and coined the theory in a \citeyear{MS11} paper, a culmination of more than a decade's worth of experimental and philosophical research \citep{Sperber01, Sperber10, MS09, Sperber00}.
Briefly, the argumentative theory of reasoning states that the main function of reasoning is to produce arguments and evaluate arguments of others, in order to stabilize communication.
Before we can tackle the theory in detail, it is useful to look at some of Mercier \& Sperber's foundations to the theory first. We already discussed their views on the evolution of communication in \cref{sec:comm-ATR}, so now we turn to their views on reasoning. We first discuss some points about reasoning they offer in \citet{Sperber01,Sperber10}, then we will discuss a dual-process theory of reasoning proposed by \citet{MS09}, and finally we will go through the \citeyear{MS11} that coined the argumentative theory of reasoning.

\section{Conclusion}
