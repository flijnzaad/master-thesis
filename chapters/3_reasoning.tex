\chapter{Why do we reason?}
\label{ch:reasoning}

Reason is considered by many to be that which sets human animals apart from their non-human fellow animals. In this chapter, we will consider reasoning from the different angles laid out in \cref{sec:evo-conclusion} --- in particular in relation to the theory under scrutiny, Mercier \& Sperber's argumentative theory of reasoning. First, we consider Mercier \& Sperber's definition of reasoning and compare it with some other definitions of reasoning. Then, we will look at reasoning in developing children and in non-human animals. Finally, we will consider the utility of reasoning, and expound the argumentative theory of reasoning.

\section{What is reasoning?}
\label{sec:reasoning-def}

\todo[inline]{This section is very unfinished, sorry!}

Before anything else can be said about reasoning, it is critical to get clear what we mean exactly when we talk about reasoning. \citet{MS11} take the following as their definition of reasoning:

\begin{quoting}
    \emph{Reasoning}, as commonly understood, refers to a very special form of inference at the conceptual level, where not only is a new mental representation (or \emph{conclusion}) consciously produced, but the previously held representations (or \emph{premises}) that warrant it are also consciously entertained.
    \hfill (p.~57)
\end{quoting}
In other words, that what distinguishes reasoning from inference is the conscious attending to the representations.
In their definition of reasoning, Mercier \& Sperber take \emph{inference} to be "the production of new mental representations on the basis of previously held representations" (ibid.). They make explicit that this definition of reasoning excludes non-human animals and preverbal children from the realm of reasoners.
We will return to the role of the position of their definition of reasoning within the argumentative theory of reasoning in \cref{sec:def-scrutiny}.

Mercier \& Sperber state that their definition of reasoning is the one that is most commonly adhered to in the psychology of reasoning. However plausible, this claim is difficult to verify since most work in psychology of reasoning does not explicitly define reasoning within the frame of their research. Especially when one moves from psychology of reasoning to the neighboring discipline of animal cognition, what is understood by 'reasoning' becomes fuzzy. As an example, \citet{Andrews15} and \citet{Call06} discuss animal reasoning at length, yet they never explicate the definition of reasoning they adhere to.

\subsection{Dual-process theories of reasoning}
Let us now mention and compare two other definitions of reasoning, one due to cognitive scientist Vinod Goel and another due to philosopher Gilbert Harman.

\subsection{Discussion/introduction of Goel}
In his \citeyear{Goel22} book "Reason and Less", Goel proposes and develops an account of \emph{tethered rationality}, where four different cognitive systems work in tandem to determine human behavior. The four systems, from evolutionarily 'oldest' to 'youngest', are the autonomic, instinctual, associative, and reasoning systems. These systems evolved 'on top of' each other, and are tightly integrated with one another --- tethered, as Goel calls it.
Goel then describes (and in doing so more or less defines) reasoning as follows:
\begin{quoting}
    Reasoning is a system for generating new beliefs from observations and/or existing beliefs and maintaining consistency of our beliefs (i.e., mental representations of the world).
    \hfill (p.~114)
\end{quoting}
In other words, according to Goel reasoning has a dual role: it generates inferences (in the way Mercier \& Sperber define them), and it ensures that our beliefs remain internally consistent or coherent.
\subsection{Discussion of Harman}
\todo[inline,caption={}]{To do:
    \begin{itemize}
        \item Compare this definition with Mercier \& Sperber's and draw a conclusion about it
        \item Discuss Harman's reasoned change in view
    \end{itemize}
}
\subsection{Discussion of Stenning \& van Lambalgen, or Oaksford \& Chater}

\section{Reasoning in children}
\label{sec:reasoning-dev}

\todo[inline]{Possible references for this section (most I have yet to read): \citet{Bubikova23}, \citet{Pontecorvo10}, \citet{Tomasello14}? It was hard to find general references on development of reasoning; most of them seem to be pretty specific, approaching it from for example education science.}

\section{Reasoning in non-human animals}
\label{sec:reasoning-nha}

Before we consider experimental findings on reasoning or reasoning-like abilities in non-human animals, we first need to consider one aspect of Mercier \& Sperber's definition of reasoning -- consciousness -- in more detail.

\subsection{Consciousness}

Mercier \& Sperber take the defining difference between reasoning and inference to be whether or not the 'reasons' are \emph{consciously} attended to. However, it may not be as straight-forward as it seems to grasp this concept exactly. Moreover, once we start to consider animal minds, it becomes especially tricky to pinpoint what consciousness means in this definition.

In a seminal \citeyear{Block95} paper, Ned Block proposed a distinction between two types of consciousness: access-consciousness (or A-consciousness) and phenomenal consciousness (or P-consciousness). A-consciousness then concerns the ability to access one's own mental states, whereas P-consciousness concerns "the qualitative nature of experience" \citep[p.~52]{Andrews15}: what it 'feels like' to be in a certain mental state. \todo{Is this an accurate paraphrase? Check the sources again}
Block noted the relation between A-consciousness and reasoning, and stated that A-consciousness is fundamental to reasoning:
\begin{quoting}
    It is of the essence of A-conscious content to play a role in reasoning
\hfill (p.~232)
\end{quoting}
Although Mercier \& Sperber do not mention Block's distinction, it is presumably A-consciousness rather than P-consciousness that their definition of reasoning employs. In other words, in order to reason one must be able to have conscious access to the mental representations resulting from their inferential mechanisms.
Unfortunately for our current purposes however, studies into animal consciousness are primarily interested in the extent to which non-human animals possess \emph{phenomenal} consciousness\footnote{It may be noted that in his original paper, Ned Block wanted to leave open the possibility of non-human animals possessing access-consciousness \citep{Block95}.}, not access consciousness \citep{Andrews15, Carruthers18}.
Josep \citet{Call06} does discuss the capacity for \emph{reflection} in non-human animals; however, its relation to access consciousness and to reasoning (in the definition of Mercier \& Sperber) is unclear\todo{Is it? Compare this to \citet{MS09} on reflective inference}. Call discusses empirical evidence that some animals know when they are uncertain about something, that some monkeys can know if they have forgotten something, and that apes know what they have not seen. \todo{Re-paraphrase}
\todo[inline]{Missing: conclusion}

\subsection{Inference and reasoning in animals}

Many non-human animals possess the cognitive capacity for inference \todo{Reference need for this first claim?}. Monkeys are capable of performing disjunctive syllogisms \citep{Ferrigno21}; monkeys, birds, and some fish are capable of transitive inference \citep{Premack07}\footnote{Although Premack remarks that no non-human animals possess the concept of monotonicity, concluding that their capacity for transitive inference must rely on a 'hard-wired mechanism' rather than being 'based on reasoning' (p.~13864).}.
\todo[inline]{Note on Darwin, Penn et al., Premack, conclude with evolutionary origins.}

\section{The utility of reasoning}

Let us now look at the utility of reasoning, and in particular, dive head-first into the argumentative theory of reasoning.
We briefly consider classical accounts of the utility of reasoning, and then we will consider Mercier \& Sperber's account at length.

\subsection{Classical theories of reasoning}

\todo[inline]{Add to this section: take from \citet{MS09} and others}

Classical theories of reasoning, building on centuries of philosophical work, maintain that the function of reasoning is to enhance or support individual cognition \citep{MS11}.

% \begin{quoting}
%     one of the functions of System 2 is to monitor the quality of both mental operations and overt behavior
%     \hfill \citep[p.~699]{Kahneman03}
% \end{quoting}

% \begin{quoting}
%     The analytic system is primarily a control system focused on the interests of the whole person. It is the primary maximizer of an individual's \emph{personal} goal satisfaction.
%     (p.~64)
% \end{quoting}

\citet{Goel22} also more or less adheres to this classical account of what the function of reasoning is, stating that
\begin{quoting}
    the function of the reasoning mind is to allow for greater flexibility in individual behavior, and thus more finely tuned responses to environmental stimuli than can be accommodated by the autonomic, instinctive, and associative minds.
    \hfill (p.~114)
\end{quoting}

\subsection{Mercier \& Sperber on the function of reasoning}

\todo[inline]{The tone and content of this section may change slightly on what has already been said in the introduction}

The argumentative theory of reasoning is Hugo Mercier and Dan Sperber's influential, but not uncontroversial, account of the function of reasoning from an evolutionary perspective. They introduced and coined the theory in a \citeyear{MS11} paper, a culmination of more than a decade's worth of experimental and philosophical research \citep{Sperber01, Sperber10, MS09, Sperber00}.
Briefly, the argumentative theory of reasoning states that the main function of reasoning is to produce arguments and evaluate arguments of others, in order to stabilize communication.
Before we can tackle the theory in detail, it is useful to look at some of Mercier \& Sperber's foundations to the theory first.

\todo[inline]{I think the discussion of the first two papers should eventually be moved to Chapter 2, since they have more to do with the utility of communication than reasoning}

\subsubsection{\citet{Sperber01}: Testimony and argumentation}

\citet{Sperber01} considers testimony and argumentation from an evolutionary perspective and in doing so, provides important groundwork for his later work with Mercier (and others) on the relation between reasoning, argumentation and the stability of communication.

\todo[inline]{Later: check overlap with \cref{sec:arms-race}}

Testimony -- "the transmission of observed (or allegedly observed) information from one person to others" (p.~401) -- and argumentation -- "the defense of some conclusion by appeal to a set of premises that provide support for it" (ibid.) -- are two concepts central to human communication.
Sperber puts these two concepts in the perspective of evolution, and discusses in particular how they have figured in stabilizing communication over the course of evolutionary history.

A way one might look at communication is as a kind of 'cognition by proxy': through communication, an organism may access information another organism obtained from their own perception or inference. If you tell me that there is milk in the fridge, I can through this act of communication benefit from the information dervied from your perception of the milk carton in the fridge.
However, Sperber argues, at least in the case of human communication, testimony does not amount to cognition by proxy, because testimony has different effects than direct perception does. Upon receiving your testimony stating the milk is in the fridge, I am in a different cognitive state than if I would have perceived the milk carton there myself. Moreover, in human communication, interpretation and acceptance of utterances are two separate processes.\todo{And so?}

The classical account of animal signalling (i.e.\@ communication) by \citet{DawkinsKrebs78} focuses only on the side of the communicator in the story. Sperber rejects this approach, arguing that the interests of the sender cannot be the only driving force in the evolution of communication.\todo{Paraphrase better}
Sperber argues that for communication to have stabilized and continued to be stable between senders and receivers, both parties must have benefited from the action:
\begin{quoting}
    For communication to stabilize within a species, as it has among humans, both the production and the reception of messages should be advantageous. If communication were on the whole beneficial to producers of messages (by contributing to their fitness) at the expensive of receivers, or beneficial to receivers at the expense of producers, one of the two behaviors would be likely to have been selected out, and the other behavior would have collapsed by the same token
\hfill (p.~403)
\end{quoting}

\todo[inline,caption={}]{Discussion of \poscite{Sperber01} most important points for the ATR. To do:
     \begin{itemize}
        \item Give definitions of testimony and argumentation
        \item Discuss the 'cognition by proxy' claims he makes
        \item Describe his quasi-game-theoretic analysis of communication
        \item Describe the evolutionary arms race
        \item Describe his conclusions about what's responsible for the stability of communication
    \end{itemize}
}

\subsubsection{\citet{Sperber10}: Epistemic vigilance}

\todo[inline]{Make this fit with the previous section once that's done}

One of the cornerstones of the argumentative theory of reasoning is the concept of \emph{epistemic vigilance}, which Sperber and colleagues (among whom Hugo Mercier) introduced in a seminal \citeyear{Sperber10} paper.

In the paper, Sperber and colleagues note that humans are dependent on communication, and they argue that this dependence leaves humans vulnerable to being deceived by others.
They state that misinformation or deception may "reduce, cancel, or even reverse" the gains that communication might bring to the addressee (p.~360).
Consequently, the information that addressees receive from communicators is only advantageous to them to the extent that the information is genuine.
Sperber and colleagues thus conclude that for this purpose, humans have evolved a suite of cognitive mechanisms for \emph{epistemic vigilance}.

A communicative act triggers not only comprehension in the addressee, but also epistemic vigilance alongside it.
In the grand scheme of human communication, vigilance maintains a balance between honesty and dishonesty; as Sperber and colleagues put it, "the audience's vigilance limits the range of situations where dishonesty might be in the communicators' best interest" (p.~368), which results in communication being honest most of the time.
Sperber and colleagues argue that vigilance is not a nicety, something that is only invoked sometimes; they maintain that vigilance is the default disposition of interlocutors in communicative settings.

One may distinguish between vigilance towards the \emph{source} of a message, and vigilance towards the \emph{content} of the message.
In regards to the source, the authors note that reliable sources must be both competent and benevolent (and this is dependent on the context).
Empirical evidence suggests that deceiving people can be quite beneficial, since experiments from deception detection research show that people are not good at detecting lies based on non-verbal behavioral cues.
\todo[inline]{Add more}

\subsubsection{The argumentative theory of reasoning}

\todo[inline,caption={}]{To do:
    \begin{itemize}
        \item Describe the ATR in detail
        \item Discuss the empirical predictions it makes, and the evidence that \citet{MS11} bring to corroborate the predictions
        \item Possibly discuss some of the additional claims and findings from \citet{MS17}
    \end{itemize}
}
