\chapter{The argumentative theory of reasoning}
\label{ch:reasoning}

\section*{Introduction}

\section{Reasoning in non-human animals}

\section{Reasoning in children}

\section{The utility of reasoning}

\subsection{Epistemic vigilance}
\todo[inline]{This section I think belongs in the communication chapter, but I've put it here for now to make it clearer what's new. I'll move it to Chapter 2 once I start revising that chapter.}

In a seminal \citeyear{Sperber10} paper, Sperber and colleagues\footnote{Among them, notably, Hugo Mercier} introduced the concept of \emph{epistemic vigilance} to the philosophical study of communication.
In the paper, Sperber and colleagues note humans' dependence on communication, and note that this dependence is accompanied by a risk of being deceived by others.
argue that to receivers of communication, the information received is only advantageous to them to the extent that they can be sure that the information is genuine.
They thus conclude that for this purpose, humans have evolved a set of cognitive mechanisms for \emph{epistemic vigilance}.

\subsection{The utility of argumentation}

\subsection{The argumentative theory of reasoning}
