\chapter{Why do we reason?}
\label{ch:reasoning}

Reason is considered by many to be that which sets human animals apart from their non-human fellow animals. In this chapter, we will consider reasoning from the different angles laid out in \cref{sec:evo-conclusion} --- in particular in relation to the theory under scrutiny, Mercier \& Sperber's argumentative theory of reasoning. First, we consider Mercier \& Sperber's definition of reasoning and compare it with some other definitions of reasoning. Then, we will look at reasoning in developing children and in non-human animals. Finally, we will consider the utility of reasoning, and expound the argumentative theory of reasoning.

\section{What is reasoning?}
\label{sec:reasoning-def}

Before anything else can be said about reasoning, it is critical to get clear what we mean exactly when we talk about reasoning. \citet{MS11} take the following as their definition of reasoning:
\todo{And what is argumentation?}

\begin{quoting}
    \emph{Reasoning}, as commonly understood, refers to a very special form of inference at the conceptual level, where not only is a new mental representation (or \emph{conclusion}) consciously produced, but the previously held representations (or \emph{premises}) that warrant it are also consciously entertained.
    \hfill (p.~57)
\end{quoting}
In other words, that what distinguishes reasoning from inference is the conscious attending to the representations.
In their definition of reasoning, Mercier \& Sperber take \emph{inference} to be "the production of new mental representations on the basis of previously held representations" (ibid.). They make explicit that this definition of reasoning excludes non-human animals and preverbal children from the realm of reasoners.
We will return to the role of the position of their definition of reasoning within the argumentative theory of reasoning in \cref{sec:def-scrutiny}.

Mercier \& Sperber state that their definition of reasoning is the one that is most commonly adhered to in the psychology of reasoning. However plausible, this claim is difficult to verify since most work in psychology of reasoning does not explicitly define reasoning within the frame of their research. Especially when one moves from psychology of reasoning to the neighboring discipline of animal cognition, what is understood by 'reasoning' becomes fuzzy. As an example, \citet{Andrews15} and \citet{Call06} discuss animal reasoning at length, yet they never explicate the definition of reasoning they adhere to.

\subsection{Dual-process theories of reasoning}

\subsection{Other definitions of reasoning}
Let us now mention and compare two other definitions of reasoning, one due to cognitive scientist Vinod Goel and another due to philosopher Gilbert Harman.

\subsubsection{Goel}
In his \citeyear{Goel22} book "Reason and Less", Goel proposes and develops an account of \emph{tethered rationality}, where four different cognitive systems work in tandem to determine human behavior. The four systems, from evolutionarily 'oldest' to 'youngest', are the autonomic, instinctual, associative, and reasoning systems. These systems evolved 'on top of' each other, and are tightly integrated with one another --- tethered, as Goel calls it.
Goel then describes (and in doing so more or less defines) reasoning as follows:
\begin{quoting}
    Reasoning is a system for generating new beliefs from observations and/or existing beliefs and maintaining consistency of our beliefs (i.e., mental representations of the world).
    \hfill (p.~114)
\end{quoting}
In other words, according to Goel reasoning has a dual role: it generates inferences (in the way Mercier \& Sperber define them), and it ensures that our beliefs remain internally consistent or coherent.

\subsubsection{Harman}

\subsubsection{Stenning \& van Lambalgen, or Oaksford \& Chater}

\section{The utility of reasoning}
\label{sec:reason-function}

Let us now look at the utility of reasoning, and in particular, dive head-first into the argumentative theory of reasoning.
We briefly consider classical accounts of the utility of reasoning, and then we will consider Mercier \& Sperber's account at length.

\subsubsection{Classical conceptions of reasoning's function}

Classical theories of reasoning, building on centuries of philosophical work, maintain that the function of reasoning is to enhance or support individual cognition \citep{MS11}.

\begin{quoting}
    one of the functions of System 2 is to monitor the quality of both mental operations and overt behavior
    \hfill \citep[p.~699]{Kahneman03}
\end{quoting}

\begin{quoting}
    The analytic system is primarily a control system focused on the interests of the whole person. It is the primary maximizer of an individual's \emph{personal} goal satisfaction.
    (p.~64)
\end{quoting}

\citet{Goel22} also more or less adheres to this classical account of what the function of reasoning is, stating that
\begin{quoting}
    the function of the reasoning mind is to allow for greater flexibility in individual behavior, and thus more finely tuned responses to environmental stimuli than can be accommodated by the autonomic, instinctive, and associative minds.
    \hfill (p.~114)
\end{quoting}

\subsection{Introductory thoughts on reasoning by Mercier \& Sperber}

\todo[inline]{Word smoothie}

In the papers preceding their \citeyear{MS11} seminal paper introducing the ATR, Sperber, Mercier and colleagues not only consider communication but also already argumentation and reasoning. Let us now briefly attend to their findings.

In \cref{sec:Sperber01} we saw Sperber discuss the evolution of testimony and argumentation. With regards to the latter, Sperber notes that reasoning may be used individually for reflection, or in communication for dialogical argumentation. He argues that, although classically the former has been viewed as the function of reasoning, this is implausible from an evolutionary point of view. He maintains that domain-general reasoning abilities are cognitively costly and slow, and therefore could not evolved for the purpose of producing knowledge, since more specific mechanisms would better suit this function. Instead, the function of reasoning is communicative rather than individual:
\begin{quoting}
    {[Reasoning]} is an evaluation and persuasion mechanism, not, or at least not directly, a knowledge-production mechanism.
    \hfill \citep[p.~409]{Sperber01}
\end{quoting}
Moreover, as we saw in \cref{sec:Sperber01}, Sperber discusses coherence checking, developed by addressees to detect misinformation and deception. He offers two observations about coherence checking that may point us to its relation with reasoning:
\begin{quoting}
    Coherence checking involves a high processing cost, it cannot be done on a large scale because it would lead to a computational explosion, and it is fallible.
    \\ (\ldots) \\
    Coherence checking (\ldots) involves metarepresentational attention to logical and evidential relationships between representations
    \hfill \citep[p.~410]{Sperber01}
\end{quoting}
We return to this in \cref{sec:MS11}.

\todo[inline]{This bottom part should be moved down, or removed}
Moving on to \poscite{Sperber10} discussion of epistemic vigilance, we see them dedicate a section to epistemic vigilance and reasoning. They illustrate their claims about checking coherence (by the addressee) and displaying coherence (by the communicator) with an example of conversational implicature. They describe this use of implicatures as offering an "implicit argument".
They go on to state that argumentation, whether it be implicit or explicit, is a product of reasoning. Further, they argue that reasoning is a "tool for epistemic vigilance".

\subsection{Mercier \& Sperber's dual-process theory}
\label{sec:MS09}

\todo[inline]{Word smoothie}
\todo[inline]{Is this paper badly written, or am I just stupid? /hj}
In the \citeyear{MS09} paper \emph{Intuitive and reflective inferences}, Mercier and Sperber propose their own dual system theory of reasoning. They do this because they are dissatisfied with current dual system theories of reasoning, arguing for example that they are vague: they accuse these other theories of relying on "a bundle of constrasting features" rather than offering a "principled distinction" (p.~149), which therefore allows these theories some leeway in ascribing characteristic to the two systems.\todo{Comment in Ch. 4 on the hypocrisy of this?}
Their own dual system theory, apparently "in the same spirit" (p.~149) as the classical dual system theories they describe, concerns the distinction between intuitive inferences on the one hand, and reflective inferences on the other.
Intuitive inferences are reminiscent of system 1 processes, in that they happen unconsciously, and reflective inferences are reminiscent of system 2 processes, in that they happen consciously.
The authors emphasize however that their dual system theory cannot be equated with the 'standard' system 1 / system 2 distinction.
\todo{Mention what 'reasoning proper' is}
\todo[inline]{Part on massive modularity}
Mercier and Sperber introduce their dichotomy as part of a massive-modularist framework. This warrants some explanation; though I consider the modularity of the mind to be out of scope for this thesis, Mercier and Sperber's dual system theory is best explained through their views on modularity.

Mercier and Sperber are proponents of a massively modular view on the human mind, maintaining that the mind consists of a number of modules specialized to a specific domain. These modules are autonomous in their function, have distinct evolutionary and developmental histories, and they have characteristic inputs, procedures and outputs.
On this massively modular view, inferences are performed by many different domain-specific modules. Even inferences that seem to be performed by domain-general module, such as logical inferences like disjunctive syllogism, are carried out by domain-specific \emph{metarepresentational modules}: modules that performs inferences on conceptual representations, they argue.
Because these conceptual representations may belong to any domain, metarepresentational inferences appear to be domain-general. However, the authors state, this domain-generality is indirect and virtual. In the words of Mercier and Sperber,
\begin{quoting}
    Metarepresentational modules are as specialized and modular as any other kind of module. It is just that the domain-specific inferences they perform may result in the fixation of beliefs in any domain.
    \hfill \citep[p.~153]{MS09}
\end{quoting}

\subsection{Mercier \& Sperber's argumentative theory of reasoning}
\label{sec:MS11}

The argumentative theory of reasoning is Hugo Mercier and Dan Sperber's influential, but not uncontroversial, account of the function of reasoning from an evolutionary perspective. They introduced and coined the theory in a \citeyear{MS11} paper, a culmination of more than a decade's worth of experimental and philosophical research \citep{Sperber01, Sperber10, MS09, Sperber00}.
Briefly, the argumentative theory of reasoning states that the main function of reasoning is to produce arguments and evaluate arguments of others, in order to stabilize communication.
Before we can tackle the theory in detail, it is useful to look at some of Mercier \& Sperber's foundations to the theory first. We already discussed their views on the evolution of communication in \cref{sec:comm-ATR}, so now we turn to their views on reasoning. We first discuss some points about reasoning they offer in \citet{Sperber01,Sperber10}, then we will discuss a dual-process theory of reasoning proposed by \citet{MS09}, and finally we will go through the \citeyear{MS11} that coined the argumentative theory of reasoning.

In Mercier and Sperber's response to the open peer commentary that supplements their paper, the authors are a bit more explicit in the relation between coherence checking and reasoning, stating that
\begin{quoting}
    Coherence checking, we argue, is the second major heuristic used in filtering communicated information, and is at the basis of reasoning proper.
    \hfill \citep[p.~96]{MS11}
\end{quoting}
(the first heuristic being trust calibration).

\section{Conclusion}
