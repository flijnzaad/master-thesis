\section{Precursory concepts to the ATR}
\label{sec:comm-ATR}

Now that we have discussed the utility of communication, and in doing so also considered the utility of cooperation and deception, and the stability of communication, we should have our first look at some basic foundations for Mercier \& Sperber's argumentative theory of reasoning. Here, we will consider \citet{Sperber01} and \citet{Sperber10}, since they mostly concern communication; in \cref{ch:reasoning}, we will consider \citet{MS09} and \citet{MS11}, since they mostly concern reasoning.

\subsection{Sperber on the evolution of testimony and argumentation}
\label{sec:Sperber01}

In a \citeyear{Sperber01} paper, Dan Sperber analyzes testimony and argumentation from an evolutionary perspective. In doing so, he provides important groundwork for his later work with Mercier (and others) on the relation between reasoning, argumentation and the stability of communication.

Testimony and argumentation are two concepts central to human communication. Sperber borrows his definitions for these concepts from epistemologist Alvin Goldman, who defines testimony as "the transmission of observed (or allegedly observed) information from one person to others" \citep[p.~401]{Sperber01} and argumentation as "the defense of some conclusion by appeal to a set of premises that provide support for it" (ibid.).
Sperber puts these two concepts in an evolutionary perspective, and discusses in particular how they have figured in stabilizing communication over the course of evolutionary history.

\todo[inline]{Replace this example by an animal communication example}
A tempting way to look at communication is as a kind of 'cognition by proxy': through communication, one organism may access information another organism has obtained from its own perception or inference. For instance, if you tell me that there is milk in the fridge, I can through this act of communication benefit from the information derived from your perception of the milk carton in the fridge.
However, Sperber argues that, at least in the case of human communication, testimony does not amount to cognition by proxy. This is because testimony has different effects than direct perception does. Going back to our example, upon receiving your testimony stating that the milk is in the fridge, I am in a different cognitive state than if I would have perceived the milk carton there myself. Moreover, in human communication, Sperber argues that interpretation and acceptance of utterances are two separate processes: recognizing what a speaker meant by their utterance is not the same as accepting it as true.\footnote{This may very well be the case philosophically or epistemologically speaking, but psychologically speaking, they may be more intertwined than Sperber implies. In a later paper, he elaborates more on his stance, making even stronger claims about how comprehension always precedes the acceptance (or rejection) of a claim \citep[\S 3]{Sperber10}. Although this is intuitively plausible, \citet{Lewandowsky12} points out that empirical evidence suggests that for someone to comprehend an utterance, they must (at least temporarily) accept it.}
\todo[inline]{And so? Connect this to the ostensive-inferential model: is interpretation part of communication?}

The classical account of animal communication by \citet{DawkinsKrebs78} focuses only on the side of the communicator in the story, maintaining that the function of communication is to manipulate others. Sperber rejects this classical approach, arguing that the interests of the sender cannot be the only driving force in the evolution of communication.
He outlines a similar line of argumentation as we have seen in \cref{sec:S-P08}\todo{Is it too much overlap with that section?},
arguing that for communication to have stabilized and continued to be stable between senders and receivers, both parties must have benefited from the action. In other -- game-theoretic -- terms, communication must (at least in the long run) be a positive-sum game, where both senders and receivers gain from the interaction.

In the case of receiving testimony from others, the receiver gains from testimony "only to the extent that it is a source of genuine (\ldots) information" (p.~404).
\todo[inline]{Where to talk about why gaining information is itself beneficial? Here or Ch. 4?}
On the side of the production of testimony, the sender stands to gain from this testimony because
\begin{quoting}
    it allows them to have desirable effects on the receivers' attitudes and behavior. By communicating, one can cause others to do what one wants them to do and to take specific attitudes to people, objects, and so on
    \hfill (p.~404)
\end{quoting}
He later elaborates on this by saying that getting others to accept your communicated message is not intrinsically beneficial. Rather, it is \emph{indirectly} beneficial, through bringing about these 'desirable effects' in others, as a way of 'cognitive manipulation'.
We briefly return to these observations (particularly the 'desirable effects') in \cref{ch:scrutiny}.
\todo[inline]{Where exactly? Will I talk about how this paper stacks up to the other things I found and concluded about communication? If so, where? Here?}
Sperber notes that it is exactly this self-interest of the sender that renders this 'cognition by proxy' view as inapplicable to human communication.
Moreover, he concludes from these observations of his that
\begin{quoting}
    the function of communication presents itself differently for communicator and audience
    \hfill (p.~411)
\end{quoting}
\todo[inline]{Conclusion about how this fits with what I wrote?? Or should this quote and comment be somewhere else?}

Sperber goes on to cast his observations in game-theoretic terms by sketching out a payoff matrix for a one-off communicative event. In it, he considers that senders may be truthful or untruthful, and receivers may be trusting or distrusting. According to Sperber, the sender's gain amounts to whether they have the 'desired' effect on the receiver; therefore, the sender gains from the interaction if the receiver is trusting (since this means the sender's message is accepted), and loses from the interaction if the receiver is distrusting. The payoff of this event for the sender is thus independent of the truthfulness of the sender. On the side of the receiver, their payoff \emph{is} dependent on the truthfulness of the sender: the receiver gains if they accept a truthful message, loses if they accept an untruthful message, and incurs no gain nor loss if they are distrusting and thus don't accept a message (truthful or not).

Sperber notes that the optimal strategy for such a game varies with the circumstances for both players: it is not always beneficial to be truthful, nor always untruthful; nor is it beneficial to be always trusting, nor always distrusting. In other words, there is no one stable solution to this game.
This is especially the case once we move away from this simple one-off communicative event to an iterated game of communication, where not only short-term payoffs but also long-term payoffs determine the optimal strategy.\todo{Should I discuss somewhere how reputation works in mass society?}
Therefore, it is in the receiver's interest to calibrate their trust towards senders as accurately as possible; in fact, Sperber argues, this trust calibration is necessary for explaining the stability of communication.

Unlike non-human animals, humans have another way to communicate facts, other than testimony, namely \emph{argumentation}. Senders may provide receivers with reasons to accept their testimony, which the receiver may evaluate and accept or reject, independent of their trust in the sender.
Sperber sketches out the steps in what he calls the 'evaluation-persuasion arms race', i.e.\@ the chain of evolutionary adaptations that has resulted in our mechanisms for argument production and evaluation.
He argues that the first step in this 'arms race' was for the receiver to develop \emph{coherence checking}. Coherence checking involves attending to both the internal coherence of the communicated message, and the external coherence with what the receiver already believes. Coherence checking, Sperber argues, is a useful defense against the risks of deception by the sender, because lies and other false claims are often externally or internally incoherent.
The second step in the arms race was then for the sender to anticipate this coherence-checking by overtly displaying the coherence of their message to their receiver, which requires argumentative form; thus, testimony becomes argument. The next steps were on the side of the receiver to develop skills for examining these displays of coherence (i.e., arguments), and on the side of the sender to 'improve their argumentative skills'.

\subsection{Sperber and colleagues on epistemic vigilance}
\label{sec:Sperber10}

\todo[inline]{Should I cast the discussion of this paper in terms of the arms race? Or discuss it top-to-bottom? Maybe the former, since I scrutinize the paper very intensively, so you'll need to hear every step of the argument?}

% my prologue
Further building upon his \citeyear{Sperber01} views, Sperber and his colleagues (among whom, notably, Hugo Mercier) introduced the concept of \emph{epistemic vigilance} in a seminal \citeyear{Sperber10} paper. This concept constitutes a cornerstone of Sperber \& Mercier's later argumentative theory of reasoning. Therefore, a comprehensive discussion is in order here -- not in the least because this concept will be one of my targets of scrutiny in \cref{ch:scrutiny}.

% introduction
In their \citeyear{Sperber10} paper, Sperber and colleagues start by emphasizing that humans are dependent on communication, and they argue that this dependence leaves humans vulnerable to being deceived by others.
They state that misinformation or deception may "reduce, cancel, or even reverse" the gains that communication can bring to the addressee (p.~360).
Consequently, the information that an addressee receives from a communicator is only advantageous to her to the extent that the information is genuine.
Sperber and colleagues thus conclude that for this purpose, humans have evolved a suite of cognitive mechanisms for \emph{epistemic vigilance}.
Moreover, this suite of mechanisms must have evolved alongside, and is used in tandem with, abilities for ostensive-inferential communication\footnote{Slightly confusingly, Sperber et al. call the ostensive-inferential communication that we saw in \cref{sec:comm:definition} 'overt intentional communication' in this paper. However, they refer to Sperber \& Wilson's \emph{Relevance Theory}, which calls this ostensive-inferential communication. So ultimately, they are talking about the same thing.}, because they work in tandem to facilitate trust calibration on the side of the receiver.

% epistemic trust and vigilance
In order to illustrate and somewhat demarcate the concept of epistemic vigilance, Sperber and colleagues discuss some work in philosophy and psychology related to trust and vigilance. Specifically, they consider different views on the question of whether humans are 'per default' trusting or vigilant.
Of this discussion, two points stand out to me as noteworthy, especially as they pertain to the discussion to come in \cref{sec:EV-scrutiny}. The first of them concerns a characterization of vigilance that nicely captures its spirit:
\begin{quoting}
    Vigilance (unlike distrust) is not the opposite of trust; it is the opposite
of blind trust
    \hfill \citep[p.~363]{Sperber10}
\end{quoting}
\todo{Add paraphrase or argue better why this quote is relevant}
The second concerns a strong claim, which we will critically assess in \cref{sec:EV-scrutiny}:\todo{Will we assess it? Why do you mention it here? Elaborate on this}
\begin{quoting}
    in communication, it is not that we can generally be trustful and therefore need to be vigilant only in rare and special circumstances. We could not be mutually trustful unless we were mutually vigilant.
    \hfill \citep[p.~364]{Sperber10}
\end{quoting}

% comprehension and acceptance
Next, the authors move on to discussing comprehension and acceptance of utterances in communication, and how these relate to epistemic vigilance and trust.
They argue that a communicative act does not only trigger comprehension in the addressee, but it also triggers epistemic vigilance alongside it. If epistemic vigilance then "does not come up with reasons to doubt" (p.~369), this comprehension leads to acceptance.
They go on to argue that comprehension of an utterance is not "guided by a presumption of truth", as other theorists state, but rather by an "expectation of relevance" (p.~367); see \citet{SperberWilson86}. This expectation of relevance requires a 'stance of trust' of the addressee regarding the speaker.
\todo{This relates to the Gricean stuff I mentioned earlier; refer to it, here, or there?}
This stance of trust of the addressee is "tentative and labile" (p.~368), and epistemic vigilance is (as mentioned) active alongside this stance of trust.\todo{This sentence is weird (the whole paragraph is)}

% Sperber and colleagues maintain that vigilance is not a nicety, something that is only invoked sometimes; they maintain that vigilance is the default disposition of interlocutors in communicative settings.

To further explicate epistemic vigilance as a concept, Sperber and colleagues outline a distinction between vigilance towards the \emph{source} of a message (the 'who'), and vigilance towards the \emph{content} of the message (the 'what').
% vigilance towards the source
As for vigilance towards the source, they note that the reliability of a source depends on two factors: a reliable source must be competent, and a reliable source must be benevolent.
Moreover (and importantly), a receiver's vigilance towards the sender as a source of information -- in other words, the sender's perceived trustworthiness -- is dependent on the context: it varies per topic and per situation.
Because of this, it is important for a receiver to accurately calibrate her trust in the sender depending on the context.
They go on to discuss empirical evidence that corroborates that trust, and calibrating trust to the situation, is indeed important to us.
Moreover, on the other side of the coin, they note that deceiving people can be quite beneficial: experiments from deception detection research show that people are not good at detecting lies based on non-verbal behavioral cues.
They end this particular discussion by noting that more empirical research is needed about how people calibrate their trust in everyday communication, outlining some desiderata for this research.

% the development of epistemic vigilance and mindreading: SKIPPED THIS

% vigilance towards the content
Moving on now to vigilance towards the \emph{content} of a message, Sperber and colleagues restate that comprehension and epistemic vigilance are two processes that are intertwined to some extent. Specifically, they note that one mechanism of comprehension, namely the search for relevance, provides a basis for an "imperfect but cost-effective epistemic assessment" (p.~374).
They discuss belief revision and the role that coherence checking plays in it. We already saw \poscite{Sperber01} discussion of coherence checking; Sperber and colleagues now describe coherence checking a mechanism for epistemic vigilance. They note that coherence checking "takes advantage of the limited background information activated by the comprehension process itself" (p.~375). They argue that the search for relevance "automatically involves the making of inferences which may turn up inconsistencies or incoherences relevant to epistemic assessment" (p.~376).\todo{Paraphrase these quotes?}

% epistemic vigilance and reasoning
Next, the authors return to and expand upon an idea we have seen \citet{Sperber01} propose, concerning the emergence of argumentation as a demonstration of coherence. I will discuss this in much more detail in \cref{sec:exp-atr}, as this part of \citet{Sperber10} basically constitutes a rudimentary explication of the argumentative theory of reasoning.

\todo[inline]{Missing, i.e. possibly discuss: \S 8 on epistemic vigilance on a population scale}

To summarize, according to Sperber and his colleagues humans have developed a suite of mechanisms for epistemic vigilance, filtering incoming information in order to avoid being deceived by others. A communicative act triggers both comprehension and epistemic vigilance, and the epistemic assessment of the communicative act draws upon some of the inferential steps that are carried out in the search for relevance, which makes the assessment relatively cost-effective. Epistemic vigilance can be directed towards the source of a message or towards the content of the message. This amounts to the calibration of trust and coherence checking, respectively.
