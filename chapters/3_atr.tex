\chapter{The argumentative theory of reasoning}
\label{ch:atr}

\todo[inline]{Thoroughly rewrite this whole introduction. Introduce the acronym}
Reason is considered by many to be that which sets human animals apart from their non-human fellow animals. In this chapter, we will consider reasoning from the different angles laid out in \cref{sec:evo-conclusion} --- in particular in relation to the theory under scrutiny, Mercier \& Sperber's argumentative theory of reasoning. First, we consider Mercier \& Sperber's definition of reasoning and compare it with some other definitions of reasoning. Then, we will look at reasoning in developing children and in non-human animals. Finally, we will consider the utility of reasoning, and expound the argumentative theory of reasoning.

The argumentative theory of reasoning is Hugo Mercier and Dan Sperber's influential, but not uncontroversial, account of the function of reasoning from an evolutionary perspective. They introduced and coined the theory in a \citeyear{MS11} paper, a culmination of more than a decade's worth of experimental and philosophical research \citep{Sperber01, Sperber10, MS09}.
Briefly, the argumentative theory of reasoning states that the main function of reasoning is to produce arguments and evaluate arguments of others, in order to stabilize communication.
Before we can tackle the theory in detail, it is useful to look at some of Mercier \& Sperber's foundations to the theory first. We first discuss some points about reasoning they offer in \citet{Sperber01,Sperber10}, then we will discuss a dual-process theory of reasoning proposed by \citet{MS09}, and finally we will go through the \citeyear{MS11} that coined the argumentative theory of reasoning.

\section{Sperber on the evolution of testimony and argumentation}
\label{sec:Sperber01}

In a \citeyear{Sperber01} paper, Dan Sperber analyzes testimony and argumentation from an evolutionary perspective. In doing so, he provides important groundwork for his later work with Mercier (and others) on the relation between reasoning, argumentation and the stability of communication.

Testimony and argumentation are two concepts central to human communication. Sperber borrows his definitions for these concepts from epistemologist Alvin Goldman, who defines testimony as "the transmission of observed (or allegedly observed) information from one person to others" \citep[p.~401]{Sperber01} and argumentation as "the defense of some conclusion by appeal to a set of premises that provide support for it" (ibid.).
Sperber puts these two concepts in an evolutionary perspective, and discusses in particular how they have figured in stabilizing communication over the course of evolutionary history.

A tempting way to look at communication is as a kind of 'cognition by proxy': through communication, one organism may access information another organism has obtained from its own perception or inference.
For instance, if you tell me that there is milk in the fridge, I can through this act of communication benefit from the information derived from your perception of the milk carton in the fridge.
\example{Replace this example by an animal communication example}
However, Sperber argues that, at least in the case of human communication, testimony does not amount to cognition by proxy. This is because testimony has different effects than direct perception does. Going back to our example, upon receiving your testimony stating that the milk is in the fridge, I am in a different cognitive state than if I would have perceived the milk carton there myself. Moreover, in human communication, Sperber argues that interpretation and acceptance of utterances are two separate processes: recognizing what a speaker meant by their utterance is not the same as accepting it as true.\footnote{This may very well be the case philosophically or epistemologically speaking, but psychologically speaking, they may be more intertwined than Sperber implies. In a later paper, he elaborates more on his stance, making even stronger claims about how comprehension always precedes the acceptance (or rejection) of a claim \citep[\S 3]{Sperber10}. Although this is intuitively plausible, \citet{Lewandowsky12} points out that empirical evidence suggests that for someone to comprehend an utterance, they must (at least temporarily) accept it.}

The classical account of animal communication by \citet{DawkinsKrebs78} focuses only on the side of the communicator in the story, maintaining that the function of communication is to manipulate others. Sperber rejects this classical approach, arguing that the interests of the sender cannot be the only driving force in the evolution of communication.
He outlines a similar line of argumentation as we have seen in \cref{sec:S-P08},
arguing that for communication to have stabilized and continued to be stable between senders and receivers, both parties must have benefited from the action. In other -- game-theoretic -- terms, communication must (at least in the long run) be a positive-sum game, where both senders and receivers gain from the interaction.

In the case of receiving testimony from others, the receiver gains from testimony "only to the extent that it is a source of genuine (\ldots) information" (p.~404).
On the side of the production of testimony, the sender stands to gain from this testimony because
\begin{quoting}
    it allows them to have desirable effects on the receivers' attitudes and behavior. By communicating, one can cause others to do what one wants them to do and to take specific attitudes to people, objects, and so on
    \hfill (p.~404)
\end{quoting}
He later elaborates on this by saying that getting others to accept your communicated message is not intrinsically beneficial. Rather, it is \emph{indirectly} beneficial, through bringing about these 'desirable effects' in others, as a way of 'cognitive manipulation'.
Sperber notes that it is exactly this self-interest of the sender that renders this 'cognition by proxy' view as inapplicable to human communication.
Moreover, he concludes from these observations of his that
\begin{quoting}
    the function of communication presents itself differently for communicator and audience
    \hfill (p.~411)
\end{quoting}

Sperber goes on to cast his observations in game-theoretic terms by sketching out a payoff matrix for a one-off communicative event. In it, he considers that senders may be truthful or untruthful, and receivers may be trusting or distrusting. According to Sperber, the sender's gain amounts to whether they have the 'desired' effect on the receiver; therefore, the sender gains from the interaction if the receiver is trusting (since this means the sender's message is accepted), and loses from the interaction if the receiver is distrusting. The payoff of this event for the sender is thus independent of the truthfulness of the sender. On the side of the receiver, their payoff \emph{is} dependent on the truthfulness of the sender: the receiver gains if they accept a truthful message, loses if they accept an untruthful message, and incurs no gain nor loss if they are distrusting and thus don't accept a message (truthful or not).

Sperber notes that the optimal strategy for such a game varies with the circumstances for both players: it is not always beneficial to be truthful, nor always untruthful; nor is it beneficial to be always trusting, nor always distrusting. In other words, there is no one stable solution to this game.
This is especially the case once we move away from this simple one-off communicative event to an iterated game of communication, where not only short-term payoffs but also long-term payoffs determine the optimal strategy.
Therefore, it is in the receiver's interest to calibrate their trust towards senders as accurately as possible; in fact, Sperber argues, this trust calibration is necessary for explaining the stability of communication.

Unlike non-human animals, humans have another way to communicate facts, other than testimony, namely \emph{argumentation}. Senders may provide receivers with reasons to accept their testimony, which the receiver may evaluate and accept or reject, independent of their trust in the sender.
Sperber notes that reasoning may be used individually for reflection, or in communication for dialogical argumentation. He argues that, although classically the former has been viewed as the function of reasoning, this is implausible from an evolutionary point of view. He maintains that domain-general reasoning abilities are cognitively costly and slow, and therefore could not have evolved for the purpose of producing knowledge, since more specific mechanisms would better suit this function. Instead, the function of reasoning is communicative rather than individual:
\begin{quoting}
    {[Reasoning]} is an evaluation and persuasion mechanism, not, or at least not directly, a knowledge-production mechanism.
    \hfill \citep[p.~409]{Sperber01}
\end{quoting}

Sperber sketches out the steps in what he calls the 'evaluation-persuasion arms race', i.e.\@ the chain of evolutionary adaptations that has resulted in our mechanisms for argument production and evaluation.
He argues that the first step in this 'arms race' was for the receiver to develop \emph{coherence checking}. Coherence checking involves attending to both the internal coherence of the communicated message, and the external coherence with what the receiver already believes. Coherence checking, Sperber argues, is a useful defense against the risks of deception by the sender, because lies and other false claims are often externally or internally incoherent.
The relation between coherence checking and reasoning is not clear; Sperber
offers two observations that may lead us in the right direction:
\begin{quoting}
    Coherence checking involves a high processing cost, it cannot be done on a large scale because it would lead to a computational explosion, and it is fallible.
    \\ (\ldots) \\
    Coherence checking (\ldots) involves metarepresentational attention to logical and evidential relationships between representations
    \hfill \citep[p.~410]{Sperber01}
\end{quoting}
\todo[inline]{What's the purpose of this part above?}

The second step in the arms race was then for the sender to anticipate this coherence-checking by overtly displaying the coherence of their message to their receiver, which requires argumentative form; thus, testimony becomes argument.
The next steps were on the side of the receiver to develop skills for examining these displays of coherence (i.e., arguments), and on the side of the sender to 'improve their argumentative skills'.
\todo{Add forward reference to the relevant Chapter 3 section}
As Mercier and Sperber nicely describe it a decade later,
\begin{quoting}
    receivers' coherence checking creates selective pressure for communicators' coherence displays in the form of arguments, which in turn creates selective pressure for adequate evaluation of arguments on the part of receivers
    \hfill \citep[p.~96]{MS11}
\end{quoting}

\section{Mercier \& Sperber on intuitive vs. reflective inference}
\label{sec:MS09}

\todo[inline]{Rewrite this below paragraph}
In the \citeyear{MS09} paper \emph{Intuitive and reflective inferences}, Mercier and Sperber propose a dual system theory of reasoning. They do so because they are dissatisfied with current dual system theories of reasoning, accusing them of relying on "a bundle of constrasting features" rather than offering a "principled distinction" (p.~149), which therefore allows these theories some leeway in ascribing characteristic to the two systems.\todo{Comment in Ch. 4 on the hypocrisy of this?}

Mercier and Sperber introduce their dichotomy as part of a massive-modularist framework. This warrants some explanation; though I consider the modularity of the mind to be out of scope for this thesis, Mercier and Sperber's dual system theory is best explained through their views on modularity.

Mercier and Sperber are proponents of a massively modular view on the human mind, maintaining that the mind consists of a number of cognitive modules specialized to a specific domain. These modules are autonomous in their function, have distinct evolutionary and developmental histories, and they have characteristic inputs, procedures and outputs.
On this massively modular view of the mind, inferences can be performed by many different domain-specific modules. Even inferences that seem to be performed by domain-general module, such as logical inferences, are carried out by domain-specific \emph{metarepresentational modules}: modules that performs inferences on conceptual representations, Mercier and Sperber argue.
Because these conceptual representations may belong to any domain, metarepresentational inferences appear to be domain-general. However, the authors state, this domain-generality is indirect and virtual. In the words of Mercier and Sperber,
\begin{quoting}
    Metarepresentational modules are as specialized and modular as any other kind of module. It is just that the domain-specific inferences they perform may result in the fixation of beliefs in any domain.
    \hfill \citep[p.~153]{MS09}
\end{quoting}

Mercier and Sperber argue that one of the many modules of the mind is the argumentation module, which "provides us with reasons to accept conclusions" \citep[p.~155]{MS09}. The module takes as input a claim, and potentially other information relevant for evaluating this claim, and it produces as its output reasons for accepting or rejecting the claim.
The authors note that the direct output of an inferential module is \emph{intuitive}, in the sense that we accept the module's output without consciously attending to the reasons for this acceptance. This is also the case for the argumentation module. The direct ('intuitive') output of the argumentation module is then "the representation of an argument-conclusion relationship" (ibid.).
The difference then between intuitively accepting a claim and accepting it because of explicit reasons, is that in the latter case, one engages in "disembedding a conclusion from the argument that justifies it" (ibid.).
This disembedded conclusion from the direct output then constitutes the indirect output of the argumentation module.
\todo[inline]{Remark in Ch. 4 how incredibly vague this is: I've tried so hard to extract useful stuff from this paper but it's nigh infeasible}

From this view on the modularity of the mind and the argumentation module, Mercier and Sperber then develop a dualistic approach to inference, which they argue is a more principled distinction than classical dual-process theories.
Their account distinguishes between intuitive inferences on the one hand, and reflective inferences on the other. Reflective inferences are then what we would refer to as reasoning, or 'reasoning proper'.

When it comes to the exact definition of these two categories of inferences however, we unfortunately run into some ontological problems.
Namely, Mercier and Sperber define intuitive and reflective inferences in two different, seemingly incompatible ways. First, they state that
\begin{quoting}
    intuitive inferences the conclusion of which are the direct output of all inferential modules (including the argumentation module), and reflective inferences the conclusions of which are an indirect output embedded in the direct output the argumentation module \emph{(sic)}
    \hfill \citep[pp.~155--156]{MS09}
\end{quoting}
In other words, the \emph{conclusion} of an intuitive inference is the direct output of an inferential module, and the \emph{conclusion} of a reflective inference is an indirect output of the argumentation module. However, later on they state
\begin{quoting}
    Intuitive inferences are the direct output of many different modules. Reflective inferences are an indirect output of one of these modules.
    \hfill \citep[p.~156]{MS09}
\end{quoting}
We will return to these problems in \cref{sec:ont-atr}.
\todo[inline]{I don't have the time for this currently, but it seems that \citet[Chapters~3--9]{MS17} clarify some things and implicitly rectify some of these problems. I think. Read this over the summer?}
For now, let us briefly pivot to a later discussion of intuitive inference and argument in \citet[\S 1.1]{MS11}.

There, Mercier and Sperber posit that the processes that are executed by inferential modules are unconscious: though one might be aware of the output of such a process -- its conclusion -- one is unaware of the process itself. In other words, "All inferences carried out by inferential mechanisms are in this sense \emph{intuitive}" \citep[p.~58]{MS11}.
Importantly, they then highlight a distinction between inferences and arguments.
Inferences are processes: they take as input a representation, and output a representation.
Arguments on the other hand are representations themselves, resulting from inference. They are "the output of an intuitive inferential mechanism"; in particular, they are "representations of relationships between premises and conclusions" \citep[p.~58]{MS11}.
Both inferences and arguments have conclusions, but there is an ontological dissimilarity between the conclusion of an argument and the conclusion of an inference.
The conclusion of an inference is the output of the inference. Characteristically, the output of an inference is justified by the input of the inference; thus, we call this output a conclusion.
The conclusion of an argument, on the other hand, is part of the representation itself, i.e. part of the argument.

Pivoting back to Mercier and Sperber's \citeyear{MS09} discussion of intuitive and reflective inferences, they emphasize that their dual systems approach is different from the classical system 1/2 reasoning. They maintain that system 1 and system 2 operate at the same level, whereas intuitive and reflective inference do not: intuitive inferences are carried out by any module, but reflective inferences result specifically from the argumentation module -- more precisely, reflective inferences are an indirect output of the argumentation module \citep[p.~156]{MS09}.

Moving beyond their dual system approach, Mercier and Sperber go on to provide yet more foundation to their argumentative theory of reasoning by discussing the function of reflective inference -- in other words, the function of reasoning.
They state that the function of intuitive inference is less controversial (but do not explicate what it is).
With regards to the function of reflective inference, they first discuss three 'classical' views on the function of system 2 reasoning.
The first view maintains that system 2 represses system 1's impulses seeking immediate gratification, in order to obtain delayed gratification \citep{Sloman96}. Mercier and Sperber argue that this cannot be the function of reasoning, since non-human animals also possess the ability to delay gratification and moreover, empirical case studies discussed in \citet{Damasio94} suggest that abilities for delayed gratification are dissociated from abilities for reasoning.
The second view maintains that system 2 reasoning enables us to better deal with novelty \citep{EvansOver1996}. Against this, Mercier and Sperber argue that it is implausible that this is the function of reasoning, for there are other features of human cognition that better explain and support this ability. Moreover, they state that reasoning cannot be said to play a central role in memory and imagination.
These views have in common that system 2 'compensates' for the shortcomings of system 1.
This idea culminates in the third view on the function of reasoning that Mercier and Sperber discuss. This is the view that the function of reasoning is to enhance individual cognition, or even stronger, this view concerns the Cartesian conception of reasoning as 'the road to knowledge'.
They argue that this is evolutionarily implausible due to a cost-benefit issue.
Reasoning is cognitively costly, and intuitive inference is not so unreliable that using reflective inference instead would be, on the whole, advantageous.

In the remainder of the paper, the authors discuss predictions stemming from their claimed function of reasoning and empirical evidence corroborating these. We will return to these and more empirical predictions of the ATR in \cref{sec:MS11}, since Mercier and Sperber's \citeyear{MS11} paper goes into much more detail concerning these.

\section{Sperber and colleagues on epistemic vigilance}
\label{sec:Sperber10}

% my prologue
Further building upon his \citeyear{Sperber01} views, Sperber and his colleagues (among whom, notably, Hugo Mercier) introduced the concept of \emph{epistemic vigilance} in a seminal \citeyear{Sperber10} paper. This concept constitutes a cornerstone of Sperber \& Mercier's later argumentative theory of reasoning. Therefore, a comprehensive discussion is in order here -- not in the least because this concept will be one of my targets of scrutiny in \cref{ch:scrutiny}.

% introduction
In their \citeyear{Sperber10} paper, Sperber and colleagues start by emphasizing that humans are dependent on communication, and they argue that this dependence leaves humans vulnerable to being deceived by others.
They state that misinformation or deception may "reduce, cancel, or even reverse" the gains that communication can bring to the addressee (p.~360).
Consequently, the information that an addressee receives from a communicator is only advantageous to her to the extent that the information is genuine.
Sperber and colleagues thus conclude that for this purpose, humans have evolved a suite of cognitive mechanisms for \emph{epistemic vigilance}.
Moreover, this suite of mechanisms must have evolved alongside, and is used in tandem with, abilities for ostensive-inferential communication\footnote{Slightly confusingly, Sperber et al. call the ostensive-inferential communication that we saw in \cref{sec:comm:definition} 'overt intentional communication' in this paper. However, they refer to Sperber \& Wilson's \emph{Relevance Theory}, which calls this ostensive-inferential communication. So ultimately, they are talking about the same thing.}, because they work in tandem to facilitate trust calibration on the side of the receiver.

% epistemic trust and vigilance
In order to illustrate and somewhat demarcate the concept of epistemic vigilance, Sperber and colleagues discuss some work in philosophy and psychology related to trust and vigilance. Specifically, they consider different views on the question of whether humans are 'per default' trusting or vigilant.
Of this discussion, two points stand out to me as noteworthy, especially as they pertain to the discussion to come in \cref{sec:EV-scrutiny}. The first of them concerns a characterization of vigilance that nicely captures its spirit:
\begin{quoting}
    Vigilance (unlike distrust) is not the opposite of trust; it is the opposite
of blind trust
    \hfill \citep[p.~363]{Sperber10}
\end{quoting}
The second concerns a strong claim, which we will critically assess in \cref{sec:EV-scrutiny}:
\begin{quoting}
    in communication, it is not that we can generally be trustful and therefore need to be vigilant only in rare and special circumstances. We could not be mutually trustful unless we were mutually vigilant.
    \hfill \citep[p.~364]{Sperber10}
\end{quoting}

% comprehension and acceptance
Next, the authors move on to discussing comprehension and acceptance of utterances in communication, and how these relate to epistemic vigilance and trust.
They argue that a communicative act does not only trigger comprehension in the addressee, but it also triggers epistemic vigilance alongside it. If epistemic vigilance then "does not come up with reasons to doubt" (p.~369), this comprehension leads to acceptance.
They go on to argue that comprehension of an utterance is not "guided by a presumption of truth", as other theorists state, but rather by an "expectation of relevance" (p.~367); see \citet{SperberWilson86}. This expectation of relevance requires a 'stance of trust' of the addressee regarding the speaker (this relates to the Gricean cooperativeness we discussed in \cref{sec:comm:function}).
This stance of trust of the addressee is "tentative and labile" (p.~368), and epistemic vigilance is (as mentioned) active alongside this stance of trust.
\todo{Tie this to the discussion of the ostensive-inferential model in §1}

% Sperber and colleagues maintain that vigilance is not a nicety, something that is only invoked sometimes; they maintain that vigilance is the default disposition of interlocutors in communicative settings.

To further explicate epistemic vigilance as a concept, Sperber and colleagues outline a distinction between vigilance towards the \emph{source} of a message (the 'who'), and vigilance towards the \emph{content} of the message (the 'what').
% vigilance towards the source
As for vigilance towards the source, they note that the reliability of a source depends on two factors: a reliable source must be competent, and a reliable source must be benevolent.
Moreover (and importantly), a receiver's vigilance towards the sender as a source of information -- in other words, the sender's perceived trustworthiness -- is dependent on the context: it varies per topic and per situation.
Because of this, it is important for a receiver to accurately calibrate her trust in the sender depending on the context.
They go on to discuss empirical evidence that corroborates that trust, and calibrating trust to the situation, is indeed important to us.
Moreover, on the other side of the coin, they note that deceiving people can be quite beneficial: experiments from deception detection research show that people are not good at detecting lies based on non-verbal behavioral cues.
They end this particular discussion by noting that more empirical research is needed about how people calibrate their trust in everyday communication, outlining some desiderata for this research.

% the development of epistemic vigilance and mindreading: SKIPPED THIS

% vigilance towards the content
Moving on now to vigilance towards the \emph{content} of a message, Sperber and colleagues restate that comprehension and epistemic vigilance are two processes that are intertwined to some extent. Specifically, they note that one mechanism of comprehension, namely the search for relevance, provides a basis for an "imperfect but cost-effective epistemic assessment" (p.~374).
They discuss belief revision and the role that coherence checking plays in it. We already saw \poscite{Sperber01} discussion of coherence checking; Sperber and colleagues now describe coherence checking a mechanism for epistemic vigilance. They note that coherence checking "takes advantage of the limited background information activated by the comprehension process itself" (p.~375). They argue that the search for relevance "automatically involves the making of inferences which may turn up inconsistencies or incoherences relevant to epistemic assessment" (p.~376).

% epistemic vigilance and reasoning
Next, the authors return to and expand upon an idea we have seen \citet{Sperber01} propose, concerning the emergence of argumentation as a demonstration of coherence. I will discuss this in much more detail in \cref{sec:exp-atr}, as this part of \citet{Sperber10} basically constitutes a rudimentary explication of the argumentative theory of reasoning.

To summarize, according to Sperber and his colleagues humans have developed a suite of mechanisms for epistemic vigilance, filtering incoming information in order to avoid being deceived by others. A communicative act triggers both comprehension and epistemic vigilance, and the epistemic assessment of the communicative act draws upon some of the inferential steps that are carried out in the search for relevance, which makes the assessment relatively cost-effective. Epistemic vigilance can be directed towards the source of a message or towards the content of the message. This amounts to the calibration of trust and coherence checking, respectively.

\section{Mercier \& Sperber's argumentative theory of reasoning}
\label{sec:MS11}

The key features and aspects of Mercier and Sperber's theory were already laid out either in detail or in a rudimentary form in the papers we have seen in the previous sections.
In this section, I will first summarize the full argumentative theory of reasoning, and then consider the theory's empirical predictions and the corroborating evidence \citet{MS11} bring to the table.

\subsection{The ATR summarized}

In order to get a good overview of the ATR, it will be illuminating to consider the theory in something of an argument form. (The following is mostly paraphrased from the discussion of the ATR in \citep[p.~60]{MS11}.)

\begin{enumerate}[label=(\arabic*)]
    \item Humans are dependent on cooperation with other humans for survival, and communication is crucial for this: "Communication plays an obvious role in human cooperation both in the setting of common goals and in the allocation of duties and rights" \citep[p.~60]{MS11}.
    \item For communication between humans to have stabilized over the course of evolutionary history like it has, it must have been advantageous to both the senders and the receivers of messages. If it were not, the practice would have collapsed over time \citep{Sperber01}.
    \item It is advantageous for senders to be dishonest -- "communicators commonly have an interest in deceiving" \citep[p.~160]{MS09}. Frequent deception threatens the stability of communication, because it renders communication disadvantageous to the receiver \citep{Sperber01}.
    \item To protect themselves against deception, receivers need to (and have evolved the means to) exercise \emph{epistemic vigilance} in order to filter incoming information \citep{Sperber10}.
    \item In order to then get her message across to a vigilant receiver, a sender may demonstrate the coherence of her claims by offering an argument as a reason to accept her claim.
    \item This demonstration of coherence -- i.e., this argument -- is produced by reasoning, and the evaluation of the argument by the receiver is also facilitated by reasoning.
    \item Thus, the function of reasoning is the production of arguments (by the sender) and the evaluation of arguments (by the receiver). In other words, reasoning has emerged and persisted throughout evolutionary history precisely because it enables the production and evaluation of arguments, which plays a critical role in ensuring the stability of interpersonal communication, which ultimately contributes to humans’ survival.
\end{enumerate}

\subsection{Empirical predictions and evidence}

\citet{MS11} point out that evolutionary hypotheses are at risk of coming across as "just so stories" if they are not buttressed by empirical evidence. They argue:
\begin{quoting}
    To establish that reasoning has a given function, we should be able at least to identify signature effects of that function in the very way reasoning works.
    \hfill \citep[p.~60]{MS11}
\end{quoting}
Consequently, they outline four predictions that follow from the ATR and they present empirical evidence that corroborates these predictions.

The first prediction of the ATR is that reasoning is best adapted to perform tasks in argumentation. In other words, reasoning is good at producing and evaluating arguments, and it works best in an argumentative context, e.g. in group discussions.
Many classical findings from the psychology of reasoning, such as the Wason selection task \citep{Wason68}, conclude that people are poor logical reasoners. Mercier and Sperber note however that people's performance on this kind of task improves once it moved from a nonargumentative to an argumentative setting. They cite evidence that people are generally good at spotting fallacies in others' arguments, and that they are skilled at recognizing the structure of arguments.
Moreover, the authors discuss findings on group reasoning tasks, in which participants are first tasked with solving problems individually, then in a small group, and lastly individually again. On for example the Wason selection task, participants' performance improved dramatically in a group setting \citep{Moshman98}.

The second prediction is that reasoning shows a confirmation bias.

Thirdly, the ATR predicts that solitary reasoning anticipates argument, in a form of \emph{motivated reasoning}.

Lastly, the ATR predicts that reasoning in decision-making guides people not to the optimal decision, but to the decision that is most easily justified.

\section{Conclusion}
