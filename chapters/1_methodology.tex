\chapter{The chicken and the egg: the evolutionary approach}
\label{ch:evolution}
% introduction

Before we are able to answer any \emph{why}-questions about humans' cognitive capacities, some groundwork needs to be laid out. For what does it mean for some trait to evolve 'for the purpose of' another trait? Are we even justified in using this kind of terminology when it comes to evolution, a process which cannot be said to be intentional or purposeful? And what intermediate questions will we need to ask ourselves in order to ultimately answer the question of why we reason and communicate?
This chapter attempts to answer these questions. It by no means provides an overview of issues in evolutionary theory, a large field of research in its own right with widely diverging opinions on a number of specifics of the process (see \citet{Ariew02} and \citet{UllerLaland19} for overviews of topics in evolutionary theory).
This chapter will merely serve to get a number of issues 'out of the way' before we can continue our investigation into the cognitive mechanisms of reasoning and communication.

\section{Causation in evolution}
\label{sec:causation-evolution}

First, we will dip our toes into the waters of causation in evolution.

Evolutionary causation is a subfield of philosophy of biology that has continued to see widely diverging opinions \citep{Baedke2021, UllerLaland19}. In this section I will restrict focus on three topics in evolutionary causation that are of interest to this thesis: proximate vs. ultimate causation; Tinbergen's four questions; and reciprocal causation and niche construction.
The former two will serve to illuminate some basic aspects of the issues evolutionary causation concerns itself with, and the latter is of particular interest in considering the evolution of human cognition.

In evolutionary causation, one may distinguish \emph{proximate} from \emph{ultimate} causes.
Proximate causes are the \emph{immediate} influences on a trait: they explain how the trait results from the internal and external factors causing it.
Ultimate causes on the other hand are the higher-level historical and evolutionary explanation of those traits \citep{Mayr61}\todo{Add an example to illustrate?}. In other words, these two different causes answer two different explanatory questions: the proximate cause is the answer to the \emph{how}-question, whereas the ultimate cause answers the \emph{why}-question. According to Mayr, who pioneered the distinction \citep{Laland13}, one needs to answer both of these questions in order to obtain a complete understanding of the phenomenon at hand.

In an influential proposal considered by some to be an extension of this distinction, \citet{Tinbergen63} outlined four questions or problems central to the study of animal behavior. In order to fully understand a pattern of behavior, one must consider (1) the proximate causation of the behavior, (2) the lifetime development of the behavior, (3) the function\footnote{See \cref{sec:teleology} for a discussion of function in biology.} of the behavior, and (4) the evolutionary history of the behavior.

While Mayr's and Tinbergen's frameworks have had considerable explanatory value in the field of evolutionary causation, it has been argued \citep{Laland13} that biologists would be better off rejecting them in favor of a framework of \emph{reciprocal causation}.
In reciprocal causation, there is feedback between the organism and its environment: not only does the environment influence the organism through the process of natural selection, but the organism in turn influences its environment through the process of \emph{niche construction} \citep{Svensson18}, where organisms actively partake in modifying their environment and shaping their niche, usually to their own benefit.\todo{Add an example to illustrate?} It has been argued that niche construction is at least as important as natural selection in shaping evolution, and moreover that especially in evolutionary processes involving interactions between organisms, reciprocal causation is very frequent \citep{Svensson18}.
% Both concepts reject the unidirectional conception of evolutionary causation, and ascribe a causal role to the organism itself and acknowledge the existence of feedback loops.

% \section{Evolutionary approaches to human psychology}
\section{Evolutionary psychology and its issues}
\label{sec:evol-psych}
In order to answer our question, we will also need to zoom out to consider the field of evolutionary psychology as a whole. What is the merit, and the validity, of adopting an evolutionary approach in our endeavor?

The field of evolutionary psychology concerns itself with trying to understand human behavior using evolutionary theory, by looking into the past and considering how our ancestors must have adapted to their environment in order to survive and reproduce.
Researchers in the social sciences and humanities have historically been wary of using evolutionary approaches to study human behavior, because evolutionary theory has been abused for prejudiced ends in the past; see \citet[pp.~19--20]{LB02} for an overview. Moreover, evolutionary-psychological research has received the criticism that too much of it is "just-so" storytelling and post-hoc explanation of known phenomena, sometimes accompanied by a sensationalist spin on the story \citep{LB02}.
However, if these pitfalls are avoided, looking at the human psyche from the evolutionary perspective can be an illuminating endeavor.

Let us now consider some of the central concepts and assumptions of evolutionary psychology.

In order to explain humans' psychological mechanisms, evolutionary psychologists look to the concept of an \emph{environment of evolutionary adaptedness} (EEA). The EEA is the environment in which these psychological mechanisms must have come into being; usually the EEA is identified as hunting and gathering groups on the African savannah in the second half of the Pleistocene, between 1.7 million and 10,000 years ago \citep{LB02}.
The assumptions underlying the use of the concept of an EEA are that (1) our modern-day environment is too different from that of our ancestors for us to use it to explain why and how our psychological mechanisms evolved in the past; and (2), for our psychological mechanisms to be as complex as they are, they must have evolved slowly -- and thus a considerable amount of time ago, without changing significantly since the Stone Age.

There are a number of issues associated with the use of the concept of the EEA \citep{LB02}. Firstly, we do not know very much about the environment of our ancestors, so the specifics of the EEA may be filled in as is seen fit for one's purpose. Secondly, we do not know enough about the process of evolution to make assumption (2); while evolution does in general operate on a large timescale, research has shown that it can also be faster, operating on a time scale of thousands of years, or less than 100 generations \citep[pp.~190--191 and references therein]{LB02}. Thirdly, the argument can be made that for our species to have flourished and dominated in the way that it did, we must have remained adaptive to our changing modern environments after the Stone Age. Lastly, the EEA argument does not take into account reciprocal causation or niche construction.

Another topic of discussion in evolutionary psychology that is of importance to our investigation, is that of domain-specificity of the psychological mechanisms. The argument has been made \citep[p.~50]{Buss15} that these adaptive mechanisms are necessarily problem- or domain-specific, because the evolutionary process would not favor general solutions to specific problems.
However, as with the EEA, issues with this stance have been raised \citep{LB02}: the push to domain-specificity can be said to rely on overly strong assumptions about the modularity of the brain; and moreover, there is also a push to domain-generality of cognitive skills because domain-general skills are neurologically more cost-efficient than domain-specific skills.

Lastly, a criticism raised to evolutionary psychology is that not only biological evolution, but also cultural evolution can be said to have played a role in shaping human behavior and human cognitive capacities; it is a joint endeavor facilitated by nature and nurture.

\section{Teleological notions in evolutionary theory}
\label{sec:teleology}
\todo[inline]{This section needs more work and stronger argumentation: I address some concerns with this concept, but don't feel like I adequately address/counter these concerns}
Next, it is useful to scrutinize the terminology that I will be using throughout this thesis.
Biological literature frequently makes use of \emph{teleological} terminology, that is, terminology that implies goal-directedness of the processes it describes. Such terminology includes concepts like the \emph{design} of a trait, and \emph{function}, \emph{purpose}, or \emph{utility} of a trait.
At first glance, the usage of these terms in discussing evolution would seem to be inappropriate; evolution is a natural process, that is not purposefully performed by an agent, and is without any intentionality or goals. And indeed, this teleological terminology has its origin in pre-Darwinian and creationist views on the origin of species.

However, explanations in terms of goals and function have considerable instrumental value in describing evolutionary processes. Throughout this thesis, I will be adhering to the conception of teleological explanations of \citet{Ayala99}, which is as follows:
\begin{quoting}
    Teleological explanations account for the existence of a certain feature in a system by demonstrating the feature’s contribution to a specific property or state of the system, in such a way that this contribution is \emph{the reason why the feature or behaviour exists at all}.
    \hfill (p.~13)
\end{quoting}
In this respect, the evolutionary process of adaptation merits a teleological explanation: the function of a trait (its 'contribution to a specific property or state of the system') is the reason that the trait exists, because it exists as a consequence of natural selection.
\todo{What about reciprocal causation and niche construction? Or cultural learning?}

The distinction between proximate and ultimate causes we saw in \cref{sec:causation-evolution} can be applied to teleological explanation as well, yielding the distinction between proximate and ultimate \emph{ends} of features. The proximate end is then the 'immediate' function the feature serves, and the ultimate end is the reproductive success of the organism.

A footnote to this account is that not all features of organisms can be explained teleologically; only if the feature has arisen and persisted as a direct result of natural selection, is a teleological explanation in place.
\todo{This "direct result of natural selection" is very vague, address this?}

In general, teleological explanations in biology are quite controversial: not only is the usage of the specific terminology itself debated \citep[p.~27 and references therein]{Ayala99}, the concept has been criticized for its apparent lack of formalization and insufficient argumentative persuasiveness \citep[p.~83]{Baedke2021}.

\section{An evolutionary foundation}
Now that we have gathered the puzzle pieces, let us return focus to our field of investigation and lay its groundwork. In particular, before we can continue with our evolutionary approach to human reasoning and communication in the next chapters, we will need to outline and justify the assumptions made.

It has become apparent in this chapter that for none of the topics in evolutionary theory discussed here consensus has been reached among its practitioners. Since the purpose of this thesis will not be to provide a complete causal framework for the evolution of reasoning and communication, we will be able to evade some of the issues plaguing the frameworks discussed in this chapter. We will proceed cautiously, using the concepts outlined without needing to account for their shortcomings.

\paragraph{Why can we consider these cognitive capacities from an evolutionary perspective?}

In choosing to explore human reasoning and communication from the evolutionary perspective, I am working on the assumption that it is appropriate to the endeavor: reasoning and communication are two key features of the homo sapiens that must have arisen at some point in our evolutionary history, shaped by natural selection, niche construction, and perhaps (cultural) learning. Moreover, it is justified to assume that the cognitive abilities underlying human reasoning and communication are heritable, either genetically or culturally.

\paragraph{What do we mean when we say 'evolution'?}
\todo[inline]{More discussion on biology vs. culture would be in place in this chapter: I think adding things from \citet{Heyes18} would complete this rather vague argument, and maybe \citet{Laland13} and \citet{LB02}.}
So far, we have mostly considered the concepts and issues surrounding \emph{biological} evolution, and the role of \emph{culture} has remained largely undiscussed.
In this thesis, I will use the term 'evolution' and associated terminology to talk about the development of cognitive capacities in humans over time, remaining agnostic about whether this evolution is due to biological processes like natural selection, or cultural processes like cultural learning.

\paragraph{What are our assumptions about the evolutionary process?}
I accept the concept of the EEA in a general sense despite the flaws discussed in \cref{sec:evol-psych}; for our purposes, we need not commit to any strong assumptions about the nature and properties of the EEA. The most important assumption I will make is that homo sapiens throughout history has been dependent on strong social groups for survival.
\todo{Missing: strong argument for making this assumption; see \citet[p.~178]{LB02} for doubts about this assumption}

Moreover, I will adopt a reciprocal causal framework along the lines of \citet{Laland13} and \citet{Svensson18}, maintaining that humans throughout history have not only been the object of natural selection, but that they have also been the subject in shaping their environments through the formation of social groups and development of technology.

\paragraph{What does it entail for one trait to evolve for the purpose of another?}
\todo[inline]{I think this could be stronger}
In using \poscite{Ayala99} conception of teleological notions, we can speak of some feature of an organism having a certain function or purpose, that ultimately contributes to its reproductive success. In the case of our investigation, we are interested in what evolutionary benefits one feature (reasoning) may have to another feature (communication). In order to answer this question, we should first consider what the function of communication is to humans; only then can we consider whether the function of reasoning could be to advance communication and in doing so improve the reproductive fitness of humans.

\paragraph{What questions will we need to answer?}
In exploring cognitive capacities in more detail from an evolutionary perspective, we will loosely follow \poscite{Tinbergen63} questions by considering the following aspects of reasoning and communication:
\begin{itemize}
    \item How does the capacity to reason/communicate develop throughout childhood?
    \item What would advancements in reasoning/communication contribute to the reproductive success of humans? In other words, what is the function of reasoning/communication?
    \item What is the evolutionary history of reasoning/communication? To what extent and how does it present in our nearest evolutionary relatives?
\end{itemize}
One may note that we will skip Tinbergen's first question about what the proximate causation of the behavior in question is; I consider the neurological mechanisms underlying reasoning and communication to be out of the scope of this thesis, not in the least because not nearly enough is known about the human brain and mind to be able to give a satisfactory account of the proximate causation of reasoning and communication.
