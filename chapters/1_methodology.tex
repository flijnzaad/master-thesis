\chapter{The chicken and the egg}

% introduction

Before we are able to answer any \emph{why}-questions about humans' cognitive capacities, some groundwork needs to be laid out. For what does it mean for some trait to evolve 'because of' or 'for the purpose of' another trait? Are we even justified in using this kind of terminology when it comes to evolution, which is a process which cannot be said to be intentional nor purposeful? And what questions will we need to ask ourselves in order to ultimately answer that big  \emph{why}-question?
This chapter attempts to answer these questions. It by no means provides an overview of issues in evolutionary theory and philosophy of biology; these programmes are large fields of research in their own right, with widely diverging opinions on a number of specifics of evolution; see \citet{Ariew02} and \citet{UllerLaland19} for overviews of topics in evolutionary theory and philosophy of biology.
This chapter will merely serve to get a number of issues out of the way before we can continue our investigation into the cognitive mechanisms that make us human.

\section{Evolutionary approaches to human psychology}

In order to answer our question, we start by zooming out to consider the field of evolutionary psychology. What is the merit of adopting an evolutionary approach in our endeavor?

\section{Teleological notions in evolutionary theory}

Next, it is useful to scrutinize the terminology that I will be using throughout this thesis.

\section{Causation in evolution}

Now, we will dip our toes into the topic of causation in evolution.

\section{An evolutionary foundation}

Now that we have gathered sufficient puzzle pieces, it is time to take the score and lay the groundwork.
