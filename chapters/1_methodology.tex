\chapter{The chicken and the egg: the evolutionary approach}
\label{ch:evolution}

Before we are able to answer any \emph{why}-questions about the evolution of reasoning and communication, some groundwork needs to be laid out. For what are the processes underlying evolution, and how does evolutionary causation work? What does it mean for some trait to evolve 'for the purpose of' another trait? Are we even justified in using this kind of terminology when it comes to evolution? And what intermediate questions will we need to ask ourselves in order to ultimately answer the question of why we reason and communicate?

This chapter serves to answer these and related questions. It by no means provides a comprehensive overview of the issues in evolutionary theory, since this is a vast field of research in its own right with widely diverging opinions on a number of specifics of the process of evolution \footnote{See \citet{Ariew02} and \citet{UllerLaland19} for an overview of topics in evolutionary theory and evolutionary causation.}.
The purpose of this chapter is to touch on a number of issues in the field that are relevant to our endeavor, such that we have a stronger foundation for our investigation into the evolution of reasoning and communication.

\section{Evolution: biological vs. cultural}
\label{sec:evo-bio-culture}

Although the term 'evolution' is in everyday usage most commonly interpreted as 'Darwinian evolution', or 'natural selection', the concept of evolution can be stripped down to a very broadly construed version, which may prove to be illuminating.

In general, any process of selection can be taken to consist of three consecutive steps: (1) variation, (2) sorting\footnote{\citet{Donahoe03} uses the term 'selection' for this step; I follow \citet{Heyes18} and \citet{S-P13} in using the term 'sorting', to avoid confusion with the full process of selection, which consists of the three steps outlined here.}, and (3) retention \citep{Donahoe03}. I will first discuss how each of the steps of this process are construed in standard evolutionary theory, and then we will consider how the process of cultural evolution fits this definition of selection.

Firstly, in standard evolutionary theory the processes that introduce variation are mutation and migration, where mutation concerns the changes in an organism's DNA and migration concerns the movement of (genetic material of) organisms from one population to another \citep{S-P13}.
Variation by itself is undirected \todo{That is... explain this very briefly}; it is only due to sorting that the selection process becomes directed \citep{Donahoe03}.
% \todo[inline]{Add a bit more on sorting: see \citet{Sperber10} notes for an example. Try to map it to what people already know, because they're probably familiar with the process, but just not with the term.}
% \todo[inline]{Make sure you yourself are clear on the difference between the higher-level process of selection, and the lower-level process of sorting.}

Secondly, in standard evolutionary theory the sorting process amounts to natural selection as well as genetic drift. We will focus here on the former of these two processes\footnote{In a nutshell, genetic drift can be described as a probabilistic sorting process, resulting from a sampling error due to populations being finite in size. However, genetic drift is a hotly debated concept in evolutionary theory (see \citet{Millstein21} for an overview), so I mention it here only for the sake of completeness.}.
Natural selection \todo{Confusion of selection vs. sorting here: resolve}
acts upon the variation introduced by mutation and migration such that the genes that enhance an organism's fitness persist over time in the population \citep{S-P13}. An organism's fitness corresponds to how likely they are to leave offspring in the next generation compared to organisms with a different genetic makeup. This fitness relates to both the organism's chances of survival as well as its chances of reproducing.

Thirdly, in standard evolutionary theory, the process responsible for retention is genetic inheritance, i.e. the transmission of characteristics from parents to their offspring through the genetic material (DNA) parents impart on their offspring.

Now, the evolution of \emph{culture} can also be said to operate by these principles \citep{Heyes18}. In this context, culture is understood at its core as \emph{information}: more specifically, it is information "that we inherit from others through social interaction (via certain kinds of social learning)" \citep[p.~30]{Heyes18}.
Let us now consider the cultural processes underlying each of the three steps of the selection process.

\todo[inline]{Add one big example for this explanation, that you can refer back to with each step}
Firstly, variation in culture is introduced by error and by innovation.
Secondly, sorting of behaviors and habits in culture can happen through two different routes. A behavior can be sorted (selected for) because of some property inherent to the behavior that makes it "more noticeable, learnable, or memorable than others" \citep[p.~34]{Heyes18} and thus more likely to be copied. Also, a behavior or habit may be sorted in a more 'classic' evolutionary way: a habit may be selected for because it improves the fitness of the individual, such that individuals with that habit are more likely to survive and reproduce than individuals with an alternative habit.
Thirdly, culture may be retained through cultural inheritance, that is, through mechanisms of social learning.

Whether or not the process of cultural evolution can be taken to be analogous to that of biological evolution, is not an uncontroversial issue \todo{Either find a reference, or remove this remark}, cf.~\citet{Claidiere14} for a discussion and a formal account of cultural evolution.
In this thesis however, I will follow \citet{Heyes18} in the assumption that cultural evolution is 'Darwinian'; that is, it abides by mechanisms that are analogous to those of biological evolution.
As a consequence, in discussing how reasoning and communication evolved, we may remain agnostic on the kind of evolution responsible for this evolution, since the mechanisms underlying biological and cultural evolution are assumed to be the same.
Thus, for the remainder of this thesis, I will use the term 'evolution' to talk about the development of characteristics (in our case, cognitive capacities) in humans over time, remaining agnostic about whether this development is due to biological evolution or cultural evolution.

% \thinkL{Possibly add to this section, !if it turns out to be relevant for my argument!: (1) Something about gene-culture coevolution (though, can that be said to be Darwinian?) (2) Something about the hypotheses we need to form to make this theory complete, that \citet[Chapter~2]{Heyes18} talks about: about variants and quantifiable differences between them; about routes of inheritance (vertical / oblique / horizontal); about mechanisms of inheritance.}

\section{Causation in evolution}
\label{sec:causation-evolution}

Next, we will dip our toes into the waters of causation in evolution. As it turns out, causation in evolution is not a simple notion; let us consider for example a moth whose wings provide it with camouflage due to their coloration. The camouflage of the wings is an \emph{effect} of their coloration; yet, it is precisely the camouflage the coloration provides that is the \emph{cause} of the coloration being present at all \citep{Lipton09}.

Evolutionary causation is a subfield of philosophy of biology that has continued to see widely diverging opinions \citep{Baedke2021, S-P13}. In this section I will restrict focus to four topics in evolutionary causation that are of interest to this thesis. First, I will discuss the distinction between proximate and ultimate causation. Then, I will briefly cover \poscite{Tinbergen63} questions for explaining animal behavior, which will be discussed at length in \cref{sec:tinbergen}. Thirdly, we will have a look at niche construction and reciprocal causation. Lastly, we circle back to our research question and discuss what it entails for one trait to evolve for the purpose of another.

In evolutionary causation, one may distinguish \emph{proximate} from \emph{ultimate} causes \citep{Mayr61}.
Proximate causes are the immediate influences on a trait: they explain how the trait results from the internal and external factors causing it. For example, the proximate cause of the coloration of our moth's wings would be the biochemical processes during the moth's development that result in that particular pattern of colors.
\todo{Still a bit clunkily written}
Ultimate causes, on the other hand, provide the higher-level historical and evolutionary explanation of those traits. In the case of the moth, the ultimate cause for the coloration of its wings is the enhanced fitness of the moth, due to the camouflaging properties of the coloration \citep{Lipton09}. In other words, these two different causes relate to two different explanatory questions: the proximate cause is related to the \emph{how}-question (\emph{how} did a trait come about?), whereas the ultimate cause is related to the \emph{why}-question (\emph{why} did a trait come about?). According to \citet{Mayr61}, who pioneered the distinction\footnote{See \citep{Laland13} for a discussion of the exact origins of the distinction.}, one needs to answer both of these explanatory questions about a trait in order to obtain a complete understanding of it.

In a seminal proposal considered by some to be an extension of Mayr's dichotomy \citep{Laland13}, \citet{Tinbergen63} outlined four questions central to the study of animal behavior. In order to fully understand a pattern of behavior, he argued, one must consider (1) the proximate causation of the behavior, (2) the lifetime development of the behavior, (3) the function\footnote{See \cref{sec:teleology} for a discussion of function in biology.} of the behavior, and (4) the evolutionary history of the behavior.
Since Tinbergen, other authors have grouped these four questions according to Mayr's proximate-ultimate distinction, characterizing the 'causation' and 'development' questions as proximate questions (\emph{how}-questions) and the 'function' and 'evolution' questions as ultimate questions (\emph{why}-questions) \citep{BatesonLaland13, Laland13}.
Tinbergen's framework has had an extensive and lasting influence on the study of animal behavior, and his questions continue to be used by biologists to this day \citep{BatesonLaland13}. Hence, I will let this framework inform the methodology for this thesis and will thus discuss it in greater detail in \cref{sec:tinbergen}.

\todo[inline]{Insert here discussion of costs and benefits. Make it general, don't talk yet about reasoning and communication: keep it at the same level as the rest of the chapter. Integrate the below paragraph with it.}

Lastly, we briefly touch on what it means for one trait to evolve because of, or for the purpose of, another trait. In this thesis, we are interested in what evolutionary benefits one feature (reasoning) may have to another feature (communication). In order to answer this question, we should first consider what the function\footnote{See \cref{sec:teleology} for a discussion of function in biology.} of communication is to humans; only then can we consider whether the function of reasoning could be to advance communication and in doing so, improve the fitness of humans.
% \todo{Reasoning vs. the ability to reason: delve into this somewhere. Behavior vs. the ability to exhibit behavior. The convention in the evolutionary literature is to talk about these behaviors at a higher level of abstraction: mention this somewhere.}
% \todo{This last paragraph deserves more discussion}

\section{Evolutionary psychology}
\label{sec:evol-psych}
In order to answer the question of whether reasoning evolved for the purpose of communication, we will also need to zoom out to consider the field of evolutionary psychology as a whole.
What is the merit, and the validity, of adopting an evolutionary approach to human psychology?

The field of evolutionary psychology concerns itself with trying to understand human behavior using evolutionary theory, by looking into the past and considering how our ancestors must have adapted to their environment in order to survive and reproduce.
Researchers in the social sciences and humanities have historically been wary of using evolutionary approaches to study human behavior, because evolutionary theory has been abused for prejudiced ends in the past; see \citet[pp.~19--20]{LB02} for an overview. Moreover, evolutionary-psychological research has received the criticism that too much of it is "just-so" storytelling and post-hoc explanation of known phenomena, sometimes accompanied by a sensationalist spin on the story \citep{LB02}.
However, if these pitfalls are avoided, looking at human psychology from the evolutionary perspective can be an illuminating endeavor.
Let us now consider some of the central concepts and assumptions of evolutionary psychology.

In order to explain humans' psychological mechanisms, evolutionary psychologists look to the concept of an \emph{environment of evolutionary adaptedness} (EEA). The EEA is the environment in which these psychological mechanisms must have come into being; usually the EEA is identified as hunting and gathering groups on the African savannah in the second half of the Pleistocene, between 1.7 million and ten thousand years ago \citep{LB02}.
% \todo{Are the groups the EEA or the savannah? Elaborate more on this: is it the physical environment or does it include the groups? Aren't the groups a trait that evolved? Think about this}
The assumptions underlying the use of the concept of an EEA are that (1) our modern-day environment is too different from that of our ancestors for us to use it to explain why and how our psychological mechanisms evolved in the past; and (2) for our psychological mechanisms to be as complex as they are, they must have evolved slowly; because of this, they must have evolved a considerable amount of time ago without changing significantly since the Stone Age.

There are a number of issues associated with the use of the concept of the EEA \citep{LB02}. Firstly, we do not know very much about the environment of our ancestors, so the specifics of the EEA may be filled in as is seen fit for one's purpose. Secondly, we do not know enough about the process of evolution to make assumption (2); while evolution does in general operate on a large timescale, there is empirical evidence
% \todo{Probably good to mention the nature of this evidence, because this evolutionary timeline point is important}
that the process can also be faster, operating on a timescale of thousands of years, or less than 100 generations \citep[pp.~190--191 and references therein]{LB02}. Lastly, the argument can be made that for our species to have flourished and dominated in the way that it did, we must have remained adaptive to our changing modern environments after the Stone Age.
% Lastly, the EEA argument does not take into account reciprocal causation or niche construction.
% \researchL{Elaborate on this last issue, because it seems to be important for my argument. Also, it seems like a strange point to make, because especially with humans, it seems obvious that niche construction is a thing that might play a role. Why would the EEA concept rely on assumptions of unidirectional causation?}

Despite the issues associated with the concept of the EEA, it is instrumentally valuable in reminding us to consider the state of the environment and its role in the evolutionary process. For the purposes of this thesis, we need not commit to any strong assumptions about the nature and properties of the EEA. The most important assumption I will make \todo{Add scholars or references here} is that humans throughout history have been dependent on cooperation and strong social groups for survival.

% \todo{This paragraph deserves some attention: some more stuff about cooperation, and how social bonds and cooperation can be beneficial. Because sharing food as-is is not beneficial per se. The issue with the "sharing food is beneficial" could also be resolved by adding half a sentence of explanation about social bonds being beneficial, but I'd like to be more rigorous.}
In the EEA, humans lived together in groups and relied on hunting game and gathering plants for their nutrition.
In this lifestyle, cooperation is a "necessary element of human life" \citep[p.~R448]{ApicellaSilk19} in a number of ways. Firstly, hunting is a 'high risk, high reward' endeavor: the returns are variable, but often when hunting does succeed the yield is large; sometimes even too large for the hunter and his relatives. In this case, food sharing within or between groups is beneficial.
Another way that early humans counterbalanced the variable returns of hunting was to also rely on gathering plant foods, which yielded more predictable returns. In this case cooperation through shared labor was also beneficial, since some foods required complex foraging techniques to acquire it, or required complex processing (through e.g. cooking) before consumption.
Lastly, cooperation in early humans manifested itself in 'cooperative breeding', where the responsibilities of childcare are spread among multiple caregivers. Moreover, mothers and children relied on the efforts of others for their food \citep{ApicellaSilk19}.
% \todo{Add also some things from \citet{Freeberg19}?}

% \rewriteL{The following paragraph might warrant a larger discussion about domain-specificity; see commented out comment. Right now, it's not clear that this is relevant, so delve more into this if it's relevant (and move it to the relevant spot, probably). Else, remove it}
% % define it and give examples (outside of psychology: maybe vision?), maybe with Cosmides/Tooby conception of Wason selection task as cheater detection? Find some literature on domain-specificity, because it turns out to be a bit of a hairy subject. Find definition; because opinions may vary on what features are and aren't domain-specific. Vision is a good example of domain-specificity that is less controversial than Cosmides/Tooby}
% Another topic of discussion in evolutionary psychology that is of importance to our investigation is that of domain-specificity of the psychological mechanisms. The argument has been made that these adaptive mechanisms are necessarily problem- or domain-specific, because the evolutionary process would not favor general solutions to specific problems \citep[p.~50]{Buss15}.
% However, as with the EEA, issues with this stance have been raised: the push to domain-specificity can be said to rely on overly strong assumptions about the modularity of the brain; and moreover, there is also a push to domain-generality of cognitive skills because domain-general skills are neurologically more cost-efficient than domain-specific skills \citep{LB02}.

\section{Teleological notions in evolutionary theory}
\label{sec:teleology}
\thinkL{This section needs quite some work: see notes from discussion with Karolina}
Next, it is important to scrutinize the terminology that I will be using throughout this thesis.
Biological literature frequently makes use of \emph{teleological} terminology, that is, terminology that implies goal-directedness of the processes it describes. Such terminology includes concepts like the \emph{design} of a trait, and \emph{function}, \emph{purpose}, or \emph{utility} of a trait.
At first glance, the usage of these terms in discussing evolution would seem to be inappropriate; for evolution is a process of nature, not purposefully performed by an agent, and it is thus without any intentionality or goals.
\rewriteS{Maybe reformulate this sentence again: still too convoluted?}
And indeed, this teleological terminology has its roots in pre-Darwinian conceptions of nature: it originates from Aristotle's views on nature, and it was subsequently adopted by creationist Muslim and Christian scholars \citep{Johnson05}.

In general, teleological explanations in biology are quite controversial: not only is the usage of the specific terminology itself debated \citep[p.~27 and references therein]{Ayala99}, the concept has been criticized for its apparent lack of formalization and insufficient argumentative persuasiveness \citep[p.~83]{Baedke2021}.
\thinkL{Address the controversy around teleological explanations. Talk about instrumentalism, usefulness of the concepts. Lack of formalization is not such a big problem for the purpose here maybe, but the other thing is more of a problem. Address why they won't be a problem for you. Can mention that MS assume it as well, this teleological explanation is at the heart of their thesis (quote it?), so it's their problem to defend this. I work using the same assumptions as them.}

However, explanations in terms of goals and function have considerable instrumental value in describing evolutionary processes. Throughout this thesis, I will be adhering to the conception of teleological explanations of \citet{Ayala99}, which is as follows:
\begin{quoting}
    Teleological explanations account for the existence of a certain feature in a system by demonstrating the feature’s contribution to a specific property or state of the system, in such a way that this contribution is \emph{the reason why the feature or behaviour exists at all}.
    \hfill (p.~13)
\end{quoting}
In this respect, the evolutionary process of adaptation merits a teleological explanation: the function of a trait (its 'contribution to a specific property or state of the system') is the reason that the trait exists, because it exists as a consequence of evolution.
\todo{Add example}
% \thinkL{Is this view compatible with reciprocal causation and niche construction? I think so; they're a complication for the whole picture, not necessarily for using this definition.}
% \todo{Is this view compatible with cultural learning? From the quote, it doesn't necessarily follow that it's about biology necessarily. Think about this, and after writing a section on culture, state to what extent and in what way we'll adhere to \citet{Ayala99}}

The distinction between proximate and ultimate causes we saw in \cref{sec:causation-evolution} can be applied to teleological explanation as well, yielding the distinction between proximate and ultimate \emph{ends} of features. The proximate end is then the 'immediate' function the feature serves, and the ultimate end is the reproductive success of the organism.
\todo{Add example}

% A footnote to this account is that not all features of organisms can be explained teleologically; only if the feature has arisen and persisted as a direct result of natural selection, a teleological explanation is in place.
% \thinkL{This "direct result of natural selection" is very vague/slippery; acknowledge this, and elaborate more on it if it turns out to be important for my thesis. A way to do this would be to contrast it with an indirect result. Talk about side effects?}

\section{Adopting and adapting Tinbergen's four questions}
\label{sec:tinbergen}
\rewriteL{In this section: terminology issue: trait vs. feature vs. characteristic vs. behavior. Address this earlier on in the chapter, it also relates in a way to the ability to exhibit behavior vs. the behavior itself.}
As mentioned in \cref{sec:causation-evolution}, \citet{Tinbergen63} proposed an influential framework of problems\footnote{In the literature (e.g. \citet{BatesonLaland13}, the terms 'problems' and 'questions' are used interchangeably. I will take 'problems to address' and 'questions to answer' to be synonymous, and will use these terms interchangeably.} that should be addressed if one intends to give a complete account of a behavior an animal exhibits. Let us dive more deeply into Tinbergen's framework here, as it will turn out to form a desirable foundation for the current investigations.

As mentioned before, the four problems that Tinbergen argued to be central to the study of behavior are a behavior's proximate causation, lifetime development, function, and evolutionary history.
Although these four problems were originally introduced with regards to animal behavior, the framework has since been widely adopted for analyzing any characteristic of an organism \citep{BatesonLaland13}\footnote{It can even be used to gain understanding of nonliving systems, such as traffic lights.}.

I will now discuss each of these problems in more detail, such that we can ultimately come to a set of methodological questions to guide us in investigating human communication and reasoning.

% \todo[inline]{Address some objections on whether we're justified in adopting this framework since it's about animal behavior; is human psychology not too complex for this? This objection is related to what \citet{BatesonLaland13} say about the historical process underlying it (genetic vs. non-genetic biological vs. cultural evolution). Mention what they also say about describing any system using this approach.}

\subsection{Causation}
The first Tinbergen problem is that of the mechanistic causation of the behavior; in other words, the proximate causation of the behavior. In our case, addressing this problem would entail a detailed investigation of the neurological processes underlying communicative and reasoning behaviors.
This problem, however interesting, will not be addressed in this thesis. The reason for this is that more empirical and conceptual research would be necessary in order to give a satisfactory account of the exact neurological processes underlying communicative and reasoning behavior. Although it has been emphasized that we can only gain a full understanding of a behavior if the four problems are addressed simultaneously \citep{Tinbergen63, BatesonLaland13}, I believe I am justified in leaving the proximate-causation problem for future research.

\subsection{Survival value}
The second problem that Tinbergen outlines relates to the value a behavior provides to an animal's survival: how does the behavior contribute to the chances of the animal surviving?\todo{Repetitive}

This survival value is, in teleological terms, the function of the behavior.  However, the use of the term 'function' may obscure the fact that a characteristic's function can change over time: the \emph{current} utility that a characteristic has, may not be the same as the \emph{original} utility it had \citep{BatesonLaland13}. For example, feathers originally evolved for temperature regulation in the evolutionary predecessors of birds, and were later adapted for flight \citep{Benton19, BatesonLaland13}.
\todo{Try to add also example in humans of original and current utility not lining up: fat retention?}
We will discuss in \cref{ch:communication} and \cref{ch:reasoning} what can be construed as the original and current utilities or functions of the cognitive capacities we are dealing with.
\rewriteS{Possibly change this comment after Chapter 2 and 3 are more or less finished}

As we saw in \cref{sec:evo-bio-culture}, an organism's fitness is not only determined by their chances of \emph{survival}, but also their chances of \emph{reproducing}. As a consequence of this, the survival value a trait brings to an organism is not the only reason that the trait may persist throughout evolution. A trait is also more likely to appear in future generations if it improves an organism's chances of reproducing.

In the methodological framework proposed here I will amend Tinbergen's question on survival value by broadly speaking of the \emph{utility} of a characteristic, which denotes the way the characteristic contributes to the fitness of the organism.
This leads us to the following formulation of Tinbergen's question for our purposes:

\begin{exe}
    \ex
    \begin{xlist}
        \ex
        What was the original utility of communication to humans? And what is the current utility of communication to humans?
        \ex
        What was the original utility of reasoning to humans? And what is the current utility of reasoning to humans?
    \end{xlist}
\end{exe}
\todo[inline]{The distinction is pretty relevant, but rigorous discussion of both utilities won't be necessary: only use the distinction, don't discuss it. This distinction might be an avenue of scrutiny for Mercier \& Sperber}

\subsection{Ontogeny}
The third question that is essential for gaining understanding about a behavior is the question of how the behavior emerges and changes throughout the development (ontogeny) of the animal.
\todo[inline]{This section is very short, but I don't feel like anything can/needs to be added?}

So this leads us to the following question:
\begin{exe}
    \ex
    \begin{xlist}
        \ex How does the capacity for communication develop throughout childhood?
        \ex How does the capacity for reasoning develop throughout childhood?
    \end{xlist}
\end{exe}

\subsection{Evolution}
The fourth and last problem considered by Tinbergen is that of the evolutionary history of the behavior: in order to provide a complete explanation of a behavior, one must look at how it evolved throughout history. To form hypotheses about this, one must look to whether and how the behavior presents itself in the close evolutionary relatives of the animal.

\citet{BatesonLaland13} maintain that for traits related to human cognition, this question about evolutionary history should be split up into two questions. They argue that due to the influence of not only biological evolution but also culture on the development of the trait, one should distinguish two kinds of evolutionary history, leading to the questions "Which historical processes were responsible for the [trait]?" and "How can its trajectory be explained?" \citep[p.~714]{BatesonLaland13}. However, as I concluded in \cref{sec:evo-bio-culture}, we are justified in remaining agnostic about these historical processes, so we will only take up the latter of these two questions.

This leads us to the following formulation of Tinbergen's evolutionary question:

\begin{exe}
    \ex
    \begin{xlist}
        \ex What is the evolutionary history of human communication? How can its evolutionary trajectory from our nearest evolutionary relatives to us be explained?
        \ex What is the evolutionary history of human reasoning? How can its evolutionary trajectory from our nearest evolutionary relatives to us be explained?
    \end{xlist}
\end{exe}

\subsection{A fifth Tinbergen question: observation and description}
\rewriteL{Emphasize the importance of this question here briefly, yes: but the definition should already be in the introduction, since the concepts are mentioned in the RQ and the title of the thesis. So drop this header, and probably the questions as well}
A problem that is mentioned by Tinbergen in his original paper \citeyear{Tinbergen63}, but not included as one of the core problems of his framework, and more or less never included by authors discussing his framework \citep{LB02, Laland13, AllenBekoff95}, is the problem of \emph{describing} the observed behavior.
In the case of describing reasoning and communication, this issue is akin to the problem of defining and delineating what we take to be reasoning and what we take to be communication. This is by no means a trivial issue, which is what warrants its inclusion in the set of questions we will ask ourselves in thesis:

\begin{exe}
    \ex
    \begin{xlist}
        \ex What is human communication?
        \ex What is human reasoning?
    \end{xlist}
\end{exe}

\section{Conclusion}
\label{sec:evo-conclusion}
Throughout this chapter, it has become apparent that for many of the topics in evolutionary theory discussed here, there exists no consensus among its practitioners. Since the purpose of this thesis will not be to provide a complete causal framework for the evolution of reasoning and communication, we will be able to cast aside some of the issues plaguing the frameworks discussed in this chapter. We will proceed cautiously, using the concepts outlined without needing to account in detail for their shortcomings.
Let us conclude this chapter by first gathering four key assumptions that will inform this thesis, and then restating the methodological questions that will guide its investigations.

The first one is the assumption that we are justified in wanting to explore human reasoning and communication from the perspective of evolution. Despite some of the issues raised against evolutionary psychology as a field of study \citep{LB02}, it cannot be denied that reasoning and communication are cognitive capacities that must have emerged somewhere on our evolutionary journey, through processes of selection (i.e. as a result of variation, sorting and retention).

The second assumption is that in order to answer the question of whether reasoning evolved for communication, we must consider not only reasoning but also communication in detail. This is because for reasoning to have evolved for the purpose of communication, the latter must have been evolutionarily advantageous in its own right, such that advancements in reasoning could have advanced communication to such an extent that it made communication more evolutionarily advantageous.
Moreover, as we will see, a thorough investigation of communication will illuminate the role reasoning plays in communication.

The third assumption is that throughout our evolutionary trajectory, we have been dependent on fellow humans for our survival, relying on cooperation and strong social groups. This assumption is especially important in the analysis of human communication.

The fourth and last assumption is that in our analysis we may remain agnostic about whether biological or cultural evolution is responsible for the emergence of reasoning and communication, since both kinds of evolution have the same underlying mechanisms of selection.

Lastly, regarding the questions we will address in the following two chapters: the discussion in \cref{sec:tinbergen} has yielded four questions reformulated and adapted from \poscite{Tinbergen63} framework. Here, I restate these questions in a general manner and in an order that will be most useful to the investigations in \cref{ch:communication} and \cref{ch:reasoning}.

\begin{labeling}{xxxxxxxxxxxxxxxx}
    \item [Definition] How can this cognitive capacity be defined and delineated?
    \item [Development] How does this cognitive capacity develop throughout childhood?
    \item [Evolution] What is the evolutionary history of this cognitive capacity; how can its evolutionary trajectory from our nearest evolutionary relatives to us be explained?
    \item [Utility] What are the original and current utilities of this cognitive capacity to humans?
\end{labeling}

Now that we have clarified the assumptions and defined the questions, it is time to consider the cognitive capacity that is primary in the context of our research question: communication.
