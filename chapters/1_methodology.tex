\chapter{The chicken and the egg: the evolutionary approach}
\label{ch:evolution}
\pagestyle{first}

Before we immerse ourselves in the evolutionary perspective integral to the argumentative theory of reasoning, we should start by laying out some groundwork on evolution.
This chapter by no means intends to provide a comprehensive overview of the issues in evolutionary theory, since this is a vast field of research in its own right with widely diverging opinions on a number of specifics of the process of evolution\footnote{See for example \citet{Ariew02} and \citet{UllerLaland19} for an overview of topics in evolutionary theory and evolutionary causation.}.
The purpose of this chapter is instead to provide a starting point for our analysis of the evolution of human communication, against which the ATR can subsequently be criticized.

In this chapter, we will consider the processes underlying evolution, and discuss causation in evolution. Moreover, we will touch on some considerations surrounding the evolutionary approach to human psychology, and discuss the use of teleological terminology in evolutionary theory. Lastly, this chapter will see us outlining a methodology for the investigations of \cref{ch:communication}.

\section{Evolution: biological vs. cultural}
\label{sec:evo-bio-culture}

Although the term 'evolution' is in everyday usage most commonly interpreted as 'Darwinian evolution', or 'natural selection', the concept of evolution can be stripped down to a very broadly construed version, which may prove to be illuminating.

In general, any process of selection can be taken to consist of three steps: (1) variation, (2) sorting\footnote{\citet{Donahoe03} uses the term 'selection' for this step. I follow \citet{Heyes18} and \citet{S-P13} in using the term 'sorting', to avoid confusion with the full process of selection (which consists of the three steps outlined here).}, and (3) retention \citep{Donahoe03}. I will first discuss how each of the steps of this process are construed in standard evolutionary theory, and then how the process of \emph{cultural evolution} fits this conception of selection.

Firstly, in standard evolutionary theory the processes that introduce variation are mutation and migration. Mutation concerns the changes in an organism's DNA, and migration concerns the movement of (genetic material of) organisms from one population to another \citep{S-P13}.
% Variation by itself is undirected; it is only due to sorting that the selection process gives the appearance of direction or progress towards some goal \citep{Donahoe03}.
% \todo[inline]{Add a bit more on sorting: see \citet{Sperber10} notes for an example. Try to map it to what people already know, because they're probably familiar with the process, but just not with the term.}
% \todo[inline]{Make sure you yourself are clear on the difference between the higher-level process of selection, and the lower-level process of sorting.}

Secondly, in standard evolutionary theory the sorting process amounts to natural selection, as well as genetic drift. Here we will only focus on the former of these two processes\footnote{In a nutshell, genetic drift can be described as a probabilistic sorting process, resulting from a sampling error due to populations being finite in size. However, genetic drift is a hotly debated concept in evolutionary theory \citep[see][for an overview]{Millstein21}, so I mention it here only for the sake of completeness.}.
The sorting process of natural selection\footnote{Not to be confused with the full three-step process, which is (confusingly) usually also referred to as natural selection.}
favors variants that enhance an organism's fitness and disfavors variants that diminish its fitness.
An organism's fitness corresponds to how likely they are to leave offspring in the next generation compared to organisms with a different genetic makeup. This fitness relates to both the organism's chances of survival as well as its chances of reproducing.

Thirdly, in standard evolutionary theory, the process responsible for retention is genetic inheritance, i.e. the transmission of characteristics from parents to their offspring through the genetic material (DNA) parents impart on their offspring.
As a result of these three steps of the selection process then, the genes that enhance an organism's fitness persist over time in the population, resulting in adaptation \citep{S-P13}.

Now, the evolution of \emph{culture} can also be said to operate by these principles \citep{Heyes18}. In this context, culture is understood at its core as \emph{information}; more specifically, it is information "that we inherit from others through social interaction (via certain kinds of social learning)" \citep[p.~30]{Heyes18}.
Let us now consider the cultural processes that constitute each of the three steps of the selection process, following \citet[pp.~33--34]{Heyes18}.

Firstly, variation in culture is introduced by error and by innovation. For example, an individual might secure their fishing line with four instead of three knots by accident; or they may produce this variation intentionally, in an effort to improve upon the practice.
Secondly, the sorting of behaviors and habits in culture can happen through two different routes. A behavior can be favored because of some property inherent to the behavior that makes it "more noticeable, learnable, or memorable than others" \citep[p.~34]{Heyes18} and thus more likely to be copied. For example, the four-knot fishing line might be favored if it is easier to construe than the three-knot version.
Also, a behavior or habit may be favored in a more 'classic' evolutionary way: a habit may be favored because it improves the fitness of the individual, such that individuals with that habit are more likely to survive and reproduce than individuals with an alternative habit. If the four-knot fishing line makes people more successful in catching fish, this practice increases their chances of survival and is thus favored.
Thirdly, culture may be retained through cultural inheritance: cultural practices are passed on between people through mechanisms of social learning.

Whether or not the process of cultural evolution can be taken to be analogous to that of biological evolution, is not an uncontroversial issue; see \citet{Claidiere14, Stanley21} for discussion.
In this thesis however, I will follow \citet{Heyes18} in the assumption that cultural evolution is 'Darwinian'; that is, it abides by mechanisms that are analogous to those of biological evolution.
As a consequence, in discussing how human communication and reasoning evolved, we may remain agnostic on the kind of evolution responsible for this evolution, since the mechanisms underlying biological and cultural evolution are assumed to be analogous to each other.
Thus, for the remainder of this thesis, I will use the term 'evolution' to talk about the development of traits over time, remaining agnostic about whether this development is due to biological evolution or cultural evolution.

Let me conclude this discussion of the workings of evolution with a brief detour to the field of evolutionary game theory, which (as the name implies) applies game theory to the evolution of animal behavior.
As we will see in \cref{ch:atr}, Mercier and Sperber's theory draws heavily on game-theoretic notions and concepts, in particular on \emph{cost-benefit analyses} \citep[see][]{Sperber01, Sperber10}.
In general, one may analyze an animal's behavior in terms of the costs and benefits the behavior yields to the animal. For example, hunting down a large prey animal has its costs (it takes energy and effort) and its benefits (it yields sustenance).
The payoff of the behavior can then be considered to act as a proxy for the fitness of the animal: "[payoffs] are meant to represent how much the outcome of the game increases fitness" \citep[p.~118]{McNamara10}\footnote{\citet[\S 4.4]{McNamara10} provide some caveats to this conception of payoff as proxy for fitness. However, I consider a comprehensive discussion of the game-theoretical approach to evolution to be out of scope of this thesis.}. For example, in the case of hunting game, if the behavior's benefits exceed its costs, it can be taken to increase the hunter's fitness.
\todo{Add wrap-up sentence}

\section{Causation in evolution}
\label{sec:causation-evolution}

Next, we will dip our toes into the waters of causation in evolution. As it turns out, causation in evolution is not a simple notion; consider for example a moth whose wings provide it with camouflage due to their coloration. The camouflage of the wings is an \emph{effect} of their coloration; yet, it is precisely the camouflage the coloration provides that is the \emph{cause} of the coloration being present at all \citep{Lipton09}.

% Evolutionary causation is a subfield of philosophy of biology that has continued to see widely diverging opinions \citep{Baedke2021, S-P13}. In this section I will restrict focus to four topics in evolutionary causation that are of interest to this thesis. First, I will discuss the distinction between proximate and ultimate causation. Then, I will briefly cover \poscite{Tinbergen63} questions for explaining animal behavior, which will be discussed at length in \cref{sec:tinbergen}. Thirdly, we will have a look at niche construction and reciprocal causation. Lastly, we circle back to our research question and discuss what it entails for one trait to evolve for the purpose of another.

In evolutionary causation, one may distinguish \emph{proximate} from \emph{ultimate} causes \citep{Mayr61}.
Proximate causes are the immediate influences on a trait: they explain how the trait results from the internal and external factors causing it. For example, the proximate cause of the coloration of the moth's wings would be the biochemical processes during the moth's development that result in that particular pattern of colors.
Ultimate causes, on the other hand, provide the higher-level historical and evolutionary explanation of those traits. In the case of the moth, the ultimate cause for the coloration of its wings is the enhanced fitness of the moth due to the camouflaging properties of the coloration \citep{Lipton09}. In other words, these two different causes relate to two different explanatory questions: the proximate cause is related to the \emph{how}-question (\emph{how} did a trait come about?), whereas the ultimate cause is related to the \emph{why}-question (\emph{why} did a trait come about?). According to \citet{Mayr61}, who pioneered the distinction\footnote{See \citet[p.~720]{Laland13} for a discussion of the exact origins of the distinction.}, one needs to answer both of these explanatory questions in order to obtain a complete understanding of a trait.

In a seminal proposal considered by some to be an extension of Mayr's dichotomy \citep{Laland13}, Nico \citet{Tinbergen63} outlined four questions central to the study of animal behavior. In order to fully understand a pattern of behavior, he argued, one must consider (1) the behavior's proximate causation, (2) its survival value, (3) its lifetime development, and (4) its evolutionary history.
Since Tinbergen's proposal, other authors have grouped these four questions according to Mayr's proximate-ultimate distinction, characterizing the 'causation' and 'development' questions as proximate questions (\emph{how}-questions) and the 'survival value' and 'evolutionary history' questions as ultimate questions (\emph{why}-questions) \citep{BatesonLaland13}.
Tinbergen's framework has had an extensive and lasting influence on the study of animal behavior, and his questions continue to be used by biologists to this day \citep{BatesonLaland13}. Hence, I will let this framework guide our investigation into the evolution of human communication; we will flesh this out in \cref{sec:tinbergen}.

\section{Evolutionary psychology}
\label{sec:evol-psych}
In order to get a grip on the evolution of cognitive capacities such as reasoning and communication, let us first
% answer the question of whether reasoning evolved for the purpose of communication, we will also need to
zoom out to consider the field of evolutionary psychology as a whole.
What is the merit, and the validity, of adopting an evolutionary approach to human psychology?

The field of evolutionary psychology concerns itself with trying to understand human behavior using evolutionary theory, by looking into the past and considering how our ancestors must have adapted to their environment in order to survive and reproduce.
Researchers in the social sciences and humanities have historically been wary of using evolutionary approaches to study human behavior, because evolutionary theory has been abused for prejudiced ends in the past\footnote{See \citet[pp.~19--20]{LB02} for an overview.}. Moreover, evolutionary-psychological research has received the criticism that too much of it is "just-so" storytelling and post-hoc explanation of known phenomena, sometimes accompanied by a sensationalist spin on the story \citep{LB02}.
However, if these pitfalls are avoided, looking at human psychology from an evolutionary perspective can be an illuminating endeavor.
Let us now consider some of the central concepts and assumptions of evolutionary psychology.

In order to explain humans' psychological mechanisms, evolutionary psychologists look to the concept of an \emph{environment of evolutionary adaptedness} (EEA). The EEA is the environment in which these psychological mechanisms must have come into being; usually the EEA is identified as hunting and gathering groups on the African savannah in the second half of the Pleistocene, between 1.7 million and ten thousand years ago \citep{LB02}.
% \todo{Are the groups the EEA or the savannah? Elaborate more on this: is it the physical environment or does it include the groups? Aren't the groups a trait that evolved? Think about this}
There are two key assumptions underlying the use of the concept of an EEA.
The first is that our modern-day environment is too different from that of our ancestors for us to use it to explain why and how our psychological mechanisms evolved in the past.
The second assumption is that for our psychological mechanisms to be as complex as they are, they must have evolved slowly; because of this, they must have evolved a considerable amount of time ago without changing significantly since the Stone Age.

There are a number of issues associated with the use of the concept of the EEA \citep{LB02}. Firstly, we do not know very much about the environment of our ancestors, so the specifics of the EEA may be filled in as is seen fit for one's purposes. Secondly, it may be argued that we do not know enough about the process of evolution to make the second assumption. While evolution does in general operate on a large timescale, there is empirical evidence
% \todo{Probably good to mention the nature of this evidence, because this evolutionary timeline point is important}
that the process can also be faster, operating on a timescale of thousands of years, or less than 100 generations \citep[pp.~190--191 and references therein]{LB02}. Lastly, the argument can be made that for our species to have flourished and dominated in the way that it did, we must have remained adaptive to our changing modern environments after the Stone Age.
% Lastly, the EEA argument does not take into account reciprocal causation or niche construction.
% \researchL{Elaborate on this last issue, because it seems to be important for my argument. Also, it seems like a strange point to make, because especially with humans, it seems obvious that niche construction is a thing that might play a role. Why would the EEA concept rely on assumptions of unidirectional causation?}

Despite the issues associated with the concept of the EEA, it is instrumentally valuable in reminding us to consider the state of the environment and its role in the evolutionary process. For the purposes of this thesis, we need not commit to any strong assumptions about the nature and properties of the EEA. The most important assumption I will make, following influential scholars like Michael \citet{Tomasello09}, is that humans throughout history have been dependent on cooperation and strong social groups for their survival.
% \todo{This paragraph deserves some attention: some more stuff about cooperation, and how social bonds and cooperation can be beneficial. Because sharing food as-is is not beneficial per se. The issue with the "sharing food is beneficial" could also be resolved by adding half a sentence of explanation about social bonds being beneficial, but I'd like to be more rigorous.}
In the EEA, humans lived together in groups and relied on hunting game and gathering plants for their nutrition.
In this lifestyle, cooperation is a "necessary element of human life" \citep[p.~R448]{ApicellaSilk19} in a number of ways. Firstly, hunting is a 'high risk, high reward' endeavor: the returns are variable, but when hunting does succeed the yield is often large -- sometimes even too large for the hunter and his relatives. In this case, food sharing within or between groups is beneficial.
Another way in which early humans counterbalanced the variable returns of hunting was by also relying on gathering plant foods, which yielded more predictable returns. In this case cooperation through shared labor was also beneficial, since some foods required complex foraging techniques or complex processing (e.g. cooking) before consumption.
Lastly, cooperation in early humans manifested itself in 'cooperative breeding', where the responsibilities of childcare are spread among multiple caregivers. Moreover, mothers and children relied on the efforts of others for their food \citep{ApicellaSilk19}.
We will discuss human cooperation in more detail in \cref{sec:comm:cooperation}, in particular how its evolution relates to the evolution of human communication.

% \rewriteL{The following paragraph might warrant a larger discussion about domain-specificity; see commented out comment. Right now, it's not clear that this is relevant, so delve more into this if it's relevant (and move it to the relevant spot, probably). Else, remove it}
% % define it and give examples (outside of psychology: maybe vision?), maybe with Cosmides/Tooby conception of Wason selection task as cheater detection? Find some literature on domain-specificity, because it turns out to be a bit of a hairy subject. Find definition; because opinions may vary on what features are and aren't domain-specific. Vision is a good example of domain-specificity that is less controversial than Cosmides/Tooby}
% Another topic of discussion in evolutionary psychology that is of importance to our investigation is that of domain-specificity of the psychological mechanisms. The argument has been made that these adaptive mechanisms are necessarily problem- or domain-specific, because the evolutionary process would not favor general solutions to specific problems \citep[p.~50]{Buss15}.
% However, as with the EEA, issues with this stance have been raised: the push to domain-specificity can be said to rely on overly strong assumptions about the modularity of the brain; and moreover, there is also a push to domain-generality of cognitive skills because domain-general skills are neurologically more cost-efficient than domain-specific skills \citep{LB02}.

\section{Teleological notions in evolutionary theory}
\label{sec:teleology}
Next, let us touch on a terminological issue that deserves some discussion.
Biological literature frequently makes use of \emph{teleological} terminology, that is, terminology that implies goal-directedness of the processes it describes. Such terminology includes concepts like the \emph{design} of a trait, and the \emph{function} or \emph{purpose} of a trait.
This teleological terminology has its roots in pre-Darwinian conceptions of nature: it originates from Aristotle's views on nature, and was subsequently adopted by creationist Muslim and Christian scholars \citep{Johnson05}.
At first glance, the usage of these terms in discussing evolution would thus seem to be inappropriate. Evolution is a process of nature, not purposefully performed by an agent, and it is thus without any intentionality or goals.
Indeed, teleological explanations in biology are somewhat controversial \citep[see][p.~27 for discussion]{Ayala99}.
However, explanations in terms of goals and functions have considerable \emph{instrumental} value in describing evolutionary processes. In light of this, consider the following characterization of teleological explanations:
\begin{quoting}
    Teleological explanations account for the existence of a certain feature in a system by demonstrating the feature’s contribution to a specific property or state of the system, in such a way that this contribution is \emph{the reason why the feature or behaviour exists at all}.
    \hfill \citep[p.~13]{Ayala99}
\end{quoting}
In this respect, the evolutionary process of adaptation merits a teleological explanation. A trait's survival value -- its 'contribution to a specific property or state of the system' -- is the reason that the trait has persisted throughout evolution. Returning to the moth's colored wings, their function is to provide camouflage, and this contribution to the moth's survival is the reason the coloration exists in the first place.

The distinction between proximate and ultimate causes we saw in \cref{sec:causation-evolution} can be applied to teleological explanation as well, yielding the distinction between proximate and ultimate \emph{ends} of features \citep[p.~18]{Ayala99}. The proximate end is then the 'immediate' function the feature serves (e.g. the camouflage), and the ultimate end is the reproductive success of the organism.

We can also see this in Mercier and Sperber's view that the function of reasoning is to make communication "more effective and advantageous" \citep[p.~60]{MS11}.
According to this view, the improvement of communication is the proximate end of reasoning. Inherent to this view is then the assumption that improving communication contributes to humans' fitness.
Therefore, in order to critically analyze Mercier and Sperber's argumentative theory of reasoning, we should first investigate human communication from an evolutionary perspective.

\section{Adopting and adapting Tinbergen's four questions}
\label{sec:tinbergen}
As mentioned in \cref{sec:causation-evolution}, \citet{Tinbergen63} proposed an influential framework of problems\footnote{In the literature \citep[e.g.][]{BatesonLaland13}, the terms 'problems' and 'questions' are used interchangeably. I will take 'problems to address' and 'questions to answer' to be synonymous, and will use these terms interchangeably.} that should be addressed if one intends to give a complete account of a behavior an animal exhibits.
The four problems that Tinbergen argued to be central to the study of behavior are a behavior's proximate causation, survival value, lifetime development, and evolutionary history.
Although these four problems were originally introduced with regards to animal behavior, the framework has since been widely adopted for analyzing any trait of an organism \citep{BatesonLaland13}\footnote{It can even be used to gain understanding of nonliving systems, such as traffic lights \citep{BatesonLaland13}.}.
Let me now discuss each of these problems in more detail, such that we can ultimately come to a set of questions to guide us in investigating the evolution of human communication in \cref{ch:communication}.

The first Tinbergen problem is that of the 'mechanistic' causation of the behavior; in other words, the proximate causation of the behavior. In our case, addressing this problem would entail a detailed investigation of the neurological processes underlying communicative behaviors.
This problem, however interesting, will not be addressed in this thesis. The reason for this is that more empirical and conceptual research would be necessary in order to give a satisfactory account of the exact neurological processes underlying human communication. Although \citet{Tinbergen63} emphasized that we can only gain a full understanding of a behavior if all four problems are addressed simultaneously \citep[see also][]{BatesonLaland13}, I believe I am justified in considering the proximate-causation problem to be out of scope for the current endeavor.

The second problem that Tinbergen outlines, relates to how the behavior contributes to the chances of the animal surviving.
This survival value is, in teleological terms, the function of the behavior.  However, the use of the term 'function' may obscure the fact that a trait's function can change over time: the \emph{current} utility that a trait has, may not be the same as the \emph{original} utility it had \citep{BatesonLaland13}. For example, feathers originally evolved for temperature regulation in the evolutionary predecessors of birds, and were later adapted for flight \citep{Benton19, BatesonLaland13}.
Another example of the current and original utilities of a trait not lining up, is fat retention in humans. Our ability to store energy in the form of fat originally contributed to our survival value by providing a buffer against malnutrition and fluctuating energy supplies \citep{Wells06}. In light of the obesity epidemics in present-day western societies however, this trait can hardly be considered to positively contribute to our chances of survival.
In our investigation in \cref{ch:communication}, we will focus on the \emph{original} utility of communication. Since we are interested in how human communication evolved and how the evolution of reasoning relates to this, only the original utility of communication is relevant here.

As we saw in \cref{sec:evo-bio-culture}, an organism's fitness is not only determined by their chances of \emph{survival}, but also their chances of \emph{reproducing}. As a consequence of this, the survival value a trait brings to an organism is not the only reason why the trait may persist throughout evolution. A trait is also more likely to appear in future generations if it improves an organism's chances of reproducing. In the methodological framework proposed here I will amend Tinbergen's question on survival value by broadly speaking of the \emph{utility} of a trait, which then denotes the way the trait contributes in general to the fitness of the organism.

The third question that is essential for gaining understanding about a behavior is the question of how the behavior emerges and changes throughout the development (ontogeny) of the animal. \todo{Add one sentence}

The fourth and last problem considered by Tinbergen is that of the evolutionary history of the behavior: in order to provide a complete explanation of a behavior, one must look at how it evolved throughout history. To form hypotheses about this, one must look to whether and how the behavior presents itself in the closest evolutionary relatives of the animal.
\citet{BatesonLaland13} maintain that for traits related to human cognition, this question about evolutionary history should be split up into two questions. They argue that, due to the influence of not only biological evolution but also culture on the development of a trait, one should distinguish two kinds of evolutionary history, leading to the questions "Which historical processes were responsible for the [trait]?" and "How can its trajectory be explained?" \citep[p.~714]{BatesonLaland13}. However, as I concluded in \cref{sec:evo-bio-culture}, we are justified in remaining agnostic about which processes were responsible. Therefore, we will only take up the latter of these two questions and ask ourselves how the evolutionary trajectory of human communication can be explained.

Lastly, a problem that is mentioned by Tinbergen in his original paper (\citeyear{Tinbergen63}) but is not included as one of the core problems of his framework\footnote{And more or less never included by authors discussing his framework \citep{LB02, Laland13, AllenBekoff95}.}, is the problem of \emph{describing} the observed behavior.
In the case of describing communication, this issue is akin to the problem of defining and delineating what we take to be communication. This is by no means a trivial issue, which is why I will include it in the set of questions we will ask ourselves in \cref{ch:communication}.

\section{Conclusion}
\label{sec:evo-conclusion}
Throughout this chapter, it has become apparent that for many of the topics in evolutionary theory discussed here, there exists no consensus among its practitioners. Since the purpose of this thesis will not be to provide a complete causal framework for the evolution of reasoning and communication, we will be able to cast aside some of the issues plaguing the frameworks discussed in this chapter. We will proceed cautiously, using the concepts outlined without needing to account in detail for their shortcomings.
Let us conclude this chapter by first gathering three key assumptions that will inform this thesis, and then formulating the methodological questions that will guide our investigations.

The first is the assumption that we are justified in wanting to explore human reasoning and communication from the perspective of evolution. Despite some of the issues raised against evolutionary psychology as a field of study \citep{LB02}, it cannot be denied that reasoning and communication are cognitive capacities that must have emerged somewhere on our evolutionary journey, through processes of selection (i.e. as a result of variation, sorting and retention).

The second assumption is that in our analysis we may remain agnostic about whether biological or cultural evolution is responsible for the emergence of reasoning and communication, since these types of evolution have analogous underlying mechanisms of selection.

% The second assumption is that in order to answer the question of whether reasoning evolved for communication, we must consider not only reasoning but also communication in detail. This is because for reasoning to have evolved for the purpose of communication, the latter must have been evolutionarily advantageous in its own right, such that advancements in reasoning could have advanced communication to such an extent that it made communication more evolutionarily advantageous.
% Moreover, as we will see, a thorough investigation of communication will illuminate the role reasoning plays in communication.

The third assumption is that throughout our evolutionary trajectory, we have been dependent on fellow humans for our survival, relying on cooperation and strong social groups.

Finally, in \cref{sec:tinbergen} we adapted \poscite{Tinbergen63} framework to arrive at four questions that will guide our analysis in \cref{ch:communication}. These questions may be formulated as follows (presented in the order in which we will discuss them):

\begin{labeling}{xxxxxxxxxxxxxxxx}
    \item [Definition] How can communication be defined and delineated?
    \item [Evolution] What is the evolutionary history of human communication: how can its evolutionary trajectory from our nearest evolutionary relatives to us be explained?
    \item [Development] How does communication develop throughout childhood?
    \item [Utility] What is the original utility of communication to humans?
\end{labeling}

% Now that we have explicated some assumptions and outlined these questions, it is time to get
