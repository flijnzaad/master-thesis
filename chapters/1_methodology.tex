\chapter{The chicken and the egg: the evolutionary approach}

% introduction

Before we are able to answer any \emph{why}-questions about humans' cognitive capacities, some groundwork needs to be laid out. For what does it mean for some trait to evolve 'because of' or 'for the purpose of' another trait? Are we even justified in using this kind of terminology when it comes to evolution, which is a process which cannot be said to be intentional nor purposeful? And what questions will we need to ask ourselves in order to ultimately answer that big  \emph{why}-question?
This chapter attempts to answer these questions. It by no means provides an overview of issues in evolutionary theory; this is a large field of research in its own right, with widely diverging opinions on a number of specifics of evolution (see \citet{Ariew02} and \citet{UllerLaland19} for overviews of topics in evolutionary theory).
This chapter will merely serve to get a number of issues out of the way before we can continue our investigation into the cognitive mechanisms that make us human.

\section{Causation in evolution}
\label{sec:causation-evolution}

Now, we will dip our toes into the waters of causation in evolution.

Evolutionary causation is a subfield of philosophy of biology that has continued to see wildly diverging opinions \citep{Baedke2021, UllerLaland19}. In this section I will restrict focus on three topics in evolutionary causation that are of interest to this thesis: proximate vs. ultimate causation; Tinbergen's four questions; and reciprocal causation and niche construction.
The former will serve to illuminate some basic aspects of the issues evolutionary causation concerns itself with, and the latter is of particular interest in considering the evolution of human cognition.

In evolutionary causation, one may distinguish \emph{proximate} from \emph{ultimate} causes.
Proximate causes are the immediate influences on a trait; proximate causes explain how the trait results from the internal and external factors causing it.
Ultimate causes serve more as the historical explanation of those traits \citep{Mayr61}. In a way, these two different causes answer two different explanatory questions: the proximate cause is the answer to the \emph{how}-question, whereas the ultimate cause answers the \emph{why}-question. One needs to answer both of these questions in order to provide a complete understanding of the phenomenon.
\todo{this paragraph deserves some rewriting}

By some considered to be an extension of this distinction, \cite{Tinbergen63} has influentially proposed four questions to be central to the study of animal behavior. In order to fully understand a pattern of behavior, one must consider the proximate causation of the behavior, the lifetime development of the behavior, the function\footnote{Visit \cref{sec:teleology} for an exposition of function} of the behavior, and lastly the evolutionary history of the behavior.

While Mayr's and Tinbergen's frameworks have had great explanatory value in the field of evolutionary causation, it has been argued \citep{Laland13} that biologists would be better off rejecting them in favor of a framework of \emph{reciprocal causation}.
In a framework of reciprocal causation, there is feedback between the organism and its environment: not only does the environment influence the organism through the process of natural selection, but the organism influences the environment through the process of \emph{niche construction} \citep{Svensson18}.
Niche construction which posits that organisms may actively partake in modifying their environment and shaping their niche. It has been argued that niche construction is at least as important as natural selection in shaping evolution, and that especially in evolutionary processes involving interactions between organisms, reciprocal causation is very frequent \citep{Svensson18}.
% Both concepts reject the unidirectional conception of evolutionary causation, and ascribe a causal role to the organism itself and acknowledge the existence of feedback loops.
\todo{example?}

% \section{Evolutionary approaches to human psychology}
\section{The legitimacy of evolutionary psychology}

In order to answer our question, we will first zoom out to consider the field of evolutionary psychology as a whole. What is the merit, and the validity, of adopting an evolutionary approach in our endeavor?

The field of evolutionary psychology concerns itself with trying to understand human behavior through evolutionary theory, by looking into the past and considering how our ancestors must have adapted to their environment.
Researchers in the social sciences and humanities have historically been a tad wary of using evolutionary approaches to study human behavior, because evolutionary theory has been abused for prejudiced ends in the past \citep[pp.~19--20]{LB02}. Moreover, evolutionary-psychological research has received the criticism that too much of it is "just-so" storytelling and post-hoc explanation of known phenomena, sometimes accompanied by a sensationalist spin \citep{LB02}.
However, if these pitfalls are avoided, looking at the human psyche from the evolutionary perspective can be an illuminating endeavor.
\todo{that's kind of a weird comment, but i needed to counterbalance it somehow}

In order to explain humans' evolved psychological mechanisms, evolutionary psychologists look to the concept of an \emph{environment of evolutionary adaptedness} (EEA). The EEA is the environment in which these psychological mechanisms must have come into being; usually the EEA is identified as hunting and gathering groups on the African savannah in the second half of the Pleistocene, between 1.7 million and 10,000 years ago \citep{LB02}.
The assumptions underlying the use of the concept of an EEA are that (1) our modern-day environment is too different from that of our ancestors for us to use it to explain why and how our psychological mechanisms evolved in the past; and (2), for our psychological mechanisms to be as complex as they are, they must have evolved slowly and thus at a considerable amount of time prior to the present, and they have not changed significantly since the Stone Age.

There are a number of issues associated with the use of the concept of the EEA \citep{LB02}. Firstly, we do not know very much about the environment of our ancestors, so the specifics of the EEA may be filled in as is seen fit for one's purpose. Secondly, we do not know enough about the process of evolution to make assumption (2); while evolution does in general operate on a large time scale, research has shown that it can also be faster, operating on a time scale of thousand of years, or less than 100 generations \citep[pp.~190--191 and references therein]{LB02}. Thirdly, the argument can be made that for our species to have flourished and dominated in the way that it did, we must have remained adaptive to our changing modern environments after the Stone Age. Fourthly, the EEA argument does not take into account reciprocal causation or niche construction, as discussed in \cref{sec:causation-evolution}.

For our endeavor at hand, I accept the concept of the EEA in a general sense despite its flaws; for our purposes, we need not commit to any strong assumptions about the nature and properties of the EEA. The most important assumption I will make is that homo sapiens throughout history has been dependent on strong social groups for survival.
\todo{i will need to argue here why i DO make this assumption, because it's maybe not justified? see \citet[p.~178]{LB02}}

Another topic of discussion in evolutionary psychology that is of importance to our investigation, is that of domain-specificity of the evolved psychological mechanisms. The argument has been made \citep[p.~50]{Buss15} that these adaptive mechanisms are problem- or domain-specific, because the evolutionary process would not favor general solutions to specific problems.
However, as with the EEA, there are some issues with this stance \citep{LB02}: the push to domain-specificity can be said to rely on too strong assumptions about the modularity of the brain; and moreover, there is also a push to domain-generality of cognitive skills because they are neurologically more cost-efficient.

\todo{should this last paragraph go in another section? in the last one? it's maybe not really related to this section}
Lastly, it is important to note here that not only biological evolution, but also cultural evolution can be said to have played a role in shaping human behavior and human cognitive capacities; it is a joint endeavor facilitated by nature and nurture. In this thesis, I will not attempt to delineate between what features are due to biological evolution and which are due to cultural evolution; for the purposes of this investigation, I believe I am justified in broadly considering them under the umbrella term of 'evolution'.

\section{Teleological notions in evolutionary theory}
\label{sec:teleology}

Next up, it is useful to scrutinize the terminology that I will be using throughout this thesis.
Biological literature frequently make use of teleological terminology, that is, terminology that implies goal-directedness of the processes it describes. Such terminology includes concepts like the \emph{design} of a trait, and \emph{function}, \emph{purpose}, or \emph{utility} of a trait.
At first glance, the usage of these terms in dealing with evolution would seem to be inappropriate; evolution is a natural process, that is not performed by an agent, not with any intentionality, direction, or end-goal. And indeed, this teleological terminology has its origin in pre-Darwinian and creationist views on the origin of species.

\todo{this segue needs to be rewritten}
However, they have considerable instrumental value in describing evolutionary processes. Throughout this thesis, I will be adhering to the conception of teleological explanations as proposed by \citet{Ayala99}, which is as follows:
\begin{quote}
    Teleological explanations account for the existence of a certain feature in a system by demonstrating the feature’s contribution to a specific property or state of the system, in such a way that this contribution is \emph{the reason why the feature or behaviour exists at all}.
    \\ \hfill
    % \citep[p.~13]{Ayala99}
    (p.~13)
\end{quote}
In this respect, the evolutionary process of adaptation merits a teleological explanation; the function of a trait (its 'contribution to a specific property or state of the system') is the reason that the trait exists, because it exists as a consequence of natural selection.
Natural selection can be said to be directed or oriented; the direction of evolution is determined by the functional utility of the trait to the organism.
\todo{do i need to sidenote this with the fact that it's too simplistic and/or ignores reciprocal causation and niche construction?}

The distinction between proximate and ultimate causes (\cref{sec:causation-evolution}) can be applied to teleological explanation, resulting in the distinction between proximate and ultimate \emph{ends} of features. The proximate end is then the 'immediate' function the feature serves \todo{example}, and the ultimate end is the reproductive success of the organism.
\todo{is this relevant?} Ayala distinguishes between \emph{natural} and \emph{artificial} teleology, which concerns whether the feature is a result of a natural process or rather by purposeful action, consciously intended by an agent.

A footnote to this whole story is that not all features of organisms can be explained teleologically; only if the feature has arisen and persisted as a direct result of natural selection, is a teleological explanation in place.
\todo{this "direct result of natural selection" is very vague, make a note of this?}

In general, teleological explanations in biology are quite controversial: not only is the usage of the specific terminology itself debated \citep[p.~27 and references therein]{Ayala99}, the concept has been criticized for its apparent lack of formalization and insufficient argumentative persuasiveness \citep[p.~83]{Baedke2021}.
\todo{finish this up with a rebuttal or last point}

\section{An evolutionary foundation}

Now that we have gathered the puzzle pieces, it is time to return focus to our field of investigation and lay the groundwork for it. In particular, before we can continue with executing our evolutionary approach to human reasoning and communication in the next chapters, we will need to outline and justify the assumptions made.

As we have seen in this chapter, it seems that for none of the topics in evolutionary theory discussed here consensus has been reached among its practitioners. Since the purpose of this thesis is not to provide a complete causal framework for the evolution of reasoning and communication, we will be able to cast aside some of the issues plaguing the frameworks discussed in this chapter. We will proceed using the concepts outlined, without needing to account for their shortcomings.

\paragraph{Why do we consider these cognitive capacities from an evolutionary perspective?}

In choosing to explore human reasoning and communication from the evolutionary perspective, I am working on the assumption that it is appropriate to the endeavor: reasoning and communication are two key features of the homo sapiens that must have arisen at some point in our evolutionary history, shaped by natural selection, niche construction, and perhaps (cultural) learning.
\todo{this is vague; add something about inheritance also?}

\paragraph{What do we mean when we say 'evolution'?}
\todo[inline]{discussion on biology vs. culture: see \citet{Laland13} and \citet{LB02}}

So far, we have mostly considered the concepts and issues surrounding \emph{biological} evolution, and the role of \emph{culture} has remained largely undiscussed.
In this thesis, I will use the term 'evolution' and associated terminology to talk about the development of cognitive capacities in humans over time, remaining agnostic about whether this evolution is due to biological processes like natural selection, or cultural processes like cultural learning.
\todo{doesn't this defeat the whole purpose of this chapter? would it be good to also spend time on cultural evolution and its role \citep{Heyes18}}

\paragraph{What are our assumptions about the evolutionary process?}
\todo[inline]{discussion of assumptions on EEA and rejecting unidirectional causation in favor of reciprocal causation}

\paragraph{What questions will we need to answer?}
\todo[inline]{tinbergen's questions?}
In exploring cognitive capacities in more detail from an evolutionary perspective, we will loosely follow \poscite{Tinbergen63} questions by considering the following aspects of these traits:
\begin{itemize}
    \item What might have been the starting point of this capacity?
    \item What would advancements in this capacity contribute to the reproductive success of humans?
\end{itemize}
