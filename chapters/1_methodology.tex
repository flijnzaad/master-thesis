\chapter{The chicken and the egg: exploring the evolutionary approach}

% introduction

Before we are able to answer any \emph{why}-questions about humans' cognitive capacities, some groundwork needs to be laid out. For what does it mean for some trait to evolve 'because of' or 'for the purpose of' another trait? Are we even justified in using this kind of terminology when it comes to evolution, which is a process which cannot be said to be intentional nor purposeful? And what questions will we need to ask ourselves in order to ultimately answer that big  \emph{why}-question?
This chapter attempts to answer these questions. It by no means provides an overview of issues in evolutionary theory; this is a large field of research in its own right, with widely diverging opinions on a number of specifics of evolution (see \citet{Ariew02} and \citet{UllerLaland19} for overviews of topics in evolutionary theory).
This chapter will merely serve to get a number of issues out of the way before we can continue our investigation into the cognitive mechanisms that make us human.

% \section{Evolutionary approaches to human psychology}
\section{The legitimacy of evolutionary psychology}

In order to answer our question, we will first zoom out to consider the field of evolutionary psychology as a whole. What is the merit, and the validity, of adopting an evolutionary approach in our endeavor?

The field of evolutionary psychology concerns itself with trying to understand human behavior through evolutionary theory, by looking into the past and considering how our ancestors must have adapted to their environment.
Researchers in the social sciences and humanities have historically been a tad wary of using evolutionary approaches to study human behavior, because evolutionary theory has been abused for prejudiced ends in the past \citep[pp.~19--20]{LB02}.

In order to explain humans' evolved psychological mechanisms, evolutionary psychologists look to the concept of an \emph{environment of evolutionary adaptedness} (EEA); they maintain that for our psychological mechanisms to be as complex as they are, they must have evolved slowly and thus at a considerable amount of time prior to the present. The EEA is then the environment in which these psychological mechanisms must have come to be; usually the EEA is identified as hunter-gatherer cultures on the African savannah in the Stone Age \citep{LB02}.

It is important to note here that not only biological evolution, but also cultural evolution can be said to have played a role in shaping human behavior and human cognitive capacities: this is a joint endeavor facilitated by nature and nurture. In this thesis, I will not attempt to delineate between what features are due to biological evolution and which are due to cultural evolution; for the purposes of this investigation, I believe I am justified in broadly considering them under the umbrella term of 'evolution'.

\section{Teleological notions in evolutionary theory}

Next, it is useful to scrutinize the terminology that I will be using throughout this thesis.

\section{Causation in evolution}

Now, we will dip our toes into the topic of causation in evolution.

\section{An evolutionary foundation}

Now that we have gathered sufficient puzzle pieces, it is time to take the score and lay the groundwork.
