\chapter{The chicken and the egg: exploring the evolutionary approach}

% introduction

Before we are able to answer any \emph{why}-questions about humans' cognitive capacities, some groundwork needs to be laid out. For what does it mean for some trait to evolve 'because of' or 'for the purpose of' another trait? Are we even justified in using this kind of terminology when it comes to evolution, which is a process which cannot be said to be intentional nor purposeful? And what questions will we need to ask ourselves in order to ultimately answer that big  \emph{why}-question?
This chapter attempts to answer these questions. It by no means provides an overview of issues in evolutionary theory; this is a large field of research in its own right, with widely diverging opinions on a number of specifics of evolution (see \citet{Ariew02} and \citet{UllerLaland19} for overviews of topics in evolutionary theory).
This chapter will merely serve to get a number of issues out of the way before we can continue our investigation into the cognitive mechanisms that make us human.

% \section{Evolutionary approaches to human psychology}
\section{The legitimacy of evolutionary psychology}

In order to answer our question, we will first zoom out to consider the field of evolutionary psychology as a whole. What is the merit, and the validity, of adopting an evolutionary approach in our endeavor?

The field of evolutionary psychology concerns itself with trying to understand human behavior through evolutionary theory, by looking into the past and considering how our ancestors must have adapted to their environment.
Researchers in the social sciences and humanities have historically been a tad wary of using evolutionary approaches to study human behavior, because evolutionary theory has been abused for prejudiced ends in the past \citep[pp.~19--20]{LB02}. However, evolutionary psychology has proved to be a illuminating field of study over the past decades.
\todo{that's kind of a weird comment, but i needed to counterbalance it somehow}

In order to explain humans' evolved psychological mechanisms, evolutionary psychologists look to the concept of an \emph{environment of evolutionary adaptedness} (EEA). The EEA is the environment in which these psychological mechanisms must have come into being; usually the EEA is identified as hunting and gathering groups on the African savannah in the second half of the Pleistocene, between 1.7 million and 10,000 years ago \citep{LB02}.
The assumptions underlying the use of the concept of an EEA are that (1) our modern-day environment is too different from that of our ancestors for us to use it to explain why and how our psychological mechanisms evolved in the past; and (2), for our psychological mechanisms to be as complex as they are, they must have evolved slowly and thus at a considerable amount of time prior to the present, and they have not changed significantly since the Stone Age.

There are a number of issues associated with the use of the concept of the EEA \citep{LB02}. Firstly, we do not know very much about the environment of our ancestors, so the specifics of the EEA may be filled in as is seen fit for one's purpose. Secondly, we do not know enough about the process of evolution to make assumption (2); while evolution does in general operate on a large time scale, research has shown that it can also be faster, operating on a time scale of thousand of years, or less than 100 generations \citep[pp.~190--191 and references therein]{LB02}. Thirdly, the argument can be made that for our species to have flourished and dominated in the way that it did, we must have remained adaptive to our changing modern environments after the Stone Age. Fourthly, the EEA argument does not take into account the fact that we actively partake in modifying and shaping our environment (see \cref{sec:causation-evolution} for more discussion on niche construction).

For our endeavor at hand, I accept the concept of the EEA in a general sense despite its flaws; for our purposes, we need not commit to any strong assumptions about the nature and properties of the EEA. The most important assumption I will make is that homo sapiens throughout history has been dependent on strong social groups for survival.
\todo{i will need to argue here why i DO make this assumption, because it's maybe not justified? see \citet[p.~178]{LB02}}

Another topic of discussion in evolutionary psychology that is of importance to our investigation, is that of domain specificity of the evolved psychological mechanisms. The argument has been made \citep[p.~50]{Buss15} that these adaptive mechanisms are problem- or domain-specific, because the evolutionary process would not favor general solutions to specific problems.
However, as with the EEA, there are some issues with this stance \citep{LB02}: the push to domain-specificity can be said to rely on too strong assumptions about the modularity of the brain; and moreover, there is also a push to domain-generality of cognitive skills because they are neurologically more cost-efficient.

\todo{should this last paragraph go in another section? in the last one? it's maybe not really related to this section}
Lastly, it is important to note here that not only biological evolution, but also cultural evolution can be said to have played a role in shaping human behavior and human cognitive capacities; it is a joint endeavor facilitated by nature and nurture. In this thesis, I will not attempt to delineate between what features are due to biological evolution and which are due to cultural evolution; for the purposes of this investigation, I believe I am justified in broadly considering them under the umbrella term of 'evolution'.

\section{Causation in evolution}
\label{sec:causation-evolution}

Now, we will dip our toes into the topic of causation in evolution.

\section{Teleological notions in evolutionary theory}

Next up, it is useful to scrutinize the terminology that I will be using throughout this thesis.
Biological literature frequently make use of teleological terminology, that is, terminology that implies goal-directedness of the processes it describes. Such terminology includes concepts like the \emph{design} of a trait, and \emph{function}, \emph{purpose}, or \emph{utility} of a trait.
At first glance, the usage of these terms in dealing with evolution would seem to be inappropriate; evolution is a natural process, that is not performed by an agent, not with any intentionality, direction, or end-goal. And indeed, this teleological terminology has its origin in pre-Darwinian and creationist views on the origin of species.

\todo{this segue needs to be rewritten}
However, they have considerable instrumental value in describing evolutionary processes. Throughout this thesis, I will be adhering to the conception of teleological explanations as proposed by \citet{Ayala99}, which is as follows:
\begin{quote}
    Teleological explanations account for the existence of a certain feature in a system by demonstrating the feature’s contribution to a specific property or state of the system, in such a way that this contribution is \emph{the reason why the feature or behaviour exists at all}.
    \\ \hfill
    % \citep[p.~13]{Ayala99}
    (p.~13)
\end{quote}
In this respect, the evolutionary process of adaptation merits a teleological explanation; the function of a trait (its 'contribution to a specific property or state of the system') is the reason that the trait exists, because it exists as a consequence of natural selection.
Natural selection can be said to be directed or oriented; the direction of evolution is determined by the functional utility of the trait to the organism.
\todo{do i need to sidenote this with the fact that it's too simplistic and/or ignores reciprocal causation and niche construction?}

The distinction between proximate and ultimate causes (\cref{sec:causation-evolution}) can be applied to teleological explanation resulting in the distinction between proximate and ultimate \emph{ends} of features. The proximate end is then the 'immediate' function the feature serves \todo{example}, and the ultimate end is the reproductive success of the organism.
\todo{is this relevant?} Ayala distinguishes between \emph{natural} and \emph{artificial} teleology, which concerns whether the feature is a result of a natural process or rather purposeful action, consciously intended by an agent.

A footnote to this whole story is that not all features of organisms can be explained teleologically; only if the feature has arisen and persisted as a direct result of natural selection, is a teleological explanation in place.
\todo{this "direct result of natural selection" is very vague, make a note of this?}

\section{An evolutionary foundation}

Now that we have gathered sufficient puzzle pieces, it is time to take the score and lay the groundwork.
