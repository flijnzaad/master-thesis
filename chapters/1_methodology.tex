\chapter{The chicken and the egg: the evolutionary approach}
\label{ch:evolution}
% introduction

Before we are able to answer any \emph{why}-questions about humans' cognitive capacities, some groundwork needs to be laid out. For what does it mean for some trait to evolve 'for the purpose of' another trait? Are we even justified in using this kind of terminology when it comes to evolution, a process which cannot be said to be intentional or purposeful? And what intermediate questions will we need to ask ourselves in order to ultimately answer the question of why we reason and communicate?
\largeTodo{Explicate why "why we communicate" is also an important question (maybe elsewhere)}
This chapter attempts to \smallTodo{Be confident!} answer these questions. It by no means provides an overview of issues in evolutionary theory, a large field of research in its own right with widely diverging opinions on a number of specifics of the process of evolution (see \citet{Ariew02} and \citet{UllerLaland19} for overviews of topics in evolutionary theory).
This chapter will serve to get a number of foundational issues 'out of the way' before we can continue our investigation into the cognitive mechanisms of reasoning and communication.
\largeTodo{Define evolution!}

\section{Causation in evolution}
\label{sec:causation-evolution}

First, we will dip our toes into the waters of causation in evolution.

Evolutionary causation is a subfield of philosophy of biology that has continued to see widely diverging opinions \citep{Baedke2021, UllerLaland19}. In this section I will restrict focus to three topics in evolutionary causation that are of interest to this thesis. First, I will discuss the distinction between proximate and ultimate causation. Then, I will mention \poscite{Tinbergen63} questions for explaining animal behavior, which will be discussed at length in \cref{sec:tinbergen}. Lastly, we will have a look at niche construction and reciprocal causation\smallTodo{Change this if what is discussed changes}.
The former two will serve to illuminate some basic aspects of the issues evolutionary causation concerns itself with, and the latter is of particular interest in considering the evolution of human cognition.

In evolutionary causation, one may distinguish \emph{proximate} from \emph{ultimate} causes.
Proximate causes are the \emph{immediate} influences on a trait: they explain how the trait results from the internal and external factors causing it.
Ultimate causes on the other hand are the higher-level historical and evolutionary explanation of those traits \citep{Mayr61}\largeTodo{Add an example to illustrate}. In other words, these two different causes answer two different explanatory questions \largeTodo{Apparently a category mistake: causes don't answer questions, \emph{statements} about causes answer questions. I'm not sure if I fully agree though; think about this}: the proximate cause is the answer to the \emph{how}-question, whereas the ultimate cause answers the \emph{why}-question. According to \citet{Mayr61}, who pioneered the distinction\footnote{See \citep{Laland13} for a discussion of the origins of the distinction}, one needs to answer both of these questions in order to obtain a complete understanding of the trait.
\smallTodo{Maybe reformulate this: still reads funky}

\largeTodo{After finishing \cref{sec:tinbergen}, modify this so it fits}
In an influential proposal considered by some \smallTodo{By whom? I don't remember, maybe it would be elucidating to know why and how they say so} to be an extension of this distinction, \citet{Tinbergen63} outlined four questions or problems central to the study of animal behavior. In order to fully understand a pattern of behavior, one must consider (1) the proximate causation of the behavior, (2) the lifetime development of the behavior, (3) the function\footnote{See \cref{sec:teleology} for a discussion of function in biology.} of the behavior, and (4) the evolutionary history of the behavior.

While Mayr's and Tinbergen's frameworks have had considerable explanatory value in the field of evolutionary causation, it has been argued that biologists would be better off rejecting them in favor of a framework of \emph{reciprocal causation} \citep{Laland13}.
In reciprocal causation, there is feedback between the organism and its environment: not only does the environment influence the organism through the process of natural selection, but the organism in turn influences its environment through the process of \emph{niche construction} \citep{Svensson18}, where organisms actively partake in modifying their environment and shaping their niche, usually to their own benefit.\largeTodo{Define niche and add an example to illustrate: discuss and elaborate way more!} It has been argued that niche construction is at least as important as natural selection in shaping evolution \largeTodo{Is niche construction governed by natural selection?}, and moreover that especially in evolutionary processes involving interactions between organisms, reciprocal causation is very frequent \citep{Svensson18}.
% Both concepts reject the unidirectional conception of evolutionary causation, and ascribe a causal role to the organism itself and acknowledge the existence of feedback loops.

% \section{Evolutionary approaches to human psychology}
\section{Evolutionary psychology and its issues}
\label{sec:evol-psych}
In order to answer the question of whether reasoning evolved for the purpose of communication, we will also need to zoom out to consider the field of evolutionary psychology as a whole. What is the merit, and the validity, of adopting an evolutionary approach in our endeavor?

The field of evolutionary psychology concerns itself with trying to understand human behavior using evolutionary theory, by looking into the past and considering how our ancestors must have adapted to their environment in order to survive and reproduce.
Researchers in the social sciences and humanities have historically been wary of using evolutionary approaches to study human behavior, because evolutionary theory has been abused for prejudiced ends in the past; see \citet[pp.~19--20]{LB02} for an overview\todo{Ask Karolina about feedback here: didn't understand}. Moreover, evolutionary-psychological research has received the criticism that too much of it is "just-so" storytelling and post-hoc explanation of known phenomena, sometimes accompanied by a sensationalist spin on the story \citep{LB02}.
However, if these pitfalls are avoided, looking at human psychology from the evolutionary perspective can be an illuminating endeavor.

Let us now consider some of the central concepts and assumptions of evolutionary psychology.

In order to explain humans' psychological mechanisms, evolutionary psychologists look to the concept of an \emph{environment of evolutionary adaptedness} (EEA). The EEA is the environment in which these psychological mechanisms must have come into being; usually the EEA is identified as hunting and gathering groups on the African savannah in the second half of the Pleistocene, between 1.7 million and ten thousand years ago \citep{LB02} \largeTodo{Are the groups the EEA or the savannah? Elaborate more on this: is it the physical environment or does it include the groups? Aren't the groups a trait that evolved? Think about this}.
The assumptions underlying the use of the concept of an EEA are that (1) our modern-day environment is too different from that of our ancestors for us to use it to explain why and how our psychological mechanisms evolved in the past; and (2), for our psychological mechanisms to be as complex as they are, they must have evolved slowly --- and thus a considerable amount of time ago, without changing significantly since the Stone Age.

There are a number of issues associated with the use of the concept of the EEA \citep{LB02}. Firstly, we do not know very much about the environment of our ancestors, so the specifics of the EEA may be filled in as is seen fit for one's purpose. Secondly, we do not know enough about the process of evolution to make assumption (2); while evolution does in general operate on a large timescale, research has shown that it can also be faster, operating on a time scale of thousands of years, or less than 100 generations \citep[pp.~190--191 and references therein]{LB02}. Thirdly, the argument can be made that for our species to have flourished and dominated in the way that it did, we must have remained adaptive to our changing modern environments after the Stone Age. Lastly, the EEA argument does not take into account reciprocal causation or niche construction.\largeTodo{Elaborate on this, because it seems to be important for my argument}

\largeTodo{The following section warrants a larger discussion about domain-specificity; define it and give examples (outside of psychology: maybe vision?), maybe with Cosmides/Tooby conception of Wason selection task as cheater detection? Find some literature on domain-specificity, because it turns out to be a bit of a hairy subject. Find definition; because opinions may vary on what features are and aren't domain-specific}
Another topic of discussion in evolutionary psychology that is of importance to our investigation, is that of domain-specificity of the psychological mechanisms. The argument has been made that these adaptive mechanisms are necessarily problem- or domain-specific, because the evolutionary process would not favor general solutions to specific problems \citep[p.~50]{Buss15}.
However, as with the EEA, issues with this stance have been raised: the push to domain-specificity can be said to rely on overly strong assumptions about the modularity of the brain; and moreover, there is also a push to domain-generality of cognitive skills because domain-general skills are neurologically more cost-efficient than domain-specific skills \citep{LB02}.

Lastly, a criticism raised to evolutionary psychology is that not only biological evolution, but also cultural evolution can be said\todo{This line in particular needs a citation} to have played a role in shaping human behavior and human cognitive capacities; it is a joint endeavor facilitated by nature and nurture.
\smallTodo{Add references: where did you read this? Or, if it's your own contribution, make that clear. If it's a summarizing contribution from multiple references, say that, and you don't need to cite them all -- use e.g.}

\section{Teleological notions in evolutionary theory}
\label{sec:teleology}
\largeTodo{This section needs quite some work: see notes from discussion with Karolina}
Next, it is useful to scrutinize the terminology that I will be using throughout this thesis.
Biological literature frequently makes use of \emph{teleological} terminology, that is, terminology that implies goal-directedness of the processes it describes. Such terminology includes concepts like the \emph{design} of a trait, and \emph{function}, \emph{purpose}, or \emph{utility} of a trait.
At first glance, the usage of these terms in discussing evolution would seem to be inappropriate; for evolution is a process of nature, not purposefully performed by an agent, and it is thus without any intentionality or goals.
\smallTodo{Maybe reformulate this sentence again: still too convoluted?}
And indeed, this teleological terminology has its roots in pre-Darwinian conceptions of nature: it originates from Aristotle's views on causation, and it was subsequently adopted by creationist Muslim and Christian scholars \citep{Johnson05}.

In general, teleological explanations in biology are quite controversial: not only is the usage of the specific terminology itself debated \citep[p.~27 and references therein]{Ayala99}, the concept has been criticized for its apparent lack of formalization and insufficient argumentative persuasiveness \citep[p.~83]{Baedke2021}.
\largeTodo{Address the controversy around teleological explanations. Talk about instrumentalism, usefulness of the concepts. Lack of formalization is not such a big problem for the purpose here maybe, but the other thing is more of a problem. Address why they won't be a problem for you. Can mention that MS assume it as well, this teleological explanation is at the heart of their thesis (quote it?), so it's their problem to defend this. I work using the same assumptions as them.}

However, explanations in terms of goals and function have considerable instrumental value in describing evolutionary processes. Throughout this thesis, I will be adhering to the conception of teleological explanations of \citet{Ayala99}, which is as follows:
\begin{quoting}
    Teleological explanations account for the existence of a certain feature in a system by demonstrating the feature’s contribution to a specific property or state of the system, in such a way that this contribution is \emph{the reason why the feature or behaviour exists at all}.
    \hfill (p.~13)
\end{quoting}
In this respect, the evolutionary process of adaptation merits a teleological explanation: the function of a trait (its 'contribution to a specific property or state of the system') is the reason that the trait exists, because it exists as a consequence of natural selection.
\smallTodo{Add example here}
\largeTodo{Is this view compatible with reciprocal causation and niche construction? I think so; they're a complication for the whole picture, not necessarily for using this definition.}
\largeTodo{Is this view compatible with cultural learning? From the quote, it doesn't necessarily follow that it's about biology necessarily. Think about this, and after writing a section on culture, state to what extent and in what way we'll adhere to \citet{Ayala99}}

The distinction between proximate and ultimate causes we saw in \cref{sec:causation-evolution} can be applied to teleological explanation as well, yielding the distinction between proximate and ultimate \emph{ends} of features. The proximate end is then the 'immediate' function the feature serves, and the ultimate end is the reproductive success of the organism.
\smallTodo{Add example}

A footnote to this account is that not all features of organisms can be explained teleologically; only if the feature has arisen and persisted as a direct result of natural selection, a teleological explanation is in place.
\largeTodo{This "direct result of natural selection" is very vague/slippery; acknowledge this, and elaborate more on it if it turns out to be important for my thesis. A way to do this would be to contrast it with an indirect result. Talk about side effects?}


\section{An evolutionary foundation}
\label{sec:evol-method}
Now that we have gathered the puzzle pieces, let us return focus to our field of investigation and lay its groundwork. In particular, before we can continue with our evolutionary approach to human reasoning and communication in the next chapters, we will need to outline and justify the assumptions made.

It has become apparent that for none of the topics in evolutionary theory discussed here consensus has been reached among its practitioners. Since the purpose of this thesis will not be to provide a complete causal framework for the evolution of reasoning and communication, we will be able to cast aside some of the issues plaguing the frameworks discussed in this chapter. We will proceed cautiously, using the concepts outlined without needing to account in detail for their shortcomings.
\smallTodo{Change structure to drop headings: make sure that the content of the pargraph is still clear without using the headings}

\paragraph{Why can we consider these cognitive capacities from an evolutionary perspective?}

In choosing to explore human reasoning and communication from the evolutionary perspective, I am working on the assumption that it is appropriate to the endeavor: reasoning and communication are two key features of the homo sapiens that must have arisen at some point in our evolutionary history, shaped by natural selection, niche construction, and perhaps (cultural) learning. Moreover, it is justified to assume that the cognitive abilities underlying human reasoning and communication are heritable, either genetically or culturally.

\paragraph{What do we mean when we say 'evolution'?}
\largeTodo{Add a section on culture to this chapter (where?): add things from \citet[Chapter~2: "Nature, nurture, culture"]{Heyes18}, and maybe some things from \citet{Laland13} and \citet{LB02}. In there, try to figure out if the parallel between biological and cultural evolution is strong enough that you can assume that the specific driving force behind the evolution is not important. Also, distinguish clearly between (and define) cultural evolution and cultural learning. Taught vs. inherited.}
So far, we have mostly considered the concepts and issues surrounding \emph{biological} evolution, and the role of \emph{culture} has remained largely undiscussed.
In this thesis, I will use the term 'evolution' and associated terminology to talk about the development of cognitive capacities in humans over time, remaining agnostic about whether this evolution is due to biological processes like natural selection, or cultural processes like cultural learning.

\paragraph{What are our assumptions about the evolutionary process?}
I accept the concept of the EEA in a general sense despite the flaws discussed in \cref{sec:evol-psych}; for our purposes, we need not commit to any strong assumptions about the nature and properties of the EEA. The most important assumption I will make is that homo sapiens throughout history has been dependent on strong social groups for survival.
\largeTodo{Missing: strong argument for making this assumption; see \citet[p.~178]{LB02} for doubts about this assumption}

Moreover, I will adopt a reciprocal causal framework along the lines of \citet{Laland13} and \citet{Svensson18}, maintaining that humans throughout history have not only been the object of natural selection, but that they have also been the subject in shaping their environments through the formation of social groups and development of technology.

\paragraph{What does it entail for one trait to evolve for the purpose of another?}
In using \poscite{Ayala99} definition of teleological explanation, we can speak of some feature of an organism having a certain function or purpose, that ultimately contributes to its reproductive success. In the case of our investigation, we are interested in what evolutionary benefits one feature (reasoning) may have to another feature (communication). In order to answer this question, we should first consider what the function of communication is to humans; only then can we consider whether the function of reasoning could be to advance communication and in doing so improve the reproductive fitness of humans.
\largeTodo{Elaborate a lot on this: this is a key key point of my research question, so it deserves more attention. Settle on a definition.}

\subsection{Adopting Tinbergen's four questions}
\label{sec:tinbergen}
\todo[inline]{Problems? Questions? Settle on one and be consistent throughout the chapter. Also, settle on whether these problems are about animals or organisms in general.}
As mentioned in \cref{sec:causation-evolution}, \citet{Tinbergen63} proposed an influential framework of problems that should be addressed if one intends to give a complete account of a behavior an animal exhibits. Let us dive more deeply into Tinbergen's framework here, as it will prove to form a desirable foundation for the investigations this thesis concerns itself with.

As mentioned before, the four problems that Tinbergen argued to be central to the study of behavior are causation, survival value, ontogeny, and evolution. I will now discuss each of these problems in more detail, such that we can ultimately come to a set of methodological questions to guide us in investigating human communication and reasoning.

\subsubsection{Causation}
The first problem Tinbergen outlines, is that of the mechanistic causation of the behavior, in other words the proximate causation of the behavior. In our case, addressing this problem would entail a detailed investigation of the neurological processes underlying communicative and reasoning behaviors.
This problem, however interesting, will not be addressed in this thesis. The reason for this is that more empirical and conceptual research would be necessary in order to give a satisfactory account of the exact neurological processes underlying communicative and reasoning behavior. Although one can only gain a full understanding of a behavior if the four problems are addressed simultaneously \citep{Tinbergen63, BatesonLaland13}, I believe I am justified in considering the proximate-causation problem out of scope for this thesis.

\subsubsection{Survival value}
The second Tinbergen problem relates to the value a behavior provides to an animal's survival. This survival value, in teleological terms, is the function of the behavior. However, the use of the term 'function' obscures the fact that a characteristic's function may change over time: the \emph{current} utility that a characteristic has, may not be the same as the \emph{original} utility it had \citep{BatesonLaland13}. For example, feathers originally evolved for temperature regulation in the evolutionary predecessors of birds, and were later adapted for flight \citep{Benton19, BatesonLaland13}.

\subsubsection{Ontogeny}
\subsubsection{Evolution}
\subsubsection{A fifth Tinbergen question: observation and description}
\subsection{The methodological framework for this thesis}
\todo{Change this title}

\begin{itemize}
    \item How does the capacity to reason and the capacity to communicate develop throughout childhood?
    \item What would advancements in reasoning or communication contribute to the reproductive success of humans? In other words, what is the function of reasoning and the function of communication?
    \item What is the evolutionary history of reasoning and of communication? To what extent and how do they present in our nearest evolutionary relatives?
\end{itemize}
One may note that we will skip Tinbergen's first question about what the proximate causation of the behavior in question is. I consider the neurological mechanisms underlying reasoning and communication to be out of the scope of this thesis. The most important reason for this is that we do not know nearly enough about the human brain to provide a satisfactory account of the proximate causation of reasoning and communication.
\largeTodo{Add in this section a word on how Tinbergen's questions can be split into proximate and ultimate answers: cite \citet{Rymer22}, and maybe see Wikipedia for extra references. Also mentioned by \citet{BatesonLaland13}}
