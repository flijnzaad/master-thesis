\thispagestyle{empty}
\begin{center}
    \large \textbf{Abstract}
\end{center}

\noindent Hugo Mercier and Dan Sperber's argumentative theory of reasoning (ATR) posits that the function of reasoning is \emph{argumentative}; that is, reasoning evolved in humans to facilitate the production of arguments and the evaluation of those of others. In this way, reasoning serves to improve communication, the stability of which is threatened by dishonest communicators.

This thesis critically examines the argumentative theory of reasoning
from the perspective of the evolution of human communication, and concludes that the theory is unsatisfactory in a number of ways.

After sketching out the details and assumptions of the evolutionary perspective inherent to this endeavor, we discuss the evolution of human communication in detail. In particular, we analyze the evolution of human cooperation and how the evolution of human communication relates to this.
We dissect the argumentative theory of reasoning, and subsequently critically examine its component parts.
We conclude that in general, the ATR lacks the detail needed to justify its debatable implicit assumptions. Moreover, the ATR is divorced from the broader context of human cooperative communication, which diminishes the theory's evolutionary plausibility.
Finally, we conjecture that the ATR might be salvaged by tethering it to the aforementioned cooperative context and providing the necessary details to substantiate the as of yet questionable assumptions. However, doing so may require significant modifications to the theory.
