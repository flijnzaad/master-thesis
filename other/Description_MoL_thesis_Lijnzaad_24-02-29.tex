\documentclass{article}
\begin{document}
\noindent This thesis will aim to investigate why humans reason and communicate in the way that they do, viewed from an evolutionary perspective. It will scrutinize the argumentative theory of reasoning (put forward by cognitive scientists and philosophers Hugo Mercier and Dan Sperber), which posits that reasoning has evolved in humans for the purpose of devising and evaluating arguments. In scrutinizing this theory, I will first consider literature from evolutionary theory on how causation in evolution works and how teleological terminology such as "function" or "purpose" may be used. Then, I will consider empirical research on reasoning and communication as it develops in children, and look at the extent to which these skills are present in our closest primate relatives. I will use this to draw a conclusion on the possible function of both communication and reasoning. Moreover, I will consider empirical research at the intersection of reasoning and communication -- research about pragmatic inference.
Finally, I will use all this to then draw a conclusion on whether reasoning could have evolved for communicative purposes.
\end{document}
